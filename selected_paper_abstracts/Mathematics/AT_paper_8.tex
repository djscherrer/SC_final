Neural networks can be thought of as applying a transformation to an input dataset. The way in which they change the topology of such a dataset often holds practical significance for many tasks, particularly those demanding non-homeomorphic mappings for optimal solutions, such as classification problems. In this work, we leverage the fact that neural networks are equivalent to continuous piecewise-affine maps, whose rank can be used to pinpoint regions in the input space that undergo non-homeomorphic transformations, leading to alterations in the topological structure of the input dataset. Our approach enables us to make use of the relative homology sequence, with which one can study the homology groups of the quotient of a manifold $\mathcal{M}$ and a subset $A$, assuming some minimal properties on these spaces. 
    
    As a proof of principle, we empirically investigate the presence of low-rank (topology-changing) affine maps as a function of network width and mean weight. We show that in randomly initialized narrow networks, there will be regions in which the (co)homology groups of a data manifold can change. As the width increases, the homology groups of the input manifold become more likely to be preserved. We end this part of our work by constructing highly non-random wide networks that do not have this property and relating this non-random regime to Dale's principle, which is a defining characteristic of biological neural networks.
    Finally, we study simple feedforward networks trained on MNIST, as well as on toy classification and regression tasks, and show that networks manipulate the topology of data differently depending on the continuity of the task they are trained on.