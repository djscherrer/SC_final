The majority of standard approaches to financial portfolio optimization (PO) are based on the mean-variance (MV) framework. Given a risk aversion coefficient, the MV procedure yields a single portfolio that represents the optimal trade-off between risk and return. However, the resulting optimal portfolio is known to be highly sensitive to the input parameters, i.e., the estimates of the return covariance matrix and the mean return vector. It has been shown that a more robust and flexible alternative lies in determining the entire region of near-optimal portfolios. In this paper, we present a novel approach for finding a diverse set of such portfolios based on quality-diversity (QD) optimization. More specifically, we employ the CVT-MAP-Elites algorithm, which is scalable to high-dimensional settings with potentially hundreds of behavioral descriptors and/or assets. The results highlight the promising features of QD as a novel tool in PO.

\keywords{quality-diversity \and illumination algorithm \and MAP-Elites \and portfolio optimization \and near-optimal portfolios}