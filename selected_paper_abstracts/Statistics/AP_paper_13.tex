A common problem in numerous research areas, particularly in clinical trials, is to test whether the effect of an explanatory variable on an outcome variable is equivalent across different groups. In practice, these tests are frequently used to compare the effect between patient groups, e.g. based on gender, age or treatments. Equivalence is usually assessed by testing whether the difference between the groups does not exceed a pre-specified equivalence threshold. 
    Classical approaches are based on testing the equivalence of single quantities, e.g. the mean, the area under the curve (AUC) or other values of interest. However, when differences depending on a particular covariate are observed, these approaches can turn out to be not very accurate. Instead, whole regression curves over the entire covariate range, describing for instance the time window or a dose range, are considered and tests are based on a suitable distance measure of two such curves, as, for example, the maximum absolute distance between them.
    In this regard, a key assumption is that the true underlying regression models are known, which is rarely the case in practice. However, misspecification can lead to severe problems as inflated type I errors or, on the other hand, conservative test procedures.

    In this paper, we propose a solution to this problem by introducing a flexible extension of such an equivalence test using model averaging in order to overcome this assumption and making the test applicable under model uncertainty. Precisely, we introduce model averaging based on smooth AIC weights and we propose a testing procedure which makes use of the duality between confidence intervals and hypothesis testing. We demonstrate the validity of our approach by means of a simulation study and demonstrate the practical relevance of the approach considering a time-response case study with toxicological gene expression data.