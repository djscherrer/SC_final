The use of automatic short answer grading (ASAG) models may help alleviate the time burden of grading while encouraging educators to frequently incorporate open-ended items in their curriculum. However, current state-of-the-art ASAG models are large neural networks (NN) often described as "black box", providing no explanation for which characteristics of an input are important for the produced output. This inexplicable nature can be frustrating to teachers and students when trying to interpret, or learn from an automatically-generated grade. To create a powerful yet intelligible ASAG model, we experiment with a type of model called a Neural Additive Model that combines the performance of a NN with the explainability of an additive model. We use a Knowledge Integration (KI) framework from the learning sciences to guide feature engineering to create inputs that reflect whether a student includes certain ideas in their response. We hypothesize that indicating the inclusion (or exclusion) of predefined ideas as features will be sufficient for the NAM to have good predictive power and interpretability, as this may guide a human scorer using a KI rubric. We compare the performance of the NAM with another explainable model, logistic regression, using the same features, and to a non-explainable neural model, DeBERTa, that does not require feature engineering.

\keywords{Explainable AI \and Automatic Grading \and ASAG \and Neural Additive Models.}