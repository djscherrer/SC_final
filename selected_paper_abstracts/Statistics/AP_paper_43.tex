This paper investigates the distribution of public school expenditures across U.S. school districts using a bayesian maximum entropy model.  Covering the period 2000-2016, I explore how inter-jurisdictional competition and household choice influence spending patterns within the public education sector, providing a novel empirical treatment of the Tiebout hypothesis within a statistical equilibrium framework.  The analysis reveals that these expenditures are characterized by sharply peaked and positively skewed distributions, suggesting significant socio-economic stratification. Employing Bayesian inference and Markov Chain Monte Carlo (MCMC) sampling, I fit these patterns into a statistical equilibrium model to elucidate the roles of competition, as well as household mobility and arbitrage in shaping the distribution of educational spending. The analysis reveals how the scale parameters associated with competition and household choice critically shape the equilibrium outcomes. The model and analysis offer a statistical basis for shaping policy measures intended to affect distributional outcomes in scenarios characterized by the decentralized provision of local public goods.