\noindent 
Understanding input substitution and output transformation possibilities is critical for efficient resource allocation and firm strategy. There are important examples of fixed proportion technologies where certain inputs are non-substitutable and/or certain outputs are non-transformable. However, there is widespread confusion about the appropriate modeling of fixed proportion technologies in data envelopment analysis. We point out and rectify several misconceptions in the existing literature, and show how fixed proportion technologies can be correctly incorporated into the axiomatic framework. A Monte Carlo study is performed to demonstrate the proposed solution.
\\[5mm]
\textbf{Keywords}: Data envelopment analysis; Fixed proportion technology; Production theory; Weight restrictions
\\[2mm]
\textbf{JEL Codes}: C14; C61; D24