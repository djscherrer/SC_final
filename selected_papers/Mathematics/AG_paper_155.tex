\begin{document}
\affiliation{$$_affiliation_$$}
\title{Non-existence of tensor t-structures on singular noetherian schemes}
\maketitle

 \begin{abstract}
     We show that there are no non-trivial tensor t-structures on the category of perfect complexes of a singular irreducible finite-dimensional noetherian scheme. To achieve this, we establish some technical results on Thomason filtrations and corresponding tensor t-structures.
 \end{abstract}
In this note, we prove that there exist no non-trivial tensor t-structures on the derived category of perfect complexes of a singular irreducible noetherian scheme of finite dimension. This theorem (Theorem \ref{main}) is a generalisation of the affine case, which was proven by Smith, see  \cite[Theorem 6.5]{smith2019bounded}. Both the affine result of Smith, and the non-affine result we obtain in this document, are proved via the classification of compactly generated tensor t-structures in terms of Thomason filtrations. The classification of compactly generated t-structures on $\mathbf{D}(R)$ for a commutative noetherian ring $R$ in terms of Thomason filtrations was done in \cite[Theorem 3.11]{Alonso_Tarr_o_2010}. More recently, a generalisation of this classification to compactly generated tensor t-structures on $\mathbf{D}_{\quasicoh}(X)$ for a noetherian scheme $X$ has been obtained in \cite[Theorem 4.11]{sahoo2023compactly}, which is what our proof utilises to extend to the non-affine case. 


Note that Theorem \ref{main} cannot be true in the non-affine setting without some additional hypothesis on the class of t-structures (in our case, tensor compatibility is the additional hypothesis) as there are known examples of non-trivial t-structures on $\mathbf{D}^{\perf}(X)$, for $X$ a singular variety, which arise from semi-orthogonal decompositions. This is unlike the affine case, since tensor compatibility is trivially satisfied by any t-structure on $\mathbf{D}^{\perf}(R)$.

\section{Background and Notation}
In this section we briefly recall some notation, definitions, and results used in this paper.

Let $X$ be a noetherian scheme. Throughout this note, $\mathbf{D}_{\quasicoh}(X)$ denotes the unbounded derived category of cochain complexes of $\mathcal{O}_X$-modules with quasicoherent cohomology, and $\mathbf{D}^{\perf}(X)$ denotes the derived category of perfect complexes on $X$. Let $Z\subseteq X$ be a closed subset, then we denote the corresponding full subcategories of complexes whose cohomology is supported on $Z$ by $\mathbf{D}^{}_{\quasicoh, Z}(X)$ and $\mathbf{D}^{\perf}_{Z}(X)$ respectively. Moreover, for a commutative ring $R$, $\mathbf{D}(R)$ denotes the unbounded derived category of cochain complexes of $R$-modules, and $\mathbf{D}^{\perf}(R)$ denotes the derived category of perfect complexes of $R$-modules.

\begin{Def}
    Let $X$ be a noetherian scheme, and let $(\mathcal{U}, \mathcal{V})$ be a t-structure on $\mathbf{D}_{\quasicoh}(X)$. We say that $(\mathcal{U}, \mathcal{V})$ is a \textit{tensor t-structure} if $\mathbf{D}_{\quasicoh}(X)^{\leq 0} \otimes \mathcal{U} \subseteq \mathcal{U}$, where $\mathbf{D}_{\quasicoh}(X)^{\leq 0}$ denotes the aisle of the standard t-structure on $\mathbf{D}_{\quasicoh}(X)$. Moreover, a tensor t-structure on $\mathbf{D}^{\perf}(X)$ is a tensor t-structure on $\mathbf{D}_{\quasicoh}(X)$ which restricts to a t-structure on $\mathbf{D}^{\perf}(X)$.
\end{Def}



Note that when $X=\Spec R$ , every compactly generated t-structure on $\mathbf{D}(R)$ is automatically a tensor t-structure, since the standard t-structure is compactly generated by the tensor unit $R$. 

We now state the classification of compactly generated tensor t-structures on $\mathbf{D}_{\quasicoh}(X)$ for a noetherian scheme X. Towards this end, we begin by recalling the definition of Thomason subsets and Thomason filtrations. Note that Thomason sets and filtrations can be defined in a more general setting, but we only state the definition for noetherian schemes.

\begin{Def}
    Let $X$ be a noetherian scheme, then a \emph{Thomason subset} of $X$ is precisely a specialisation closed subset. A \emph{Thomason filtration} is a sequence $\Phi = \{Z^i\}_{i \in \mathbb{Z}}$ such that each $Z^i$ is a Thomason subset of $X$, and $Z^i \supseteq Z^{i+1}$ for all $i \in \mathbb{Z}$.
\end{Def}

In \cite{sahoo2023compactly}, the authors obtain the following classification theorem for compactly generated tensor t-structures on noetherian schemes, generalising the classification of compactly generated t-structures on a noetherian ring from \cite{Alonso_Tarr_o_2010}.

\begin{citedthm}[{\cite[Theorem 4.11]{sahoo2023compactly}}]\label{Theorem 4.11 on Classification DS}
    
    Let $X$ be a noetherian scheme. Then, there is a bijective correspondence between the collection of compactly generated tensor t-structures on $\mathbf{D}_{\quasicoh}(X)$ and the collection of Thomason filtrations on $X$. 
\end{citedthm}

We give one of the maps of the bijective correspondence explicitly in the following remark, as we will make use of it later.
\begin{Rem}\label{Thomason filtration to tensor aisle}
    Let $\Phi = \{Z^i\}$ be a Thomason filtration, then the aisle of the corresponding t-structure is given by \[\mathcal{U}_\Phi \coloneq \{E \in \mathbf{D}_{\quasicoh}(X)\mid \Supp(\mathcal{H}^i(E)) \subseteq Z^i \text{ for all } i \in \mathbb{Z}\}.\]
    Given a Thomason filtration $\Phi$ we will denote the corresponding tensor t-structure by $(\mathcal{U}_\Phi, \mathcal{V}_\Phi)$ and the truncation functors by $\tau_{\Phi}^{\leq 0}$ and $ \tau_{\Phi}^{\geq 1}$.
\end{Rem}
We recall the following definition from \cite{sahoo2023compactly}.
\begin{Def}
    Given a collection of objects $\mathcal{A} \subset \mathcal{T}$, we denote the smallest cocomplete preaisle of $\mathcal{T}$ containing $\mathcal{A}$ by $\langle \mathcal{A}\rangle^{\leq 0}$. Let $X$ be a scheme and $j:U \hookrightarrow X$ an open immersion. Let $\mathcal{U}$ be a preaisle on $X$, then we define the restriction of the preaisle to $U$ by $\mathcal{U}|_U:=\langle j^\ast U\rangle^{\leq 0}$.   
\end{Def}

\begin{citedlemma}[{\cite[Lemma 4.6]{sahoo2023compactly}}]\label{Lemma 4.6 DS}
    Let $X$ be a noetherian scheme and $(\mathcal{U},\mathcal{V})$ a tensor t-structure on $\mathbf{D}_{\quasicoh}(X)$, and $U$ an open affine subscheme. Then, $(\mathcal{U}|_{U},\mathcal{V}|_{U})$ is a t-structure on $\mathbf{D}_{\quasicoh}(U)$. Further, for any $F \in \mathbf{D}_{\quasicoh}(X)$, the truncation triangle with respect to this t-structure is given by, \[j^*(\tau_{\mathcal{U}}^{\leq 0}F) \to j^*(F) \to j^*(\tau_{\mathcal{U}}^{\geq 1}F) \to j^*(\tau_{\mathcal{U}}^{\leq 0}F)[1]\]
    where $\tau_{\mathcal{U}}^{\leq 0}$ and $\tau_{\mathcal{U}}^{\geq 1}$ are the truncation functors for $(\mathcal{U},\mathcal{V})$.
\end{citedlemma}
We now state some results from \cite{smith2019bounded} which we will need in the following section.


\begin{citedlemma}[{\cite[Lemma 6.2]{smith2019bounded}}]\label{Lemma 6.2 Smith}
    Let $R$ be a noetherian ring, $\mf{p}$ a prime ideal, and $\Phi = \{Z^i\}$ a Thomason filtration such that $(\mathcal{U}_\Phi, \mathcal{V}_\Phi)$ restricts to a t-structure on $\mathbf{D}^{\perf}(R)$. We define the Thomason filtration $\Psi = \{Z^i \cap \Spec R_{\mf{p}}\}$ on $\Spec R_{\mf{p}}$. Then, the t-structure $(\mathcal{U}_\Psi, \mathcal{V}_\Psi)$ restricts to $\mathbf{D}^{\perf}(R_{\mf{p}})$.
\end{citedlemma}

\begin{citedlemma}    [{\cite[Lemma 6.3]{smith2019bounded}}]\label{Theorem 6.3 Smith}
    Let $R$ be a noetherian ring and $\Phi = \{Z^i\}$ a Thomason filtration such that $Z^i = \varnothing$ for large $i$, and there exists an integer $j$ such that $Z^j$ has a non-trivial intersection with the singular locus of $\Spec R$. Then, $(\mathcal{U}_{\Phi}, \mathcal{V}_{\Phi})$ does not restrict to a t-structure on $\mathbf{D}^{\perf}(R)$.
\end{citedlemma}
We recall the following well-known definition.
\begin{Def}
    Let $X$ be a scheme, and $Z\subset X$ be any subset. We define the \textit{height} of $Z$, denoted by $\height(Z)$, to be the infimum over the dimensions of the local rings at all $x\in Z$. 
\end{Def}
The following is a combination of \cite[Proposition 5.3 \& Corollary A.5]{smith2019bounded}. Also see the proof of \cite[Theorem 6.4]{smith2019bounded}.

\begin{Th}\label{Theorem 5.3 and A.5 Smith}
    Let $R$ be a noetherian ring. Let $\Phi$ be a Thomason filtration on $\Spec R$ such that $\height(Z^i) = h \geq 1$ for all $a \leq i \leq a+h$. Then, $H^{a+h}(\tau_\Phi^{\leq 0}R[-a])$ is not finitely generated as an $R$-module.
\end{Th}



\section{Proof of the main result}
\begin{Lemma}\label{Thomason filtrations restrict nicely}
    Let $X$ be a notherian scheme and $U$ an open affine subscheme. Let $\Phi=\{Z^i\}$ be a Thomason filtration on $X$. Let $\Phi'=\{Z^i \cap U\}$ be the restricted Thomason filtration on $U$. Then, $\mathcal{U}_{\Phi'}=\mathcal{U}_\Phi|_U$. 
\end{Lemma}
\begin{proof}
    We first prove the easier inclusion, $\mathcal{U}_{\Phi'}\supseteq\mathcal{U}_\Phi|_U$. It is easy to see that that $j^*(\mathcal{U}_\Phi) \subseteq \mathcal{U}_{\Phi'}$ as the support condition is satisfied trivially, see Remark \ref{Thomason filtration to tensor aisle}. As $\mathcal{U}_{\Phi'}$ is an aisle, it further contains $\mathcal{U}_\Phi|_U$, which is the cocomplete pre-aisle generated by $j^*(\mathcal{U}_\Phi)$. 

    Let $U = \Spec(R)$. Then, for the other inclusion, it is enough to show that for each prime $\mathfrak{p} \in Z^i \cap U \subseteq \Spec(R)$, there is a complex $K\in \mathbf{D}^{\perf}(X)$ with $K[-i]\in \mathcal{U}_\Phi$ such that $j^*(K)=K(\mf{p})$. Note that the Koszul complex $K(\mf{p})$ is supported on the closed subset $V(\mf{p}) \subseteq U$. Hence, it lies in $\mathbf{D}_{\quasicoh, V(\mf{p})}(U)$. Now, by \cite[Corollary 3.5 and Remark 3.6]{neeman2020bounded}, there is an equivalence $j_* : \mathbf{D}_{\quasicoh, V(\mf{p})}(U) \leftrightarrows \mathbf{D}_{\quasicoh, V(\mf{p})}(X) : j^*$, which restricts on the compacts to the equivalence, $j_! = j_* : \mathbf{D}^{\perf}_{ V(\mf{p})}(U) \leftrightarrows \mathbf{D}^{\perf}_{V(\mf{p})}(X) : j^*$. We define $K$ to be $j_*(K(\mf{p}))$, which is a perfect complex by the above discussion. Then, $K[-i]$ lies in $\mathcal{U}_\Phi$. Finally, note that $K(\mf{p}) \cong j^*(j_*(K(\mf{p}))) = j^*(K)$, which is what we needed. 
\end{proof}


\begin{Lemma}\label{restricting to affines}
       Let $X$ be a noetherian scheme. Let $\Phi=\{Z^i\}$ be a Thomason filtration on $X$ corresponding to a tensor t-structure on $\mathbf{D}_{\quasicoh}(X)$ which restricts to a t-structure on $\mathbf{D}^{\perf}(X)$. Let $U$ be an affine open subset of $X$. Then the t-structure corresponding to the filtration $\Phi'=\{Z^i \cap \Spec R\}$ on $\mathbf{D}(R)$ restricts to $\mathbf{D}^{\perf}(R)$.
\end{Lemma}

\begin{proof}
    By Lemma \ref{Lemma 4.6 DS}, $\mathcal{U}_\Phi|_U$ is an aisle and, for any $F \in \mathbf{D}_{\quasicoh}(X)$, the truncation triangle with respect to this aisle is given by, $j^*(\tau_{\Phi}^{\leq 0}F) \to j^*(F) \to j^*(\tau_{\Phi}^{\geq 1}F) \to j^*(\tau_{\Phi}^{\leq 0}F)[1]$. By Lemma \ref{Thomason filtrations restrict nicely}, we know that $\mathcal{U}_\Phi|_U = \mathcal{U}_{\Phi'}$, where $\Phi'$ is the Thomason filtration given by $\{Z^i \cap U\}$. 

    Now, we need to show that this aisle restricts to an aisle on $\mathbf{D}^{\perf}(R)$. That is, we need to show that the truncation triangles with respect to $\mathcal{U}_{\Phi'}$ respect perfect complexes. Let $U = \Spec(R)$. Then, we already know that $j^*(\tau_{U}^{\leq 0}\mathcal{O}_X) \to R \to j^*(\tau_{U}^{\geq 1}\mathcal{O}_X) \to j^*(\tau_{U}^{\leq 0}\mathcal{O}_X)[1]$ is the truncation triangle for $R$. As $j^*$ preserves perfect complexes, we get that $j^*(\tau_{U}^{\leq 0}\mathcal{O}_X) = \tau_{\Phi'}^{\leq 0}R$ and $j^*(\tau_{U}^{\geq 1}\mathcal{O}_X) = \tau_{\Phi'}^{\geq 1}R$ are perfect complexes. Note that $\tau_{\Phi'}^{\leq 0}$ and $\tau_{\Phi'}^{\geq 1}$ respects summands, extensions, and shifts. As $R$ is a classical generator for $\mathbf{D}^{\perf}(R)$, we get that the truncation triangles respect perfect complexes of $\mathbf{D}(R)$.
\end{proof}


\begin{Lemma}\label{restricting to local rings}
    Let $X$ be a noetherian scheme. Let $\Phi=\{Z^i\}$ be a Thomason filtration on $X$ corresponding to a tensor t-structure on $\mathbf{D}_{\quasicoh}(X)$ which restricts to $\mathbf{D}^{\perf}(X)$. Let $x\in X$ be some point. Then the filtration $\Phi'=\{Z^i \cap \Spec \mathcal{O}_{x}\}$  on $\Spec \mathcal{O}_{x}$ corresponds to a t-structure on $\mathbf{D}(\mathcal{O}_{x})$ which restricts to $\mathbf{D}^{\perf}(\mathcal{O}_{x})$.
\end{Lemma}

\begin{proof}
     From Lemma \ref{restricting to affines} we can reduce the problem to the case where $X$ is affine. Moreover, the affine case is already covered by \ref{Lemma 6.2 Smith}.
\end{proof}

\begin{Lemma}\label{singular point}
   Let $X$ be a singular noetherian scheme. Let $\Phi=\{Z^i\}$ be a Thomason filtration on $X$ such that there exists an $r$ with $Z^r=\varnothing$ and an $s$ with $Z^s$ containing a singular point of $X$. Then the compactly generated tensor t-structure on $\mathbf{D}_{\quasicoh}(X)$ corresponding to this Thomason filtration does not restrict to $\mathbf{D}^{\perf}(X)$.
\end{Lemma}

\begin{proof}
    Suppose for contradiction that the t-structure does restrict to the $\mathbf{D}^{\perf}(X)$. Now pick any affine open set $\Spec R$ containing the singular point in question. By Lemma \ref{restricting to affines} the filtration $\Phi'=\{Z^i \cap \Spec R\}$ restricts to $\mathbf{D}^{\perf}(R)$, so the question is local. Now we apply Lemma \ref{Theorem 6.3 Smith}, which gives us a contradiction.
\end{proof}

\begin{Lemma}\label{height}
    Let $X$ be a noetherian scheme and let $\Phi=\{Z^i\}$ be a Thomason filtration on $X$ corresponding to a tensor t-structure on $\mathbf{D}_{\quasicoh}(X)$ which restricts to $\mathbf{D}^{\perf}(X)$. If $\dim X \geq h\geq 1$ then there are at most $h$ many consecutive $Z^i$'s of height $h$.
\end{Lemma}

\begin{proof}
    Suppose for a contradiction that there are $h+1$ many consecutive $Z^i$'s of height $h$, that is, there exists an $a$ such that for $i$ in the interval $[a, a+h]$, each $Z^i$ has height $h\geq 1$. Take a point $x\in Z^{a+h}$ which is minimal, that is, the point $x$ corresponds to a prime ideal $\mathfrak{p}$ in some affine open set $\Spec R$ such that $\height(\mf{p})=\height(Z^{a+h})$.  Consider the Thomason filtration $\Phi'=\{Z^i \cap \Spec R\}$ on $\Spec R$. By our assumption combined with Lemma \ref{restricting to affines} we can see that the t-structure $(\mathcal{U}_{\Phi'}, \mathcal{V}_{\Phi'})$ restricts to $\mathbf{D}^{\perf}(R)$. The Thomason filtration $\Phi'=\{Z^i \cap \Spec R\}$ on $\Spec R$ also has the property that the height of $Z^i 
    \cap \Spec R$ is equal to $h$ for all $i\in[a, a+h]$, since the minimal point $x$ of $Z^{a+h}$ was assumed to be in $\Spec R$ and such filtrations are defined to be decreasing.
    
    From Theorem \ref{Theorem 5.3 and A.5 Smith}, we see that this property of the filtration implies that $H^{a+h}(\tau_\Phi^{\leq 0}R[-a])$ is infinitely generated over $R$, and thus $\tau_\Phi^{\leq 0}R[-a]$ cannot be a perfect complex. Therefore the t-structure $(\mathcal{U}_{\Phi'}, \mathcal{V}_{\Phi'})$ cannot restrict to $\mathbf{D}^{\perf}(R)$, which is a contradiction.
\end{proof}


\begin{Th}\label{main}
    Let $X$ be a singular irreducible finite-dimensional noetherian scheme. Let $(\mathcal{U}, \mathcal{V})$ be a tensor t-structure on $\mathbf{D}^{\perf}(X)$, then either $\mathcal{U}=\varnothing$ or  $\mathcal{U}=\mathbf{D}^{\perf}(X)$.
\end{Th}

\begin{proof}
To achieve this, we classify the possible Thomason filtrations on $X$ which can correspond to tensor t-structures on $\mathbf{D}_{\quasicoh}(X)$ which restrict to $\mathbf{D}^{\perf}(X)$.

If $\dim X=0$, then $X$ is just a single point. In which case there are only three possible Thomason filtrations on $X$, the constant filtration associated to $\varnothing$, the constant filtration associated to $X$, and the standard filtration up to shifting. Lemma \ref{singular point} contradicts the possibility of the standard filtration being associated to a tensor t-structure which restricts to $\mathbf{D}^{\perf}(X)$. The other two filtrations correspond to the two trivial tensor t-structures.


Let $d=\dim X \geq 1$, and let $\Phi=\{Z^i\}$ be a Thomason filtration on $X$ corresponding to a tensor t-structure on $\mathbf{D}_{\quasicoh}(X)$ which restricts to $\mathbf{D}^{\perf}(X)$. Since $X$ is finite-dimensional, the heights of the non-empty $Z^i$ must be bounded between $0$ and $\dim X$. Lemma \ref{height} shows us that there can only $h$ many $Z^i$ of height $h$ for $\dim X \geq h\geq 1$, therefore we can see that there can only be at most $d(d+1)/2$ non-empty $Z^i$ of height $\dim X \geq h\geq 1$. Combining these two observations we see that the Thomason filtration must be bounded, in the sense that it must start with either $\varnothing$ or $X$, and end with either $\varnothing$ or $X$. In other words, the only permitted Thomason filtrations are of one of three forms, the constant filtration associated to the empty set: \begin{align*}
        \dots \supseteq \varnothing \supseteq \dots \supseteq \varnothing \supseteq \dots,
    \end{align*}
    the constant filtration associated to the entire space:
     \begin{align*}
        \dots \supseteq X \supseteq \dots \supseteq X \supseteq \dots,
    \end{align*}
    or some intermediate filtration with finitely many terms in the middle
         \begin{align*}
        \dots \supseteq X \supseteq Z^{a + d(d+1)/2} \supseteq \dots \supseteq Z^{a + 1}  \supseteq \varnothing \supseteq \dots.
    \end{align*}
    Now since $X$ is assumed to be singular, Lemma \ref{singular point} excludes the possibility of this intermediate filtration, as it would give us a contradiction. Therefore we can see that the only possibilities are the constant filtrations associated to either $\varnothing$ or $X$. These filtrations correspond to the two trivial tensor t-structures, so we are done.
\end{proof}

\begin{Rem}
    If $X$ is a finite-dimensional noetherian scheme and $Z\subseteq X$ is an irreducible closed subset, it is an obvious question to wonder if there are any tensor t-structures on the category $\mathbf{D}_Z^{\perf}(X)$ whenever $Z$ is not contained in the regular locus of $X$. Indeed, the details work out nicely with suitable modifications, and are to appear in forthcoming work by the authors.
\end{Rem}

{\bf Acknowledgements.}  K. Manali
Rahul was supported by the ERC Advanced Grant 101095900-TriCatApp,
the Australian Research Council Grant DP200102537, and is a recipient of the AGRTP scholarship. The work of C. J. Parker was partially supported by the same ERC Advanced Grant during his stay as a guest at the University of Milan, as well as the Deutsche Forschungsgemeinschaft (SFB-TRR 358/1 2023 - 491392403).


\bibliographystyle{alpha}
\bibliography{references}


{\footnotesize
\medskip

Rudradip Biswas,

Mathematics Institute, Zeeman Building, University of Warwick, Coventry CV4 7AL, United Kingdom.


{\tt Email: Rudradip.Biswas@warwick.ac.uk (R.Biswas)}

\smallskip

Kabeer Manali Rahul,

Center for Mathematics and its Applications, Mathematical Science Institute, Building 145,
The Australian National

University, Canberra, ACT 2601, Australia

{\tt Email: kabeer.manalirahul@anu.edu.au (K. Manali Rahul)}

\smallskip


Chris J. Parker,

Fakult\"at f\"ur Mathematik, Universit\"at Bielefeld, 33501, Bielefeld, Germany

{\tt Email: cparker@math.uni-bielefeld.de (C.Parker)}



\end{document}
