\begin{document}
\affiliation{$$_affiliation_$$}
\title{Some Observations on projective Stiefel Manifolds}
\begin{abstract}
    We compute the \textit{upper characteristic rank} of the projective Stiefel manifolds over $\bb R, \bb C$ and $\bb H$ and of the flip Stiefel manifolds. We provide bounds for the \textit{cup lengths} of these spaces. We also provide necessary conditions for existence of $S^3$-map between quarternionic Stiefel manifolds using Fadell-Husseini index.
\end{abstract}
\maketitle

\section*{Introduction}
This article concerns two independent problems of computing the \textit{upper characteristic rank }of the projective Stiefel manifolds and that of flip Stiefel manifolds, and determining the existence of $S^3$-equivariant maps between quaternionic Stiefel manifolds.

The \textit{characteristic rank} of a closed smooth manifold $M$ was introduced by Korba\v{s} in \cite{Korbaš2010} as a homotopy invariant of the manifold while providing an upper bound for the cup length $cup(M)$ of $M$. Naolekar and Thakur in \cite{NT2014} defined the general notion of characteristic rank of vector bundles over connected CW-complexes. The characteristic rank \textit{charrank}$(\xi)$ of a vector bundle $\xi$ over a CW-complex $Y$ is the largest integer $r\leq \dim (Y)$ such that all the cohomology classes in $H^i(Y; \bb Z_2), i\leq r$ can be expressed as a polynomial in the Stiefel-Whitney classes of $\xi$. The characteristic rank of a closed smooth manifold $M$ is the characteristic rank of the tangent bundle $TM$ of $M$. The \textit{upper characteristic rank} of $Y$ is defined to be $\max\{charrank(\xi) |~\textit{$\xi$ is a bundle over $Y$.}\}$.

The upper characteristic rank of a CW-complex is an important invariant and has implications to the $\bb Z_2$-cup length, Betti number and LS-category (see \cite{KNT2012}, \cite{korbas2014}). Although the computation of upper characteristic rank is in general a complex problem and not much seems to be known, there has been some interest in computing these invariants in recent times. In \cite{Korbaš2015}, \cite{PETROVIC2017}, the authors determined the characteristic rank of the canonical vector bundle over the oriented Grassmann manifold $\tilde{G}_{k,n}$ of $k$-dimensional subspaces in $\bb R^n$ for specific values of $k$ and $n$ and determined the $\bb Z_2$-cup-length for these spaces. For $n\ge 8$, in \cite{BC2020}, Basu and Chakraborty have calculated the upper characteristic rank of $\tilde{G}_{n,3}$ to be the characteristic rank of the oriented tautological bundle over $\tilde{G}_{n,3}$. For the product of spheres, real and complex projective spaces, the Dold manifolds and the stunted projective spaces, the characteristic rank of vector bundles was computed in \cite{NT2014}. In \cite{KNT2012} the authors have given a complete description of the upper characteristic rank of the Stiefel manifolds $\bb FV_{n,k}$ where $\bb F=\bb R, \bb C$ or $\bb H$, except for few cases where they could provide lower bound for upper characteristic rank (see Theorem \ref{KNT_main}).   

 Recall that the Stiefel manifold $\bb FV_{n, k}$ is the space of $k$-orthonormal frames in $\bb F^n$. The projective Stiefel manifolds $\bb FX_{n, k}$ is obtained as quotient of $\bb FV_{n, k}$ by the free left action, given by $z(w_1, \dots, w_k)=(zw_1, \dots, zw_k)$, of the group of absolute value 1 elements of $\bb F$. On the other hand, the flip Stiefel manifolds were recently introduced by Basu \textit{et al.} in \cite{BFG2023}. Let $C_n$ be the cyclic group of order $n$ and $C_2= \lra{a}$. The flip Stiefel manifold $FV_{n, 2k}$ is defined to be the quotient of $\bb RV_{n,2k}$ by the action $a(x_1,x_2,\ldots, x_{2k-1}, x_{2k}) =(x_2, x_1,\ldots, x_{2k}, x_{2k-1})$ of $\bb Z_2$.

In section \ref{ucr}, after describing the necessary mod 2 and integral cohomology algebras, we determine the upper characteristic rank of projective Stielfel manifolds and of $FV_{n, 2k}$. Although the description of the cohomology algebra of $\bb HX_{n, k}$ has been calculated by Zhubanov and Popelenskii in \cite{Zhu-Pop2022}, we give a short proof of the description of $H^*(\bb HX_{n,k}, \bb Z_2)$ in Theorem \ref{Quaternion_cohomology}, which we require for Section \ref{S^3_index}. We assume $k>1$ and $2k\geq 2$ for $\bb RX_{n,k}$ and for $FV_{n, 2k}$ throughout. Our computations are complete except for a few cases where we were able to provide lower bounds for the upper characteristic rank. The main theorems of Section \ref{ucr} are as follows. Let $N_1=\min\big{\{}j~|~n-k+1\leq j\leq n \textit{, and $\binom{n}{j}$ is odd}\big{\}}$ and $N_2= \min\big{\{}j~|~n-2k+1\leq j\leq n \textit{ and $\binom{k+j-1}{j}$ is odd}\big{\}}$.
\begin{Theorem}\label{main1}
    Let $X$ be either the projective Stiefel manifold $\bb RX_{n, k}$ or the flip Stiefel manifold $FV_{n, 2k}$, and $c=1$ or $2$ respectively as $X=\bb RX_{n, k}$ or $FV_{n, 2k}$. The upper characteristic rank of $X$ is as follows.
    
          (a) Let $n - ck\neq 1,2,4$ or $8$. If $N_c\neq n-ck+1$ then $ucharrank(X)= n-ck-1$. If $N_c=n-ck+1$, then $ucharrank(X)\geq n-ck$, the equality holds if $n-ck$ is odd and $n-ck+1$ is not a power of 2, or $n-ck$ is even. \\
          (b) Let $n - ck = 1$. Then $ucharrank(\bb RX_{n,k})=3$, if $n\equiv 2, 3\pmod 4$, and $ucharrank(FV_{n, 2k})\geq 2$ for $n\equiv 0,2\pmod 4$. The $ucharrank(X)= 0$ otherwise. \\
          (c) If $n-ck = 2$, then $ucharrank(\bb RX_{n,k})=2$, if $n\equiv 3\pmod 4$, and $ucharrank(FV_{n, 2k})= 2$ for $n\equiv 0\pmod 4$. The $ucharrank(X)= 1$ otherwise. \\
          (d) If $n-ck = 4$ then $ucharrank(X)\geq 3$. If in addition $N_c\neq 6$, then $ucharrank(X)\leq 4$. For $n\equiv 1,3\pmod 4$ and $ucharrank(\bb RX_{n, k}) =4$ and $ucharrank(FV_{n, k}) =4$ for $n\equiv2\pmod 4$. Further, $N_c\neq6$ if $c=1$ and $n\not\equiv2\pmod4$, and $c=2$ and $n\not\equiv 0\pmod 4$.\\
          (e) If $n-ck=8$ then $ucharrank(X)\geq 7$.  If in addition $N_c\neq 10$, then $ucharrank(X)\leq 8$, and $ucharrank(X)=8$ if $N_c=9$.
    \end{Theorem}
    In the case of the complex and quaternionic projective Stiefel manifolds, we found the following.
\begin{Theorem}\label{main2}
    (a) Let $k<n$. The upper characteristic rank of $\bb CX_{n,k}$ is $2(n-k)+2$ if $\binom{n}{n-k+1}$ is odd and $2(n-k)$ if $\binom{n}{n-k+1}$ is even.
    (b) The upper characteristic rank of $\bb HX_{n,k}$ is $4(n-k)+6$ if $\binom{n}{n-k+1}$ is odd and $4(n-k)+2$ if $\binom{n}{n-k+1}$ is even. 
\end{Theorem}

Let $G$ be a topological group. A map $f\colon X\to Y$ between two $G$ spaces $X$ and $Y$ is a $G$-equivariant map (or $G$-map), if $f(g\cdot x)=g\cdot f(x); \forall g\in G, \forall x\in X$. Recall that, in this language, the Borsuk-Ulam theorem states that there is no $\bb Z_2$-equivqriqnt maps between $S^n$ and $S^{n-1}$. The existence of $G$-equivariant maps between two $G$-spaces are related to Borsuk-Ulam type theorems (see \cite{FH1988}), and hence it is an interesting problem to look for such $G$-equivariant maps between $G$-spaces. Let $S^3$ be the group of the unit quaternions. In Section \ref{S^3_index}, after recalling the Fadell-Husseini index (see \cite{FH1988}) associated to a space $X$, we use it for computing some necessary conditions for existence of $S^3$-equivariant map between quaternionic Stiefel manifolds and certain $4n-1$-dimensional spheres. Similar problems have been studied by authors in \cite{PETROVIC2017} and \cite{BK21} for complex and real Stiefel manifolds. Our main result of Section \ref{S^3_index} is as follows. 




Let $G$ be a topological group. A map $f\colon X\to Y$ between two $G$ spaces $X$ and $Y$ is a $G$-equivariant map (or $G$-map), if $f(g\cdot x)=g\cdot f(x); \forall g\in G, \forall x\in X$. Let $S^3$ be the group of the unit quaternions. Computing neceassary and sufficient conditions for existence of certain $G$-maps between real and complex stiefel manifolds using Fadell-Husseini index have been described in \cite{Ha05}, \cite{Pe13}, \cite{BK21}.   In Section \ref{S^3_index} we calculate some necessary conditions for existence of $S^3$-equivariant map between quaternionic Stiefel manifolds and certain $4n-1$-dimensional spheres. The main result of this section is the following. 
\begin{Theorem}
(a) If there exists $S^3$-map between $\bb HV_{n,k}\to \bb HV_{m,l}$,  then $n-k\leq m-l$. Moreover if $n-k=m-l$, $\binom{n}{n-k+1}\vert \binom{m}{m-l+1}$.\\
 (b) There exists $S^3$-map $f:~Sp(n)\to Sp(m)$ iff $n~|~m$.\\
    (c) If there exists $S^3$-map between $ S^{4n-1}\to \bb HV_{m,l}$, then $n\leq m-l+1$.\\
    (d) If there exists an $S^3$-map $\bb HV_{n,k}\to  S^{4m-1}$, then $m\geq n-k+1$.
\end{Theorem}


The upper characteristic rank of the real projective Stiefel manifolds $\bb RX_{n, k}$ were independently obtained by the author in \cite{dasgupta2024} for some particular values of $n-k$, in fact for $n-k=5, 6$ or $\ge 9$. The article also deals with the complex case. The present article, however, concerns all possible values of $n-k$ and, moreover, our approach is different from that of \cite{dasgupta2024}.
\section{Upper Characteristic Rank}\label{ucr}
\subsection{Real Projective Stiefel Manifolds}\label{Real_case}
 Let $X$ and $c$ be as in theorem \ref{main1}. Let $\xi$ be the line bundle associated to the double covering $\bb RV_{n, ck}\to X$ and $\ga$ be the Hopf line bundle over $BC_2$. Consider the fibration
                         \[\bb RV_{n,ck}\to BO_{n-ck}\to BO_n.\]
 The classifying map of the bundle $n\ga$ induces a corresponding fibration over $BC_2$ 
        \[\bb RV_{n,ck}\xrightarrow{i} Y_{n, ck}\xrightarrow{p} BC_2,\eqno(1)\]\label{f1}
where the space $Y_{n, ck}$ is homotopy equivalent to the space $X$(see \cite{GH68}), and henceforth we do not differentiate between $Y_{n, ck}$ and $X$.
Borel in \cite{Bo53} showed that the cohomology algebra of $\bb RV_{n,ck}$ with $\bb Z_2$ coefficients is
                \[H^*(\bb RV_{n,ck}; \bb Z_2) = V(z_{n-ck},\dots, z_{n-1}).\]
Here, $V(z_{n-ck},\dots, z_{n-1})$ denotes the commutative, associative algebra with unit over $\bb Z_2$ with the classes $z_{n-ck},z_{n-ck+1},\cdots,z_{n-1}$ simple system of generators such that $z_i^2=z_{2i}$ if $2i\leq n-1$ and $z_i^2=0$ otherwise. The dimensions of $z_i$ are $i$.\\
The action of Steenrod algebra on $H^*(\bb RV_{n, ck};\bb Z_2)$ is given by (see \cite{Bo53})
\[Sq^i(z_q)=\binom{q}{i} z_{q+i} ~~~~~\text{for $i\leq q.$}\eqno(2)\]\label{sqV} 
The mod 2 cohomology algebra of $X$ has the following description. Let $c=1, 2$ and $N_c$ be defined as before theorem $\ref{main1}$.
\begin{theorem}\label{rpsm_cohomology}{\em (\cite[Theorem 2.1]{GH68}; \cite[Theorem 3.5]{BFG2023})} 
Suppose $ck<n$.  Let $x$ be the generator of the algebra $H^*(BC_2; \bb Z_2)$ of dimension 1. Then 
    \[H^*(X; \bb Z_2) = \bb Z_2[y]/\lra{y^{N_c}}\otimes V(y_{n-ck},\dots,\hat{y}_{N_c-1},\dots y_{n-1}),\]
    where $y=p^*(x)$ and $y_j=i^*({z_{j}})$ and $n-ck\leq j\leq n-1$.\hfill$\square$
\end{theorem}

As the class $y$ is pullback of the class $x$, $i^*(y)=0$. Hence,
\begin{remark}
The class $z_{N_c-1}$ is not in the image of $i^*$ which implies $i^*$ is not surjective.
\end{remark}
Before proving the Theorem \ref{main1}, we recall the upper characteristic rank of the projective Stiefel manifolds, which we will use on several occasions in the course of proving Theorem \ref{main1}.
 \begin{theorem}{\em (\cite[Theorem 1.1]{KNT2012})}\label{KNT_main}
     Let $X=\bb FV_{n, k}$ with $1<k<n$ when $\bb F=\bb R$ and $1<k\leq n$ when $\bb F=\bb C, \bb H$.\\
     (a) If $\bb F=\bb R$, then
     \[ucharrank(X)=\begin{cases}
         n-k-1 & \textit{if $n-k\neq 1, 2, 4, 8,$}\\
         2     & \textit{if $n-k= 1$ and $n\geq 4$,}\\
         2     & \textit{if $n-k= 2$,}\\
         4     & \textit{if $n-k= 4$ and $k=2$.}
     \end{cases}\]
     (b) If $\bb F=\bb R, k>2$ and $n-k=4$, then $ucharrank(X)\leq 4$.\\
     (c) If $\bb F=\bb R$ and $n-k=8$, then $ucharrank(X)\leq 8$.\\
     (d) If $\bb F=\bb C$, then
     \[ucharrank(X)=\begin{cases}
         2     & \textit{if $k=n$,}\\
         2(n-k)     & \textit{if $k<n$.}\\
     \end{cases}\]
     (e) If $\bb F=\bb H$, then $ucharrank(X)= 4(n-k)+2$.\\
 \end{theorem}



\begin{proof}[Proof of Theorem \ref{main1}]
    Let us denote $ucharrank(\bb RV_{n, ck})$ by $\fk r$. \\

    Assume first  $n-ck\neq 1, 2, 4$ or $8$, and $N_c\neq n-ck+1$. Then $N_c> n-ck+1$. There could be a vector bundle $\alpha$ over $X$ with $w_{n-ck}(\alpha)=y_{n-ck}$. Then $w_{n-ck}(i^*(\alpha))=z_{n-ck}$ which gives a contradiction as $\fk r = n-ck-1$ by theorem \ref{KNT_main}. Therefore, $ucharrank(X_{n,k})\leq  n-k-1$. Also, all the cohomology classes in $H^*(X_{n,k}; \bb Z_2)$ of degree less than $n-ck$ are obtained as the Stiefel-Whitney classes of $p^*(\gamma )$ where $\gamma$ denotes the canonical line bundle over $BC_2$. Therefore, $ucharrank(X)= n-ck-1$ in this case.
    
    For $N_c=n-ck+1$. Observing the spectral sequence associated to \eqref{f1} there is no degree $N_c-1$ class in the $H^*(X)$ coming from the $H^*(V_{n,ck})$. Since all the cohomology classes in $H^*(X; \bb Z_2)$  degree less than $n-ck+1$ are obtained as the Stiefel-Whitney classes of $p^*(\gamma)$, $ucharrank(X)\ge n-ck$. 
    
    Let $n-ck$ be odd, and $n-ck+1$ be not a power of 2. If there is a vector bundle over $X$ with $(n-ck+1)$th Stiefel-Whitney class $y_{n-ck+1}$, then pullback of this bundle on $V_{n,ck}$ has first nontrivial Stiefel-Whitney class in such a dimension that is not a power of 2. This is impossible, so there cannot be such a bundle, proving $ucharrank(X)=n-ck$.
    
 Let $n-ck$ be even and $ucharrank(X) = n-ck+1$ if possible. We show that there must be a vector bundle $\beta$, over $V_{n, ck}$ with $w_{n-ck}(\beta)=z_{n-ck}$ which will contradict the value of $\fk r$. Let $\alpha$ be a vector bundle over $X$ with $w_{n-ck+1}(\alpha)=y_{n-ck+1}$. Denote by $\beta$ the pullback bundle $i^*\alpha$. Then $w_{n-ck+1}(\beta) = z_{n-ck+1}$. If possible, let $w_{n-ck}(\beta) = 0$, then by Wu's formula, $0=Sq^1(w_{n-ck}(\beta))=(n-ck-1)w_{n-ck+1}(\beta)=w_{n-ck+1}(\beta)$. This is a contradiction as $w_{n-ck+1}(\beta)=z_{n-ck+1}\neq 0$. Therefore, $w_{n-ck}(\beta)$ must be $z_{n-ck}$, as claimed. . Hence, we have $ucharrank(X)=n-ck$. This completes the proof of $(a)$.\\

    


    Let now $n-ck$ be $1$. Then $\fk r=2$. Since $H^1(X; \bb Z_2)\cong \bb Z_2\oplus \bb Z_2$, if $N_c\neq 2$, the upper characteristic rank is positive only when $N_c=2$ , $i.e.$,  the class $y_1$ is trivial in the cohomology of $X$. Note that $N_c=2$ if and only if $n\equiv 2, 3\pmod 4$ for $\bb RX_{n,k}$ and $n\equiv 0, 2\pmod 4$ for $FV_{n, 2k}$.  We write $\bb RV_{ck+1, ck}$ for $\bb RV_{n,ck}$.
    
    Now, for $N_c=2$, as $H^1(X;\bb Z_2)$ is generated by the Stiefel-Whitney class of $p^*(\ga)$, the $ucharrank(X)\geq 1$. We show that the homomorphism $\rho_2\colon H^2(X;\bb Z)\to H^2(X;\bb Z_2)$ is surjective. Since elements of $H^2(X; \bb Z)$ are in one-to-one correspondence with the Chern class of complex line bundles over $X$, this will prove that $y_2$ can be obtained as the second Whitney class of some orientable vector bundle $\zeta$ over $X$. Hence, all the classes in $H^*(X; \bb Z_2)$ can be obtained as polynomials in the Stiefel-Whitney classes of the bundle $p^*(\ga)\oplus \zeta$. Therefore, $ucharrank(X)\geq 2$.
    
    Consider the homotopy long exact sequence induced by $\bb RV_{ck+1,ck}\xrightarrow{i} X\xrightarrow{p} \bb RP^{\infty}$. Since $\pi_j(BC_2)=0; j>1$, hence $\pi_j(X)\cong \pi_j(\bb RV_{ck+1, ck}); j>1$, and $\pi_1(\bb RV_{ck+1, ck})=\bb Z_2$, we have the following exact sequence
                \[0\to \bb Z_2\to \pi_1(X)\to \bb Z_2\to 0\]
                        
    First, we show that $\pi_1(X)=\bb Z_4 $. If $\pi_1(X)=\bb Z_2\oplus \bb Z_2$, then $H^1(X; \bb Z)=\bb Z_2\oplus \bb Z_2$, which contradicts that $H^1(X; \bb Z_2)$ is $\bb Z_2$(see \ref{rpsm_cohomology}). Therefore, $\pi_1(X) = \bb Z_4 $. Since $H_2(BC_4, \bb Z)=0$ and $\pi_2(X)\cong \pi_2(\bb RV_{ck+1, ck})\cong 0$, the right exact sequence $\pi_2(X)\to H_2(X; \bb Z)\to H_2(BC_4, \bb Z)\to 0$ yields $H_2(X; \bb Z)\cong \pi_2(X)\cong 0$ , and hence $H^2(X; \bb Z)$ is $\bb Z_4 $. Hence, the Bockstein long exact sequence
\[ 0\xrightarrow{} H^1(X; \bb Z_2)\xrightarrow{\delta} H^2(X; \bb Z)\to H^2(X; \bb Z)\xrightarrow{\rho_2} H^2(X; \bb Z_2)\rightarrow\cdots\]

becomes
            \[0\to\bb Z_2\to \bb Z_4 \to \bb Z_4 \to\bb Z_2\to\cdots.\]

Since $\delta$ is injective, $\rho_2$ must be surjective.

Finally, we show that $ucharrank(\bb RX_{n+1, n})< 3$. If possible, let $ucharrank(X_{n+1, n})\geq 3$ and $\alpha$ be a vector bundle over $X$ with $w_3(\alpha) = y_3$. We show that whatever $w_2(\alpha)$ be, $y_2$ or $0$, $w_3(\alpha)$ must be trivial. As $H^1(X; \bb Z_2)=\bb Z_2\lra{y}$ for $N_c=2$, by taking a direct sum with $p^*(\ga)$, we may assume $w_1(\alpha)=0$, and the Wu's formula becomes $Sq^1(w_2(\alpha))=w_1(\alpha)w_2(\alpha)+ w_3(\alpha)=w_3(\alpha)$. Now, if $w_2(\alpha)=0$ then clearly $w_3(\alpha)=0$. And, if $w_2(\alpha)=y_2$, by theorem 2.8 of \cite{GH68}, $w_3(\alpha)=Sq^1(w_2(\alpha))=Sq^1(y_2)=2y_3=0$. Hence, such an $\alpha$ does not exist.\\


Now, let us consider the case when $n-ck=2$. Then $\fk r=2$ and $N_c\ge 3$.

If $n=4m+3$ for $c=1$ and $n=4m$ for $c=2$, then $N_c=3$. Hence $H^i(X)\cong \bb Z_2\lra{y^i}; i=1, 2$ and, therefore, $ucharrank(X)\geq 2$. We show $ucharrank(X)<3$. If not, there is a bundle $\alpha$ over $X$ such that $w_3(\alpha) = y_3$. After taking a direct sum with $p^*(\ga)$, we may assume $w_1(\alpha)=0$. As before, we show whatever $w_2(\alpha)$ be, $w_3(\alpha)=0$, leading to a contradiction. Since $w_1(\alpha)=0$, $Sq^1(w_2(\alpha))=w_3(\alpha)$, if $w_2(\alpha)=0$, so is $w_3(\alpha).$ If $w_2(\alpha)=y^2$, then $w_3(\alpha)=Sq^1(y^2)=y^3=0.$ Hence, this bundle $\alpha$ cannot have a nontrivial $w_3$, as the first nontrivial Stiefel-Whitney class cannot be in odd dimension, a contradiction. Therefore, if $N_c=3$, then $ucharrank(X)=2$.




If  $n\not\equiv 3\pmod 4$, then $N_c>3$ and the class $y_{2}$ is nontrivial in $H^2(X_{n,k}; \bb Z_2)$. We show that $y_2$ cannot be in the image of $\rho_2\colon H^2(X;\bb Z)\to H^2(X;\bb Z_2)$. It follows that there cannot be no orientable and, hence, also no non-orientable vector bundle over $X$ with the second Stiefel-Whitney class $y_2$.


First note that $\pi_1(X)\cong \pi_1(\bb RV_{ck+2,ck})=\bb Z_2$, as $ck>1$, and therefore $H_1(X; \bb Z)\cong \bb Z_2$. Since $\bb RV_{n,ck}\to X$ is a covering projection, $\pi_2(X)\cong \pi_2(\bb RV_{ck+2,ck})=\bb Z$. The following right exact sequence gives a surjection of $\pi_2(X)$ and $ H_2(X; \bb Z)$.
\[\pi_2(X)\to H_2(X; \bb Z)\to H_2(\pi_1(X), \bb Z)\to 0,\]

as $H_2(\pi_1(X), \bb Z)=0$. Therefore, $H_2(X; \bb Z)$ is either $\bb Z$ or finite abelian. We show that for either possibility of $H_2(X; \bb Z)$ we must have $y_2\notin\im(\rho_2)$.

If $H_2(X; \bb Z)=\bb Z$, then $H^2(X; \bb Z)=\bb Z\oplus\bb Z_2$ and the Bockstein long exact sequence
\[ \cdots\rightarrow H^1(X; \bb Z)\xrightarrow{} H^1(X; \bb Z_2)\xrightarrow{} H^2(X; \bb Z)\xrightarrow{m_2} H^2(X; \bb Z) \xrightarrow{\rho_2} H^2(X; \bb Z_2)\rightarrow\cdots,\]
becomes
\[ \cdots\to \bb Z_2\xrightarrow{\cong}  \bb Z_2\xrightarrow{0} \bb Z\oplus \bb Z_2\xrightarrow{\cong}  \bb Z\oplus \bb Z_2 \xrightarrow{\rho_2}  \bb Z_2\oplus\bb Z_2\rightarrow\cdots.\]

Hence $\rho_2$ is trivial, and the image of $\rho_2$ does not contain $y_2$.

Else if $H_2(X; \bb Z)$ is finite abelian, then $H^2(X; \bb Z)\cong \bb Z_2$. Again, we consider the Bockstein long exact sequence.
\[0\xrightarrow{} \bb Z_2\xrightarrow{\cong} \bb Z_2\xrightarrow{0} \bb Z_2\xrightarrow{\rho_2} \bb Z_2\oplus\bb Z_2\to\cdots\]

Hence $\rho_2$ is not surjective. Since $H^2(X; \bb Z_2)$ is generated by $y^2$ and $y_2$, and $y^2$ is the Whitney class of the vector bundle $p^*(\ga)$, hence the image of $\rho_2$ does not contain $y_2$. Hence  (c) is proved.

For $n-ck=4$, since $N_c\geq 5$, therefore $H^i(X)\cong \bb Z_2\lra{y^i}; i=1, 2, 3$, and hence $ucharrank(X)\geq 3$.

We show that if $N\neq 6$, then $ucharrank(X)\leq 4$. Let, if possible, a vector bundle $\alpha$ over $X$ exists such that $w_5(\alpha) = y_5$. Then if $\beta = i^*(\alpha), w_5(\beta) = z_5$. Also, $w_4(\beta)$ must be zero as $\fk r=4$. But then, $0=Sq^1(w_4(\beta))=w_1(\beta)w_4(\beta)+w_5(\beta)=w_5(\beta)$. This is a contradiction as it implies $z_5=0$. Since $y_4$ vanishes for $N_c=5$, $i.e.$, $n\equiv 1,3 \pmod4$ for $\bb RX_{n+4, n}$ and $n\equiv 2 \pmod4$ for $FV_{n, 2k}$, the $ucharrank(X)$ is exactly $4$. Hence $(d)$.\\

The proof of $(e)$ is parallel to that of $(c)$.
\end{proof}
\subsection{Complex and Quaternionic Projective Stiefel Manifold}\label{complex}
We denote the complex and quaternionic Stiefel manifolds respectively by $\bb CV_{n,k}$ and $\bb HV_{n,k}$ and the corresponding projective Stiefel manifolds by $\bb CX_{n,k}$ and $\bb HX_{n,k}$. The mod 2 cohomology algebra of $\bb CV_{n,k}$ and $\bb HV_{n, k}$ are well known and has the following descriptions.( See \cite{Bo53}.)

\[H^*(\bb CV_{n,k}; \bb Z_2)\cong \Lambda_{\bb Z_2}(z'_{n-k+1}, z'_{n-k+2}, \cdots, z'_{n}),\quad   |z'_i|=2i-1,\eqno(3)\]

\[H^*(\bb HV_{n,k}; \bb Z_2)\cong \Lambda_{\bb Z_2}(z''_{n-k+1}, z''_{n-k+2}, \ldots, z''_{n}),~~~  |z''_i|=4i-1.\eqno(4)\]

We have again the fibrations, which are the complex and quaternionic analogue of \eqref{f1}

\[\bb CV_{n,k}\xrightarrow{i'} \bb CX_{n,k}\xrightarrow{p'} BS^1,\eqno(5)\]\label{f2}  
\[\bb HV_{n,k}\xrightarrow{i''} \bb HX_{n,k}\xrightarrow{p''} BSp(1),\eqno(6)\]\label{f3}

which determines the cohomology algebra of complex and quaternionic projective Stiefel manifolds.
\begin{theorem}{\em (\cite[Theorem A]{Ru69},\cite[Theorem 5]{Zhu-Pop2022})}
Let $k<n$ and $N=\min\{j~|~n-k+1\leq j< n$, such that $ \binom{n}{j} \not\equiv 0 \pmod 2\}$. Let $x'$ be the generator of the algebra $H^*(BS^1; \bb Z_2)$ of dimension $2$ and $x''$ be the generator of the algebra $H^*(BSp(1); \bb Z_2)$ of dimension $4$. Then 
\[H^*(\bb  CX_{n,k}; \bb Z_2)\cong \bb Z_2[y']/\lra{(y')^N}\otimes \Lambda_{\bb Z_{2}}(y'_{n-k+1},\cdots, \hat{y'}_{N}, \cdots, y'_{n}),\quad   |y'_i|=2i-1,\]

\[H^*(\bb  HX_{n,k}; \bb Z_2)\cong \bb Z_2[y'']/\lra{(y'')^N}\otimes \Lambda_{\bb Z_{2}}(y''_{n-k+1}, \cdots,\hat{y''}_{N},\cdots,y''_{n}),\quad   |y_i|=4i-1,\]

where $(p')^*(x')=y', (p'')^*(x'')=y''$ and $(i')^*(y'_i)=z'_i, (i'')^*(y''_i)=z''_i$.
\end{theorem}

We will give proof for the mod 2 cohomology algebra of the quaternionic projective Stiefel manifold. The proof is similar and along the lines of the real case as given in \cite{GH68}. We recall some general facts about the universal quaternionic bundle and its associated characteristic classes. The integral cohomology algebra of classifying space of the symplectic group $Sp(n)$ is the polynomial algebra generated by the universal symplectic classes $k_i$ of degree $4i$, associated with universal $n$-dimensional bundle over $BSp(n)$.
\[H^*(BSp(n);\bb Z)=\bb Z[k_1,...k_n].\]
 The $\mod 2$ reduction of the symplectic classes are the Stiefel-Whitney classes of underlying real bundle. Thus 
 \[H^*(BSp(n);\bb Z_2)=\bb Z_2[w_4,...w_{4n}].\] 
 
\begin{theorem}{\em (\cite[Theorem 5]{Zhu-Pop2022})}\label{Quaternion_cohomology}
Let $k<n$ and $N=\min\{j~|~n-k+1\leq j< n$, such that $ \binom{n}{j} \not\equiv 0 \pmod 2\}$. Let $\omega$ be the generator of the algebra $H^*(BSp(1); \bb Z_2)$ of dimension $4$. Then 
\[H^*(\bb  HX_{n,k}; \bb Z_2)\cong \bb Z_2[y]/\lra{y^N}\otimes \Lambda_{\bb Z_{2}}(y_{n-k+1}, \cdots,\hat{y}_{N},\cdots,y_{n}),\quad   |y_i|=4i-1,\] where $p^*(\omega)=y$ and $i^*(y_i)=z_i$.
\end{theorem}

\begin{proof}
We have the pullback square of fiber bundles associated with \eqref{f3}
\[
\xymatrix{
\bb HV_{n,k}\ar@{=}[rr]\ar[d] & &\bb HV_{n,k}\ar[d]\\
\bb HX_{n,k}\ar[rr] \ar[d] & &BSp(n-k)\ar[d]\\
 BSp(1)\ar[rr]^{i} & &BSp(n).
}
\]
Here, the map $i$ classifies the bundle $\oplus_{n}\gamma$ where $\gamma$ is the canonical quaternionic line bundle over $BSp(1)$. In the Serre spectral sequence of the right hand fibration  $d_{4i}(z_i)=w_{4i}$ and
\begin{align*}
    w(\oplus_{n}\gamma)&=(1+\omega)^n\\
               &=\sum_{i=0}^{i=n}\binom{n}{i}\omega^i.
\end{align*}
Here $\omega$ is the pullback of the class $w_4$ under the map $i^*$. This implies \[d_{4i}=\binom{n}{i}\omega^i.\] So the first non-zero differential will appear at $E_N$ and all the non-zero power of $\omega$ will vanish on $E_{n+1}$ page by multiplicative property of the Serre spectral sequence. Hence 
\begin{align*}
    E_2&=E_N\\
    E_{N+1}&=E_{\infty}=\bb Z_2[y]/\lra{y^N}\otimes \Lambda_{\bb Z_{2}}(y_{n-k+1}, \cdots, y_{n}).
\end{align*}
Since we are in $\pmod 2$ coefficient there is no extension problem. Hence the proof is complete.
\end{proof}

We are now ready to prove Theorem \ref{main2}.

\begin{proof}[Proof of Theorem \ref{main2}]
    Recall that $ucharrank(\bb CV_{n,k})$ and $ucharrank(\bb HV_{n,k})$ are respectively $2(n-k), k<n$ and $4(n-k)+2$.

    Let $\binom{n}{n-k+1}$ be even, $i.e.$, $N\neq 2(n-k)+1$. If $ucharrank(\bb CX_{n,k})\geq 2(n-k)+1$, then there is a vector bundle over $\bb CX_{n,k}$ with $w_{2(n-k)+1}=y_{2(n-k)+1}$. Then $w_{2(n-k)+1}(i^*(\alpha))=z_{2(n-k)+1}$. Since $\bb CV_{n,k}$ has first nontrivial cohomology in dimension $2(n-k)+1$, which is not a power of $2$, this is impossible. So, $ucharrank(\bb CX_{n,k})\leq 2(n-k)$. Now, since $charrank(i^*(\gamma))=2(n-k)$ the upper characteristic rank of $\bb CX_{n,k}$ is $2(n-k)$. 
    
    If $\binom{n}{n-k+1}$ be odd, $i.e.$, $N=2(n-k)+1$, then $H^{2(n-k)+1}(\bb CX_{n,k}; \bb Z_2)=0=H^{2(n-k)+2}(\bb CX_{n,k}; \bb Z_2)$. If $ucharrank(\bb CX_{n, k})\ge 2(n-k)+3$, the bundle $\beta$ with $w_{2(n-k)+3}=y_{2(n-k)+3}$ pulls back to a bundle $\alpha$ over $\bb CV_{n, k}$ whose first Stiefel-Whitney class appear in degree $2(n-k)+1$ or $2(n-k)+3$, which is impossible.Therefore, $ucharrank(\bb CX_{n, k})=2(n-k)+2$ in this case.\\

    The proof of quaternionic case is identical.
\end{proof}



We conclude this section with the following observation about the cup length of $\bb RX_{n, k}$. The $\bb Z_2$-cup length, denoted by $cup(Y)$, of a space $Y$ is the greatest integer $j$ such that there are classes $y_i \in H^*
(Y; \bb Z_2)$, deg$(y_i) \ge 1$, such that the cup product $y_1y_2\cdots y_j\neq 0$.  First, we recall the following theorem from \cite{NT2014}.
\begin{theorem}{\em \cite[Theorem 1.2]{NT2014}}\label{TN}
    Let $Y$ be a connected closed smooth manifold of dimension $d$ and $\xi$ be a vector bundle over $Y$ satisfying the following: there exists $j, j\leq charrank(\xi)$, such that every monomial $w_{i_1}(\xi)\ldots w_{i_r}(\xi), 0\leq i_p\leq j$, in dimension $d$ vanishes. Then, 
                        \[cup(Y)\leq 1+\frac{d-j-1}{r_y}.\]
    Here $r_y$ is the smallest positive integer such that $\tilde{H}^{r_y}(Y; \bb Z_2)\neq 0.$
\end{theorem}
\begin{theorem}
    Let $n\neq k, d$ be the dimension of $\bb RX_{n, k}$ and $\binom{n}{n-k+1}$ be odd.Let $k>1$, then $cup(X_{n,k})\leq d-N$.
\end{theorem}
\begin{proof}
    Let us consider the vector bundle $\xi_0$ over $\bb RX_{n, k}$ associated to the double covering $\bb RV_{n, k}\to \bb X_{n, k}$. Then $w_1(\xi_0)=y$ and all the monomials $w_{i_1}(\xi_0)\ldots w_{i_r}(\xi_0)=y^{\sum_{p=1}^r i_p}$, which vanished if and only if $\sum_{p=1}^r i_p\geq N_1$. Hence for $d\geq N_1$ all the monomials $w_{i_1}(\xi_0)\ldots w_{i_r}(\xi_0), 0\leq i_p\leq N$ vanish. Since $\binom{n}{n-k+1}$ is odd, $N_1=n-k+1$, and, as $n-k\geq 1,d\geq N_1$ hold if $k>1$. Therefore, we may take $j=N_1$.

    Now, $r_x=1$ since $\tilde{H}^1(\bb RX_{n, k}; \bb Z_2)\neq 0$. Hence the theorem follow from Theorem \ref{TN}.
\end{proof}
\begin{remark}
    For $k>1, N_2=n-2k+1$ and $N=n-k+1$, the same set of arguments applies to the flip Stiefel manifolds, complex and quaternionic projective Stiefel manifolds, providing upper bounds $d-N_2, d-2N+1$ and $d-4N+1$ for cup lengths respectively.
\end{remark}
\section{$S^3$-Equivariant Maps}\label{S^3_index}
Given a $G$-space we define the homotopy orbit $X_{hG}:=EG\times_{G}X$. Here $EG$ is the total space for the universal $G$-bundle. Note that if the $G$-action on $X$ is free $X_{hG}\equiv X/G$. Given a $G$-space $X$ we have a canonical fibration \[X\to X_{hG}\xrightarrow{p} BG.\]
We define Fadell-Husseini index associated to $X$ for a suitable coefficient ring $R$ to be \[\Index_{G}(X;R)=\ker p^*:H^*(BG;R)\to H^*(X_{hG};R).\] One of the interesting properties of the index is the monotonicity property. A $G$-map between two $G$-spaces $X$ and $Y$ will imply  \[\Index_G(Y)\subset \Index_G(X).\]
We will Denote by $\Index_G^q(X;R)=\Index_G(X;R)\cap H^q(X;R)$. Various cases have been considered in real and complex Stiefel manifolds to rule out $\bb Z_2$ and $S^1$-equivariant maps between them in the articles \cite{Pe13} and \cite{Pe97}. We here considered the quaternionic Stiefel manifolds and $S^3$ action on it.
\subsection{Index of odd-dimensional sphere}
Consider the free action of $S^1$ to the odd dimensional sphere $S^{2n-1}$ by complex multiplication which results the fibration \[S^{2n-1}\to \bb CP^n\to \bb CP^{\infty}. \] The only non-trivial differential for the Serre spectral sequence associated to the above fibration is at $E_{2n}$ page. The differential takes the generator of $H^{2n-1}(S^{2n-1};\bb Z_2)$ to the $2n$-dimensional generator of $H^{*}(\bb CP^{\infty};\bb Z_2)$. Thus, 
\begin{proposition}
$\Index_{S^1}(S^{2n-1};\bb Z_2)$ is the ideal $\lra{\alpha^{2n}}$ in  $H^*(BS^1;\bb Z_2)$. Here $\alpha$ denotes the two dimensional generator of the cohomology ring $H^*(BS^1;\bb Z_2)$.
\end{proposition}
Using similar argument on free $S^3$ action on the odd dimensional sphere $S^{4n-1}$ we obtain the following. 
\begin{proposition}\label{index}
$\Index_{S^3}(S^{4n-1})$ is the ideal $\lra{\alpha^{4n}}$ in  $H^*(BS^3; \bb Z_2)$. Here $\alpha$ denotes the four dimensional generator of the cohomology ring $H^*(BS^3; \bb Z_2)$.
\end{proposition}
\begin{theorem}\label{indst}
 a) $\Index_{S^3}(\bb HV_{n,k};\bb Z_2)=\lra{\omega^N}$,
 where $\omega$ is the generators of $\pmod 2$ cohomology algebra of $BS^3$ and $|\omega|=4$.\\
 b) $\Index_{S^3}^{4(n-k+1)}(\bb HV_{n,k};\bb Z)=\bb Z[\binom{m}{m-l+1}k^{n-k+1}]$, where $k$ is the generator of the integral cohomology algebra of $BS^3$ and $|k|=4$.
\end{theorem}
\begin{proof}
From fibration (6) and using Serre spectral sequence associated to it, Theorem (3.2) readily gives part (a) of the theorem.\\

For part (b) we again use Serre spectral sequence associated to fibration (6) and the pullback square in the proof of the Theorem \ref{Quaternion_cohomology}. Recall that $H^*(\bb HV_{n,k};\bb Z)=\Lambda_{\bb Z}(z_{n-k+1}, z_{n-k+2}, \ldots, z_{n})$ where  $|z_i|=4i-1$. The generator $z_{n-k+1}$ is transgressive, translating the proof of the Theorem \ref{Quaternion_cohomology} in integral set up we get \[d_{4(n-k+1)}(z_{n-k+1})=\binom{n}{n-k+1}k^{n-k+1}.\] This completes the proof.
\end{proof}
 
We will use the index computations to rule out the existence of $S^3$-equivariant maps between quaternionic Stiefel manifolds and certain $S^3$-spheres. 
\begin{theorem}
If there exists $S^3$-map between $\bb HV_{n,k}\to \bb HV_{m,l}$,  then $n-k\leq m-l$. Moreover if $n-k=m-l$, $\binom{n}{n-k+1}\vert \binom{m}{m-l+1}$.
\end{theorem}
\begin{proof}
Existence of such $S^3$-equivariant map will imply 
\[\Index_{S^3} (\bb HV_{m,l})\subset \Index_{S^3} (\bb HV_{n,k}). \] From part (b) of the Theorem \eqref{indst} the proof readily follows. 
\end{proof}
\begin{proposition}
There exists $S^3$-map $f:~Sp(n)\to Sp(m)$ iff $n~|~m$.
\end{proposition}
\begin{proof}
The second part of the Theorem (4.4) implies if there exists $S^3$-map $f:~Sp(n)\to Sp(m)$ then $n$ must divide $m$.\\
For the `if' part, note that $m=nk$ for some $k\in N^+$  implies that any $M\in Sp(n)$ can embedded in $Sp(m)$. This can be done by putting $M$ as $n\times n$ block matrix in the diagonal and putting everywhere $O$ otherwise. This completes the proof.
\end{proof}
We conclude with the following immediate consequences of the Proposition \ref{index} and Theorem \ref{indst}. 
\begin{theorem}
a) If there exists $S^3$-map between $ S^{4n-1}\to \bb HV_{m,l}$, then $n\leq m-l+1$.\\
b)If there exists an $S^3$-map $\bb HV_{n,k}\to  S^{4m-1}$, then $m\geq n-k+1$.
c) There exists $S^3$-map between $S^{4n-1}\to S^{4m-1}$ iff $n\leq m$.
\end{theorem} 
{\bf Acknowledgments:} We would like to express our gratitude to Aniruddha Naolekar for introducing us to the problem and for the productive discussion we had with him during our stay at the Indian Statistical Institute Bangalore Centre while working on this project.
\section{}
We do not analyze or generate any datasets, because our work proceeds with in a theoretical and mathematical approach.
\bibliography{main}
\bibliographystyle{siam}
\end{document}
