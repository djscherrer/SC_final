\begin{document}
\affiliation{$$_affiliation_$$}
\title{Slice rank and analytic rank for trilinear forms}
\begin{abstract}
In this note, we present an elementary proof of the fact that the slice rank of a trilinear form over a finite field is bounded above by a linear expression in the analytic rank. The existing proofs by Adiprasito-Kazhdan-Ziegler and Cohen-Moshkovitz both rely on results of Derksen via geometric invariant theory. A novel feature of our proof is that the linear forms appearing in the slice rank decomposition are obtained from the trilinear form by fixing coordinates.
\end{abstract}
\maketitle

\section{The theorem}
Let $\k = \F_q$ be a finite field and let $U,V,W$ be finite dimensional vector spaces over $\k.$  
\begin{definition}\label{rk-def} 
For a trilinear form $f:U\times V\times W\to\k$ we are interested in two kinds of rank.
\begin{itemize}
    \item The \emph{slice rank} of $f$ is 
    \[
    \srk(f) = \min\left\{r: f = \sum_{i=1}^r \alpha_i\cdot h_i\right\},
    \]
    where $\alpha_i,h_i$ are linear and bilinear, respectively, in disjoint sets of variables.
    \item The \emph{analytic rank} of $f$ is 
    \[
    \ark(f) = -\log_q \frac{|Z|}{|U\times V|},
    \]
    where $Z = \{(u,v)\in U\times V: f(u,v,\cdot) \equiv 0\}.$
\end{itemize}
    
\end{definition}

The definition of slice rank goes back to the work of Schmidt \cite{S} on systems of polynomial equations. It was reidiscovered in \cite{T} and used to give a reformulation of Ellenberg and Gijswijt's work on the capset problem. Analytic rank was introduced in \cite{GW}. The inequality $\ark(f)\le \srk(f)$ is straightforward, see \cite{KZ} or \cite{L}. In the other direction, Adiprasito-Kazhdan-Ziegler \cite{AKZ} and Cohen-Moshkovitz \cite{CM-cubics} independently proved that $\srk(f) \ll \ark(f).$ Their proofs use results of Derksen \cite{D} which rely on the powerful tools of geometric invariant theory. The goal of this note is to give an elementary proof.

\begin{theorem}\label{main} For any finite field $\k =\F_q$ and trilinear form $f,$ we have
$$\srk(f) \le 5\cdot \ark(f)+4\cdot \log_q(\ark(f)+1)+29.$$
In addition, the linear forms in the corresponding slice rank decomposition are obtained from $f$ by fixing coordinates.
\end{theorem}

\begin{remark}
    This is slightly weaker than the best current bound $\srk(f) \le 3\cdot\ark(f),$ obtained in \cite{AKZ}.
\end{remark}

Theorem \ref{main} is the first nontrivial case of a more general conjecture relating analytic rank to another kind of rank, \emph{partition rank}. To state the general conjecture, we need some definitions. Let $V_1,\ldots,V_d$ be finite-dimensional vector spaces over $\k$ and let $f:V_1\times\ldots\times V_d\to \k$ be a multilinear form.

\begin{definition}
    \begin{itemize}
    \item The \emph{partition rank} of $f$ is 
    \[
    \prk(f) = \min\left\{r: f = \sum_{i=1}^r g_i\cdot h_i\right\},
    \]
    where $g_i,h_i$ are multilinear in disjoint sets of variables. Note that if $d=3$ this agrees with the slice rank.
    \item The \emph{analytic rank} of $f$ is 
    \[
    \ark(f) = -\log_q \frac{|Z|}{|V_1\times\ldots\times V_{d-1}|},
    \]
    where $Z = \{(x_1,\ldots,x_{d-1})\in V_1\times\ldots\times V_{d-1}: f(x_1,\ldots,x_{d-1},\cdot) \equiv 0\}.$
\end{itemize}
    
\end{definition}

Again, the inequality $\ark(f)\le \prk(f)$ is not difficult, see \cite{KZ} and \cite{L}.  Lovett \cite{L} and Adiprasito-Kazhdan-Ziegler \cite{AKZ} conjectured that the reverse inequality also holds, up to a constant factor.

\begin{conjecture}
    There exists a constant $C_d$ such that for any finite field and multilinear form $f$ we have 
    \[
    \prk(f) \le C_d\cdot \ark(f).
    \]
\end{conjecture}

The current best bound in this direction is 
$$\prk(f) \le C_d\cdot \ark(f) \cdot \log^{d-1}(1+\ark(f)),$$
due to Moshkovitz-Zhu \cite{MZ}.

\subsection{Acknowledgements} The author would like to thank Daniel Altman, Guy Moshkovitz and Tamar Ziegler for helpful discussions.

\section{Proof of theorem \ref{main}}

Let $r=\ark(f).$ For $u\in U$ we write $f[u]$ for the bilinear form $f[u]:V\times W\to \k $ obtained by fixing the $U$ coordinate. For $v\in V$ we write $f\langle v\rangle$ for the bilinear form $f\langle v\rangle: U\times W\to \k$ obtained by fixing the $V$ coordinate. For a finite set $X,$ denote $\E_{x\in X} = \frac{1}{|X|} \sum_{x\in X}.$ Our first lemma is inspired by a recent observation of Moshkovitz-Zhu \cite{MZ}.
\begin{lemma}
    \[
    \E_{(u,v)\in Z} q^{\rk f[u]} = \E_{(u,v)\in Z} q^{\rk f\langle v\rangle} = q^r.
    \]
\end{lemma}

\begin{proof}
    A straightforward computation. Let
    $$p(u) = \frac{|\{v\in V:(u,v)\in Z\}|}{|Z|} = q^{-\rk f[u]} \frac{|V|}{|Z|}$$
    be the marginal probability mass function for $u\in U.$ Then 
    \begin{align*}
        \E_{(u,v)\in Z} q^{\rk f[u]} &=  \sum_u p(u)\cdot  q^{\rk f[u]} \\
        &= \sum_u \frac{|V|}{|Z|} = \frac{|U|\cdot|V|}{|Z|} = q^r.  
    \end{align*}
    The proof for $\rk f\langle v\rangle$ is identical.
\end{proof}

This allows us to find a large subspace $U'\subset U$ where $\rk f[u]$ is almost surely small.

\begin{lemma}\label{subspace}
    There exists a subspace $U'\subset U$ with $\codim U' \le r+1$ and 
    \[
    \P_{u\in U'} (\rk f[u] > r+s) < q^{1-s}
    \]
    for every $s>0.$ The linear forms defining $U'$ are obtained from $f$ by fixing coordinates.
\end{lemma}

\begin{proof}
     By linearity of expectation,
    \[
    \E_{(u,v)\in Z} (q^{\rk f[u]}+q^{\rk f\langle v\rangle}) = 2q^r \le q^{r+1}.
    \]
    Therefore, there must exist some $v_0\in V$ with
    \[
    q^{\rk f\langle v_0\rangle}+\E_{u\in Z(v_0)}  q^{\rk f[u]} \le q^{r+1},
    \]
    where $Z(v_0) = \{u\in U: (u,v_0)\in Z\}.$ Choosing $U' = Z(v_0),$ we have \linebreak $\codim U' = \rk f\langle v_0\rangle \le r+1$ and 
    \[
    \P_{u\in U'} (\rk f[u] > r+s) = \P_{u\in U'} (q^{\rk f[u]} > q^{r+s}) < q^{1-s} 
    \]
    by Markov's inequality.
\end{proof}

The final ingredient is a lemma of Shpilka-Haramaty \cite{HS} regarding subspaces of bilinear forms of bounded rank. We include the proof for the reader's convenience.

\begin{lemma}\label{low-rk-der}
    If $g:U\times V\times W\to k$ is a trilinear form with
    $$\P_{u\in U} (\rk g[u] > t) < \frac{q-1}{2qt}$$ for some positive integer $t$ then $\srk (g) \le 4t.$ Moreover, the linear forms in the slice rank decomposition are obtained by fixing coordinates of $g.$
    \end{lemma}

    \begin{proof}
         Let $A = \{u\in U: \rk g[u] \le t \}.$ Our assumption is that $\frac{|A|}{|U|} > 1-\frac{q-1}{2qt}.$ The proof proceeds by induction on $t.$
         
         \textbf{Base case $t=1:$} Assume there is some $u_0\in A$ with $\rk g[u_0] = 1$ (otherwise $g\equiv 0$ and there is nothing to prove). Writing $g[u_0] = \alpha(v)\beta(w),$ we claim that for all $u\in A\cap (A-u_0)$ the bilinear form $g[u]$ is contained in the ideal $(\alpha,\beta).$  Indeed, if $g[u] = \gamma(v)\delta(w)$ with $\gamma \not\in \text{span}(\alpha)$ and $\delta\not\in\text{span}(\beta)$ then 
         \[
         \rk g[u_0+u] = \rk(\alpha(v)\beta(w)+\gamma(v)\delta(w)) = 2,
         \]
         contradicting the fact that $u_0+u \in A.$ Therefore, if $V_0 = \{v: \alpha(v) = 0\}$ and $W_0 = \{w: \beta(w) = 0\}$ and $\tilde g = g\restriction_{U\times V_0\times W_0},$ we get that $\tilde g[u] = 0$ whenever $u\in A\cap (A-u_0),$  a set of density greater than $ 1- 2\cdot \frac{q-1}{2q} = \frac{1}{q}.$ This implies $\tilde g = 0$ so $g\in(\alpha,\beta)\implies \srk(g) \le 2.$ 

         
         \textbf{Inductive step:} We may assume that there is some $u_0\in A$ with $\rk g[u_0] = t,$ otherwise we're done by the inductive hypothesis. Write $g[u_0] = \sum_{i=1}^t \alpha_i(v)\beta_i(w)$ and set $V_0 = \{v:\alpha_i(v) = 0\ \forall i\in[t]\},$ $W_0 = \{w:\beta_i(w) = 0\ \forall i\in[t]\}.$ We claim that the form $\tilde g = g\restriction_{U\times V_0\times W_0}$ satisfies $\rk \tilde g[u] \le t/2 $ for all \linebreak $u\in A\cap (A-u_0).$ This will complete the proof because this set of $u$'s has density greater than $ 1-2\cdot \frac{q-1}{2qt} \ge 1- \frac{q-1}{2q\lfloor t/2\rfloor}, $ so the inductive hypothesis implies 
         \[
         \srk (\tilde g) \le 4\lfloor t/2\rfloor \le 2t \implies \srk(g) \le 4t.
         \]
        To prove the claim about $\rk \tilde g[u],$ suppose it has rank $s$ and write $\tilde g[u] = \sum_{i=1}^s \gamma_i(v)\delta_i(w).$ Note that $\alpha_i,\beta_i,\gamma_i,\delta_i$ are  linearly independent. Choosing bases, we may identify $V=\k^n,W=\k^m$ and $$ \gamma_i = v_i,\ \delta_i = w_i,\  \alpha_i = v_{s+i},\ \beta_i = w_{s+i}.$$ This yields an equation
         \[
         g[u] = \sum_{i=1}^s v_iw_i +\sum_{j=1}^t  v_{s+j} \phi_j (w)+\sum_{j=1}^t  w_{s+j} \tau_j (v).
         \]
         Decomposing $$\phi_j(w) = \sum_{i=1}^s a_{i,j} w_i + \phi'(w_{s+1},\ldots,w_m),\ \tau_j(v) = \sum_{i=1}^s b_{i,j} v_i + \tau'(v_{s+1},\ldots,v_n),$$ 
         we get 
         \[
         g[u] = \sum_{i=1}^s  \gamma'_i(v) \delta'_i(w) +q(v_{s+1},\ldots,v_n,w_{s+1},\ldots,w_m),
         \]
         where 
         \[
         \gamma'_i(v) = v_i+\sum_{j=1}^t a_{i,j} v_{s+j},\ \delta'_i(w)= w_i+\sum_{j=1}^t b_{i,j} w_{s+j}.
         \]
         The collection of linear forms $\gamma'_1,\ldots,\gamma'_s,v_{s+1},\ldots,v_n$ spans $V^*$ and so is linearly independent, likewise for $\delta'_1,\ldots,\delta'_s,w_{s+1},\ldots,w_m.$ This means that 
         \begin{equation}\label{rk-q}
             t\ge \rk g[u] = s+ \rk(q).
         \end{equation}
         Since $u+u_0\in A,$ we get 
         \begin{align*}
             t &\ge \rk(g[u]+g[u_0]) = \rk \left(\sum_{i=1}^s  \gamma'_i(v) \delta'_i(w)+\sum_{i=1}^t v_{s+i} w_{s+i}+q\right) \\
             &= s+\rk\left(\sum_{i=1}^t v_{s+i} w_{s+i}+q\right) \ge s+t-\rk(q) \ge 2s
         \end{align*}
         where the last inequality used \eqref{rk-q}. This proves that $\rk\tilde g[u] = s\le t/2,$ completing the proof of the lemma.
    \end{proof}
    Now we are ready to put everything together.

    \begin{proof}[Proof of theorem \ref{main}]
    Let $U'$ be the subspace of lemma \ref{subspace} and let \linebreak $g = f\restriction_{U'\times V\times W}.$ For $s = \lceil\log_q (r+1)\rceil+6,$ we get 
    \begin{align*}
        \P_{u\in U'} (\rk f[u] > r+s) &< q^{1-s} \le \frac{1}{q^5(r+1)}  \\
        &< \frac{1}{4(r+s)} \le \frac{q-1}{2q}\cdot \frac{1}{r+s},
    \end{align*}
    where the inequality at the start of the second line follows from 
    \[
    q^5(r+1) > 4r+8r+28 > 4(r+\lceil\log_q (r+1)\rceil+6).
    \]
    By lemma \ref{low-rk-der}, we deduce that $\srk(g) \le 4(r+s) $ and so
    \[
    \srk(f) \le r+1+4(r+s) \le 5r+4\log_q(r+1)+29.
    \]
    \end{proof}

    The proof of theorem \ref{main} is asymmetric in its treatment of $U,V,W.$ While only $r+1$ $U$-linear forms are needed, we require $\approx 2r$ linear forms on each of $V,W.$ One might hope that a more symmetric treatment would require only $r+1$ linear forms in each of $U,V,W.$ This would essentially match the best bound currently known.

    \begin{question}
        Is there an elementary proof that $\srk(f) \le 3\cdot \ark(f)+o(\ark(f))$? 
    \end{question}

\bibliographystyle{plain}
\bibliography{refs}
\end{document}