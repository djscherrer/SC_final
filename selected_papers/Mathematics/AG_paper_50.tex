\begin{document}
\affiliation{$$_affiliation_$$}
\title{Free curves in Fano hypersurfaces must have high degree}
\begin{abstract}
The purpose of this note is to show that the minimal \(e\) for which every
smooth Fano hypersurface of dimension \(n\) contains a free rational curve of
degree at most \(e\) cannot be bounded by a linear function in \(n\) when
the base field has positive characteristic. This is done by providing a
super-linear bound on the minimal possible degree of a free curve in certain
Fermat hypersurfaces.
\end{abstract}
\maketitle

\thispagestyle{empty}
\section*{Introduction}
The geometry of smooth projective Fano varieties is controlled by the rational
curves they contain. Seminal work \cite{KMM, Campana} of
Koll\'ar--Miyaoka--Mori and Campana show that, over a field of characteristic
\(0\), every smooth projective Fano variety \(X\) contains a rational curve
\(\varphi \colon \PP^1 \to X\) that, informally speaking, can be deformed to
pass through \(r+1\) general points of \(X\) for any chosen \(r \geq 0\): in
other words, \(X\) is \emph{separably rationally connected}. The precise
condition on the curve \(\varphi\) is that
\(\mathrm{H}^1(\PP^1,\varphi^*\mathcal{T}_X \otimes \sO_{\PP^1}(-r-1)) = 0\);
by way of terminology, \(\varphi\) is said to be \emph{free} or \emph{very
free} when \(r = 0\) or \(r = 1\), respectively. See \cite{Kollar,
Debarre:HDAG} for a presentation of this theory.

Whether smooth projective Fano varieties over a field of positive
characteristic are separably rationally connected is a long-standing open
question. Results are fragmentary even for smooth Fano hypersurfaces in
projective space: The general Fano hypersurface is separably rationally
connected by \cite{Zhu, CZ14, Tian, CR17}; notably, the work of Tian reduces
the problem of separable rational connectedness to \emph{separable
uniruledness}---that is, the existence of a free rational curve---a problem
that often is simpler because free curves typically have significantly lower
degree than very free curves. More recently, \cite[Theorems
3.10 and 3.24]{STZ} shows that all smooth Fano hypersurfaces with degree
less than the characteristic are separably rationally connected and even that,
up to a minor condition, such hypersurfaces always contain either free lines or
conics. See also \cite[Theorem 34]{LP} and \cite[Theorem 1.5]{BS} for related
results.

The main result of this note is that, nevertheless and contrary to experience,
the minimal \(e\) such that there exists a free rational curve of degree
\(\leq e\) on every smooth Fano hypersurface cannot be bounded by a linear
function in the dimension (or degree); contrast this with the fact that every
smooth Fano hypersurface in characteristic \(0\) contains either a free line or
conic.

\begin{IntroTheorem*}
For any algebraically closed field \(\kk\) of characteristic \(p > 0\),
\[
\limsup_{n \to \infty} \frac{1}{n}
\inf\Set{ e \in \mathbf{Z} |
\begin{array}{c}
\text{for every smooth Fano hypersurface of dimension \(n\) over \(\kk\)} \\
\text{there exists a free rational curve of degree \(\leq e\)}
\end{array}}
=
\infty.
\]
\end{IntroTheorem*}

It suffices to describe one sequence of increasingly high-dimensional
hypersurfaces without low degree free curves, and this is given by the Fermat
of degree \(q+1\) in \(\PP^{q+1}\) with \(q \coloneqq p^\nu\) for
\(\nu \geq 1\). These hypersurfaces are exceptional and exemplify many positive
characteristic phenomena: see \cite{Shen} for closely related work and
\cite[pp. 7--11]{thesis} for a general survey regarding these hypersurfaces.
What is fascinating is that these hypersurfaces contain many, many rational
curves---they are unirational!---and so the challenge is to develop techniques
to study their spaces of rational curves: see \cite{fano-schemes,
qbic-threefolds} for work in this direction. The method is to exploit a certain
tension arising from the curious differential geometry of these hypersurfaces:
the structure of the equation implies that, on the one hand, free curves cannot
lie in a hyperplane and, on the other hand, there are unexpected constraints on
the coordinate functions of the curve.

\smallskip
\noindent\textbf{Acknowledgements. ---}
This note originates from a question posed to me long ago by Aise Johan
de Jong; much gratitude for the many discussions and interest over the years.
Thanks also to Jason Starr and Remy van Dobben de Bruyn with whom I shared
helpful conversations on this topic. I was supported by a Humboldt Research
Fellowship during the preparation of this note.

\section*{Free curves in the Fermat \texorpdfstring{\(q\)}{q}-bic}
In what follows, let
\(X \coloneqq \mathrm{V}(T_0^{q+1} + \cdots + T_{q+1}^{q+1})\) be the Fermat
hypersurface of degree \(q+1\) in \(\PP^{q+1}\). The \emph{embedded tangent
bundle} \(\mathcal{E}_X\) is the vector bundle on \(X\) whose fibre at a point
\(x\) is the linear space underlying the embedded tangent space of \(X\) at
\(x\); it fits into a short exact sequence
\[
0 \to \sO_X \to \mathcal{E}_X \to \mathcal{T}_X \to 0.
\]
The extension class is the pullback via the tangent map
\(\mathcal{T}_X \to \mathcal{T}_{\PP^{q+1}}\rvert_X\) of the class of the Euler
sequence, and so there is another short exact sequence
\[
0 \to \mathcal{E}_X \to \sO_X(1)^{\oplus q+2} \to \mathcal{N}_{X/\PP^{q+1}} \to 0
\]
where the second map is
\((\psi_0,\ldots,\psi_n) \mapsto \sum\nolimits_{i=0}^{q+1} T_i^q \cdot \psi_i\).
Remarkably, as already observed by \cite[Equation (2)]{Shen}, this sequence
twisted down by \(\sO_X(-1)\) is isomorphic to the pullback of the dual Euler
sequence by the \(q\)-power Frobenius morphism \(\Fr\). In particular, there is
an isomorphism
\[
\mathcal{E}_X(-1) \cong \Fr^*(\Omega^1_{\PP^{q+1}}(1)\rvert_X).
\]

Let \(\varphi \colon \PP^1 \to X\) be a nonconstant morphism of degree
\(e = mq + r\), where \(m,r \in \mathbf{Z}\) and \(0 \leq r \leq q-1\); to
simplify notation, \(\varphi\) will sometimes be viewed as a morphism into
\(\PP^{q+1}\). Viewing \(\mathcal{E}_X\) as an extension of \(\mathcal{T}_X\),
it follows that \(\varphi\) is free if and only if
\(\mathrm{H}^1(\PP^1, \varphi^*\mathcal{E}_X \otimes \sO_{\PP^1}(-1)) = 0\).
Combined with the isomorphism above and the projection formula, this implies
that, if \(\varphi\) is free,
\begin{align*}
0
= \mathrm{H}^1(\PP^1, \varphi^*\mathcal{E}_X \otimes \sO_{\PP^1}(-1))
& = \mathrm{H}^1(\PP^1, \varphi^*\Fr^*(\Omega^1_{\PP^{q+1}}(1)\rvert_X) \otimes \sO_{\PP^1}(e-1)) \\
& = \mathrm{H}^1(\PP^1, \varphi^*(\Omega^1_{\PP^{q+1}}(1)) \otimes \Fr_*\sO_{\PP^1}(e-1)).
\end{align*}
Since
\(\Fr_*\sO_{\PP^1}(e-1) \cong \sO_{\PP^1}(m)^{\oplus r} \oplus \sO_{\PP^1}(m-1)^{\oplus q-r}\),
this shows that:

\begin{Lemma}\label{vanishing-H1}
If \(\varphi \colon \PP^1 \to X\) is free, then
\(\mathrm{H}^1(\PP^1, \varphi^*\Omega^1_{\PP^{q+1}} \otimes \sO_{\PP^1}(e + m - 1)) = 0\).
\qed
\end{Lemma}

The converse also holds. Since
\(\chi(\PP^1, \varphi^*\Omega^1_{\PP^{q+1}} \otimes \sO_{\PP^1}(e+m-1)) = (q+2)m - e - m = m - r\),
this gives a relation between \(m\) and \(r\) when \(\varphi\) is free,
recovering \cite[Theorem 1.7]{Shen}:

\begin{Lemma}\label{r-at-most-m}
If \(\varphi \colon \PP^1 \to X\) is free, then \(r \leq m\). \qed
\end{Lemma}

The following gives a geometric restriction on free curves in \(X\), and stands
in stark contrast to the fact, see \cite[V.4.4]{Kollar}, that a general
smooth Fano hypersurface contains either a free line or conic:


\begin{Lemma}\label{free-nondegen}
If \(\varphi \colon \PP^1 \to X\) is free, then \(\varphi(\PP^1)\) is not
contained in any hyperplane.
\end{Lemma}

\begin{proof}
Identifying \(\mathcal{E}_X = \ker(\sO_X(1)^{\oplus q+2} \to \sO_X(q+1))\) and
using that \(\varphi\) is free shows that
\[
\dim\mathrm{H}^0(\PP^1, \varphi^*\mathcal{E}_X \otimes \sO_{\PP^1}(-1)) = e.
\]
If \(\varphi\) were contained in a hyperplane \(\PP^q \subset \PP^{q+1}\), then
\(\varphi^*(\Omega^1_{\PP^{q+1}}(1))\) would split as
\(\varphi^*(\Omega^1_{\PP^q}(1)) \oplus \sO_{\PP^1}\). Combined with the fact
that \(\mathcal{E}_X(-1) \cong \Fr^*(\Omega^1_{\PP^{q+1}}(1))\), this would
imply that
\[
e =  \dim\mathrm{H}^0(\PP^1, \varphi^*\mathcal{E}_X \otimes \sO_{\PP^1}(-1)) =
\dim\mathrm{H}^0(\PP^1, \Fr^*\varphi^*(\Omega^1_{\PP^q}(1)) \otimes \sO_{\PP^1}(e-1)) + e.
\]
Therefore \(\Fr^*\varphi^*(\Omega^1_{\PP^q}(1)) \otimes \sO_{\PP^1}(e-1)\)
could not have global sections; in fact, freeness of \(\varphi\) means that it
can have no cohomology whatsoever, and so it must be isomorphic to
\(\sO_{\PP^1}(-1)^{\oplus q}\). However, the Euler sequence for
\(\varphi^*(\Omega^1_{\PP^q}(1))\) implies that
\[
\Fr^*\varphi^*(\Omega^1_{\PP^q}(1)) \otimes \sO_{\PP^1}(e-1) \cong
\bigoplus\nolimits_{i = 0}^q \sO_{\PP^1}(e-qa_i-1)
\]
for some integers \(a_i \geq 0\) such that \(a_0 + \cdots + a_q = e\). So
if \(e = qa_i\) for all \(i = 0,\ldots, q\), then \(a_0 = \cdots = a_q = e = 0\),
meaning that \(\varphi\) would be constant: contradiction!
\end{proof}

\begin{Remark}\label{frobenius-decomposition}
Viewing \(\varphi\) as a morphism into \(\PP^{q+1}\) and letting
\((\varphi_0: \cdots :\varphi_{q+1})\) be its components,
\parref{free-nondegen} means that the \(\varphi_i\) are linearly independent in
\(\mathrm{H}^0(\PP^1,\sO_{\PP^1}(e))\). Already, this implies \(e \geq q+1\).
Now, that \(\varphi\) factors through \(X\) means that
\(\sum_{i = 0}^{q+1} \varphi_i^q \cdot \varphi_i = 0\). Upon choosing homogeneous
coordinates \((S_0:S_1)\) for \(\PP^1\), there is a unique decomposition
\[
\varphi_i =
\sum\nolimits_{j = 0}^r \zeta_{ij}^q \cdot S_0^j S_1^{r-j} +
\sum\nolimits_{k = 1}^{q-r-1} \eta_{ik}^q \cdot S_0^{r+k} S_1^{q-k}
\]
where \(\zeta_{ij} \in \mathrm{H}^0(\PP^1, \sO_{\PP^1}(m))\) and
\(\eta_{ik} \in \mathrm{H}^0(\PP^1, \sO_{\PP^1}(m-1))\). Substituting this into
the equation of \(X\) and using the fact that \(q\) is a power of the ground
field characteristic shows that
\[
0 =
\sum\nolimits_{j = 0}^r \Big(\sum\nolimits_{i = 0}^{q+1} \varphi_i \zeta_{ij}\Big)^q \cdot S_0^j S_1^{r-j} +
\sum\nolimits_{k = 1}^{q-r-1} \Big(\sum\nolimits_{i = 0}^{q+1} \varphi_i \eta_{ik}\Big)^q \cdot S_0^{r+k} S_1^{q-k}
\]
and so, upon looking at exponents modulo \(q\),
\(\sum_{i = 0}^{q+1} \varphi_i\zeta_{ij} = \sum_{i = 0}^{q+1} \varphi_i \eta_{ik} = 0\) for all \(j\) and \(k\). These
relations are at odds with linear independence of the \(\varphi_i\), and give
strong restrictions on the degree \(e\):
\end{Remark}

\begin{Theorem}\label{free-bound}
If \(\varphi \colon \PP^1 \to X\) is free, then \(q+1 \leq m^2 + m + r\).
\end{Theorem}

\begin{proof}
Consider the linear map
\(
\Phi \colon
\mathrm{H}^0(\PP^{q+1}, \sO_{\PP^{q+1}}(1)) \to
\mathrm{H}^0(\PP^1,\sO_{\PP^1}(e))
\) defining \(\varphi \colon \PP^1 \to \PP^{q+1}\), so that
the \(i\)-th coordinate \(T_i\) maps to \(\varphi_i\). This has rank \(q+2\) by
\parref{free-nondegen}. Let \(\Fr \colon \PP^1 \to \PP^1\) be
the \(q\)-power Frobenius morphism, and identify the target of \(\Phi\) as
\[
\mathrm{H}^0(\PP^1, \sO_{\PP^1}(e)) \cong
\mathrm{H}^0(\PP^1, \Fr_*\sO_{\PP^1}(e)) \cong
\mathrm{H}^0(\PP^1, \sO_{\PP^1}(m))^{\oplus r+1} \oplus
\mathrm{H}^0(\PP^1, \sO_{\PP^1}(m-1))^{\oplus q-r-1}.
\]
Let \(\Phi_1\) and \(\Phi_2\) be the linear map obtained by post-composing
\(\Phi\) with projection to \(\mathrm{H}^0(\PP^1, \sO_{\PP^1}(m))^{\oplus r+1}\)
and \(\mathrm{H}^0(\PP^1, \sO_{\PP^1}(m-1))^{\oplus q-r-1}\), respectively.
Elementary linear algebra gives
\[
q+2 = \rank\Phi \leq \rank \Phi_1 + \rank \Phi_2 \leq (r+1)(m+1) + \rank\Phi_2,
\]
so it remains to bound the rank of \(\Phi_2\).

Let \(\Phi_{2,k} \colon \mathrm{H}^0(\PP^{q+1},\sO_{\PP^{q+1}}(1)) \to \mathrm{H}^0(\PP^1,\sO_{\PP^1}(m-1))\)
be the further projection of \(\Phi_2\) to the \(k\)-th component of its
target, so that \(\Phi_{2k}(T_i) = \eta_{ik}\), notation as in
\parref{frobenius-decomposition}. The discussion of
\parref{frobenius-decomposition} means that the \(\Phi_{2k}\) lie in the kernel
of the linear map
\begin{align*}
\Hom(\mathrm{H}^0(\PP^{q+1},\sO_{\PP^{q+1}}(1)), \mathrm{H}^0(\PP^1, \sO_{\PP^1}(m-1)))
& \longrightarrow
\mathrm{H}^0(\PP^1, \sO_{\PP^1}(e + m - 1)) \\
\Psi & \longmapsto \sum\nolimits_{i = 0}^{q+1} \varphi_i \cdot \Psi(T_i).
\end{align*}
Comparing with the Euler sequence shows that the kernel is
\(\mathrm{H}^0(\PP^1, \varphi^*\Omega^1_{\PP^{q+1}} \otimes \sO_{\PP^1}(e+m-1))\),
and this has dimension \(m-r\) by \parref{vanishing-H1}. Thus at most \(m - r\)
of the \(q-r-1\) components \(\Phi_{2,k}\) of \(\Phi_2\) are linearly independent.
Since each \(\Phi_{2,k}\) has rank at most \(m\), this implies that
\(\rank\Phi_2 \leq (m-r)m\). Combining with the inequality above gives
\[
q+2 \leq (r+1)(m+1) + (m-r)m = m^2 + m + r + 1
\]
which, upon subtracting \(1\) from either side, gives the result.
\end{proof}

Combining \parref{r-at-most-m} with \parref{free-bound} gives
\(\sqrt{q+2} - 1 \leq m+1\). Writing \(e = mq + r\) and grossly underestimating
yields a super-linear bound on the minimal degree of a free curve in \(X\):

\begin{Theorem}\label{superlinear-bound}
If \(\varphi \colon \PP^1 \to X\) is free, then \(e \geq q^{3/2} - q\).
\qed
\end{Theorem}

The strongest restrictions on possible degrees of free curves are provided by
\parref{r-at-most-m} and \parref{free-bound}. They give the following for the first
few prime powers:
\begin{center}
\begin{tabular}{l|c|c|c|c|c|c|c|c|c|c|c|c|c|c|c|c|c|c}
\(q\) & \(2\) & \(3\) & \(4\) & \(5\)  & \(7\)  & \(8\)  & \(9\)  & \(11\) & \(13\) & \(16\) & \(17\) & \(19\) & \(23\) & \(25\) & \(27\) & \(29\) & \(31\) & \(32\) \\
\hline
\(e_{\mathrm{min}}\) & \(3\) & \(6\) & \(8\) & \(10\) & \(16\) & \(24\) & \(27\) & \(33\) & \(41\) & \(64\) & \(68\) & \(76\) & \(96\) & \(125\) & \(135\) & \(145\) & \(157\) & \(163\)
\end{tabular}
\end{center}
Free curves achieving these lower bounds are known to exist in the first few
cases: see \cite[p.6]{Madore}, \cite[p.69]{Conduche}, and \cite{BDENY} for
\((q,e) = (2,3), (3,6), (4,8)\), respectively. As far as I know, no free
curves are known to exist when \(q \geq 5\). Finally, observe that the
arguments of \parref{vanishing-H1} further imply that if \(\varphi \colon \PP^1
\to X\) is free, then it is \(r\)-free, in the sense that
\(\mathrm{H}^1(\PP^1, \varphi^*\mathcal{T}_X \otimes \sO_{\PP^1}(-r-1)) =
0\), so that for many \(q\) above, a minimal possible free curve is
automatically very free.

\bibliographystyle{amsalpha}
\bibliography{main}
\end{document}
