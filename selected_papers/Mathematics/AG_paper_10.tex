\begin{document}
\affiliation{$$_affiliation_$$}
\title{The scalar product formula \\  for parahoric Deligne--Lusztig induction}
\maketitle

\begin{abstract}
  Parahoric Deligne--Lusztig induction gives rise to positive-depth representations of parahoric subgroups of $p$-adic groups. The most fundamental basic question about parahoric Deligne--Lusztig induction is whether it satisfies the scalar product formula. We resolve this conjecture for all split-generic pairs $(T,\theta)$---in particular, for all characters $\theta$ if $T$ is elliptic.  
\end{abstract}

\section{Introduction}\label{sec:introduction}

In the early 2000s, Lusztig established \cite{Lus04} an analogue of Deligne--Lusztig induction for algebraic groups arising as jet schemes $\bbG_r$ of connected reductive groups $\bbG$ over finite fields $\FF_q$. Lusztig defined, for the jet scheme $\bbT_r$ of any maximal torus $\bbT$ of $\bbG$, a functor
\begin{equation*}
  R_{\bbT_r, \bbB_r}^{\bbG_r} \from \cR(\bbT_r(\FF_q)) \to \cR(\bbG_r(\FF_q))
\end{equation*}
where $\bbB_r$ is the jet scheme of a Borel subgroup  over $\overline \FF_q$ which contains $\bbT_{\overline \FF_q}$ and $\cR$ denotes the Grothendieck ring. Lusztig proved that if a character $\theta \from \bbT_r(\FF_q) \to \overline \QQ_\ell^\times$ satisfies a genericity condition, then for any $(\bbT_r', \bbB_r', \theta')$, 
\begin{equation}\label{eq:scalar product}
  \langle R_{\bbT_r,\bbB_r}^{\bbG_r}(\theta), R_{\bbT_r',\bbB_r'}^{\bbG_r}(\theta') \rangle = \sum_{w \in W_{\bbG_r}(\bbT_r,\bbT_r')(\FF_q)} \langle \theta, {}^w \theta' \rangle.
\end{equation}
In particular, this formula proves that $R_{\bbT_r,\bbB_r}^{\bbG_r}(\theta)$ is independent of the choice of $\bbB_r$ and that $R_{\bbT_r,\bbB_r}^{\bbG_r}(\theta)$ is irreducible if $\theta$ is generic and has trivial stabilizer in the Weyl group. These results were extended by Stasinski \cite{Sta09} to mixed-characteristic jet schemes and by the author and Ivanov \cite{CI21-RT} to algebraic groups---also denoted by $\bbG_r$---arising from Moy--Prasad quotients of parahoric subgroup schemes associated to unramified maximal tori $T$ of connected reductive groups $G$ over non-archimedean local fields $F$. 

From the perspective of the representation theory of $p$-adic groups $G(F)$, this more general setting of $\bbG_r$ arising from parahoric subgroups is essential. When $T \subset G$ is elliptic, the author and Oi \cite{CO21} proved that under the aforementioned genericity condition and a largeness condition on $q$, the representations $R_{\bbT_r,\bbB_r}^{\bbG_r}(\theta)$ give rise to $L$-packets of toral supercuspidal representations in the sense of \cite{Ree08,DS18}. A serious obstruction to proving such a comparison result for regular supercuspidal representations \cite{Kal19} beyond the toral setting is establishing \eqref{eq:scalar product} in general, which is arguably the most fundamental basic question about the functor $R_{\bbT_r,\bbB_r}^{\bbG_r}$:

\begin{scalarconjecture*}
  Fix $(\theta,\bbT_r, \bbB_r)$. For all $(\theta,\bbT_r', \bbB_r')$, the formula \eqref{eq:scalar product} holds.
\end{scalarconjecture*}

When $r = 0$, it is a classical theorem of Deligne and Lusztig \cite{DL76} that the scalar product formula holds for all $(\theta,\bbT_0, \bbB_0)$. For $r>0$, this conjecture is obviously false as stated, the simplest example of which was discussed in \cite{CI21-RT}: when $\bbT_r$ is the jet scheme associated to the split torus of $\bbG$, then $R_{\bbT_r,\bbB_r}^{\bbG_r}(\theta) = \Ind_{\bbB_r(\FF_q)}^{\bbG_r(\FF_q)}(\tilde \theta)$ where $\tilde \theta = \theta \circ \pr$ for $\pr \from \bbB_r(\FF_q) \to \bbT_r(\FF_q)$. If $\theta$ factors through a character on $\bbT_0$ in general position, then $R_{\bbT_r,\bbB_r}^{\bbG_r}(\theta)$ is not irreducible for any $r > 0$.

As mentioned above, for $r>0$, thanks to \cite{Lus04,Sta09,CI21-RT} the Scalar Product Conjecture is known to be true for arbitrary $T$ if $\theta$ satisfies a genericity condition which we call $(T,G)$-generic\footnote{In general this is weaker than the analogous notion in \cite{Yu01}---see Section \ref{subsec:generic characters} for more comments.} (it is a strong nontriviality condition on the restriction of $\theta$ to $\ker(\bbT_r^\sigma \to \bbT_{r-1}^\sigma)$). Outside this setting, results are more sparse: when $G$ is an inner form of $\GL_n$, this was proved by the author and Ivanov in \cite{CI_loopGLn}, the techniques of which were vastly generalized by work of Dudas and Ivanov in \cite{DI20}, which established the Scalar Product Conjecture for $T$ Coxeter under a mild root-theoretic assumption on $q$ ($q > 5$ suffices).

In the present paper, we establish a novel approach and prove:

\begin{maintheorem}\label{thm:scalar product}
  The Scalar Product Conjecture holds if $(\theta,\bbT_r,\bbB_r)$ is split-generic.
\end{maintheorem}

For a fixed maximal torus $T \subset G$, the proportion of split-generic characters $\theta$ of $\bbT_r$ depends on the ``degree of ellipticity'' of $T$: on one extreme, if $T$ is the split torus, then $\theta$ is split-generic if and only if it is $(T,G)$-generic, and on the other extreme, if $T$ is elliptic, then all $\theta$ are split-generic. We see therefore that the Main Theorem includes all previously known progress towards the Conjecture and also explains  the spectrum of dependence on $(\theta,\bbT_r)$. We expect that this result is sharp: If $(\theta,\bbT_r)$ is not split-generic, then there exists a triple $(\theta',\bbT_r',\bbB_r')$ for which the Scalar Product Formula does not hold. We note: 
\begin{displaycorollary}
  For $T$ elliptic, $R_{\bbT_r,\bbB_r}^{\bbG_r}(\theta)$ is 
  irreducible if and only if $\Stab_{W_{\bbG_r(\FF_q)}(\bbT_r)}(\theta) = \{1\}.$
\end{displaycorollary}

The new approach to the Scalar Product Conjecture presented in this paper is to describe $R_{\bbT_r,\bbB_r}^{\bbG_r}(\theta)$ in terms of a sequence of parahoric Lusztig inductions associated to a(ny) \textit{Howe factorization} of $\theta$. The present paper appears to be the first work to define parahoric Lusztig induction, but the definition is a natural generalization of classical Lusztig induction \cite{Lus76} in the parahoric setting of \cite{CI21-RT}. We will need several general properties of parahoric Lusztig induction, which we establish in Section \ref{sec:parahoric lusztig}. Of these results, Proposition \ref{prop:twisting} is the most nontrivial (for example, see Remark \ref{rem:green insufficient} for a discussion on a proof of this fact in $r=0$ that fails for $r>0$).

This description of $R_{\bbT_r,\bbB_r}^{\bbG_r}(\theta)$ in terms of parahoric Lusztig inductions will certainly illuminate the relation to the author's recent work with Bezrukavnikov \cite{BC24} constructing generic character sheaves on parahoric group schemes.


The first serious calculation comes in Section \ref{sec:generic mackey} in establishing a \textit{generic Mackey formula} (Theorem \ref{thm:generic mackey}). In general, a Mackey formula should relate the Lusztig induction functors and their adjoints, giving a formula for the composition ${}^* R_{\bbL_r,\bbQ_r}^{\bbG_r} \circ R_{\bbM_r,\bbP_r}^{\bbG_r}$. The conjectural Mackey formula of course \textit{contains} the Scalar Product Conjecture, and even in the classical setting $r=0$, establishing the Mackey formula is well documented to be difficult: it has been resolved in many (but not all) special cases by work of \cite{DL83,DM20,BMM93,BM11,Tay18,Lus20a}. In the special case that (at least) one of $L$ or $M$ is a torus, the $r=0$ formula can be obtained by a single argument due to Deligne--Lusztig \cite{DL83} and Lusztig (see \cite{DM20}). We will prove (Theorem \ref{thm:generic mackey}) a formula for ${}^*R_{\bbT_r,\bbB_r}^{\bbG_r} \circ R_{\bbM_r,\bbP_r}^{\bbG_r}$ under a genericity condition governed by $M$; the proof combines Lusztig's $r=0$ argument together with techniques established in \cite{Lus04,Sta09,CI21-RT}. 
\begin{displaytheorem}\label{thm:display mackey}
  Let $\rho$ be any representation of $\bbM_r^\sigma$ which is $(M,G)$-generic. Then
  \begin{equation*}
    {}^*R_{\bbT_r, \bbB_r}^{\bbG_r} \circ R_{\bbM_r, \bbP_r}^{\bbG_r}(\rho)  = \sum_{w \in \bbT_r^\sigma \backslash \cS(\bbT_r, \bbM_r)^\sigma/\bbM_r^\sigma} {}^* R_{{}^w\bbT_r, {}^w\bbB_r \cap \bbM_r}^{\bbM_r}(\ad(w^{-1})^*\rho).
  \end{equation*}
\end{displaytheorem}

The overarching idea of this paper is the Scalar Product Conjecture can be resolved by iteratively leveraging generic Mackey formulas. This target iteration directs us to the problem of describing $R_{\bbT_r,\bbB_r}^{\bbG_r}(\theta)$ as the composition of parahoric Lusztig inductions. This relies on Kaletha's work \cite{Kal19} generalizing Howe's $\GL_n$ work \cite{How77} factorizing characters of maximal tori: given a character $\theta$ of $\bbT_r^\sigma$, Kaletha proves that one can write down a sequence of characters $\phi_{-1}, \phi_0, \ldots, \phi_d$ of increasing depth, where each $\phi_i$ is a character of $(\bbG_r^i)^\sigma$ for an increasing sequence of Levi subgroups $G^{-1} = T \subseteq G^0 \subset \cdots \subset G^d = G$. (The characters $\phi_i$ are not uniquely determined by $\theta$, but their depths and the associated Levi subgroups $G^i$ are.) Given a Howe factorization $\vec \phi$ of a character $\theta$ of $\bbT_r^\sigma$, we may define a virtual $\bbG_r^\sigma$-representation $r_{\bbT_r,\bbB_r}^{\bbG_r}(\vec \phi; \vec P)$ of $\bbG_r^\sigma$ obtained by inflating to a larger depth, tensoring by a generic character, and applying parahoric Lusztig induction. We prove (Proposition \ref{prop:Howe}) that if $(\theta,\bbT_r,\bbB_r)$ is split-generic, then $R_{\bbT_r,\bbB_r}^{\bbG_r}(\theta) \cong r_{\bbT_r,\bbB_r}^{\bbG_r}(\vec \phi)$. The reason this isomorphism holds is due to the following theorem:

\begin{displaytheorem}\label{thm:display depth}
  Assume $\bfT$ is elliptic. If $\theta \from \bbT_r^\sigma \to \overline \QQ_\ell^\times$ is a character which factors through $\bbT_s^\sigma$ for some $s < r$, then we have an isomorphism of virtual $\bbG_r^\sigma$-representations
  \begin{equation*}
    R_{\bbT_r,\bbB_r}^{\bbG_r}(\theta) \cong R_{\bbT_s,\bbB_s}^{\bbG_s}(\theta).
  \end{equation*}
\end{displaytheorem}

Theorem \ref{thm:display depth} follows from Theorem \ref{thm:fiber cohomology}, the true crux of this paper. Consider the $\overline \FF_q$-schemes $X_{\bbT_r,\bbB_r}^{\bbG_r}$ defining the functor $R_{\bbT_r,\bbB_r}^{\bbG_r}$. Theorem \ref{thm:fiber cohomology} calculates the cohomology of the fibers of
\begin{equation}\label{eq:r to r-1}
  X_{\bbT_r,\bbB_r}^{\bbG_r} \to X_{\bbT_{r-1},\bbB_{r-1}}^{\bbG_r}.
\end{equation}
This allows us to obtain (see  Corollary \ref{cor:infinite level}) a definition of $\ell$-adic homology groups for the infinite-depth parahoric Deligne--Lusztig variety $X_{\bbT_\infty,\bbB_\infty}^{\bbG_\infty}$. This for example endows any sufficiently well understood  $p$-adic Deligne--Lusztig space (conjectured by Lusztig \cite{Lus79} and studied by Ivanov \cite{I23arc,I23orbit} in Coxeter cases) with $\ell$-adic homology groups and shows that they encode the same representations as $X_{\bbT_\infty,\bbB_\infty}^{\bbG_\infty}$ (see Remarks \ref{rem:relation to lusztig} and \ref{rem:indirect depth compat} for more comments in this direction). Prior to this paper, Theorem \ref{thm:display depth} was known only in the setting that $G$ is an inner form of $\GL_n$ \cite{Lus79,Boy12,CI21-MA}, and in these cases, the fibers of \eqref{eq:fiber cohomology} are disjoint unions of a fixed affine space $\bbA^N$. For general $G$, while it is conceivable that this also happens, to establish Theorem \ref{thm:display depth} we prove the weaker statement that the fibers of \eqref{eq:r to r-1} share the same \textit{cohomology} as disjoint unions of $\bbA^N$ (Theorem \ref{thm:fiber cohomology}).  

After establishing Theorems \ref{thm:display mackey} and \ref{thm:display depth}, the Main Theorem is simple to prove: using the isomorphism $R_{\bbT_r,\bbB_r}^{\bbG_r}(\theta) \cong r_{\bbT_r}^{\bbG_r}(\vec \phi)$ from Proposition \ref{prop:Howe} (which depends on Theorem \ref{thm:display depth}), it is equivalent to calculate the inner product
\begin{equation*}
  \langle r_{\bbT_r}^{\bbG_r}(\vec \phi), R_{\bbT_r',\bbB_r'}^{\bbG_r}(\theta') \rangle,
\end{equation*}
which we do by applying the generic Mackey formula (Theorem \ref{thm:display mackey}) to successively peel off layers in the Howe factorization. This is the content of Section \ref{sec:scalar}.

The techniques in this paper have direct analogues for the functors arising from the Drinfeld stratification $X_{\bbT_r,\bbB_r}^{\bbL_r\bbG_r^+}$ of parahoric Deligne--Lusztig varieties (see Definition \ref{def:drinfeld}) in the sense of \cite{CI21-SM}. We explain the minor modifications required to do this and  establish the Scalar Product Conjecture for Drinfeld strata in Section \ref{sec:drinfeld}.


Allow us to mention an immediate application of the results of this paper. Assume that $T$ is elliptic, the setting of the Corollary above and of Theorem \ref{thm:display depth}. In forthcoming work with M.\ Oi, under a largeness condition on $q$, we determine $R_{\bbT_\infty,\bbB_\infty}^{\bbG_\infty}(\theta)$ in terms of Yu's construction \cite{Yu01} of tame supercuspidal representations. In particular, we can then describe $R_{\bbT_\infty,\bbB_\infty}^{\bbG_\infty}(\theta)$ in terms of Kaletha's local Langlands correspondence for regular supercuspidal representations \cite{Kal19}, thereby removing the torality assumption required in our previous work \cite{CO21}. Our methods vitally depend on the scalar product formula. 





\setcounter{tocdepth}{1}

\tableofcontents


\section{Notation}\label{sec:notation}

Let $F$ be a non-archimedean local field and let $\breve F$ denote the maximal unramified extension of $F$. We write $\cO_F$ and $\breve \cO$ for the ring of integers of $F$ and $\breve F$. Write $k_F \cong \bbF$ and $k$ for the residue fields of $F$ and $\breve F$; note that $k$ is an algebraic closure of $k_F$. Choose a uniformizer $\varpi$ of $F$. For any finite group $\G$, we write $\cR(\G)$ for the representation ring of $\G$, with coefficients in $\overline \QQ_\ell$ for $\ell \neq \Char(k_F)$.


Let $\bfG$ be a connected reductive group over $\breve F$ and $\bfT \hookrightarrow \bfG$ a split torus. We denote by $\Phi(\bfG,\bfT)$ its corresponding root system. Choose a point $\x$ in the apartment of $\bfT$ and fix a positive integer $r > 0$. By Bruhat--Tits theory and a construction of Yu \cite{Yu15}, we have an associated smooth affine $\breve \cO$-model $\cG_{\x,r}$ of $\bfG$ such that $\cG_{\x,r}(\breve \cO)$ is the $r$th Moy--Prasad filtration subgroup \cite{MP94,MP96} of the parahoric group subgroup $\cG_{\x,0}(\breve \cO) \subset G(\breve F)$. Following \cite[Section 2.5]{CI21-RT}, we consider the perfectly of finite type smooth affine group scheme $\bbG_{s:r+}$ representing the perfection of the functor
\begin{equation}
  R \mapsto \cG_{\x,s}(\bbW(R))/\cG_{\x,r+}(\bbW(R)),
\end{equation}
where $R$ is any $k$-algebra. Here, $\bbW$ denotes the Witt ring associated to $F$ if $F$ has characteristic $0$ and $\bbW(R) = R[\![\varpi]\!]$ if $F$ has positive characteristic. As in \cite[Section 2.6]{CI21-RT}, associated to any closed subgroup scheme $\bfH$ of $\bfG$, we have an associated closed subgroup scheme $\bbH_{s:r+}$ of $\bbG_{s:r+}$. Abusing notation, we define
\begin{equation*}
  \bbG_r \colonequals \bbG_{0:r+}.
\end{equation*}
We denote by $W_{\bbG_r}(\bbT_r)$ the absolute Weyl group of $\bbG_r$. 

Throughout this paper, we assume that $\bfG$ and $\bfT$ each arise as the base-change of a connected reductive group $G$ and a torus $T$ defined over $F$. We then have associated Frobenius endomorphisms $\sigma \from \bfG \to \bfG$ and $\sigma \from \bbG_r \to \bbG_r$ stabilizing $\bfT$ and $\bbT_r$ respectively. We use the superscript ${}^\sigma$ to denote the $\sigma$-fixed points, so that for example $\bbG_r^\sigma$ is a quotient of a parahoric subgroup of $G(F)$ and $\bbT_r^\sigma$ is a subquotient of $T(F)$. If $T$ satisfies some property, we will say that $\bfT$ satisfies that property \textit{over $F$}.

\section{Parahoric Lusztig induction}\label{sec:parahoric lusztig}

\subsection{Definitions}

Completely analogously to parahoric Deligne--Lusztig induction as defined in \cite{Lus04,Sta09,CI21-RT}, we may define parahoric Lusztig induction.

\begin{definition}
  Let $\bfM$ be a $F$-rational Levi subgroup of $\bfG$ containing $\bfT$ and let $\bfP$ be a parabolic subgroup of $\bfG$ with Levi component $\bfM$. Let $\bfN$ denote the unipotent radical of $\bfP$. Define the parahoric Lusztig variety to be
  \begin{equation*}
    X_{\bbM_r, \bbP_r}^{\bbG_r} \colonequals \{x \in \bbG_r : x^{-1}\sigma(x) \in \sigma(\bbN_r)\}.
  \end{equation*}
  Note that this has a natural action of $\bbM_r^\sigma \times \bbG_r^\sigma$ given by
  \begin{equation*}
    (m,g) \from x \mapsto gxm^{-1}.
  \end{equation*}
  We point out to the reader that we may not have $\sigma(\bbN_r) = \bbN_r$. Let $n$ be a positive integer such that $\sigma^n(\bbN_r) = \bbN_r$; then $X_{\bbM_r,\bbP_r}^{\bbG_r}$ is defined over $\FF_{q^n}$.
\end{definition}

\begin{definition}\label{def:induction}
  We define the functor
  \begin{equation*}
    R_{\bbM_r, \bbP_r}^{\bbG_r} \from \cR(\bbM_r^\sigma) \to \cR(\bbG_r^\sigma)
  \end{equation*}
  by the formula
  \begin{equation*}
    R_{\bbM_r,\bbP_r}^{\bbG_r}(\chi)(g) = \frac{1}{|\bbM_r(\FF_r)|} \sum_{m \in \bbM_r^\sigma} \tr((g,m) ; H_c^*(X_{\bbM_r, \bbP_r}^{\bbG_r}, \overline \QQ_\ell)) \cdot \overline{\chi(m)}.
  \end{equation*}
  The adjoint functor 
  \begin{equation*}
    {}^* R_{\bbM_r,\bbP_r}^{\bbG_r} \from \cR(\bbG_r^\sigma) \to \cR(\bbM_r^\sigma)
  \end{equation*}
  is given by the formula
  \begin{equation*}
    {}^* R_{\bbM_r,\bbP_r}^{\bbG_r}(\psi)(m) = \frac{1}{|\bbG_r^\sigma|} \sum_{g \in \bbG_r^\sigma} \tr((g,m); H_c^*(X_{\bbM_r,\bbP_r}^{\bbG_r}, \overline \QQ_\ell)) \cdot \overline{\psi(g)}.
  \end{equation*}
\end{definition}

\subsection{Properties}

We present several natural properties of parahoric Lusztig functors.

\begin{proposition}[transitivity]\label{prop:transitivity}
  Let $\bfQ \subset \bfP$ be two parabolic subgroups of $\bfG$ and let $\bfL \subset \bfM$ be $F$-rational Levi subgroups of $\bfQ$ and $\bfP$ respectively. Then
  \begin{equation*}
    R_{\bbM_r, \bbP_r}^{\bbG_r} \circ R_{\bbL_r, \bbM_r \cap \bbQ_r}^{\bbM_r} = R_{\bbL_r, \bbQ_r}^{\bbG_r}.
  \end{equation*}
\end{proposition}

\begin{proof}
  We have Levi decompositions  $\bfQ = \bfL \ltimes \bfV$ and $\bfP = \bfM \ltimes \bfN$ which induces a Levi decomposition $\bfQ \cap \bfM = \bfL \ltimes (\bfV \cap \bfM)$ of the parabolic $\bfQ \cap \bfM$ in $\bfM$. We may consider the three functors $R_{\bbM_r,\bbP_r}^{\bbG_r},R_{\bbL_r, \bbQ_r \cap \bbM_r}^{\bbM_r}, R_{\bbL_r,\bbQ_r}^{\bbG_r}$. We would like to show that there is an isomorphism
  \begin{equation*}
    H_c^*(X_{\bbM_r,\bbP_r}^{\bbG_r}, \overline \QQ_\ell) \otimes_{\overline \QQ_\ell[\bbM_r(\bbF_r)]} H_c^*(X_{\bbL_r, \bbQ_r \cap \bbM_r}^{\bbM_r}, \overline \QQ_\ell) \cong H_c^*(X_{\bbL_r, \bbQ_r}^{\bbG_r}, \overline \QQ_\ell).
  \end{equation*}
  To do this, we will prove that we have a $(\bbG_r^\sigma \times \bbL_r^\sigma)$-equivariant isomorphism of varieties
  \begin{equation*}
    X_{\bbM_r,\bbP_r}^{\bbG_r} \times_{\bbM_r^\sigma} X_{\bbL_r,\bbQ_r \cap \bbM_r}^{\bbM_r} \to X_{\bbL_r,\bbQ_r}^{\bbG_r}
  \end{equation*}
  given by restricting the multiplication map $\bbG_r \times \bbM_r \to \bbG_r$. 
  To see surjectivity, choose any $y \in X_{\bbL_r,\bbQ_r}^{\bbG_r}$ and first observe that $\sigma(\bbV_r) = (\sigma(\bbV_r) \cap \bbM_r)(\sigma(\bbV_r) \cap \sigma(\bbN_r))$. Hence we may write $y^{-1} \sigma(y) = mn$. Surjectivity of the Lang map implies that we may choose $m_0 \in \bbM_r$ such that $m_0^{-1} \sigma(m_0) = m$. Then $g_0 \colonequals ym_0^{-1}$ has the property $g_0^{-1}\sigma(g_0) = m_0 y^{-1} \sigma(y) \sigma(m_0)^{-1} = \sigma(m_0) n \sigma(m_0)^{-1} \in \sigma(m_0) \sigma(\bbN_r) \sigma(m_0)^{-1} = \sigma(\bbN_r).$ Hence we see that $y$ has a preimage $(g_0,m_0)$. To see injectivity, it suffices to prove that if $g,g' \in \bbG_r$ satisfy $g^{-1}\sigma(g), g'{}^{-1} \sigma(g') \in \sigma(\bbN_r)$ and $m,m' \in \bbM_r$ satisfy $m^{-1}\sigma(m), m'{}^{-1}\sigma(m') \in \sigma(\bbV_r) \cap \bbM_r$ and are such that $gm = g'm'$, then $g \in g'\bbM_r^\sigma$. Let $\gamma \colonequals g'{}^{-1} g = m^{-1} m' \in \bbM_r$. Then setting $g^{-1}\sigma(g) = n$ and $g'{}^{-1}\sigma(g') = n'$, we have $\sigma(\gamma) = \sigma(g'{}^{-1}g) = (g'n')^{-1}(gn) = n'{}^{-1} \gamma n$. Therefore $\gamma^{-1} \sigma(\gamma) = \gamma^{-1} n'{}^{-1} \gamma n \in \sigma(\bbN_r)$ (since $\bbM_r$ normalizes $\sigma(\bbN_r)$) and therefore $\gamma^{-1} \sigma(\gamma) \in 
  \sigma(\bbN_r) \cap \bbM_r = \{1\}$. This proves that we have an isomorphism on points and the proposition follows.
\end{proof}


Proposition \ref{prop:transitivity} has the following special case:

\begin{lemma}\label{lem:pInd factor}
  Let $\bfV$ be the largest $\sigma$-stable subgroup scheme of $\bfN$; then  $\bfV$ is the unipotent radical of a $F$-rational parabolic $\bfQ$ with Levi component $\bfL$. Then
  \begin{equation*}
    R_{\bbM_r,\bbP_r}^{\bbG_r} = \Ind_{\bbQ_r^\sigma}^{\bbG_r^\sigma} \circ \Inf_{\bbL_r^\sigma}^{\bbQ_r^\sigma} \circ R_{\bbM_r, \bbL_r \cap \bbP_r}^{\bbL_r}.
  \end{equation*}
\end{lemma}

\begin{proof}
  If $\bfP$ is $F$-rational, then $R_{\bbM_r,\bbP_r}^{\bbG_r}$ is simply given by parabolic induction:
  \begin{equation}\label{eq:pInd}
    R_{\bbM_r,\bbP_r}^{\bbG_r} = \Ind_{\bbP_r^\sigma}^{\bbG_r^\sigma} \circ \Inf_{\bbM_r^\sigma}^{\bbP_r^\sigma}.
  \end{equation}
  This follows from the surjectivity of the Lang map on $\bbN_r$: if $g \in \bbG_r$ is such that $g^{-1} \sigma(g) \in \bbN_r$, then there exists an $n \in \bbN_r$ such that $n^{-1} \sigma(n) = g^{-1} \sigma(g)$ so that we have an isomorphism
  \begin{equation*}
    X_{\bbM_r,\bbP_r}^{\bbG_r}/\bbN_r \cong \bbG_r^\sigma/\bbN_r^\sigma.
  \end{equation*}
  Now the lemma follows from \eqref{eq:pInd} together with Proposition \ref{prop:transitivity}.
\end{proof}


\begin{lemma}\label{lem:invariants}
  For any $s \leq r$, 
  we have a commutative diagram
  \begin{equation*}
    \begin{tikzcd}
      \cR(\bbM_r^\sigma) \ar{rr}{R_{\bbM_r,\bbP_r}^{\bbG_r}} \ar{d}[left]{(-)^{\bbM_{s+:r+}^\sigma}} && \cR(\bbG_r^\sigma) \ar{d}{(-)^{\bbG_{s+:r+}^\sigma}}\\
      \cR(\bbM_{s}^\sigma) \ar{rr}{R_{\bbM_{s},\bbP_{s}}^{\bbG_{s}}} && \cR(\bbG_{s}^\sigma)
    \end{tikzcd}
  \end{equation*}
  where the vertical arrows are given by taking invariants.
\end{lemma}

\begin{proof}
  Consider the surjective map $X_{\bbM_r,\bbP_r}^{\bbG_r} \to X_{\bbM_s,\bbP_s}^{\bbG_s}$. For any $\bar g \in X_{\bbM_s,\bbP_s}^{\bbG_s}$, choose a lift $g \in X_{\bbM_r,\bbP_r}^{\bbG_r}$ and write $\sigma(u) = g^{-1} \sigma(g)$. Then the fiber over $\bar g$ is isomorphic to
  \begin{equation*}
    \{g_r \in \bbG_{s+:r+} : (g_r g)^{-1}\sigma(g_r g) \in \sigma(\bbN_r)\} 
    = \{g_r \in \bbG_{s+:r+} : g_r^{-1}\sigma(g_r) \in g \sigma(\bbN_r) g^{-1}\},
  \end{equation*}
  where the equality holds since $g \sigma(\bbN_r) \sigma(g)^{-1} = g \sigma(\bbN_r) \sigma(u)^{-1} g^{-1} = g \sigma(\bbN_r) g^{-1}$. Hence the fibers of $\bbG_{s+:r+}^\sigma\backslash X_{\bbM_r,\bbP_r}^{\bbG_r} \to X_{\bbM_s,\bbP_s}^{\bbG_s}$ are each isomorphic to $\sigma(\bbN_{s+:r+}),$ an affine space. Therefore we see that as virtual representations of $\bbG_r^\sigma \times \bbM_r^\sigma$,
  \begin{equation*}
    H_c^*(X_{\bbM_r,\bbP_r}^{\bbG_r}, \overline \QQ_\ell)^{\bbG_{s+:r+}^\sigma} \cong H_c^*(X_{\bbM_s,\bbP_s}^{\bbG_s}, \overline \QQ_\ell). \qedhere
  \end{equation*}
\end{proof}


\begin{lemma}\label{lem:der}
  Let $\bfG^\der$ be the derived subgroup of $\bfG$ and consider the associated subgroups $\bbG_r^{\der}$, $\bbM_r^{\der}$, $\bbP_r^{\der},$ and $\bbN_r^{\der} = \bbN_r$. We have an isomorphism
  \begin{equation*}
    \bigsqcup_{\tau \in \bbT_r^\sigma/(\bbT_r^{\der})^\sigma} X_{\bbM_r^{\der}, \bbP_r^\der}^{\bbG_r^\der} \cdot \tau \to X_{\bbM_r,\bbP_r}^{\bbG_r}.
  \end{equation*}
\end{lemma}

\begin{proof}
  Since $\bbN_r^{\der} = \bbN_r$, it is clear that the map in question is an inclusion. It remains to show surjectivity. If $x \in X_{\bbM_r,\bbP_r}^{\bbG_r}$, then by definition we have $x^{-1} \sigma(x) \in \sigma(\bbN_r) = \sigma(\bbN_r^{\der})$, and so in particular $x \bbG_r^\der = \sigma(x) \bbG_r^{\der} \in (\bbG_r/\bbG_r^{\der})^\sigma$. By \cite[Lemma 3.1.3]{Kal19}, we have $(\bbG_r/\bbG_r^{\der})^\sigma = \bbT_r^\sigma/(\bbT_r^\der)^\sigma$, which therefore implies that $x = y \cdot \tau$ for some $y \in \bbG_r^\der$ and $\tau \in \bbS_r^\sigma$. To conclude, we note that $y^{-1} \sigma(y) = \tau x^{-1} \sigma(x) \tau^{-1} \in \tau \sigma(\bbN_r) \tau^{-1} = \sigma(\bbN_r^\der)$.
\end{proof}





We now use Lemma \ref{lem:der} to establish the behavior of the functor $R_{\bbM_r,\bbP_r}^{\bbG_r}$ under twisting.


\begin{proposition}\label{prop:twisting}
  Let $\tilde \phi \from \bbG_r^\sigma \to \overline \QQ_\ell^\times$ be any character and write $\phi = \tilde \phi|_{\bbT_r^\sigma}$. Then for any $\chi \in \cR(\bbM_r^\sigma)$,
  \begin{equation*}
    R_{\bbM_r,\bbP_r}^{\bbG_r}(\chi \otimes \phi) \cong R_{\bbM_r,\bbP_r}^{\bbG_r}(\chi) \otimes \tilde \phi^{-1}.
  \end{equation*}
\end{proposition}

\begin{proof}
  We use the Deligne--Lusztig fixed-point formula \cite[Theorem 3.2]{DL76} and Lemma \ref{lem:der}. By definition,
  \begin{equation}\label{eq:RTG def}
    R_{\bbM_r,\bbP_r}^{\bbG_r}(\chi \otimes \phi)(g)
    = \frac{1}{|\bbM_r^\sigma|} \sum_{m \in \bbM_r^\sigma} \tr((g,m); H_c^*(X_{\bbM_r,\bbP_r}^{\bbG_r},\overline \QQ_\ell)) \cdot \overline{\chi(m)} \cdot \phi(m)^{-1}.
  \end{equation}
  Let us work with the summand corresponding to $m$. Under the isomorphism in Lemma \ref{lem:der}, the action of $(g,m)$ on $x \cdot \tau$ for $x \in X_{\bbM_r^\der,\bbP_r^\der}^{\bbG_r^\der}$ is given by
  \begin{equation}\label{eq:der action}
    (g,m) \cdot (x \cdot \tau) = g_0 (\tau_g x \tau_g^{-1}) m_0^{-1} \cdot \tau_m^{-1} \tau_g \tau,
  \end{equation}
  where $g_0 \in (\bbG_r^{\der})^\sigma$ and $\tau_g \in \bbT_r^\sigma$ are such that $g = g_0 \tau_g$ and $m_0 \in (\bbM_r^{\der})^\sigma$ and $\tau_m \in \bbT_r^\sigma$ are such that $\tau_g \tau m \tau^{-1} \tau_g^{-1} = \tau_m m_0$. Applying \eqref{eq:der action}, we see that if $\tau_g \tau m \tau^{-1} \tau_g^{-1} \bbG_r^{\der} \neq g \bbG_r^{\der}$, then $(g,m)$  permutes the copies of $X_{\bbM_r^\der,\bbP_r^\der}^{\bbG_r^\der}$ with no fixed copies. Therefore
  \begin{equation*}
    \tr((g,m); H_c^*(X_{\bbM_r,\bbP_r}^{\bbG_r}, \overline \QQ_\ell)) = 0 \qquad \text{if $\tau_g \tau m \tau^{-1} \tau_g^{-1} \bbG_r^{\der} \neq g \bbG_r^{\der}$.}
  \end{equation*}
  Therefore, the summand of \eqref{eq:RTG def} corresponding to $m$ becomes
  \begin{equation*}
    \tr((g,m); H_c^*(X_{\bbM_r,\bbP_r}^{\bbG_r}, \overline \QQ_\ell)) \cdot \overline{\chi(m)} \cdot \tilde\phi(g)^{-1},
  \end{equation*}
  since $\tilde\phi$ is a character of $\bbG_r^\sigma$. This implies that we may factor out $\tilde\phi(g)^{-1}$ in \eqref{eq:RTG def}, and the lemma follows.
\end{proof}




\begin{remark}\label{rem:green insufficient}
  In the $r=0$ case, Proposition \ref{prop:twisting} follows from the Deligne--Lusztig character formula expressing $R_{\bbT_0,\bbB_0}^{\bbG_0}(\theta)$ in terms of $\theta$ and a Green function (which does not depend on $\theta$) \cite[Theorem 4.2]{DL76}. There is an analogous formula in the $r>0$ case, proved by exactly the same method as in \textit{op.\ cit.} However, the ``Green function'' that arises depends on $\theta|_{\bbT_{0+:r+}^\sigma}$, which makes this approach insufficient to prove Proposition \ref{prop:twisting}.
\end{remark}

\section{Generic Mackey formula for a torus}\label{sec:generic mackey}

\subsection{Generic characters and Howe factorizations}\label{subsec:generic characters}

\begin{definition}[$(\bfM,\bfG)$-generic character]
  A character $\phi$ of $\bfM(F)$ is \textit{$(\bfM,\bfG)$-generic of depth $r$} if for all $\alpha \in \Phi(\bfG,\bfT) \smallsetminus \Phi(\bfM,\bfT)$, we have $\phi|_{N_{E/F}(\alpha^\vee(E_r^\times))} \neq \triv$, where $E$ is a splitting field of $\bfT$. We say a representation $\rho$ of $\bbM_r^\sigma$ is $(\bfM,\bfG)$-generic if the restriction $\rho|_{\bbM_{r:r+}^\sigma}$ is the restriction of a sum of $(\bfM,\bfG)$-generic characters of depth $r$.
\end{definition}

By \cite[Lemma 3.6.8]{Kal19}, this exactly means that $\phi$ satisfies GE1 of \cite[\S8]{Yu01}. Note that in our definition of genericity, we \textit{do not} require condition GE2 of \textit{op.\ cit.} This distinction only affects finitely many primes $p$ as GE2 is automatic if $p$ is not bad for $\bfG$ and does not divide the order of $|\pi_1(\widehat \bfG_{\der})|$ (see \cite[Remark 3.4]{CO21} and \cite[\S4]{Kaletha} for more details). As a consequence of our genericity definition not requiring GE2, Kaletha's result (see Theorem \ref{thm:Howe} below) on the existence of Howe factorizations for characters of maximal tori holds for all $p$.

\begin{definition}[Howe factorization]\label{def:Howe}
  Set $\bfG^{-1} = \bfT$. 
  A \textit{Howe factorization} of $(\theta,\bfT)$ is a sequence of characters $\phi_i \from \bfG^i(F) \to \bbC^\times$ for $i = -1,0, \ldots, d$ with the following properties:
  \begin{enumerate}
    \item $\theta = \prod_{i=-1}^d \phi_i|_{T(F)}$.
    \item For all $0 \leq i \leq d$, the character $\phi_i$ is trivial on $G_{\sc}^i(F)$.
    \item For all $0 \leq i < d$, the character $\phi_i$ has depth $r_i$ and is $(\bfG^i,\bfG^{i+1})$-generic. For $i = d$, we take $\phi_d = 1$ if $r_d = r_{d-1}$ and has depth $r_d$ otherwise. For $i = -1,$ the character $\phi_{-1}$ is trivial if $\bfG^0 = \bfT$ and otherwise satisfies $\phi_{-1}|_{T(F)_{0+}} = 1$.
  \end{enumerate}
  We call $d$ the \textit{Howe factorization length} of $(\theta,\bfT)$.
\end{definition}

Of essential importance in this paper is the following result of Kaletha:

\begin{theorem}[{\cite[Proposition 3.6.7]{Kal19}}]\label{thm:Howe}
  Any character $\theta$ of $T(F)$ has a Howe factorization $(\phi_{-1}, \ldots, \phi_d)$.
\end{theorem}

\begin{definition}\label{def:rTG}
  Given a Howe factorization $\vec \phi = (\phi_{-1},\ldots, \phi_d)$, choose a nested sequence of parabolic subgroups $\bfP^{i-1} \subset \bfG^{i}$ with Levi component $\bfG^{i-1}$ so that we have
  \begin{equation*}
    \begin{tikzcd}
      \bfT = \bfG^{-1} \ar[phantom]{r}{\subseteq} \ar[phantom]{d}[sloped]{\subsetneq} & \bfG^0 \ar[phantom]{r}{\subsetneq}\ar[phantom]{d}[sloped]{\subsetneq} & \bfG^1 \ar[phantom]{r}{\subsetneq}\ar[phantom]{d}[sloped]{\subsetneq} & \cdots \ar[phantom]{r}{\subsetneq} & \bfG^{d-1} \ar[phantom]{r}{\subsetneq}\ar[phantom]{d}[sloped]{\subsetneq} & \bfG^d = \bfG \ar[phantom]{d}[sloped]{=} \\
      \bfB = \bfP^{-1} \ar[phantom]{r}{\subseteq} & \bfP^0 \ar[phantom]{r}{\subsetneq} & \bfP^1 \ar[phantom]{r}{\subsetneq} & \cdots \ar[phantom]{r}{\subsetneq} & \bfP^{d-1} \ar[phantom]{r}{\subsetneq} & \bfG
    \end{tikzcd} 
  \end{equation*}
  Define for $0 \leq i \leq d$:
  \begin{equation*}
    r_{\bbT_{r_i}}^{\bbG_{r_i}^{i}}(\phi_{-1}, \ldots, \phi_i; \vec \bfP) = \Inf_{\bbG_{r_{i-1}}^{i \sigma}}^{\bbG_{r_i}^{i \sigma}}\left(R_{\bbG_{r_{i-1}}^{i-1},\bbP_{r_{i-1}}^{i-1}}^{\bbG_{r_{i-1}}^{i}}\left(r_{\bbT_{r_{i-1}}}^{\bbG_{r_{i-1}}^{i-1}}(\phi_{-1}, \ldots, \phi_{i-1}; \vec \bfP)\right)\right) \otimes \phi_i.
  \end{equation*}
  We write
  \begin{equation*}
    r_{\bbT_r}^{\bbG_r}(\vec \phi; \vec \bfP) \colonequals r_{\bbT_{r_d}}^{\bbG_{r_d}^{d}}(\phi_{-1}, \ldots, \phi_d; \vec \bfP).
  \end{equation*}
\end{definition}

\subsection{Generic Mackey formula}\label{subsec:generic mackey}

Set $\cS(\bbT_r, \bbM_r) = \{x \in \bbG_r(\overline \FF_q) : x^{-1} \bbT_r x \subset \bbM_r\}$. We have an identification $\bbT_r(\overline \FF_q) \backslash \cS(\bbT_r,\bbM_r) / \bbT_r(\overline \FF_q) \cong \bbT_0(\overline \FF_q) \backslash \cS(\bbT_0,\bbM_0) / \bbT_0(\overline \FF_q)$ and the generalized Bruhat decomposition
\begin{equation*}
  \bbG_0 = \bigsqcup_{w \in \bbT_0(\overline \FF_q) \backslash \cS(\bbT_0, \bbM_0)/\bbM_0(\overline \FF_q)} \bbU_0 \dot w \bbM_0 \bbN_0
\end{equation*}
pulls back to a decomposition
\begin{equation*}
  \bbG_r = \bigsqcup_{w \in \bbT_0(\overline \FF_q) \backslash \cS(\bbT_r, \bbM_r)/\bbM_0(\overline \FF_q)} \bbG_{r,w},
\end{equation*}
where
\begin{equation*}
  \bbG_{r,w} \colonequals \bbU_r \dot w \bbM_r \bbN_r = \bbB_r \bbK_{w,0+:r+} \dot w \bbP_r, \qquad \bfK_w \colonequals \bfU^- \cap \dot w \bfN^- \dot w^{-1}.
\end{equation*}

The main theorem of this section will be a formula relating the parahoric Lusztig and Deligne--Lusztig inductions $R_{\bbM_r \subset \bbP_r}^{\bbG_r}$ and $R_{\bbT_r \subset \bbB_r}^{\bbG_r}$. 

\begin{theorem}[Generic Mackey formula]\label{thm:generic mackey}
  Let $\rho$ be any representation of $\bbM_r^\sigma$ which is $(\bfM,\bfG)$-generic. Then
  \begin{equation*}
    {}^*R_{\bbT_r, \bbB_r}^{\bbG_r} \circ R_{\bbM_r, \bbP_r}^{\bbG_r}(\rho)  = \sum_{w \in \bbT_r^\sigma \backslash \cS(\bbT_r, \bbM_r)^\sigma/\bbM_r^\sigma} {}^* R_{{}^w\bbT_r, {}^w\bbB_r \cap \bbM_r}^{\bbM_r}(\ad(w^{-1})^*\rho).
  \end{equation*}
\end{theorem}



In Section \ref{sec:scalar}, we will apply the following reformulation of Theorem \ref{thm:generic mackey}:

\begin{corollary}\label{cor:generic mackey}
  Let $\rho$ be a $(\bfM,\bfG)$-generic representation of $\bbM_r^\sigma$ and let $\theta$ be any character of $\bbT_r^\sigma$. Then
  \begin{equation*}
    \langle R_{\bbT_r,\bbB_r}^{\bbG_r}(\theta), R_{\bbM_r,\bbP_r}^{\bbG_r}(\rho) \rangle_{\bbG_r^\sigma} = \sum_{w \in \bbT_r \backslash \cS(\bbT_r, \bbM_r)^\sigma/\bbM_r} \langle R_{\bbT_r,\bbB_r \cap {}^w \bbM_r}^{{}^w \bbM_r}(\theta), \ad(w^{-1})^* \rho \rangle_{{}^w \bbM_r^\sigma}.
  \end{equation*}
\end{corollary}

\begin{proof}
  Hence for any $(\bfM,\bfG)$-generic representation $\rho$ of $\bbM_r^\sigma$ and any character $\theta$ of $\bbT_r^\sigma$, we have
  \begin{align*}
    \langle R_{\bbT_r,\bbB_r}^{\bbG_r}(\theta), R_{\bbM_r,\bbP_r}^{\bbG_r}(\rho) \rangle_{\bbG_r^\sigma} 
    &= \langle \theta, {}^* R_{\bbT_r,\bbB_r}^{\bbG_r}(R_{\bbM_r,\bbP_r}^{\bbG_r}(\rho)) \rangle_{\bbT_r^\sigma} \\
    &= \sum_{w \in \bbT_r \backslash \cS(\bbT_r, \bbM_r)^\sigma/\bbM_r}\langle \theta, {}^* R_{\bbT_r,\bbB_r \cap {}^w \bbM_r}^{{}^w \bbM_r}(\ad(w^{-1})^*(\rho)) \rangle_{\bbT_r^\sigma} \\
    &= \sum_{w \in \bbT_r \backslash \cS(\bbT_r, \bbM_r)^\sigma/\bbM_r} \langle R_{\bbT_r, \bbB_r \cap {}^w \bbM_r}^{{}^w \bbM_r}(\theta), \ad(w^{-1})^* \rho \rangle_{{}^w \bbM_r^\sigma},
  \end{align*}
  where the first and third equalities hold by adjointness and the second equality holds by Theorem \ref{thm:generic mackey}.
\end{proof}

We will prove Theorem \ref{thm:generic mackey} over the course of the next three subsections, culminating with Section \ref{subsec:generic mackey proof}. The calculation proceeds by analyzing the cohomology of the fiber product $X_{\bbT_r, \bbB_r}^{\bbG_r} \times_{\bbG_r^\sigma} X_{\bbM_r, \bbP_r}^{\bbG_r}$. We have an isomorphism
\begin{align*}
  X_{\bbT_r, \bbB_r}^{\bbG_r} \times_{\bbG_r^\sigma} X_{\bbM_r, \bbP_r}^{\bbG_r} &\to \{(x,x',y) \in \sigma(\bbU_r) \times \sigma(\bbN_r) \times \bbG_r : x \sigma(y) = yx'\} \equalscolon \Sigma, \\
  (g,g') &\mapsto (g^{-1}\sigma(g), g'{}^{-1}\sigma(g'), g^{-1} g'), 
\end{align*}
where $\bfU$ is the unipotent radical of $\bfB$ and $\bfN$ is the unipotent radical of $\bfP$. Note that this isomorphism is $(\bbT_r^\sigma \times \bbM_r^\sigma)$-equivariant with respect to the action on $\Sigma$ given by
\begin{equation*}
  (t,m) \from (x,x',y) \mapsto (txt^{-1}, mx'm^{-1}, tym^{-1}).
\end{equation*}
For each double coset $w \in \bbT_r(\overline \FF_q) \backslash \cS(\bbT_r, \bbM_r)/\bbM_r(\overline \FF_q)$, set
\begin{equation*}
  \Sigma_w \colonequals \{(x,x',y) \in \Sigma : y \in \bbG_{r,w}\}.
\end{equation*}
It is clear that each $\Sigma_w$ is $(\bbT_r^\sigma \times \bbM_r^\sigma)$-stable.


\begin{lemma}\label{lem:Sigma hat}
  The cohomology of
  \begin{equation*}
    \widehat \Sigma_w \colonequals \{(x,x',u,u',z,\mu) \in \sigma(\bbU_r) \times \sigma(\bbN_r) \times \bbU_r \times \bbN_r \times \bbK_{w,0+:r+} \times \bbM_r : x \sigma(z \dot w \mu) = u z \dot w \mu u' x'\}
  \end{equation*}
  is isomorphic as a $(\bbT_r^\sigma \times \bbM_r^\sigma)$-module to that of $\Sigma_w$. This isomorphism is induced by the affine fibration $\widehat \Sigma_w \to \Sigma_w$ given by composing the isomorphism
  \begin{equation*}
    (x,x',u,u',z,\mu) \mapsto (x\sigma(u)^{-1}, x'\sigma(u'), u,u',z,\mu)   
  \end{equation*}
  with the affine fibration
  \begin{equation*}
    (x,x',u,u',z,\mu) \mapsto (x,x',uz \dot w \mu u').
  \end{equation*}
  Both these maps are $(\bbT_r^\sigma \times \bbM_r^\sigma)$-equivariant, where the action on $\widehat \Sigma_w$ is given by
  \begin{equation}\label{eq:action}
    (m,t) \from (x,x',u,u',z,\mu) \mapsto (txt^{-1}, mx'm^{-1}, tut^{-1}, mu'm^{-1}, tzt^{-1}, \dot w^{-1} t \dot w \mu m^{-1}).
  \end{equation}
\end{lemma}


Define
\begin{align*}
  \widehat \Sigma_w' &\colonequals \{(x,x',u,u',z,\mu) \in \widehat \Sigma_w : z \neq 1\}, \\
  \widehat \Sigma_w'' &\colonequals \{(x,x',u,u',z,\mu) \in \widehat \Sigma_w : z = 1\}.
\end{align*}
Theorem \ref{thm:generic mackey} will follow as a corollary (see Section \ref{subsec:generic mackey proof}) after we show that the cohomology of $\widehat \Sigma_w'$ does not contribute to the generic Mackey formula (Proposition \ref{prop:Sigma'}, proved in Section \ref{subsec:Sigma' proof}) and the cohomology of $\widehat \Sigma_w''$ is equal to the $w$-summand on the right-hand side of the Mackey formula (Proposition \ref{prop:Sigma''}, proved in Section \ref{subsec:Sigma'' proof}).

\begin{proposition}\label{prop:Sigma'}
  Let $\psi \from \bbM_{r:r+}^\sigma \to \overline \QQ_\ell^\times$ be $(\bfM,\bfG)$-generic. If $w$ has a representative in $\cS(\bbM_r, \bbT_r)^\sigma$ and for all $i \geq 0$,
  \begin{equation*}
    H_c^i(\widehat \Sigma_w', \overline \QQ_\ell)_{(\psi)} = 0,
  \end{equation*}
  where $H_c^i(\widehat \Sigma_w', \overline \QQ_\ell)_{(\psi)}$ is the subspace on which $\bbM_{r:r+}^\sigma$ acts by $\psi$.
\end{proposition}

\begin{proposition}\label{prop:Sigma''}
  If $w$ has a representative in $\cS(\bbT_r, \bbM_r)^\sigma$, then we have isomorphisms of virtual $(\bbT_r^\sigma \times \bbM_r^\sigma)$-representations
  \begin{equation*}
    \sum_{i \geq 0} (-1)^i H_c^i(\widehat \Sigma_w'', \overline \QQ_\ell) \cong \sum_{i \geq 0} (-1)^i H_c^i(X_{{}^w \bbT_r, {}^w \bbB_r \cap \bbM_r}^{\bbM_r}, \overline \QQ_\ell).
  \end{equation*}
  where $\bbM_r^\sigma$ acts on $X_{\bbT_r, \bbB_r \cap {}^w\bbM_r}^{{}^w \bbM_r}$ through $\ad(w) \from \bbM_r^\sigma \to {}^w \bbM_r^\sigma$.
  If $w$ does not have a representative in $\cS(\bbT_r, \bbM_r)^\sigma$, then $\sum_{i \geq 0} (-1)^i H_c^i(\widehat \Sigma_w'', \overline \QQ_\ell) = 0$.
\end{proposition}


\subsection{Proof of Proposition \ref{prop:Sigma'}}\label{subsec:Sigma' proof}

The proof is a natural generalization of the arguments in \cite{Lus04,Sta09,CI21-RT}. Following \cite[Section 3.5, esp.\ (3.7)]{CI21-RT}, we have a stratification into locally closed subsets
\begin{equation*}
  \bbK_{w,0+:r+} \smallsetminus \{1\} = \bigsqcup_{1 \leq a \leq r} \bigsqcup_{I \in \cX} \bbK_{w,r}^{a,I},
\end{equation*}
where $\cX$ is the set of nonempty subsets of $\{\beta \in \Phi(\bfG,\bfT) \smallsetminus \Phi(\bfM,\bfT) : \bbU_{\beta,r} \subset \bbK_{w,r}\}$ where $\bfU_\beta$ is the root subgroup of $\bfG$ corresponding to $\beta \in \Phi(\bfG,\bfT)$. By pulling back along the natural projection $\widehat \Sigma_w' \to \bbK_{w,0+:r+} \smallsetminus \{1\}$, we have an induced stratification
\begin{equation*}
  \widehat \Sigma_w' = \bigsqcup_{a,I} \widehat \Sigma_w^{\prime, a, I}.
\end{equation*}

Fix a pair $(a,I)$ with $1 \leq a \leq r$ and $I \in \cX$.  Consider the morphism
\begin{equation*}
  \widehat \Sigma_w^{\prime, a, I} \to \bbM_{0}, \qquad (x,x',u,u',z,\mu) \mapsto \mu \bbM_{0+:r+}.
\end{equation*}
Let $\widehat \Sigma_{w,\bar\mu}^{\prime, a, I}$ denote the fiber over $\bar \mu = \mu\bbM_{0+:r+} \in \bbM_{0}$.

\begin{lemma}\label{lem:w mu action}
  Let $\alpha \in \Phi(\bfG,\bfT)$ be such that $-\alpha \in I$. Then
  $\widehat \Sigma_{w,\bar \mu}^{\prime, a, I}$ has an action of the algebraic group
  \begin{equation*}
    \cH_{\bar \mu} \colonequals \{m \in \bbM_{r:r+} : m \sigma(m)^{-1} \in \mu^{-1} \dot w^{-1} \bbT_{r:r+}^\alpha \dot w \mu\}.
  \end{equation*}
\end{lemma}

\begin{proof}
  Choose any $z \in \bbK_{w,r}^{a,I}$. For any $\xi \in \bbU_{\alpha,r-a:r+}$, consider the commutator $[\xi^{-1},z^{-1}] = \xi^{-1} z^{-1} \xi z$. By \cite[Proposition 3.8]{CI21-RT} together with the fact that $\xi$ and $z$ are both normalized by $\dot w \bbM_r \dot w^{-1}$, the construction of $\bbK_{w,r}^{a,I}$ ensures that $[\xi^{-1},z^{-1}] \in [\bbU_{\alpha,r-a:r+},\bbK_{w,r}^{a,I}]$ takes values in $\bbT_{r:r+}^\alpha(\dot w \bbN_{r:r+} \dot w^{-1})$, where $\bfT^\alpha$ is rank-1 subtorus of $\bfT$ contained in group generated by $\bfU_\alpha$ and $\bfU_{-\alpha}$. In particular, we may now define 
  \begin{equation*}
    [\xi^{-1},z^{-1}] = \tau_{\xi,z} \cdot \omega_{\xi,z}, \qquad \text{where $\tau_{\xi,z} \in \bbT_{r:r+}^\alpha$ and $\omega_{\xi,z} \in \dot w \bbN_{r:r+} \dot w^{-1}$.}
  \end{equation*} 
  Moreover, the assignment $\xi \mapsto \tau_{\xi,z}$ defines a map $\lambda_z \from \bbU_{\alpha,r-a:r+} \to \bbT_{r:r+}^\alpha$ which factors through an isomorphism $\bbU_{\alpha,r-a:(r-a)+} \cong \bbT_{r:r+}^\alpha$. Fix a section $s_z \from \bbT_{r:r+}^\alpha \to \bbU_{\alpha,r-a:r+}$ of $\lambda_z$.

  For notational convenience, write $\cH \colonequals \cH_{\bar \mu}$. For $m \in \cH$, consider the function
  \begin{equation}\label{eq:m action}
    f_{m}(x,x',u,u',z,\mu) = (x \sigma(\xi), \hat x', u, \sigma(m)^{-1}u'\sigma(m), z, \mu \sigma(m))
  \end{equation}
  for $(x,x',u,u',z,\mu) \in \widehat \Sigma_{w,\bar \mu}^{\prime,a,I}$, where
  \begin{equation*}
    \xi = s_z(\dot w \mu m \sigma(m)^{-1} \mu^{-1} \dot w^{-1}) \in \bbU_{\alpha,r-a:r+} \subset \bbU_r \cap \dot w \bbN_r \dot w^{-1}
  \end{equation*}
  and $\hat x'$ is defined by 
  $x \sigma(\xi) \sigma(z) \sigma(\dot w) \sigma(\mu) \sigma^2(m) = u z \dot w \mu u' \sigma(m) \hat x'$.   
  
  It is a quick argument to see that $f_{m'} \circ f_{m} = f_{mm'}$. Indeed, in the first coordinate, this amounts to observing that $\bbM_{r:r+}$ is commutative, and in coordinates 3 thorugh 6, it is obvious. It follows from this that $f_{m'} \circ f_{m} = f_{mm'}$ also holds in the second coordinate. Hence to see that $f_m$ defines an action on $\widehat \Sigma_{w,\bar \mu}'$, it remains to show that the image under $f_m$ of any $(x,x',u,u',z,\mu) \in \widehat \Sigma_{w, \bar \mu}'$ lies in $\widehat \Sigma_{w, \bar \mu}'$. To do this amounts to showing $\hat x' \in \sigma(\bbN_r)$, and we spend the rest of the proof doing this.

  The argument to show $\hat x' \in \sigma(\bbN_r)$ is exactly the same as in \cite[p.\ 7]{Lus04}. We provide it here for completeness. The statement
  \begin{equation*}
    x \sigma(\xi) \sigma(z \dot w \mu \sigma(m)) \in u z \dot w \mu u' \sigma(m) \sigma(\bbN_r)
  \end{equation*}
  holds if and only if
  \begin{equation*}
    x \sigma(z) \sigma(\xi) \sigma(\tau_{\xi,z}) \sigma(\omega_{\xi,z}) \sigma(\dot w \mu \sigma(m)) \in u z \dot w \mu u' \sigma(m) \sigma(\bbN_r) 
  \end{equation*}
  since by definition $\xi z = z \xi \tau_{\xi, z} \omega_{\xi, z}$ where $\tau_{\xi, z} \in \bbT_{r:r+}^\alpha$ and $\omega_{\xi,z} \in \dot w \bbN_{r:r+} \dot w^{-1}$. By definition, we have $x\sigma(z) = u z \dot w \mu u' x' \sigma(\mu)^{-1} \sigma(\dot w)^{-1}$, so the previous statement holds if and only if
  \begin{equation*}
    x' \sigma(\mu)^{-1} \sigma(\dot w)^{-1} \sigma(\xi) \sigma(\tau_{\xi, z}) \sigma(\omega_{\xi,z}) \sigma(\dot w \mu \sigma(m)) \in \sigma(m) \sigma(\bbN_r).
  \end{equation*}
  By construction, $x' \in \sigma(\bbN_r)$ and $\sigma(\dot w^{-1}) \sigma(\omega_{\xi,z}) \sigma(\dot w) \in \sigma(\bbN_r)$ and $\sigma(\dot w^{-1}) \sigma(\xi) \sigma(\dot w) \in \sigma(\bbN_r)$. Since $\bbM_r$ normalizes $\bbN_r$, the previous statement holds if and only if
  \begin{equation*}
    \sigma(\mu)^{-1} \sigma(\dot w)^{-1} \sigma(\tau_{\xi,z}) \sigma(\dot w) \sigma(\mu \sigma(m)) \in \sigma(m) \sigma(\bbN_r),
  \end{equation*}
  and projecting to the Levi component $\bbM_r$, we see that the previous statement holds if and only if
  \begin{equation*}
    \mu^{-1} \dot w^{-1} \tau_{\xi,z} \dot w \mu = m \sigma(m)^{-1},
  \end{equation*}
  which follows from the definition of $\xi$.
\end{proof}

We can find $n \geq 1$ such that $\sigma^{n}(\mu^{-1} \dot w^{-1} \bbT_{r:r+}^\alpha \dot w \mu) = \mu^{-1} \dot w^{-1} \bbT_{r:r+}^\alpha \dot w \mu$. Then we have a morphism
\begin{equation*}
  \cN_\sigma^{\sigma^n} \from \mu^{-1} \dot w^{-1} \bbT_{r:r+}^\alpha \dot w \mu \to \cH, \qquad m \mapsto m \sigma(m) \sigma^{2}(m) \cdots \sigma^{n-1}(m)
\end{equation*}
since
\begin{equation*}
  \cN_\sigma^{\sigma^n}(m) \sigma(\cN_\sigma^{\sigma^n}(m))^{-1} =  \cN_\sigma^{\sigma^n}(m\sigma(m)^{-1}) = t' \sigma^n(m)^{-1} \in \mu^{-1} \dot w^{-1} \bbT_{r:r+}^\alpha \dot w \mu,
\end{equation*}
where the second equality holds since $\bbM_{r:r+}$ is commutative. 

\begin{lemma}\label{lem:H^0}
  The intersection $\cH^0 \cap \bbM_{r:r+}^\sigma$ contains $\cN_\sigma^{\sigma^n}((\mu^{-1} \dot w^{-1} \bbT_{r:r+}^\alpha \dot w \mu)^{\sigma^n})$.
\end{lemma}

\begin{proof}
Since $\bbT_{r:r+}^\alpha$ is connected, its image in $\cH$ under $\cN_\sigma^{\sigma^n}$ must also be connected. If $m \in (\mu^{-1} \dot w^{-1} \bbT_{r:r+}^\alpha \dot w \mu)^{\sigma^n}$, then $\cN_\sigma^{\sigma^n}(m)$ is $\sigma$-stable, so the desired conclusion follows.
\end{proof}

By Lemma \ref{lem:w mu action}, the connected algebraic group $\cH^0$ acts on $H_c^i(\widehat \Sigma_{w,\bar\mu}^{\prime, a, I}, \overline \QQ_\ell)$, and by general principles this action must be trivial. Hence by Lemma \ref{lem:H^0}, we know that the finite group $\cN_\sigma^{\sigma^n}((\mu^{-1} \dot w^{-1} \bbT_{r:r+}^\alpha\dot w \mu)^{\sigma^n})$ acts trivially on $H_c^i(\widehat \Sigma_{w,\bar\mu}^{\prime, a, I}, \overline \QQ_\ell)$. On the other hand, by construction, we have $\dot w^{-1} \cdot \alpha \notin \Phi(\bfM,\bfT)$, so the $(\bfM,\bfG)$-genericity of $\psi$ implies that $\psi \circ \cN_\sigma^{\sigma^n}$ is nontrivial on $\cN_\sigma^{\sigma^n}((\mu^{-1} \dot w^{-1} \bbT_{r:r+}^\alpha \dot w \mu)^{\sigma^n})$. Therefore,
\begin{equation*}
  H_c^i(\widehat \Sigma_{w, \bar \mu}^{\prime, a, I}, \overline \QQ_\ell)_{(\psi)} = 0 \qquad \text{for all $i \geq 0$}.
\end{equation*} 
Since $\bar \mu, a, I$ are all chosen arbitrarily, the conclusion of the proposition follows.

\subsection{Proof of Proposition \ref{prop:Sigma''}}\label{subsec:Sigma'' proof}
  By the Deligne--Lusztig fixed-point formula \cite[Theorem 3.2]{DL76}, if $H$ is any algebraic torus which acts on $\widehat \Sigma_w''$ compatibly with the action of $\bbM_r^\sigma \times \bbT_r^\sigma$, then we have an isomorphism
  \begin{equation*}
    \sum_{i \geq 0} (-1)^i H_c^i(\widehat \Sigma_w'', \overline \QQ_\ell) \cong     \sum_{i \geq 0} (-1)^i H_c^i((\widehat \Sigma_w'')^{H}, \overline \QQ_\ell)
  \end{equation*}
  of virtual $(\bbM_r^\sigma \times \bbT_r^\sigma)$-representations. In this proof, we will construct such an algebraic torus (we will call it $\bar H_w^0$) and show that either $(\widehat \Sigma_w'')^{\bar H_w^0}$ is empty or has cohomology equal (up to an even shift) to the cohomology of the parahoric Deligne--Lusztig variety $X_{\bbT_r, \bbB_r \cap {}^w \bbM_r}^{{}^w \bbM_r}$. 

  Recall that $(x,x',u,u',1,\mu) \in \widehat \Sigma_w''$ if and only if $x\sigma(\dot w \mu) = u \dot w \mu u' x'$. Then a straightforward calculation shows that equation \eqref{eq:action} also defines an action of
  \begin{equation*}
    H_w \colonequals \{(t,m) \in \bbT_r \times Z(\bbM_r) : 
    t^{-1}\sigma(t) = \sigma(\dot w) m^{-1}\sigma(m) \sigma(\dot w)^{-1}\}
  \end{equation*}
  on $\widehat \Sigma_w''$. Let $\bar H_w$ denote the image of $H_w$ under the surjection $\bbT_r \times Z(\bbM_r) \to \bbT_0 \times Z(\bbM_0)$. Then the identity component $\bar H_w^0$ is an algebraic torus.

  \begin{claim}\mbox{}
    \begin{enumerate}[label=(\alph*)]
      \item The projection map $\bar H_w^0 \to Z(\bbM_0)$ has image containing $Z(\bbM_0)^0$.
      \item If $w$ has a representative $\dot w$ in $\cS(\bbT_r,\bbM_r)^\sigma$, then $(\widehat \Sigma_w'')^{\bar H_w^0} \cong S_w$, where
      \begin{equation*}
        S_w \colonequals \{(u,\mu) \in (\bbU_r \cap \dot w \bbM_r \dot w^{-1}) \times \dot w \bbM_r \dot w^{-1} : u \mu \sigma(\mu)^{-1} \in \sigma(\bbU_r)\}.
      \end{equation*}
      Otherwise, $(\widehat \Sigma_w'')^{\bar H_w^0} = \varnothing.$
    \end{enumerate}
  \end{claim}

  \begin{proof}[Proof of Claim]
    For (a): Let $m \in Z(\bbM_r)^0$. Then $m^{-1}\sigma(m) \in Z(\bbM_r)^0 \subset \bbT_r$ and of course $\sigma(\dot w)^{-1} m^{-1} \sigma(m) \sigma(\dot w) \in \bbT_r$, so by the surjectivity of the Lang map, there is some $t \in \bbT_r$ such that $(t,m) \in H_w$. Hence the image of $H_w$ in the projection to $Z(\bbM_r)$ contains $Z(\bbM_r)^0$, and the same is true of $\bar H_w \to Z(\bbM_0)$. But now the connectedness of $Z(\bbM_0)^0$ implies that $\bar H_w^0$ projects surjectively onto $Z(\bbM_0)^0$.

    For (b): We compute on $\overline \FF_q$-points. Assume $(\widehat \Sigma_w'')^{\bar H_w^0} \neq \varnothing$ and let $\dot w$ be any representative of $w$. Then in particular there exists a $\mu \in \bbM_r$ such that $\dot w^{-1} t \dot w m \mu = \mu$ for all $(t,m) \in \bar H_w^0$, which implies that $\dot w^{-1} t \dot w = m$ for all $(t,m) \in \bar H_w^0$. On the other hand, by (a), this implies that
    \begin{equation*}
      \bar H_w^0 = \{(\dot w m  \dot w^{-1}, m) : m \in Z(\bbM_0)^0\}.
    \end{equation*}
    This then implies that for any $(x,x',u,u',1,\mu) \in (\widehat \Sigma_w'')^{\bar H_w^0}$, the elements $x,u$ centralize $\dot w Z(\bbM_0)^0 \dot w^{-1}$ and the elements $x',u'$ centralize $Z(\bbM_0)^0$. Since we have $Z_{\bbG_r}(Z(\bbM_0)^0) = \bbM_r$, we see that $x,u \in \dot w \bbM_r \dot w^{-1}$ and $x',u' \in \bbM_r$. Since $\bbN_r \cap \bbM_r = \{1\}$, this implies $x' = u' = 1$ and $x\sigma(\dot w \mu) = u \dot w \mu$. This implies that $w$ can be represented by an element $\dot w'$ such that $\sigma(\dot w')\dot w'{}^{-1} \in \bbM_r$, which implies that the double coset $w$ has a representative in $\cS(\bbT_r, \bbM_r)^\sigma$. From this argument, plus a simple elementary manipulation of terms, we now see (b) of the Claim.
  \end{proof}

  Part (b) of the Claim implies the last sentence of Proposition \ref{prop:Sigma''}. Now assume that $w$ has a representative $\dot w \in \cS(\bbT_r,\bbM_r)^\sigma$. We can see that the multiplication map 
  \begin{equation*}
    S_w \to {}^w \bbM_r, \qquad (u,\mu) \mapsto (u \mu)^{-1}
  \end{equation*}
  has image exactly equal to $X_{\bbT_r,\bbB_r \cap {}^w \bbM_r}^{{}^w \bbM_r}$ and fibers isomorphic to $\bbU_r \cap {}^w \bbM_r$, an affine space. Moreover, the action of $(t,m) \in \bbT_r^\sigma \times {}^w\bbM_r^\sigma$ on $(u,\mu)$ is $(tut^{-1}, t \mu \dot w m^{-1} \dot w^{-1})$, which under the above multiplication map gets sent to $(u \mu)^{-1} \mapsto \dot w m \dot w^{-1} (u \mu)^{-1} t^{-1}$, which is exactly the $(\bbT_r^\sigma \times {}^w \bbM_r^\sigma)$-action on $X_{\bbT_r, \bbB_r \cap {}^w \bbM_r}^{{}^w \bbM_r}$.

  
  \subsection{Proof of Theorem \ref{thm:generic mackey}}\label{subsec:generic mackey proof}

  Let $\rho$ be a $(\bfM,\bfG)$-generic representation of $\bbM_r^\sigma$. The desired result follows directly from Propositions \ref{prop:Sigma'} and \ref{prop:Sigma''}; we spell it out in detail here. For any $t \in \bbT_r^\sigma$, we have
  \begin{align*}
    {}^* {}&{}R_{\bbT_r,\bbB_r}^{\bbG_r} \circ R_{\bbM_r,\bbP_r}^{\bbG_r}(\rho)(t) \\
    &= \frac{1}{|\bbM_r^\sigma|} \sum_{m \in \bbM_r^\sigma} \rho(m)^{-1} \Tr((t,m); H_c^*(\Sigma, \overline \QQ_\ell)) \\
    &= \sum_{w \in \bbT_r \backslash \cS(\bbT_r, \bbM_r)^\sigma / \bbM_r} \frac{1}{|\bbM_r^\sigma|} \sum_{m \in \bbM_r^\sigma} \rho(m)^{-1} \Tr((t,m); H_c^*(\widehat \Sigma_w, \overline \QQ_\ell)) \\
    &= \sum_{w \in \bbT_r \backslash \cS(\bbT_r, \bbM_r)^\sigma / \bbM_r} \frac{1}{|\bbM_r^\sigma|} \sum_{m \in \bbM_r^\sigma} \rho(m)^{-1} \Tr((t,m); H_c^*(\widehat \Sigma_w'', \overline \QQ_\ell)) \\
    &= \sum_{w \in \bbT_r \backslash \cS(\bbT_r, \bbM_r)^\sigma / \bbM_r} \frac{1}{|\bbM_r^\sigma|} \sum_{m \in \bbM_r^\sigma} \rho(m)^{-1} \Tr((\dot w m \dot w^{-1},t); H_c^*(X_{\bbT_r, \bbB_r \cap {}^w \bbM_r}^{{}^w \bbM_r}, \overline \QQ_\ell)) \\
    &= \sum_{w \in \bbT_r \backslash \cS(\bbT_r, \bbM_r)^\sigma / \bbM_r} {}^* R_{\bbT_r, \bbB_r \cap {}^w \bbM_r}^{{}^w \bbM_r}(\ad(w)^{-1}(\rho))(t).
  \end{align*}
  where the first equality follows from the isomorphism $X_{\bbT_r,\bbB_r}^{\bbG_r} \times_{\bbG_r^\sigma} X_{\bbM_r,\bbP_r}^{\bbG_r} \cong \Sigma$, the second equality follows from Lemma \ref{lem:Sigma hat}, the third equality follows from Proposition \ref{prop:Sigma'}, the fourth equality follows from Proposition \ref{prop:Sigma''}, and the last equality holds by definition.

  
\section{Parahoric Deligne--Lusztig varieties for elliptic tori}\label{sec:depth compatibility}

It is natural to ask how the parahoric Deligne--Lusztig functors $R_{\bbT_r,\bbB_r}^{\bbG_r}$ are compatible as $r$ varies. From the surjectivity of the Lang map, it follows that the morphism
\begin{equation*}
  \tilde \pi \from X_{\bbT_r,\bbB_r}^{\bbG_r} \to X_{\bbT_{r-1},\bbB_{r-1}}^{\bbG_{r-1}}
\end{equation*}
is surjective. The technical effort of this section is in proving the following theorem:

\begin{theorem}\label{thm:fiber cohomology}
  Let $N = \#\Phi(\bfG,\bfT)/2$. 
  For any point $x \in X_{\bbT_{r-1},\bbB_{r-1}}^{\bbG_{r-1}}$,
  \begin{equation*}
    H_c^i(\tilde \pi^{-1}(x), \overline \QQ_\ell)^{\bbT_{r:r+}^\sigma} = \begin{cases}
      \overline \QQ_\ell^{\oplus \#\bbU_{r:r+}^\sigma} & \text{if $i = 2N$,} \\
      0 & \text{otherwise.}
    \end{cases}
  \end{equation*} 
  Moreover, $\sigma^n$ acts on $H_c^{2d}(\tilde \pi^{-1}(x), \overline \QQ_\ell)^{\bbT_{r:r+}^\sigma}$ by multiplication by $q^{nN}$.
\end{theorem}

We prove this theorem in Section \ref{subsec:fiber cohomology}. The techniques we employ also work to calculate the cohomology of the fibers of the depth-lower projections of parahoric Lusztig varieties $X_{\bbM_r,\bbP_r}^{\bbG_r}$. The answer is the same as in Theorem \ref{thm:fiber cohomology}, with $(M,P,N)$'s replacing $(T,B,U)$'s (preserving font).

Theorem \ref{thm:fiber cohomology} has some important immediate corollaries. If $\bfT$ is elliptic over $F$, then $\bbU_{r:r+}^\sigma = \{1\}$, and so we obtain the following result as a corollary of Theorem \ref{thm:fiber cohomology}.

\begin{theorem}\label{thm:level lower}
  If $\bfT$ is elliptic over $F$, then we have $(\bbG_r^\sigma \times \bbT_r^\sigma)$-equivariant isomorphisms 
  \begin{equation}\label{eq:level lower}
    H_c^i(X_{\bbT_r,\bbB_r}^{\bbG_r}, \overline \QQ_\ell)^{\bbT_{r:r+}^\sigma} \cong H_c^{i+2N}(X_{\bbT_{r-1},\bbB_{r-1}}^{\bbG_{r-1}}, \overline \QQ_\ell(N)) \qquad \text{for all $i \geq 0$}.
  \end{equation}
\end{theorem}

In the above, $(N)$ denotes the Tate twist. By Theorem \ref{thm:level lower}, we immediately obtain the following corollaries.

\begin{corollary}\label{cor:depth compatibility}
  Assume $\bfT \subset \bfG$ is elliptic over $F$ and fix $s < r$. For any character $\theta \from \bbT_s^\sigma \to \overline \QQ_\ell^\times$, we have an isomorphism of virtual $\bbG_r^\sigma$-representations
  \begin{equation*}
    R_{\bbT_r,\bbB_r}^{\bbG_r}(\theta) \cong R_{\bbT_s,\bbB_s}^{\bbG_s}(\theta).
  \end{equation*}
\end{corollary}

Following \cite{Lus79}, define $H_i(S, \overline \QQ_\ell) \colonequals H_c^{2\dim(S)-i}(S, \overline \QQ_\ell(\dim(S)))$ for any smooth $\overline \FF_q$-scheme of pure dimension. 

\begin{corollary}\label{cor:infinite level}
  Assume $\bfT \subset \bfG$ is elliptic over $F$. We have a natural embedding
  \begin{equation*}
    H_i(X_{\bbT_{r-1},\bbB_{r-1}}^{\bbG_{r-1}}, \overline \QQ_\ell) \hookrightarrow H_c^i(X_{\bbT_r,\bbB_r}^{\bbG_r}, \overline \QQ_\ell).
  \end{equation*}
  For $X_{\bbT_\infty,\bbB_\infty}^{\bbG_\infty} \colonequals \varprojlim_r X_{\bbT_r,\bbB_r}^{\bbG_r}$, setting
  \begin{equation*}
    H_i(X_{\bbT_\infty,\bbB_\infty}^{\bbG_\infty}, \overline \QQ_\ell) \colonequals \varinjlim_r H_i(X_{\bbT_r,\bbB_r}^{\bbG_r}, \overline \QQ_\ell)
  \end{equation*}
  therefore defines $\ell$-adic homology groups for the infinite-dimensional $\overline \FF_q$-scheme $X_{\bbT_\infty,\bbB_\infty}^{\bbG_\infty}$. Moreover, on the category of smooth representations of $\bbT_\infty^\sigma$, it makes sense to define a functor $R_{\bbT_\infty,\bbB_\infty}^{\bbG_\infty}$ analogously to Definition \ref{def:induction}, and for any character $\theta$ of $\bbT_r^\sigma$, we have an equality of $\bbG_\infty^\sigma$-representations
  \begin{equation*}
    R_{\bbT_\infty,\bbB_\infty}^{\bbG_\infty}(\theta) = R_{\bbT_r,\bbB_r}^{\bbG_r}(\theta)
  \end{equation*}
\end{corollary}

\begin{remark}\label{rem:relation to lusztig}
  In 1979, Lusztig conjectured \cite{Lus79} that there should exist reasonable $p$-adic Deligne--Lusztig spaces. Lusztig studied this in \textit{op.\ cit. }for $\bfG$ the norm-1 elements of division algebras, and this was later formalized and generalized by Boyarchenko \cite{Boy12} to $\bfG$ a division algebra. For other inner forms of $\GL_n$, this was studied by the author and Ivanov \cite{CI21-MA,CI_loopGLn}. In these settings, representation-theoretic calculations proceed by establishing:
  \begin{enumerate} 
    \item The $p$-adic Deligne--Lusztig space is a disjoint union of infinite-dimensional parahoric Deligne--Lusztig varieties $X_\infty$.
    \item $\ell$-adic homology groups of $X_\infty$ can be defined as a direct limit of $\ell$-adic homology groups of finite-depth parahoric Deligne--Lusztig varieties $X_r$.
  \end{enumerate}
  For $\GL_n$, elliptic unramified maximal tori are automatically Coxeter, but this is no longer the case for general connected reductive groups $\bfG$; on the other hand, progress on (1) has only been made in the Coxeter setting. For $\bfG = \GSp$ and $\bfT$ Coxeter, Takamatsu established (1) in \cite{T23}. For $\bfG$ unramified of classical type and $\bfT$ Coxeter, Ivanov proved (1) in \cite{I23arc,I23orbit}. In all these settings, the parahoric schemes $X_\infty$ are examples of $X_{\bbT_\infty,\bbB_\infty}^{\bbG_\infty}$, hence Corollary \ref{cor:infinite level} resolves (2) and endows the above studied infinite-dimensional $p$-adic Deligne--Lusztig spaces with $\ell$-adic homology groups. This therefore generalizes the definition of homology in \cite{Lus79,Boy12,CI21-MA,CI_loopGLn} to Ivanov's setting in \cite{I23orbit} and relates results on the cohomology of finite-depth parahoric Deligne--Lusztig varieties---for example of the author and Oi \cite{CO21}---to the setting of Lusztig's 1979 conjecture.
\end{remark}

\begin{remark}\label{rem:indirect depth compat}
  We offer an indirect alternate argument to the discussion in Remark \ref{rem:relation to lusztig}. Another way to endow the $p$-adic Deligne--Lusztig spaces in Ivanov's decomposition result \cite{I23orbit} (for $\bfG$ of classical type and Coxeter $\bfT$) with $\ell$-adic homology groups is to use Dudas--Ivanov \cite{DI20} (scalar product formula for Coxeter $\bfT$ and $q>5$) in tandem with the results of the author with Oi \cite{CO21} (arbitrary $\bfT$, $q \gg 0$, mild regularity condition on $\theta$) which characterizes the irreducible representations $R_{\bbT_r}^{\bbG_r}(\theta)$. (As mentioned in the introduction, in forthcoming work, the author and Oi will remove the mild regularity condition in \cite{CO21}.) \textit{A postiori}, we then obtain $R_{\bbT_r}^{\bbG_r}(\theta) \cong R_{\bbT_{r+1}}^{\bbG_{r+1}}(\theta)$ when $q \gg 0$.
\end{remark}

\subsection{The cohomology of the fibers of $\tilde \pi$}\label{subsec:fiber cohomology}

The purpose of this section is to prove Theorem \ref{thm:fiber cohomology}. The simplest reason for this theorem to hold would be if $\tilde \pi^{-1}(x)/\bbT_{r:r+}^\sigma \cong \bbA^d$ (as usual, up to perfection). This is the case when $\bfG$ is a division algebra \cite[Lemma 4.7]{Boy12} and when $\bfG$ is more generally any inner form of $\GL_n$ \cite[Proposition 7.6]{CI21-MA}. While this is true that at least for some $x \in X_{\bbT_{r-1},\bbB_{r-1}}^{\bbG_{r-1}}$ (for example if the image of $x$ in $\bbG_0$ is $\FF_q$-rational), despite our best efforts over several years, we were not able to prove this isomorphism for arbitrary $x$. In the following, we focus instead on the statement of Theorem \ref{thm:fiber cohomology}, which requires only a calculation about the cohomology of $\tilde \pi^{-1}(x)$, not its explicit geometry.

For notational convenience, let us prove the theorem for $X_{\bbT_r,\sigma^{-1}(\bbB_r)}^{\bbG_r}$. Choose any $\tilde x \in X_{\bbT_r,\sigma^{-1}(\bbB_r)}^{\bbG_r}$ over $x \in X_{\bbT_{r-1},\sigma^{-1}(\bbB_{r-1})}^{\bbG_{r-1}}$. Denote by $\bar x$ the image of $x$ in $\bbG_0$. By definition, for any $x_r \in\bbG_{r:r+}$, we have that $\tilde x x_r \in X_{\bbT_r,\sigma^{-1}(\bbB_r)}^{\bbG_r}$ if and only if $(\tilde x x_r)^{-1} \sigma(\tilde x x_r) \in \bbU_r$. We have
\begin{align*}
  (\tilde x x_r)^{-1} \sigma(\tilde x x_r) \in \bbU_r 
  &\Longleftrightarrow x_r^{-1} \tilde x^{-1} \sigma(\tilde x) \sigma(x_r) \sigma(\tilde x)^{-1} \tilde x \in \bbU_{r:r+} \\
  &\Longleftrightarrow (\tilde x x_r \tilde x^{-1})^{-1} \sigma(\tilde x x_r \tilde x^{-1}) \in \tilde x \bbU_{r:r+} \tilde x^{-1}.
\end{align*}
Note that $\tilde x \bbU_{r:r+} \tilde x^{-1}$ only depends on $x$ (in fact, only on $\bar x$); hence we write $x \bbU_{r:r+} x^{-1}$ for this subgroup. We have shown that we have an isomorphism
\begin{equation*}
  \tilde \pi^{-1}(x) \cong \{x_r \in\bbG_{r:r+} : x_r^{-1} \sigma(x_r) \in x \bbU_{r:r+} x^{-1}\}.
\end{equation*}
The Lang map
\begin{equation*}
  \bbG_{r:r+} \to\bbG_{r:r+}, \qquad g_r \mapsto g_r^{-1} \sigma(g_r)
\end{equation*}
restricts to a morphism
\begin{equation*}
  \tilde \varphi \from \tilde \pi^{-1}(x) \to x \bbU_{r:r+} x^{-1},
\end{equation*}
and so we see that
\begin{equation*}
  \tilde \varphi_! \overline \QQ_\ell = \bigoplus_{\chi \in (G_{r:r+}^\sigma)^\wedge} i^* \cL_\chi
\end{equation*}
where we write $i \from x \bbU_{r:r+} x^{-1} \hookrightarrow\bbG_{r:r+}$. The morphism $\tilde \pi$ factors through the quotient $\tilde \pi^{-1}(x) \to \tilde \pi^{-1}(x)/(x\bbT_{r:r+} x^{-1})^\sigma \equalscolon \pi^{-1}(x)$; write
\begin{equation*}
  \varphi \from \pi^{-1}(x) \to x \bbU_{r:r+} x^{-1}
\end{equation*}
for the map induced by $\tilde \varphi$. Then 
\begin{equation*}
  \varphi_! \overline \QQ_\ell = \bigoplus_{\substack{\chi \in (G_{r:r+}^\sigma)^\wedge\, \text{s.t.} \\ \chi|_{(h\bbT_{r:r+}h^{-1})^\sigma} = \triv}} i^* \cL_\chi.
\end{equation*}
By construction, each sheaf $i^* \cL_\chi$ is a multiplicative local system on the connected algebraic group $x \bbU_{r:r+} x^{-1}$ and therefore we see that
\begin{equation*}
  H_c^i(x \bbU_{r:r+} x^{-1}, i^* \cL_\chi) = \begin{cases}
    \overline \QQ_\ell & \text{if $i = 2N$ and $i^* \cL_\chi \cong \overline \QQ_\ell$,} \\
    0 & \text{otherwise.}
  \end{cases}
\end{equation*}
From this, we may conclude
\begin{equation}\label{eq:fiber cohomology}
  H_c^i(\pi^{-1}(x), \overline \QQ_\ell) = \bigoplus_\chi H_c^i(x \bbU_{r:r+} x^{-1}, i^* \cL_\chi) = \begin{cases}
    \#A & \text{if $i = 2N$,} \\
    0 & \text{otherwise,}
  \end{cases}
\end{equation}
where
\begin{equation*}
  A = \{\chi \in (G_{r:r+}^\sigma)^\wedge : \text{$\chi|_{(x\bbT_{r:r+} x^{-1})^\sigma} = \triv$, and $i^* \cL_\chi \cong \overline \QQ_\ell$}\}.
\end{equation*}

Let $m$ be such that $\sigma^m(x) = x$ and $\sigma^m(\bbU_{r:r+}) = \bbU_{r:r+}$. Then the multiplicative local system $i^* \cL_\chi$ is $\sigma^m$-equivariant and under the sheaves-to-functions correspondence, $i^* \cL_\chi$ corresponds to the character
  \begin{equation*}
    t_{i^* \cL_\chi} \from (x \bbU_{r:r+} x^{-1})^{\sigma^m} \to \overline \QQ_\ell^\times, \qquad x y_r x^{-1} \mapsto \chi(\Nm(xy_rx^{-1})),
  \end{equation*}
  where
  \begin{equation*}
    \Nm \from\bbG_{r:r+}^{\sigma^m} \to\bbG_{r:r+}^{\sigma}, \qquad g_r \mapsto g_r \sigma(g_r) \cdots \sigma^{m-1}(g_r).
  \end{equation*}
  Since $i^* \cL_\chi \cong \overline \QQ_\ell$ if and only if $t_{i^* \cL_\chi} = \triv$, we see that
  \begin{equation*}
    |A| = [G_{r:r+}^\sigma : (x\bbT_{r:r+}x^{-1})^\sigma \cdot \Nm(x \bbU_{r:r+}^{\sigma^m} x^{-1})].
  \end{equation*}
  Although we will see that $|A|$ does not depend on $x$, we will see that the subgroups $(x\bbT_{r:r+}x^{-1})^\sigma$ and $\Nm(x \bbU_{r:r+}^{\sigma^m} x^{-1})$ very much do depend on $x$. Fix an ordering on $\Phi^+ \colonequals \Phi^+(\bfG,\bfT)$ and write
  \begin{equation*}
    \bar x^{-1} \sigma(\bar x) = u = \prod_{\alpha \in \Phi^+} u_\alpha \in \bbU_{0:0+}.
  \end{equation*}
  Furthermore, since both $\sigma \from\bbG_{r:r+}^{\sigma^m} \to\bbG_{r:r+}^{\sigma^m}$ and $\ad(x)(g_r) = x g_r x^{-1}$ are $\FF_q$-linear automorphisms on the $\FF_q$-vector space $\bbG_{r:r+}^{\sigma^m}$, calculating the size of $(x\bbT_{r:r+} x^{-1})^\sigma$ and $\Nm(x \bbU_{r:r+}^{\sigma^m} x^{-1})$ is equivalent to calculating their dimensions as $\FF_q$-vector spaces.

\begin{proposition}\label{prop:|A| elliptic}
  If $\bfT \subset \bfG$ is elliptic over $F$, then $|A| = 1$.
\end{proposition}

\begin{proof}
  To prove that $|A| = 1$, we will show that $(x \bbT_{r:r+} x^{-1})^\sigma$ and $\Nm(x \bbU_{r:r+}^{\sigma^m} x^{-1})$ complementary $\FF_q$-vector spaces of $\bbG_{r:r+}^\sigma$.
  \begin{claim}\label{claim:xTx}
    $\dim_{\FF_q} (x \bbT_{r:r+} x)^\sigma = \dim_{\FF_q} \bbT_{r:r+}^\sigma - \dim_\bbR \Span_\bbR\{\alpha \in \Phi^+ : \text{$u_{\sigma^i(\alpha)} \neq 1$ for some $i$}\}.$
  \end{claim}

  \begin{proof}[Proof of Claim \ref{claim:xTx}]
    Let $t \in \bbT_{r:r+}$ be such that $\sigma(xtx^{-1}) = xtx^{-1}$. This is equivalent to $t = u \sigma(t) u^{-1} \in \sigma(t) U_r$, which implies $t = \sigma(t)$. Therefore we see that
    \begin{equation}\label{eq:xTx}
      (x\bbT_{r:r+} x^{-1})^\sigma = x\{t \in \bbT_{r:r+}^\sigma : u = t^{-1} u t\}x^{-1}.
    \end{equation}
    Since $t^{-1} u_\alpha t = \alpha(t) u_\alpha$, we see that the right-hand side has the desired dimension.
  \end{proof}

  \begin{claim}\label{claim:xUx} \mbox{}
    \begin{enumerate}
      \item $\Nm(x \bbU_{r:r+}^{\sigma^m} x^{-1}) \cap (x \bbT_{r:r+} x^{-1})^\sigma = \{1\}$.
      \item $\dim_{\FF_q} \Nm(x\bbU_{r:r+}^{\sigma^m} x^{-1}) \geq \dim_{\FF_q} \Nm(\bbU_{r:r+}^{\sigma^m}) + \dim_\bbR \Span_\bbR\{\alpha \in \Phi^+ : \text{$u_{\sigma^i(\alpha)} \neq 1$ for some $i$}\}.$
    \end{enumerate}
  \end{claim}

  \begin{proof}[Proof of Claim \ref{claim:xUx}]
    First observe that for any $y_r \in \bbU_{r:r+}^{\sigma^m}$,
    \begin{equation*}
      \Nm(xy_r x^{-1}) = x \Nm(y_r u) x^{-1} = x (y_r u) \cdot \sigma(y_r u) \cdots \sigma^{m-1}(y_r u) x^{-1}.
    \end{equation*}
    It is sufficient to show that the projection of $\Nm(xy_rx^{-1})$ to $x\bbT_{r:r+}x^{-1}$ has zero intersection with $(x \bbT_{r:r+} x^{-1})^\sigma$. Computing $\Nm(y_r u) \in x^{-1} \bbG_{r:r+} x = \bbG_{r:r+}$ in terms of $u = \prod_{\alpha \in \Phi^+} u_\alpha$ shows us that contribution to $\pr_{x \bbT_{r:r+} x^{-1}}$ comes from moving $u_\alpha$ across $u_\beta$:
    \begin{align*}
      \pr_{x \bbT_{r:r+} x^{-1}}(\Nm(xy_rx^{-1})) 
      &= \pr_{\bbT_{r:r+}}(\Nm(y_r u)) \\
      &\subset \left(\begin{gathered}\text{subgroup of $\bbT_{r:r+}$ generated by $\bbT_{r:r+}^\beta$} \\ \text{for all $\beta \in \Span_{\bbR}\{\alpha \in \Phi^+ : \text{$u_{\sigma^i(\alpha)} \neq 1$ for some $i$}\}$}\end{gathered}\right).
    \end{align*}
    By \eqref{eq:xTx}, we know that the subgroup on the right-hand side has trivial intersection with $(x \bbT_{r:r+} x^{-1})^\sigma$, and this shows (1).

    It is clear that (2) follows from the assertion that 
    \begin{equation}\label{eq:beta inequality}
      \dim_{\bbF_q} \Nm(\bbU_{r:r+}^{\sigma^m} u u_\beta) \geq \dim_{\bbF_q} \Nm(\bbU_{r:r+}^{\sigma^m} u) + 1,
    \end{equation}
    where $u_\beta \in \bbU_{\beta,0:0+}^{\sigma^m} \smallsetminus \{1\}$ and $\beta \in \Phi^+$ does not lie in $\Span_{\bbR}\{\alpha \in \Phi^+ : \text{$u_{\sigma^i(\alpha)} \neq 1$ for some $i$}\}$. For any $y_r \in \bbU_{r:r+}^{\sigma^m}$, we have
    \begin{align*}
      (y_r u u_\beta) \cdot \sigma(y_r u u_\beta) \cdots \sigma^{m-1}(y_r u u_\beta) = \Nm(y_r u) \cdot X_m(y_r),
    \end{align*}
    where we set $X_1(y_r) = u_\beta$ and
    \begin{equation*}
      X_{i+1}(y_r) = \sigma^i(y_r)^{-1} \cdot X_i \cdot \sigma^i(y_r) \cdot \sigma^i(u_\beta).
    \end{equation*}
    Observe that $X_i \in \bbG_{r:r+} \cdot u_\beta \sigma(u_\beta) \cdots \sigma^{i-1}(u_\beta)$. Then it makes sense to consider the projection of $X_i$ to $\bbT_{r:r+} \cdot u_\beta \sigma(u_\beta) \cdots \sigma^{i-1}(u_\beta)$. Writing
    \begin{equation*}
      T_i(y_r) = u_\beta \sigma(u_\beta) \cdots \sigma^{i-2}(u_\beta) \cdot \sigma^{i-1}(y_r) \cdot \sigma^{i-2}(u_\beta)^{-1} \cdots \sigma(u_\beta)^{-1} u_\beta^{-1}, 
    \end{equation*}
    we have that
    \begin{equation*}
      \pr_{\bbT_{r:r+} \cdot u_\beta \sigma(u_\beta) \cdots \sigma^{m-1}(u_\beta)}(X_m(y_r)) = T_1 \cdot T_2 \cdots T_m.
    \end{equation*}

    Since $\bfT \subset \bfG$ is elliptic over $F$ by definition, there exists a $\delta \in \Phi^+$ such that $\sigma^i(\delta) = -\beta$. Let $l$ be the smallest positive integer for which $\sigma^l(\beta) = \beta$; we may assume $1 < i < l$. Then the map $y_r \mapsto \pr_{\bbT_{r:r+} \cdot u_\beta \sigma(u_\beta) \cdots \sigma^{m-1}(u_\beta)}(X_m(y_r))$ determines a map
    \begin{equation*}
      \varphi \from \FF_{q^m} \to \FF_{q^m}, \qquad y_\delta(z) \mapsto \pr_{\bbT_{r:r+}^\delta \cdot u_\beta \sigma(u_\beta) \cdots \sigma^{m-1}(u_\beta)}(X_m(y_\delta(z))).
    \end{equation*}
    By construction, we may write
    \begin{equation*}
      \varphi(z) = \sum_{i=1}^{m-1} a_i z^{q^i}.
    \end{equation*}
    Let us show that $a_l \neq 0$: the $z^{q^l}$ contribution to $\varphi(z)$ comes from 
    \begin{equation*}
      T_{l+1}(y_\delta(z)) = u_\beta \sigma(u_\beta) \cdots \sigma^{l-1}(u_\beta) \cdot \sigma^{l}(y_\delta(z)) \cdot \sigma^{l-1}(u_\beta)^{-1} \cdots \sigma(u_\beta)^{-1} u_\beta^{-1},
    \end{equation*}
    and this necessarily has a nontrivial $\bbT_{r:r+}^\delta$ component since by construction $\sigma^l(\delta) = -\sigma^{l-i}(\beta)$. If $\varphi(z^{q^l}) = \varphi(z)$ for all $z \in \FF_{q^m}$, this would force $a_l = 0$ since $\varphi$ has at most degree $q^{m-1}$. Therefore we may conclude that 
    \begin{equation*}
      \varphi(L_{q^l}(z)) \not\equiv 0.
    \end{equation*}
    On the other hand, note that 
    \begin{equation*}
      \Nm(y_\delta(z) u) = 1 \quad \Longleftrightarrow \quad \Nm(xy_\delta(z) x^{-1}) = 1 \quad \Longleftrightarrow \quad \Tr_{\FF_{q^m}/\FF_{q^l}}(z) = 0.
    \end{equation*}
    This shows that 
    \begin{equation}\label{eq:ker neq}
      \ker(z \mapsto \Nm(y_\delta(z) u)) \not\subseteq \ker(z \mapsto \pr_{\bbT_{r:r+} \cdot u \beta \sigma(u_\beta) \cdots \sigma^{m-1}(u_\beta)}).
    \end{equation}
    Since $\pr_{\bbT_{r:r+}^\delta}(\Nm(y_\delta(z) u)) = \{1\}$ by the argument in (1), we see that \eqref{eq:ker neq} proves \eqref{eq:beta inequality}.
  \end{proof}

  Claims \ref{claim:xTx} and \ref{claim:xUx} imply that
  \begin{align*}
    \dim_{\FF_q} ((x\bbT_{r:r+} x^{-1})^\sigma \cdot \Nm(x\bbU_{r:r+}^{\sigma^m} x^{-1})) 
    &= \dim_{\FF_q}(x\bbT_{r:r+} x^{-1})^\sigma + \dim_{\FF_q} \Nm(x\bbU_{r:r+}^{\sigma^m} x^{-1}) \\
    &\geq \dim_{\FF_q} \bbT_{r:r+}^\sigma + \dim_{\FF_q} \Nm(\bbU_{r:r+}^{\sigma^m}).
  \end{align*}
  If $\bfT \subset \bfG$ is elliptic over $F$, then for every $\alpha \in \Phi(\bfG,\bfT)$, there exists an $i$ such that $\sigma^i(\alpha) \in \Phi^+$. From this, it follows that $\Nm(\bbU_{r:r+}^{\sigma^m}) = (\bbU_{r:r+} \bbU_{r:r+}^-)^\sigma$; indeed, for any $\alpha \in \Phi(\bfG,\bfT)$, 
  \begin{equation*}
    \pr_\alpha(\Nm(y_r)) = \sum_{\text{$\beta$ s.t.\ $\sigma^i(\beta) = \alpha$}} \Tr_{\FF_{q^m}/\FF_{q^{l_\alpha}}}(y_\beta),
  \end{equation*} 
  where $l_\alpha$ is the smallest positive integer for which $\sigma^{l_\alpha}(\alpha) = \alpha$. Therefore the previous inequality is
  \begin{equation*}
    \dim_{\FF_q}((x\bbT_{r:r+} x)^\sigma \cdot \Nm(x \bbU_{r:r+}^{q^m} x^{-1})) \geq \dim_{\FF_q}\bbG_{r:r+}^\sigma.
  \end{equation*}
  But of course the opposite inequality is automatic by construction, and so $|A| = 1$.
\end{proof}


Proposition \ref{prop:|A| elliptic} with \eqref{eq:fiber cohomology} implies Theorem \ref{thm:fiber cohomology} in the case that $\bfT \subset \bfG$ is elliptic over $F$, which then implies Theorem \ref{thm:level lower}. We will use this to prove Theorem \ref{thm:fiber cohomology} in the general case. If $\bfT \subset \bfG$ is not elliptic over $F$, then by Lemma \ref{lem:pInd factor}, for any $\theta$ of depth $<r$, we have 
\begin{align*}
  R_{\bbT_r,\bbB_r}^{\bbG_r}(\theta) 
  &= \Ind_{\bbQ_r^\sigma}^{\bbG_r^\sigma}(\Inf_{\bbL_r^\sigma}^{\bbQ_r^\sigma}(R_{\bbT_r,\bbL_r \cap \bbB_r}^{\bbL_r}(\theta))) \\
  \Inf_{\bbG_s^\sigma}^{\bbG_r^\sigma} R_{\bbT_s,\bbB_s}^{\bbG_s}(\theta) 
  &= \Ind_{\bbQ_s^\sigma}^{\bbG_s^\sigma}(\Inf_{\bbL_s^\sigma}^{\bbQ_s^\sigma}(R_{\bbT_s,\bbL_s \cap \bbB_s}^{\bbL_s}(\theta))) \\
  &= \Ind_{\bbQ_r^\sigma \bbG_{s+:r+}^\sigma}^{\bbG_r^\sigma}(\Inf_{\bbL_s^\sigma}^{\bbQ_r^\sigma \bbG_{s+:r+}^\sigma}(R_{\bbT_s,\bbL_s \cap \bbB_s}^{\bbL_s}(\theta))) \\
  &= \Ind_{\bbQ_r^\sigma \bbG_{s+:r+}^\sigma}^{\bbG_r^\sigma}(\Inf_{\bbL_r^\sigma}^{\bbQ_r^\sigma \bbG_{s+:r+}^\sigma}(R_{\bbT_r,\bbL_r \cap \bbB_r}^{\bbL_r}(\theta))),
\end{align*}
where $s = r-1$ and the last equality holds by Theorem \ref{thm:level lower}. This shows that
\begin{equation*}
  \dim R_{\bbT_r,\bbB_r}^{\bbG_r}(\theta) = |U_{r:r+}^\sigma| \cdot \dim R_{\bbT_s,\bbT_s}^{\bbG_s}(\theta).
\end{equation*} 
On the other hand, \eqref{eq:fiber cohomology} implies that
\begin{equation}
  \dim H_c^i(X_r, \overline \QQ_\ell)^{\bbT_{r:r+}^\sigma} = |A| \cdot \dim H_c^{i-2N}(X_{r-1}, \overline \QQ_\ell), \qquad \text{for all $i \geq 0$}.
\end{equation}
Therefore $|A| = |U_{r:r+}^\sigma|$, which now completes the proof of Theorem \ref{thm:fiber cohomology}.

\section{The scalar product formula for parahoric Deligne--Lusztig induction}\label{sec:scalar}

\begin{definition}
  Let $\theta \from \bbT_r^\sigma \to \overline \QQ_\ell^\times$ be any character. Then the restriction $\theta|_{\bbT_{r:r+}^\sigma}$ agrees with the restriction of a $(\bfM,\bfG)$-generic character for some Levi subgroup $\bfM$ of $\bfG$. 
  We say that  $(\theta,\bbT_r)$ is \textit{split-generic} if $\bfT$ is elliptic over $F$ as a torus of $\bfM$. \end{definition}

We prove the main theorem of the paper:

\begin{theorem}\label{thm:scalar product formula}
  Let $(\theta,\bbT_r,\bbB_r)$ be split-generic. For any $(\theta',\bbT_r',\bbB_r')$,
  \begin{equation*}
    \langle R_{\bbT_r,\bbB_r}^{\bbG_r}(\theta), R_{\bbT_r',\bbB_r'}^{\bbG_r}(\theta') \rangle_{\bbG_r^\sigma} = \sum_{w \in W_{\bbG_r}(\bbT_r,\bbT_r')^\sigma} \langle \theta, \ad(w^{-1})^* \theta' \rangle_{\bbT_r^\sigma}.
  \end{equation*}
  In particular, $R_{\bbT_r,\bbB_r}^{\bbG_r}(\theta)$ is independent of the choice of $\bfB$.
\end{theorem}

Let us state an immediate corollary in the setting that $\bfT \subset \bfG$ is elliptic over $F$, the notation here being as in Corollary \ref{cor:depth compatibility}.

\begin{corollary}
  Let $\bfT \subset \bfG$ be elliptic over $F$. 
  \begin{enumerate}
    \item The functor $R_{\bbT_\infty}^{\bbG_\infty} \colonequals R_{\bbT_\infty,\bbB_\infty}^{\bbG_\infty}$ is independent of the choice of $\bfB$.
    \item $R_{\bbT_\infty}^{\bbG_\infty}(\theta)$ is irreducible if and only if $\Stab_{W_{\bbG_\infty}(\bbT_\infty)^\sigma}(\theta) = \{1\}.$
  \end{enumerate}
\end{corollary}


\subsection{Proof of the scalar product formula}

We first note the following proposition, which comes as an easy corollary of several results we have established in this paper.

\begin{proposition}\label{prop:Howe}
  Let $(\theta,\bbT_r,\bbB_r)$ be split-generic and let $\vec \phi$ be any Howe factorization and choose any accompanying sequence $\vec \bfP$ of parabolic subgroups. Then
  \begin{equation*}
    R_{\bbT_r,\bbB_r}^{\bbG_r}(\theta) = r_{\bbT_r}^{\bbG_r}(\vec \phi; \vec \bfP).
  \end{equation*}
\end{proposition}

\begin{proof}
  Since $(\theta,\bbT_r,\bbB_r)$ is split-generic by assumption, we may apply Theorem \ref{thm:level lower} at each intermediate step. Hence we have
  \begin{align*}
    R_{\bbG_{r_1}^{1}, \bbP_{r_1}^{1}}^{\bbG_{r_1}^{2}}(R_{\bbT_{r_0}', \bbB_{r_0}'}^{\bbG_{r_0}^{1}}(\theta_0') \otimes \theta_1') 
    &= R_{\bbG_{r_1}^{1}, \bbP_{r_1}^{1}}^{\bbG_{r_1}^{2}}(R_{\bbT_{r_1}', \bbB_{r_1}'}^{\bbG_{r_1}^{1}}(\theta_0') \otimes \theta_1')\\
    &= R_{\bbG_{r_1}^{1}, \bbP_{r_1}^{1}}^{\bbG_{r_1}^{2}}(R_{\bbT_{r_1}', \bbB_{r_1}'}^{\bbG_{r_1}^{1}}(\theta_0' \otimes \theta_1')) \\
    &= R_{\bbT_{r_1}', \bbB_{r_1}'}^{\bbG_{r_1}^{2}}(\theta_0' \otimes \theta_1')
  \end{align*}
  where the first equality holds by Theorem \ref{thm:level lower}, the second equality holds by Proposition \ref{prop:twisting}, and the third equality holds by Proposition \ref{prop:transitivity}. Continuing this, we see the desired equality.
\end{proof}

With Proposition \ref{prop:Howe} in mind, Theorem \ref{thm:scalar product formula} follows from calculating the inner product $\langle R_{\bbT_r,\bbB_r}^{\bbG_r}(\theta), r_{\bbT_r'}^{\bbG_r}(\vec \phi'; \vec \bfP') \rangle,$ which we do in Proposition \ref{prop:half scalar product} below. The final assertion of Theorem \ref{thm:scalar product formula} about independence of the choice of $\bfB$ follows from the scalar product formula using the same trick as in \cite[Corollary 2.4]{Lus04}: the inner product of $R_{\bbT_r,\bbB_r}^{\bbG_r}(\theta) - R_{\bbT_r,\bbB_r'}^{\bbG_r}(\theta)$ with itself is equal to zero.

\begin{proposition}\label{prop:half scalar product}
  Let $\vec \phi'$ be any Howe factorization of $\theta'$ and choose any accompanying sequence $\vec \bfP$ of parabolic subgroups. Then 
  \begin{equation*}
    \langle R_{\bbT_r,\bbB_r}^{\bbG_r}(\theta), r_{\bbT_r,\bbB_r}^{\bbG_r}(\vec \phi'; \vec \bfP') \rangle_{\bbG_r^\sigma} = \sum_{w \in W_{\bbG_r}(\bbT_r,\bbT_r')^\sigma} \langle \theta, \ad(w) \theta' \rangle_{\bbT_r^\sigma}.
  \end{equation*}
\end{proposition}

\begin{proof}
  We induct on the length $d'$ of $\vec \phi'$. The base case is $d' = 0$. We have
  \begin{align*}
    \langle R_{\bbT_r,\bbB_r}^{\bbG_r}(\theta), r_{\bbT_r'}^{\bbG_r}(\vec \phi'; \bfP') \rangle_{\bbG_r^\sigma} 
    &= \langle R_{\bbT_r,\bbB_r}^{\bbG_r}(\theta), \Inf_{\bbG_{r_0'}^\sigma}^{\bbG_r^\sigma}(R_{\bbT_{r_0'}',\bbB_{r_0'}'}^{\bbG_{r_0}}(\phi_{-1}')) \otimes \phi_0' \rangle_{\bbG_r^\sigma} \\
    &= \langle R_{\bbT_r,\bbB_r}^{\bbG_r}(\theta) \otimes \phi_0'{}^{-1}, \Inf_{\bbG_{r_0'}^\sigma}^{\bbG_r^\sigma}(R_{\bbT_{r_0'}',\bbB_{r_0'}'}^{\bbG_{r_0}}(\phi_{-1}')) \rangle_{\bbG_r^\sigma} \\
    &= \langle R_{\bbT_r,\bbB_r}^{\bbG_r}(\theta \otimes \phi_0'{}^{-1}), \Inf_{\bbG_{r_0'}^\sigma}^{\bbG_r^\sigma}(R_{\bbT_{r_0'}',\bbB_{r_0'}'}^{\bbG_{r_0}}(\phi_{-1}')) \rangle_{\bbG_r^\sigma} \\
    &= \langle R_{\bbT_{r_0},\bbB_{r_0}}^{\bbG_{r_0}}(\theta \otimes \phi_0'{}^{-1}), R_{\bbT_{r_0'}',\bbB_{r_0'}'}^{\bbG_{r_0}}(\phi_{-1}') \rangle_{\bbG_{r_0}^\sigma}.
  \end{align*}
  By construction, $\phi_{-1}$ is $(\bfT,\bfG)$-generic of depth $r_0$, so we may apply the generic Mackey formula (Corollary \ref{cor:generic mackey}, which in this special case is the same as \cite[Theorem 1.1]{CI21-RT}) to obtain
  \begin{align*}
    \langle R_{\bbT_r,\bbB_r}(\theta), r_{\bbT_r'}^{\bbG_r}(\vec \phi'; \vec \bfP') \rangle_{\bbG_r^\sigma} 
    &= \sum_{w \in W_{\bbG_{r_0}}(\bbT_{r_0},\bbT_{r_0}')^\sigma} \langle \theta \otimes \phi_0'{}^{-1}, \ad(w^{-1})^* \phi_{-1}' \rangle_{\bbT_{r_0}^\sigma} \\
    &= \sum_{w \in W_{\bbG_{r}}(\bbT_{r},\bbT_{r}')^\sigma} \langle \theta, \ad(w^{-1})^* \theta' \rangle_{\bbT_{r}^\sigma}
  \end{align*}
  where in the last equality note that since $\phi_0'$ is a character of $\bbG_r^\sigma$, it is obviously invariant under pullback by $\ad(w)$.

  Now assume that the proposition holds for any $\theta'$ with Howe factorization length $d'$; we must show that the proposition holds for $\vec \phi'$ of length $d'+1$. We have
  \begin{align*}
    \langle R_{\bbT_r,\bbB_r}^{\bbG_r}(\theta), r_{\bbT_r,\bbB_r}^{\bbG_r}(\vec \phi'; \vec \bfP') \rangle_{\bbG_r^\sigma} 
    &= \langle R_{\bbT_r,\bbB_r}^{\bbG_r}(\theta), \Inf_{\bbG_{s_{d'}}^\sigma}^{\bbG_{r}^\sigma}(R_{\bbG_{s_{d'}}^{d'},\bbP_{s_{d'}}^{d'}}^{\bbG_{s_{d'}}}(r_{\bbT_{s_{d'}}'}^{\bbG_{s_{d'}}^{\prime(d')}}(\vec \phi_{\leq d'}'; \vec \bfP_{\leq d'}'))) \otimes \phi_{d'+1}' \rangle_{\bbG_r^\sigma} \\
    &= \langle R_{\bbT_r,\bbB_r}^{\bbG_r}(\theta) \otimes \phi_{d'+1}^{\prime-1}, \Inf_{\bbG_{s_{d'}}^\sigma}^{\bbG_{r}^\sigma}(R_{\bbG_{s_{d'}}^{d'},\bbP_{s_{d'}}^{d'}}^{\bbG_{s_{d'}}}(r_{\bbT_{s_{d'}}'}^{\bbG_{s_{d'}}^{d'}}(\vec \phi_{\leq d'}'; \vec \bfP_{\leq d'}')))\rangle_{\bbG_r^\sigma} \\
    &= \langle R_{\bbT_r,\bbB_r}^{\bbG_r}(\theta \otimes \phi_{d'+1}^{\prime-1}), \Inf_{\bbG_{s_{d'}}^\sigma}^{\bbG_{r}^\sigma}(R_{\bbG_{s_{d'}}^{d'},\bbP_{s_{d'}}^{\prime (d')}}^{\bbG_{s_{d'}}}(r_{\bbT_{s_{d'}}}^{\bbG_{s_{d'}}^{d'}}(\vec \phi_{\leq d'}'; \vec \bfP_{\leq d'}')))\rangle_{\bbG_r^\sigma} \\
    &= \langle R_{\bbT_{s_{d'}},\bbB_{s_{d'}}}^{\bbG_{s_{d'}}}(\theta \otimes \phi_{d'+1}^{\prime-1}), R_{\bbG_{s_{d'}}^{d'},\bbP_{s_{d'}}^{\prime(d')}}^{\bbG_{s_{d'}}}(r_{\bbT_{s_{d'}}'}^{\bbG_{s_{d'}}^{d'}}(\vec \phi_{\leq d'}'; \vec \bfP_{\leq d'}'))\rangle_{\bbG_{s_{d'}}^\sigma},
  \end{align*}
  where the third equality holds by the twisting lemma (Proposition \ref{prop:twisting}) and the fourth equality holds by in invariants lemma (Lemma \ref{lem:invariants}). Applying the generic Mackey formula (Corollary \ref{cor:generic mackey}) now gives
  \begin{align*}
    &\langle R_{\bbT_r,\bbB_r}^{\bbG_r}(\theta), r_{\bbT_r'}^{\bbG_r}(\vec \phi'; \vec \bfP') \rangle_{\bbG_r^\sigma} \\
    &= \sum_{w \in \bbT_{s_{d'}} \backslash \cS(\bbT_{s_{d'}},\bbG_{s_{d'}}^{d'})^\sigma/\bbG_{s_{d'}}^{d'}} \langle R_{\bbT_{s_{d'}}, \bbB_{s_{d'}} \cap {}^w \bbG_{s_{d'}}^{d'}}^{\bbG_{s_{d'}}^{d'}}(\theta \otimes \phi_{d'+1}^{\prime-1}), \ad(w^{-1})^* r_{\bbT_{s_{d'}}'}^{\bbG_{s_{d'}}^{d'}}(\vec \phi_{\leq d'}'; \vec \bfP_{\leq d'}') \rangle_{{}^w (\bbG_{s_{d'}}^{d'})^\sigma}.
  \end{align*}
  By the inductive hypothesis, each summand on the right-hand side is equal to
  \begin{align*}
    \sum_{v \in W_{{}^w \bbG_{s_{d'}}^{\prime (d')}}(\bbT_{s_{d'}},\bbT_{s_{d'}}')^\sigma} {}&{} \langle \theta \otimes \phi_{d'+1}^{\prime-1}, \ad(v^{-1})^* \ad(w^{-1})^* \theta' \otimes \phi_{d'+1}^{\prime-1} \rangle_{\bbT_{s_{d'}}^{\prime\sigma}} \\
    &= \sum_{v \in W_{{}^w \bbG_r^{\prime(d')}}(\bbT_r,\bbT_r')^\sigma} \langle \theta, \ad(v^{-1})^* \ad(w^{-1})^* \theta' \rangle_{\bbT_r^{\prime\sigma}}.
  \end{align*}
  The desired formula in the proposition now follows.
\end{proof}

\section{Variations: the Drinfeld stratification}\label{sec:drinfeld}

The methods in this paper can be mildly modified to yield results on the cohomology of the Drinfeld stratification of $X_{\bbT_r,\bbB_r}^{\bbG_r}$.

\begin{definition}[Drinfeld stratification]\label{def:drinfeld}
  The \textit{Drinfeld stratum} of $X_{\bbT_r,\bbB_r}^{\bbG_r}$ associated to a Levi subgroup $\bfL$ of $\bfG$ which contains $\bfT$, is the disjoint union
  \begin{equation*}
    \bigsqcup_{\gamma \in \bbG_r^\sigma/(\bbL_r \bbG_{0+:r+})^\sigma} \gamma \cdot X_{\bbT_,\bbB_r}^{\bbL_r\bbG_r^+}, \qquad \text{where $X_{\bbT_r,\bbB_r}^{\bbL_r\bbG_r^+} \colonequals \{x \in \bbL_r \bbG_{0+:r+} : x^{-1}\sigma(x) \in \sigma(\bbU_r)\}$}.
  \end{equation*}
  It is stable under the natural $(\bbT_r^\sigma \times \bbG_r^\sigma)$-action on $X_{\bbT_r,\bbB_r}^{\bbG_r}$. Denote by 
  \begin{equation*}
    R_{\bbT_r,\bbB_r}^{\bbL_r\bbG_r^+} \from \cR(\bbT_r^\sigma) \to \cR((\bbL_r\bbG_{0+:r+})^\sigma)
  \end{equation*}
  the functor corresponding to $X_{\bbT_r,\bbB_r}^{\bbL_r\bbG_r^+}$ in analogy with Definition \ref{def:induction}.
\end{definition}

\begin{theorem}\label{thm:drinfeld}
  Let $(\theta,\bbT_r,\bbB_r)$ be split-generic. For any triple $(\theta',\bbT_r',\bbB_r')$,
  \begin{equation*}
    \langle R_{\bbT_r,\bbB_r}^{\bbL_r\bbG_r^+}(\theta), R_{\bbT_r',\bbB_r'}^{\bbL_r\bbG_r^+}(\theta') \rangle_{(\bbL_r\bbG_{0+:r+})^\sigma} = \sum_{w \in W_{\bbL_r\bbG_r^+}(\bbT_r,\bbT_r')^\sigma} \langle \theta, \ad(w)^* \theta' \rangle_{\bbT_r^\sigma}.
  \end{equation*}
\end{theorem}

\begin{proof}
  The properties of parahoric Deligne--Lusztig induction presented in Section \ref{sec:parahoric lusztig} have direct analogues for $R_{\bbT_r,\bbB_r}^{\bbL_r\bbG_r^+}$ and the proofs go through with only notational changes. The same is the case for Section \ref{sec:generic mackey} and especially Theorem \ref{thm:generic mackey}, the generic Mackey formula. Here, note that the generalized Bruhat decomposition in Section \ref{subsec:generic mackey} should be intersected with $\bbL_r \bbG_r^+ \subset \bbG_r$. The crux then is to see that the fiber calculations in Section \ref{sec:depth compatibility}. But this is again straightforward---since the Drinfeld stratification on $X_{\bbT_r,\bbB_r}^{\bbG_r}$ is defined by pullback from a stratification on $X_{\bbT_0,\bbB_0}^{\bbG_0}$, the fiber cohomology calculations required to establish Theorem \ref{thm:level lower} for a Drinfeld stratum is a special case of Theorem \ref{thm:fiber cohomology}. (A particular case to keep in mind is the closed Drinfeld stratum $X_{\bbT_r,\bbB_r}^{\bbT_r\bbG_r^+}$. This stratum lies over the locus $\bbG_0^\sigma \subset X_{\bbT_0,\bbB_0}^{\bbG_0}$ which corresponds to $u = 1$ in the notation of Section \ref{subsec:fiber cohomology}. The proof of Proposition \ref{prop:|A| elliptic} proves that the fibers of $X_{\bbT_r,\bbB_r}^{\bbT_r\bbG_r^+} \to X_{\bbT_{r-1},\bbB_{r-1}}^{\bbT_{r-1}\bbG_{r-1}^+}$ in fact all have the \textit{same} cohomology, even before taking $\bbT_{r:r+}^\sigma$-fixed points. Moreover, this phenomenon does not happen for any other stratum.)
\end{proof}


\bibliographystyle{my_amsalpha}
\bibliography{References}







\end{document}


\subsection{}

It now only remains to show that the split-genericity assumption on $(\theta,\bbT_r,\bbB_r)$ is sharp. 

\begin{proposition}
  If $(\theta,\bbT_r,\bbB_r)$ is not split-generic, then \eqref{eq:scalar product} fails for $(\theta',\bbT_r',\bbB_r') = (\theta,\bbT_r,\bbB_r)$.
\end{proposition}

\begin{proof}
  Let $\bfP$ denote the parabolic subgroup containing $\bfB$ whose unipotent radical is given by $\bfU \cap \sigma(\bfU) \cap \cdots \cap \sigma^{n-1}(\bfU)$; write $\bfM$ for its Levi component. Fix a Howe factorization $\vec \phi$ of $\theta$. If $\phi_d \neq \triv$, then
  \begin{equation*}
    R_{\bbT_r,\bbB_r}^{\bbG_r}(\theta) = R_{\bbT_r,\bbB_r}^{\bbG_r}(\theta \phi_d^{-1}|_{\bbT_r^\sigma}) \otimes \phi_d.
  \end{equation*}
  Then by Lemma \ref{lem:pInd factor}, we have
  \begin{equation*}
    R_{\bbT_r,\bbB_r}^{\bbG_r}(\theta) = \Ind_{\bbP_r^\sigma}^{\bbG_r^\sigma}(\Inf_{\bbM_r^\sigma}^{\bbP_r^\sigma}(R_{\bbT_r, \bbB_r \cap \bbM_r}^{\bbM_r}( \theta \phi_d^{-1}|_{\bbT_r^\sigma}))) \otimes \phi_d
  \end{equation*}
  and by Theorem \ref{thm:level lower}, the $\bbM_r^\sigma$-representation $R_{\bbT_r,\bbB_r \cap \bbM_r}^{\bbM_r}(\theta \phi_d^{-1}|_{\bbT_r^\sigma})$ has depth $s < r$. Therefore we see that
  \begin{equation*}
    R_{\bbT_r,\bbB_r}^{\bbG_r}(\theta) = \rho \oplus (\Ind_{\bbP_r^\sigma \bbG_{s+:r+}^\sigma}^{\bbG_r^\sigma}(R_{\bbT_s,\bbB_s}^{\bbG_s}(\theta \phi_d^{-1}|_{\bbT_r^\sigma})) \otimes \phi_d)
  \end{equation*}
  for some nonzero virtual representation of $\bbG_r^\sigma$ with no nontrivial homomorphisms to $R_{\bbT_s,\bbB_s}^{\bbG_s}(\theta \phi_d^{-1}|_{\bbT_r^\sigma}) \otimes \phi_d$. Since
  \begin{equation*}
    \Inf_{\bbG_s^\sigma}^{\bbG_r^\sigma} R_{\bbT_s,\bbB_s}^{\bbG_s}(\theta \phi_d^{-1}|_{\bbT_r^\sigma}) = \Ind_{\bbP_r^\sigma \bbG_{s+:r+}^\sigma}^{\bbG_r^\sigma}(R_{\bbT_s,\bbB_s}^{\bbG_s}(\theta \phi_d^{-1}|_{\bbT_r^\sigma})),
  \end{equation*}
  we see that
  \begin{align*}
    \langle R_{\bbT_r,\bbB_r}^{\bbG_r}(\theta), R_{\bbT_r,\bbB_r}^{\bbG_r}(\theta) \rangle &= \langle \rho \oplus (R_{\bbT_s,\bbB_s}^{\bbG_s}(\theta \phi_d^{-1}|_{\bbT_r^\sigma}) \otimes \phi_d), \rho \oplus (R_{\bbT_s,\bbB_s}^{\bbG_s}(\theta \phi_d^{-1}|_{\bbT_r^\sigma}) \otimes \phi_d) \rangle \\
    &> \langle R_{\bbT_s,\bbB_s}^{\bbG_s}(\theta \phi_d^{-1}|_{\bbT_r^\sigma}), R_{\bbT_s,\bbB_s}^{\bbG_s}(\theta \phi_d^{-1}|_{\bbT_r^\sigma}) \rangle.
  \end{align*}
  
  
  By construction, we have
  \begin{equation*}
    R_{\bbT_r,\bbB_r}^{\bbG_r}(\theta) = R_{\bbG_r^{d-1},\bbP_r^{d-1}}^{\bbG_r}(R_{\bbT_r,\bbB_r \cap \bbG_r^{d-1}}^{\bbG_r^{d-1}}(\theta \phi_d)
  \end{equation*}
\end{proof}


\newpage

\begin{proof} 
  \textcolor{red}{This proof is so crazy... need to rewrite completely!!}
  By the existence of Howe factorizations, there exists a twisted Levi subgroup $\bfM$ and a $(\bfM,\bfG)$-generic character $\phi$ of $\bbM_r^\sigma$ such that $\theta \otimes \phi^{-1}|_{\bbT_r^\sigma}$ has depth $s < r$. By Proposition \ref{prop:transitivity}, for any parabolic $\bfP$ of $\bfG$ with Levi component $\bfM$, we have
\begin{equation*}
  R_{\bbT_r,\bbB_r}^{\bbG_r}(\theta) = R_{\bbM_r,\bbP_r}^{\bbG_r}(R_{\bbT_r,\bbB_r \cap \bbM_r}^{\bbM_r}(\theta)) = R_{\bbM_r,\bbP_r}^{\bbG_r}(R_{\bbT_r, \bbB_r \cap \bbM_r}^{\bbM_r}(\theta \otimes \phi^{-1}) \otimes \phi),
\end{equation*}
where the last equality holds by Proposition \ref{prop:twisting}.

\begin{lemma}\label{lem:pInd generic}
  Let $\rho$ be a representation of $\bbP_r^\sigma$ of depth $s$ and $\phi$ be an $(\bfM,\bfG)$-generic character of $\bbM_r^\sigma$. Then $\Ind_{\bbP_r^\sigma}^{\bbM_r^\sigma}(\rho) \otimes \phi$ is $(\bfM,\bfG)$-generic.
\end{lemma}


\begin{proof}[Proof of Lemma \ref{lem:pInd generic}]
  Let $\psi \from \bbM_r^\sigma \to \overline \QQ_\ell^\times$ be arbitrary of depth $r$. Then by Frobenius reciprocity
  \begin{equation*}
    \langle \psi, \Ind_{\bbP_r^\sigma}^{\bbM_r^\sigma}(\rho) \otimes \phi \rangle = \langle \psi|_{\bbP_r^\sigma}, \rho \otimes \phi|_{\bbP_r^\sigma} \rangle,
  \end{equation*}
  and this is automatically zero if $\psi$ is not $(\bfM,\bfG)$-generic.
\end{proof}

Because of Lemma \ref{lem:pInd generic}, we may apply the generic Mackey formula (Corollary \ref{cor:generic mackey}):
\begin{align*}
  \langle R_{\bbT_r,\bbB_r}^{\bbG_r}(\theta), R_{\bbM_r}^{\bbG_r}(R_{\bbT_r,\bbB_r}^{\bbM_r}(\theta \otimes \phi^{-1}) \otimes \phi) \rangle 
  &= \langle R_{\bbT_r,\bbB_r}^{\bbM_r}(\theta), R_{\bbT_r,\bbB_r}^{\bbM_r}(\theta \otimes \phi^{-1}) \otimes \phi \rangle \\
  &= \langle R_{\bbT_r,\bbB_r}^{\bbM_r}(\theta \otimes \phi^{-1}), R_{\bbT_r,\bbB_r}^{\bbM_r}(\theta \otimes \phi^{-1}) \rangle.
\end{align*}
Since by assumption $\theta$ is not split-generic, this means that we have
\begin{equation*}
  R_{\bbT_r,\bbB_r}^{\bbM_r}(\theta \otimes \phi^{-1}) = \pInd_{\bbL_r^\sigma \cap \bbM_r^\sigma}^{\bbM_r^\sigma}(R_{\bbT_r,\bbB_r}^{\bbL_r \cap \bbM_r}(\theta \otimes \phi^{-1})),
\end{equation*}
where $R_{\bbT_r,\bbB_r}^{\bbL_r \cap \bbM_r}(\theta \otimes \phi^{-1})$ has depth $s$ by Theorem \ref{thm:level lower}. Furthermore, by Theorem \ref{thm:level lower} (\textcolor{red}{need slightly stronger statement here}), we have
\begin{equation*}
  \pInd_{\bbL_r^\sigma \cap \bbM_r^\sigma}^{\bbM_r^\sigma}(R_{\bbT_r,\bbB_r}^{\bbL_r \cap \bbM_r}(\theta \otimes \phi^{-1})) = \Inf_{\bbM_s^\sigma}^{\bbM_r^\sigma}(\pInd_{\bbL_s^\sigma \cap \bbM_s^\sigma}^{\bbM_s^\sigma}(R_{\bbT_s}^{\bbL_s \cap \bbM_s}(\theta \otimes \phi^{-1}))) + \rho'
\end{equation*}
for a nonzero $\rho'$. Hence by Proposition \ref{prop:half scalar product} applied to $R_{\bbT_r,\bbB_r}^{\bbL_r \cap \bbM_r}$, we have
\begin{align*}
  \langle R_{\bbT_r,\bbB_r}^{\bbM_r}(\theta \otimes \phi^{-1}), R_{\bbT_r,\bbB_r}^{\bbM_r}(\theta \otimes \phi^{-1}) \rangle_{\bbM_r^\sigma}
  &= C + \sum_{w \in W_{\bbM_r}(\bbT_r)^\sigma} \langle \theta \otimes \phi^{-1}, \ad(w)^*(\theta \otimes \phi^{-1}) \rangle_{\bbT_r^\sigma},
\end{align*}
where $C = \langle \rho', \rho' \rangle > 0$.
\end{proof}




\section{}

Consider the space $\overline \QQ_\ell[\bbG_r(\FF_q)]^{\bbG_r(\FF_q)}$ of class functions on $\bbG_r(\FF_q)$. For a Levi subgroup $\bfL \subset \bfG$ containing $\bfT$, define
\begin{equation*}
  V_L \colonequals \Span_{\overline \QQ_\ell}\{\Theta_{R_{T_r}^{G_r}(\theta)} : \text{$\theta$ is $(\bfL,\bfG)$-generic}\}
\end{equation*}

\begin{remark}
  If we wanted to do this for the larger space
  \begin{equation*}
    \Span_{\overline \QQ_\ell}\{\Theta_{R_{L_r}^{G_r}(\rho \otimes \theta)} : \text{$\theta \in \widehat L_r(\FF_q)$ is $(\bfL,\bfG)$-generic and $\rho$ has depth $<r$}\},
  \end{equation*} 
  then we'd need to know that this space is equal to the span of all trace-of-Frobenius functions arising from objects in $D_{G_r}^\psi(G_r)$ for $\psi$ varying over all $(\bfL,\bfG)$-generic elements.
\end{remark}

\begin{theorem}
  \langle {}^*R_{L_r \subset P_r}^{G_r}(R_{T_r \subset B_r}^{G_r}(\theta)), R_{T_r \subset B_r}^{G_r}(\theta) \rangle
\end{theorem}

\newpage






