\begin{document}
\affiliation{$$_affiliation_$$}
\title{Real plane separating $(M-2)$-curves of degree $\MakeLowercase{d} and totally real pencils of degree $\MakeLowercase{d-3}$}
\begin{abstract}
It is well known that a non-singular real plane projective curve of degree five with five connected components is separating if and only if its ovals are in non-convex position. In this article, this property is set into a different context and generalised to all real plane separating $(M-2)$-curves. 


   \end{abstract}
\maketitle
\tableofcontents
\section{Introduction}
Let $C$ be any smooth complex compact algebraic curve equipped with an anti-holomorphic involution $\sigma:C \rightarrow C$, i.e. a smooth real algebraic compact curve. If the real points $C(\mathbb R)$ of $C$ separates its complex points $C (\mathbb C)$, i.e. $C (\mathbb C) \setminus C(\mathbb R)$ is disconnected, we say that $C$ is of \textit{type I} or \textit{separating}. By Harnack-Klein's inequality \cite{Harn76, Klei73}, the number $l$ of connected components of $C(\mathbb R)$ is bounded by the genus $g$ of $C$ plus one . For any fixed $0 \leq i \leq g+1$, if $l$ equals $g+1-i$, we say that $C$ is an $(M-i)$-curve. The number $l$ is related to the property of separateness of the curve. For example, if $C$ is separating, then $l$ has the parity of $g+1$. Or, if $C$ is an $M$-curve, then $C$ is separating. In this article, we work with non-singular real algebraic plane projective separating $(M-2)$-curves.

First of all, let us present some general features of separating curves. If $C$ is of type I, the two  halves of $C (\mathbb C)\setminus C(\mathbb R )$ induce two opposite orientations on $C(\mathbb R )$ called \textit{complex orientations}; \cite{Rokh74}. Looking at complex orientations of separating real curves embedded in some ambient surface has lead to remarkable progress in the study of their topology and a refinement of their classifications. One of the first results relating topology, complex orientations and properties of separating plane curves, is Rokhlin's complex orientations formula \cite{Rokh74}, \cite{Mish75}, and one of the more recent is \cite[Theorem 1.1]{Orev21}, where Orevkov shows that there are finer relations for the numbers which intervene in the complex orientations formula. An important role in \cite{Orev21} is played by separating morphisms.

\begin{defn}
\label{defn: separating}
We say that a real morphism $f$ from a smooth real algebraic compact curve $C$ to the complex projective line $\mathbb P^1_{\mathbb C} $ is \textit{separating} if $f^{-1}(\mathbb P^1 (\mathbb R ) )= C (\mathbb R )$. 
\end{defn}
According to Ahlfors \cite[\S4.2]{Ahlf50}, there exists a separating morphism $f: C \rightarrow \mathbb C \mathbb P^1$ if and only if $C$ is of type I. We call \textit{separating gonality} of $C$, and we denote it with $\text{sepgon}(C)$, the minimal possible value for the degree of a separating morphism of $C$. Observe that the separating gonality has always, as lower bound, the number of real connected components of $C(\mathbb R)$.

Actually, there is a more general definition of separating morphisms including real morphisms between any real algebraic varieties of same dimension. We direct the interested reader to \cite{KumSha20} and \cite{KumLeTMan22}. In the current paper, we need Definition \ref{defn: separating}, only.

The study of smooth real curves of type I and their separating morphisms has been carried out mainly from two points of view: on the one hand from that of abstract curves; on the other hand from that of curves embedded in some ambient surface; e.g. \cite{Huis01}, \cite{Gaba06},\cite{CopHui13}, \cite{Copp13}, \cite{Copp14}, \cite{KumShaw20}, \cite{Orev21}. 
Any real $M$-curve of genus $g$ admits a separating morphism of degree $g+1$, because of Riemann-Roch theorem. In general, for some fixed integers $i,k$ such that $1 \leq i \leq g+1$ and $k \geq g+1-i$, given a real separating $(M-i)$-curve $C$,  it is not evident, a priori, how to construct a separating morphism $f: C \rightarrow \mathbb P^1_{\mathbb C}$ of degree $k$. 

On the other hand, \cite[Theorem 7.1]{Gaba06} states that a genus $g$ real separating curve with $l$ real connected components admits a separating morphism of degree at most $\frac{g+ l+1}{2}$. 

From now on, unless otherwise stated, we call \textbf{real plane (separating) curve} any real algebraic plane projective (separating) curve.


The real locus of a non-singular real plane curve is homeomorphic to a disjoint union of circles embedded in $\mathbb P^2 (\mathbb{R})$. One can embed a circle in $\mathbb P^2(\mathbb{R})$ in two different ways: as an \textit{oval}, i.e. realising the trivial-class in $H_1( \mathbb P^2 (\mathbb{R}); \mathbb{Z}/2\mathbb{Z})$, otherwise as a \textit{pseudo-line}. A non-singular real plane curve of even degree has a non-negative number ovals only; otherwise exactly one pseudo-line and ovals (possibly none).

$\mathbb{P}^{2}(\mathbb{R})$ is separated by an oval in two disjoint non-homeomorphic connected components: a disk, called \textit{interior} of the oval; a M\"obius band, called \textit{exterior} of the oval. 



In the study of the topology of real plane curves, the following it is well known.

\begin{lem}
	\label{lem: quintic}
A non-singular real plane curve $C_5$ of degree $5$ with five connected components is separating if and only if its ovals are in non-convex position (see Definition \ref{defn: non-convex}).
\end{lem}
 \begin{figure}[h!]
\begin{picture}(100,105)
\put(-25,-3){\includegraphics[width=0.25\textwidth]{non_convex.png}}
\end{picture}
\caption{Arrangement of a triplet $(\mathbb P^2 (\mathbb R), C(\mathbb R), S_1 \cup S_2 \cup S_3)$ as in Definition \ref{defn: non-convex}, where the $S_i$'s are the three segments.}
\label{fig: non_convex}
\end{figure} 
\begin{defn}
	\label{defn: non-convex}
	Let $C$ be a non-singular real plane curve of degree $5$ with five connected components. We say that its ovals are in non-convex position if three of the ovals of $C(\mathbb R)$ are such that, chosen a point in the interior of each of them, once one traces three segments containing pair-wisely the points and such that every segment does not cross the pseudo-line of $C(\mathbb R)$, the fourth oval is contained inside the triangle cut out by these segments; see Fig. \ref{fig: quintic}.
\end{defn}
A possible construction of a real plane separating quintic with five connected components can be found in \cite[pag. 36, Fig. 19]{Viro07}.

Since in the proof of \Cref{lem: quintic} resides the germ of the main theorem of this article, \Cref{thm: unico}, we report it briefly. To do so, we first recall the following definitions.

\begin{defn}
 	\label{defn: totally_real_pencil}
 	Let $C$ be a non-singular real algebraic plane projective curve of type I. We say that $C$ admits a totally real pencil of curves of degree $k$ if there exists an integer $k$ such that there are $f,g \in \mathbb R [x,y,z]_{k}$ and $V(\lambda f + \mu g) \cap C$ consists of real points only for all $\lambda, \mu \in \mathbb R$ not both zero.
 \end{defn}
 
 
\begin{defn}
	\label{defn: pos_neg_oval}
An oval $\mathcal O$ of a non-singular real plane curve of odd degree is called positive if $[\mathcal O]= -2[J]$ in $H^1(N)$, where $J$ is the pseudo-line of the curve and $N$ the closure of the non-orientable component of $\mathbb P^2(\mathbb R) \setminus \mathcal O$. Otherwise an oval is called negative. 
\end{defn}

\begin{proof}[Proof of \Cref{lem: quintic}]
 Assume that $C_5$ is separating and, for the sake of contradiction, suppose that the ovals are in convex position. Then, applying \cite[Theorem 1]{Fied83}, one can show that two ovals must be negative and two positive, which is in contradiction with the complex orientations formula \cite{Mish75}. Now, assume that the ovals are in non-convex position. In order to prove that $C_5$ is separating it is enough to prove that $C_5$ admits a totally real pencil; see \Cref{defn: totally_real_pencil}. Let us consider any real pencil of conics with base locus a point for each oval of $C_5(\mathbb R)$. 
 First, remark that every real conic in the real projective plane is convex (because of B\'ezout Theorem). Then, the non-convexity respectively of the ovals of $C_5(\mathbb R)$ and the fact that every conic is convex, forces the pencil to be totally real for the curve $C_5$; indeed, each real conic of the pencil is obliged to intersect all real connected components of $C_5$.
\end{proof} 

A priori, Lemma \ref{lem: quintic} is uniquely an observation concerning separating plane quintics with four ovals. It is not clear that one may expect to have some generalisation of it to other real plane separating curves. On the other hand, a remarkable fact, from the proof of \Cref{lem: quintic}, is that \textit{any non-singular real plane projective separating curve of degree $5$ with five connected components admits a totally real pencil of conics.} Surprisingly, this property can be generalised to all real plane separating $(M-2)$-curves; see \Cref{thm: unico}. Indeed, the generalisation of \Cref{lem: quintic} to separating plane $(M-2)$-curves $C_d$ of degree $d$ is that every curve $C_d$ admits totally real pencils of curves of degree $(d-3)$.
\begin{thm}
	\label{thm: unico}
	Let $C$ be a non-singular real plane $(M-2)$-curve of degree $d \geq 4$ and of type I. Let $g=\frac{(d-1)(d-2)}{2}$ denote the genus of $C$. Then, the curve $C$ admits infinitely many totally real pencils of degree $d-3$ with $g-1$ or $g-2$ base points.
	\end{thm}

We prove \Cref{thm: unico} in Section \ref{sec: proofs} splitting it in two statements, Propositions \ref{prop: sepgon_m-2_g-1} and \ref{prop: sepgon_m-2_g}. 
	
 \begin{rem}
 	\label{rem: tot_real_pencil_no_base_points}
 	In \cite{Touz13}, there have been constructed totally real pencils of rational cubics for real plane separating $(M-2)$-sextics realising two particular topological arrangements in the real projective plane. On the other hand, the points of the base locus of such totally real pencils do not belong to the curves, therefore the obtained separating morphisms of such sextics have degree 18.
 
 	\end{rem}


\subsection*{Acknowledgements}
I would like to thank Erwan Brugall\'e. I thank also Athene Grant and the DFG, German Research Foundation (Deutsche Forschungsgemeinschaft), Project- ID 286237555, TRR 195, ``Symbolic Tools in Mathematics and their Application".	
	\section{Proofs and examples}
	\label{sec: proofs}
	Since the separating gonality of a smooth real $(M-2)$-curve of genus $g$ and of type I is either $g-1$ or $g$ \cite[Theorem 7.1]{Gaba06}, in order to prove Theorem \ref{thm: unico}, we split it in Proposition \ref{prop: sepgon_m-2_g-1} and \ref{prop: sepgon_m-2_g}.
	\begin{prop}
	\label{prop: sepgon_m-2_g-1}
	Let $C$ be a non-singular real plane $(M-2)$-curve of degree $d \geq 4$ and of type I. Assume that $\text{sepgon}(C)=g-1$, where $g=\frac{(d-1)(d-2)}{2}$ is the genus of $C$. Then, the curve $C$ admits infinitely many totally real pencils of degree $d-3$ with $g-1$ base points.
\end{prop}
To prove \Cref{prop: sepgon_m-2_g-1}, we apply \cite[Theorem 3.2]{Orev21}. Therefore we report here the statement restricted to the case of real plane curves.

\begin{thm}(\cite[Theorem 3.2]{Orev21})
\label{thm: stepan}
Let $C$ be a non-singular real plane separating curve. Let $D$ be a real divisor belonging to the linear system $|C+K_{\mathbb P^2_{\mathbb C}}|$. Assume that $D$ has not $C$ as a component. We may always write $D=2D_0 + D_1$ with $D_1$ a reduced curve and $D_0$ an effective divisor. Let us fix a complex orientation on $ C(\mathbb R)$ and an orientation on $\mathbb P^2(\mathbb R) \setminus ( C(\mathbb R) \cup D_1(\mathbb R))$ which changes each time we cross $ C(\mathbb R) \cup D_1(\mathbb R)$ at its smooth points. The latter orientation induces a boundary orientation on $C(\mathbb R) \setminus ( C(\mathbb R) \cap D_1)$. Let $f: C \rightarrow \mathbb C \mathbb P^1$ be a separating morphism. Then it is impossible that, for some $p \in  \mathbb P^1(\mathbb R)$, the set $f^{-1}(p) \setminus \supp(D)$ is non-empty and the two orientations coincide at each point of the set.
\end{thm}


\begin{proof}[Proof of \Cref{prop: sepgon_m-2_g-1}] 

	Since $\text{sepgon}(C)=g-1$, there exists a separating morphism $f: C \rightarrow \mathbb P^1$ of degree $g-1$. Therefore for any fixed $p \in \mathbb P^1(\mathbb R)$, every point $p_i$ in $f^{-1}(p)$ belongs to a distinct connected component $C_i$ of $C(\mathbb R)$, where $1 \leq i \leq g-1$. 
	
In the following, we show that any real pencil of curves of degree $(d-3)$ passing through $g-2$ points of $f^{-1}(p)$ contains indeed  the all $f^{-1}(p)$. In particular, this would imply that any such pencil must be totally real for the curve $C$. 

	
	Fixed $\frac{(d-1)(d-2)}{2}-1 = g-1$ real points, thanks to Riemann-Roch theorem, there always exists at least one real curve of degree $(d-3)$ passing through such configuration. Moreover, a configuration of $g-2$ points defines a pencil of curves of degree $(d-3)$. 
	
	
For any fixed $p \in \mathbb P^1(\mathbb R)$, pick any configuration $\mathcal P$ of $g-2$ distinct points $p_{j_1},\dots, p_{j_{g-2}}$ belonging to $f^{-1}(p)$. Applying the notation in Theorem \ref{thm: stepan}, take $D_1$ as the curve containing $\mathcal P$ and an additional real point $q$, different from $p_{j_{g-1}}$. Remark that $D_1$ must be reduced; otherwise $D_1$ can be written as a product of two real curves $A^2B$, where $A$ and $B$ have degree respectively $s$ and $d-3-2s$ with $1 \leq s \leq \lfloor \frac{d-3}{2} \rfloor$. But, this leads to a contradiction. In fact, because of the choice of the $g-2$ points, the curve $C$ must intersect $A \cup B$ in at least $2(g-2)$ points, which is not possible by B\'ezout theorem. 

Let us fix an orientation on $\mathbb P^2(\mathbb R) \setminus ( C(\mathbb R) \cup D_1(\mathbb R))$ which changes each time we cross $ C(\mathbb R) \cup D_1(\mathbb R)$ at its smooth points. The latter orientation induces a boundary orientation $\mathfrak O$ on $C(\mathbb R) \setminus ( C(\mathbb R) \cap D_1)$.

Suppose, for the sake of contradiction, that $p_{j_{g-1}}$ does not belong to $D_1$. Then, the set $f^{-1}(p) \setminus \supp(D_1)= \{p_{j_{g-1}}\}$ is non-empty and, up to fix one of the two complex orientations on $ C(\mathbb R)$, such complex orientation and the orientation $\mathfrak O$ coincide at $p_{j_{g-1}}$. This implies that the separating morphism $f: C \rightarrow \mathbb P^1$ cannot exist and it contradicts \cite[Theorem 7.1]{Gaba06}.

 Therefore, any real curve of degree $(d-3)$ passing through $\mathcal P$ is obliged to contain also the point $p_{j_{g-1}}$. In particular, the configuration $\mathcal P$ defines a totally real pencil of curves of degree $(d-3)$.

Moreover, such totally real pencil has exactly $g-1$ base points on $C$. Indeed $\mathcal B_p \cap C=f^{-1}(p)$, where $\mathcal B_p$ denotes the base locus of the pencil. The fact that $\mathcal B_p \cap C$ contains $f^{-1}(p)$ comes from the construction and the equality must hold as well; otherwise there would exist at least another real point $r$ (respectively two complex conjugated points $s,\overline s$) belonging to $(\mathcal B_p \cap C )\setminus f^{-1}(p)$, and such $r$ would belong to some connected component $C_h$ and the intersection of $C$ with a real curve $A$ of the pencil passing through another real point $q \in C_h \setminus \{r, p_h \}$ would be of at least $4+2(g-2)=2g=d(d-3)+2$ points (respectively the intersection of $C$ with a real curve $A$ of the pencil passing through another point $q$ would be of at least $2+2(g-1)=2g=d(d-3)+2$ points). A contradiction with the theorem of B\'ezout. 
\end{proof}

\begin{prop}
	\label{prop: sepgon_m-2_g}
	Let $C$ be a non-singular real plane $(M-2)$-curve of degree $d \geq 4$. Assume that $\text{sepgon}(C)=g$, where $g=g(C)=\frac{(d-1)(d-2)}{2}$ is the genus of $C$. Then, the curve $C$ admits infinitely many totally real pencils of degree $d-3$ with $g-2$ base points.
\end{prop}

To prove \Cref{prop: sepgon_m-2_g}, we need to recall the definition of $(M-2)$-curves of \textit{special type} and Theorem 4.4 stated in \cite{Copp14}.

\begin{defn}(\cite[Definition 2.2]{Copp14})
\label{defn: special_type}
	A real separating $(M-2)$-curve $C$ of genus $g$ is of special type if there exists a connected component $\tilde C$ of $C(\mathbb R)$ such that for each real morphism $f: C \rightarrow \mathbb P^1$ of degree $g$ having odd parity on each connected component $C_j \not= \tilde C$ of $C(\mathbb R)$, one has $f(\tilde C)= \mathbb P^1(\mathbb R)$.
\end{defn}

\begin{thm}(\cite[Theorem 4.4]{Copp14})
	\label{thm: coppens_m-2}
	Let $C$ be a real separating $(M-2)$-curve of genus $g$ not of special type, then $\text{sepgon}(C)=g-1$.
\end{thm}
\begin{proof}[Proof of \Cref{prop: sepgon_m-2_g}]
	Since $\text{sepgon}(C)=g$, there exists a separating morphism $f: C \rightarrow \mathbb P^1$ of degree $g$. Therefore for any fixed $p \in \mathbb P^1(\mathbb R)$, the points $p_1, \dots, p_{g-2}$ of $f^{-1}(p)$ belongs to distinct connected components $C_i$'s of $C(\mathbb R)$, where $1 \leq i \leq g-2$, and the remaining points $p_{g-1},p_g$ belongs to the same connected component $C_{g-1}$.
	
First of all remark that, thanks to \Cref{thm: stepan}, for any fixed $p \in \mathbb P^1 (\mathbb R)$, any real curve of degree $(d-3)$ passing through $g-1$ points $p_{j_1},\dots, p_{j_{g-1}}$ of $f^{-1}(p)$ is obliged to contain also $p_{j_{g}}$; see the proof of Proposition \ref{prop: sepgon_m-2_g-1}.

On the other hand, thanks to Theorem \ref{thm: coppens_m-2}, the fact that $\text{sepgon}(C)\not = g-1$, implies that $C$ is of special type (see Definition \ref{defn: special_type}). Now, there are two possibilities: either $\tilde C= C_i$ for some $i \not = g-1$ or  $\tilde C= C_{g-1}$. Since $f$ is a real morphism of degree $g$ having odd parity on $C_1, \dots, C_{g-2}$, it can only be that $\tilde C= C_{g-1}$. Let $s: C \rightarrow \mathbb P^1$ be the real morphism associated to the pencil of curves of degree $(d-3)$ passing through the points $p_1, \dots, p_{g-2}$. Since $C$ is of special type and the degree of $s_{|_{C_i}}$ is odd, for all $i\not = g-1$, it follows that $s(C_{g-1})= \mathbb P^1(\mathbb R)$ and, by construction, the morphism $s$ must be separating and the pencil totally real for the curve $C$. Moreover, the base locus of such pencil contains exactly $g-2$ points on $C$.

\end{proof}


Let us consider a non-singular real plane separating curve $C_5$ of degree $5$ with five connected components. As proved in \cite[Example 2.2]{Manz23}, applying \Cref{thm: stepan}, one can show that $C_5$ cannot have separating gonality equal to $5$. Therefore $\text{sepgon}(C_5)=6$. Moreover, applying \Cref{thm: stepan} once again we observe the following.

\begin{exa}
\label{exa: quintic}
	 Let us prove that all separating morphisms of degree $6$ of $C_5$ must have odd degree on the three negative ovals and degree two either on the pseudo-line or on the positive oval; see \Cref{defn: pos_neg_oval}.
	 \begin{figure}[h!]
\begin{picture}(100,109)
\put(-25,-10){\includegraphics[width=0.7\textwidth]{disegno6.png}}
\end{picture}
\caption{$(\mathbb P^2 (\mathbb R), C(\mathbb R), L(\mathbb R))$ of Example \ref{exa: quintic}. Double arrows denote $\mathfrak O$, simple arrows the fixed complex orientation of $C(\mathbb R)$ and $\bullet$ the points in $f^{-1}(p)$.}
\label{fig: quintic}
\end{figure} 
For the sake of contradiction, let us suppose that there exists a separating morphism $f: C \rightarrow \mathbb P^1_{\mathbb C}$ of degree $6$ such that $f$ has degree $2$ when restricted to a negative oval of$C(\mathbb R)$. Then, fix some $p \in \mathbb P^1 (\mathbb R)$ and apply \Cref{thm: stepan} taking as $D_0$ the line passing through	the two points of $f^{-1}(p)$ belonging to the positive oval and the pseudo-line. Up to a choice of the orientation $\mathfrak O$ (double arrows in Fig. \ref{fig: quintic}), we get a contradiction with Theorem \ref{thm: stepan} and therefore such $f$ cannot exist. On the other hand sepgon$(C)=6$. This means that all separating morphisms of $C$ must have odd degree on the three negative ovals.	 
\end{exa}
\begin{rem}
\label{rem: sharpness}
For real plane quintics as in Example \ref{exa: quintic}, one of the two bounds of \cite[Theorem 1.1]{Orev21} is sharp. More in general, any real separating plane $(M-2)$-curve of genus $g$, for which one of the two bounds in \cite[Theorem 1.1]{Orev21} is sharp, must have separating gonality equal to $g$ (in fact, one can apply the same argument of \Cref{exa: quintic}). 
\end{rem}

\begin{que}
\begin{itemize}
\item[]
	\item For any integer $k \geq 2$, is there a non-singular real plane separating $(M-2)$-curves of degree $4k+1$ for which the bound on the left of (2) in \cite[Theorem 1.1]{Orev21} is sharp?

Observe that, because of \cite[Remark 1.8]{Orev21}, for all $k \geq 1$, such bounds can never be sharp for real separating plane $(M-2)$-curves of degree $4k+3$. Moreover, for any real plane separating curve of odd degree the bound on the right of (2) in \cite[Theorem 1.1]{Orev21} can never be sharp.
	\end{itemize}
	\begin{itemize}
\item There exist two real plane $(M-2)$-curves of degree $d$ which have the same arrangement in the real projective plane, up to homeomorphism of $\mathbb P^2 (\mathbb R)$, but with different separating gonality?
	\end{itemize}
\end{que}



In \cite{KumShaw20}, the separating morphisms of real separating curves are studied as follows. Let a smooth real separating algebraic compact curve $C$ consist of $l$ real connected components $C_1, \dots, C_l$. Let $f: C \rightarrow \mathbb P^1_{\mathbb C}$ be any separating morphism of $C$. Set $d_i(f) \in \mathbb N$ the degree of the covering map $f|_{C_i}: C_i \rightarrow \mathbb P^1 (\mathbb R ) $ and set $d(f):=(d_1(f), \dots, d_l(f))$. The set $\text{Sep}(C)$ of all such degree partitions forms a semigroup, called \textit{separating semigroup}.


Here, we report a remark on the separating semigroup of real separating $(M-2)$-curves.
\begin{lem}
	\label{lem: sep_semigroup_m-2}
Let $C$ be a smooth real compact separating $(M-2)$-curve of genus $g$. Then $\text{Sep}(C) \supseteq (4,3,\dots,3)+\mathbb N^{g-1}$. Moreover
\begin{enumerate}
	\item[(1)] $\text{Sep}(C) \supseteq (3,\dots,3)+\mathbb N^{g-1}$, if $\text{sepgon}(C)=g-1$.
	\item[(2)] $\text{Sep}(C) \supseteq (4,2,\dots,2)+\mathbb N^{g-1}$, if $\text{sepgon}(C)=g$.
\end{enumerate}

\end{lem}

\begin{proof}
First, thanks to \cite[Theorem 2.5]{Huis03}, one has that any real divisor $D$ on $C$ such that $\text{deg}(D)+k \geq 2g-1$, is non-special, where $k$ is the number of connected components of $C(\mathbb R)$ such that the degree of $D$ restricted to each of those is odd.

There exists a real divisor $\tilde D$ on $C$ associated to a separating morphism $f: C \rightarrow \mathbb P^1$ with $\text{deg}(f)=\text{sepgon}(C)$, which is either $g$ or $g-1$ \cite[Theorem 7.1]{Gaba06}.

If $\text{deg}(f)=g-1$, it means that $(1,\dots, 1) \in$ Sep$(C)$. Moreover, since Sep$(C)$ is a semigroup \cite[Proposition 2.1]{KumShaw20}, there exists a separating morphism $\tilde f$ of $C$ of degree $3g-3$ with partition degree $(3, \dots, 3)$. Therefore, the associated divisor $\tilde D$ is non-special and, by \cite[Proposition 3.2 and Remark 3.3]{KumShaw20}, the separating semigroup of $C$ contains $(3, \dots, 3) + \mathbb N^{g-1}$.

Otherwise, if $\text{deg}(f)=g$, it means that $(2,1,\dots, 1) \in$ Sep$(C)$ and, analogously, there exists a separating morphism $\tilde f$ of $C$ of degree $2g$ with partition degree $(4,2, \dots, 2)$. Therefore, the associated divisor $\tilde D$ is non-special, and the separating semigroup of $C$ contains $(4, 2, \dots, 2) + \mathbb N^{g-1}$.

\end{proof}
\begin{rem}
In \cite[Example 2.8]{KumShaw20}, it is observed, via an example, that the separating semigroup of real separating curves is not symmetric in general. Here we give another example. Pick a real plane separating quintic $C$ with five connected components. A linear system of rank 2 on a curve of genus bigger equal to 3 is unique; \cite[A.18]{ACGH85}. So, one can label the pseudo-line, the positive oval and the negative ones of $C(\mathbb R)$ respectively as $X_1,\dots,X_5$. The element $(2,1,1,1,1)$ has been constructed in proof of \Cref{lem: quintic}. But, because of \Cref{exa: quintic}, not all permutations of $(2,1,1,1,1)$ belong to Sep($C$). In fact, only $(1,2,1,1,1)$ may also exist.	
	\end{rem}

\begin{rem}
	\label{rem: other_surfaces}
	The interested reader can investigate, analogously to the case of plane curves, further constructions and applications of Theorems \ref{thm: stepan} and \ref{thm: coppens_m-2} to real curves embedded in other ambient surfaces; see e.g. \cite{DegKha00}, \cite{GudShu80}, \cite{Mikh98}, \cite{Manz21}, \cite{Manz22}, \cite{Orev03}.
\end{rem}
\bibliographystyle{alpha}
\bibliography{biblio.bib}

\par\nopagebreak
Matilde Manzaroli, \textsc{Universit\"{a}t T\"{u}bingen, Germany}\par\nopagebreak
 \textit{E-mail address}: \texttt{ matilde.manzaroli'at'uni-tuebingen.de}
\end{document}
