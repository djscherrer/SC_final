\begin{document}
\affiliation{$$_affiliation_$$}
\title{Optimal Bias-Correction and Valid Inference in High-Dimensional Ridge Regression: A Closed-Form Solution}
\maketitle
			\vspace{-1cm}
			
			\begin{abstract}
   Ridge regression is an indispensable tool in big data econometrics but suffers from bias issues affecting both statistical efficiency and scalability. We introduce an iterative strategy to correct the bias effectively when the dimension $p$ is less than the sample size $n$. For $p>n$, our method optimally reduces the bias to a level unachievable through linear transformations of the response. We employ a Ridge-Screening (RS) method to handle the remaining bias when $p>n$, creating a reduced model suitable for bias-correction. Under certain conditions, the selected model nests the true one, making RS a novel variable selection approach. We establish the asymptotic properties and valid inferences of our de-biased ridge estimators for both $p< n$ and $p>n$, where $p$ and $n$ may grow towards infinity, along with the number of iterations. Our method is validated using simulated and real-world data examples, providing a closed-form solution to bias challenges in ridge regression inferences.				\vspace{0.5cm}
				
				\noindent\textbf{Keywords:} Ridge Regression, Bias Correction, High-Dimension, Ridge Screening, Inference, Bias-Variance Trade-off
				
				
			\end{abstract}

		\end{titlepage}
		
		

\setcounter{page}{2}
  
  \abovedisplayskip=0.1pt
\belowdisplayskip=0.1pt
Regularization theory was one of the first signs of the existence of intelligent inference.{---Vladimir N. Vapnik (p.9, \cite{vapnik2013nature})}
		\section{Introduction}
Ridge regression, or more formally $\ell_2$-regularization estimation, is a fundamental tool in econometrics, statistics, and machine learning with applications in various fields of science, technology, engineering, mathematics, medicine, social sciences, and humanities. The idea of $\ell_2$-regularization appeared in the early  1940s  for the stability of inverse problems; see \cite{tikhonov1943stability}.
It was first introduced to data analysis by \cite{hoerl1959optimum}  and later formulated in \cite{hoerl1970aridge,hoerl1970bridge} for providing a robust solution to some of the persistent challenges encountered in traditional linear regression techniques; see \cite{hoerl1985ridge} for a nice review. Emerging as a fundamental technique in predictive modeling, ridge regression addresses issues such as multicollinearity and overfitting, which commonly afflict predictive models dealing with high-dimensional data. Since its inception, ridge regression's practical adoption persists due to its superior performance over the least-squares estimator in various scenarios, evident in applications across neuroscience, chemistry, biology, and economics; see \cite{leonard2023large}, \cite{zahrt2019prediction}, \cite{otwinowski2014inferring}, \cite{giannone2021economic}, and \cite{abadie2019choosing}, among others, underscoring its empirical effectiveness. From a shrinkage perspective, the ridge estimator also dominates the least-squares solutions in the sense that its mean-squared errors (MSEs) can be smaller, which provides a reasonable explanation  on the empirical effectiveness of ridge estimators.  See \cite{theobald1974generalizations}, \cite{athey2019machine}, \cite{hastie2020ridge}, \cite{hansen2022econometrics}, and a comprehensive introduction to ridge regression in  \cite{van2023lecture}.




The ridge estimator offers a closed-form expression that simplifies both theoretical and empirical analyses. It aligns with the dense modeling techniques of \cite{giannone2021economic}, which acknowledge the potential significance of all explanatory variables for prediction. Empirical studies, such as those in \cite{giannone2021economic}, indicate that dense models generally tend to outperform the sparse ones in out-of-sample economic prediction performance. Similarly, \cite{abadie2019choosing} find that the ridge estimators dominate the lasso and the pre-testing estimators in terms of the risks when the effects of different predictors on the dependent variable are ``smoothly distributed”. 
These results suggest that ridge estimators indeed constitute a crucial tool in econometric modeling and economic forecasting, especially in the big data era.


However, as highlighted in Section 2.8 of \cite{athey2019machine}, constructing valid confidence intervals remains a challenge for many regularized methods, including ridge regression, even in asymptotic settings. This long-standing challenge in ridge-type regression involves at least two critical aspects within a linear regression framework:
(a) performing hypothesis tests on specific linear combinations of the regression coefficients using the  ridge estimators; and
(b) deriving confidence or prediction intervals based on the ridge estimators in empirical applications.
The primary reason for these challenges is that the inherent bias of ridge estimators poses significant challenges, compromising both statistical efficiency and scalability across various applications. To date, the feasibility of conducting statistical inferences and hypothesis testing on ridge-type estimators without imposing additional structural constraints remains largely unexplored in the literature. This complexity arises from the ridge estimator's intrinsic bias, which complicates direct statistical inferences despite its elegant closed-form expression.
As a result, research addressing the inference challenges of ridge regression in high-dimensional settings is limited, leading to its widespread application across disciplines without comprehensive theoretical investigations.






To the best of our knowledge, there are only a few works in the literature concerning the bias and inference of ridge estimators under different scenarios. Under a sparse structure, \cite{shao2012estimation} proposed a threshold ridge regression method and proved its consistency. The method therein actually estimates the projected coefficient vector rather than the true one in the linear model. \cite{dobriban2018high} derived the limit of the predictive risk of ridge regression and regularized discriminant analysis in a dense random effects model. \cite{buhlmann2013statistical} proposed to use the lasso to correct the bias of ridge estimators. However, the estimation depends on the existence of an initial estimator which needs to be accurate enough. \cite{zhang2022ridge} adopted a similar approach as that in \cite{shao2012estimation} and proposed a threshold ridge regression and a bootstrap method to make inferences. However, the method therein still estimates the projected coefficient vector rather than estimating the true one, and the remaining bias may not be asymptotically negligible in general.









In this paper, we introduce a systematic approach tailored specifically for mitigating the bias in ridge-type estimators for high-dimensional linear regression models. Leveraging the closed-form expression of the ridge estimators, the bias term can also be established in an analytic form. Although the bias term involves the true parameters which are unknown in practice, we found that replacing the true parameters with the ridge estimators  turns out to be an effective way to mitigate the bias.
Therefore, the proposed method employs an iterative bias-correction strategy, and the bias can be reduced substantially when the number of iterations is sufficiently large.
 Notably, it achieves complete bias correction if the covariate dimension $p$ is smaller than the sample size $n$, and can reduce considerable bias if $p$ surpasses $n$. We show that our bias-correction method is an optimal one in the sense that the bias can be completely corrected (asymptotically) when $p< n$ with a sufficient number of the proposed bias-correction iterations, and the remaining bias of the de-biased ridge estimator  when $p>n$ is unattainable through any linear transformations of the response vector.

To further combat the remaining bias in the de-biased ridge estimators when $p>n$, we introduce a novel ridge-screening (RS) method for covariate selection prior to applying our bias correction procedure. The RS approach constructs a restricted model that inherently encompasses the true model as a subset. This is based on the assumption that only a subset of the covariates holds significance in the linear model. Specifically, we postulate that the number of significant covariates should be less than the sample size, which can appropriately diverge relative to the dimension 
$p$ and sample size 
$n$. Crucially, we demonstrate that under certain mild conditions, the true model is inherently nested within the selected one, establishing RS as a novel variable selection approach that offers independent value beyond bias correction. Leveraging this restricted model, our bias-correction procedure can be further applied to the restricted model and effectively rectifies the bias in the resulting ridge estimators



The derivation of the methodology and theory is mainly based on a fixed design, which is similar to the setting in \cite{hansen2022modern}, and the results throughout this paper are valid for random regressors by conditioning on the design matrix. This paper rigorously establishes the asymptotic properties and provides valid inferences of our estimators for both $p< n$ and $p>n$ under some relaxed and intuitive conditions. Furthermore, we delve into the bias-variance trade-off of our de-biased ridge estimators, examining its relationship with the number of iterations in the bias-correction procedure. To validate our methodology, we provide empirical evidence using both simulated and real-world data.  The prediction intervals constructed by the proposed method are indeed satisfactory for forecasting the U.S. macroeconomic series using factor-augmented regression. Moreover, the RS method can further enhance the coverage rates of these prediction intervals. These empirical validations highlight the practical efficacy and adaptability of our approach across a wide range of high-dimensional regression settings.


The contributions of this paper are multi-fold. First, the proposed bias-correction method is simple and easy to implement. In fact, the proposed approach is a systematic procedure and the resulting de-biased estimator has a closed-form expression. Second, the optimality of the proposed bias-correction procedure consists of two aspects: (a) it achieves complete bias correction when the covariate dimension $p$ is smaller than the sample size $n$; and (b) the remaining bias of the de-biased ridge estimator in the scenario when $p>n$ is unattainable through any linear transformations of the response vector. These results distinguish our work from the existing ones that only part of the bias can be corrected in most of the aforementioned literature. Third, we also propose a novel variable selection procedure, namely the ridge-screening (RS) method, which screens out some insignificant variables based on the de-biased ridge estimator due to its optimality.  Fourth, we also establish the asymptotic properties of the de-biased ridge estimators in both scenarios when $p< n$ and $p>n$, and it is shown that the de-biased ridge estimators are asymptotically normal, which provides valid inference methods for the ridge estimators. Fifth, we develop a procedure to construct confidence and prediction intervals in ridge regressions using our proposed de-biased estimators and associated inference methods. Finally, we establish the bias-variance trade-off of our proposed approach both theoretically and through validation with simulated data. It's important to note that, unlike the scenario described in Section 2.8 of \cite{athey2019machine} where many inference approaches for regularized machine learning methods compromise predictive performance, our proposed procedure focuses  on correcting the bias and rendering the estimators suitable for inferences, without adversely affecting their predictive performance.
 
We highlight that the asymptotic framework adopted in this paper is slightly different from traditional approaches in the literature. Typically, in conventional frameworks, the asymptotic properties are established when the dimension and/or the sample size are approaching infinity. However, in our study, most of the asymptotic results are derived under a different scenario that the number of iterations in the bias-correction procedure tends towards infinity for any given configuration of the dimension $p$ and the sample size $n$. This configuration can be sufficiently large to encompass the framework of big data analysis. One of the primary motivations behind this choice is to demonstrate the validity of our proposed procedure by explicitly providing the exact bias and covariance terms of the de-biased estimators in this paper. This approach is also reasonable because it reflects common scenarios encountered in practical data analysis, where datasets often have fixed dimensions and sample sizes.  Without this setting, the dimensions of the bias and covariance terms of the de-biased estimators would expand to infinity if we considered $n$ and $p$ approaching infinity, making them challenging to formulate and describe theoretically. Moreover, our asymptotic results remain valid as $n$ and $p$ approach infinity. This holds true so long as we initially allow the number of iterations to increase towards infinity at a moderate rate. In this asymptotic manner, we can symbolically retain the forms of the bias and covariance terms for a growing configuration of $(p, n)$. Therefore, it's important to emphasize that our chosen asymptotic framework does not undermine the validity of the proposed bias-correction procedures.


The rest of the paper is organized as follows: Section \ref{sec2} introduces the ridge estimation and its bias-correction procedure in scenarios when $p< n$ and $p>n$, where a ridge-screening method is also introduced for variable selection.  Following this, Section \ref{sec2} also presents the inference methodologies for the proposed de-biased estimators. Section \ref{sec3} examines the proposed approach's finite-sample performance via Monte-Carlo simulations. Section~\ref{sec40} provides an empirical application of the proposed method, and Section \ref{sec4}  offers conclusive insights. All proofs and derivations for the asymptotic results are available in an online Appendix.



{\bf Notation:}  We use the following notation. For a $p\times 1$ vector
$\bu=(u_1,..., u_p)'$, $\|\bu\|_2=\sqrt{\sum_{i=1}^p|u_i|^2}$ is the $\ell_1$-norm and $\|\bu\|_\infty=\max_{1\leq i\leq p}|u_i|$ is the $\ell_\infty$-norm. $\bI_p$ denotes the $p\times p$ identity matrix. For a matrix $\bH$, its operator norm is $\|\bH
\|_2=\sqrt{\lambda_{\max} (\bH' \bH ) }$, where
$\lambda_{\max} (\cdot) $ denotes the largest eigenvalue of a matrix, and $\|\bH\|_{\min}$ is the square root of the minimum non-zero eigenvalue of $\bH\bH'$. $|\bH|$ denotes the absolute value of $\bH$ elementwisely. The superscript ${'}$ denotes the 
transpose of a vector or matrix. We also use the notation $a\asymp b$ to denote $a=O(b)$ and $b=O(a)$ or $a$ and $b$ have the same order.


		
		\section{Models and Methodology} \label{sec2}
		
		\subsection{High-dimensional Linear Regression}\label{model_overview}
	Let $\{(\bx_1,y_1),...,(\bx_n,y_n)\}$ be a given sample of centered observable data. We consider the problem of estimating a $p$-dimensional vector $\bbeta$ from the following linear model:
 \begin{equation}\label{hlm}
     y_i=\bx_i'\bbeta+\ve_i,\,\,i=1,...,n,
 \end{equation}
where $y_i$ is a scalar response variable, $\bx_i$ is a $p$-dimensional covariate vector, and $\ve_i$ is a random error term with mean zero and finite variance.  Similar to the setting in \cite{hansen2022modern} and Ch. 29 of \cite{hansen2022econometrics}, we treat the $n\times p$ design matrix $\bX=(\bx_1,...,\bx_n)'$ consisting of $p$ covariates as a fixed one. But the estimation results throughout the paper remain valid for random regressors by conditioning on the design matrix data $\bX$. Note that Model (\ref{hlm}) can be expressed in vector form
\begin{equation}\label{v:hlm}
    \by=\bX\bbeta+\bve,
\end{equation}
where $\by=(y_1,...,y_n)'$ is an $n$-dimensional response vector,  and $\bve=(\ve_1,...,\ve_n)'$ is an $n$-dimensional vector of errors with $E(\bve)={\bf 0}$ and $\cov(\bve)=\bSigma_\ve$, where $\bSigma_\ve$ is a diagonal matrix with positive and bounded diagonal elements.  
If $p< n$ and $\bX'\bX$ is an invertible matrix, the least-squares estimator for $\bbeta$ is 
\begin{equation}\label{lse}
    \wh\bbeta_{lse}=(\bX'\bX)^{-1}\bX'\bY.
\end{equation}
Note that the least-squares estimator $\wh\bbeta_{lse}$ is only well-defined if $(\bX'\bX)^{-1}$ exists. In a high-dimensional setting, if the columns of the design matrix $\bX$ are linearly dependent, for example, this is obviously true when $p>n$, this collinearity among the columns implies that $\bX'\bX$ is singular, rendering $\wh\bbeta_{lse}$ an invalid estimator. To make the least-squares estimator in (\ref{lse}) a well-defined quantity, we modify the definition in (\ref{lse}) as
\begin{equation}\label{lse:mp}
    \wh\bbeta_{lse}=(\bX'\bX)^{+}\bX'\bY,
\end{equation}
where $(\bX'\bX)^{+}$ is the Moore-Penrose generalized inverse. It is not hard to see that the estimator in (\ref{lse:mp}) reduces to the one in (\ref{lse})
if $p<n$ and $\bX'\bX$ is invertible. Therefore, we will denote the estimator in (\ref{lse:mp}) as the least-squares solution throughout this article.


\subsection{Ridge Regression}
The ridge regression estimator was first proposed by \cite{hoerl1959optimum}; see the review article of \cite{hoerl1985ridge}. It essentially comprises of an ad-hoc fix to resolve the singularity issue of $\bX'\bX$ in the presence of many covariates. Suppose $(\bX'\bX+\lambda\bI_p)$ is invertible for a given $\lambda>0$, the ridge estimator is defined as
\begin{equation}\label{rg:e}
    \wh\bbeta(\lambda)=(\bX'\bX+\lambda\bI_p)^{-1}\bX'\by,
\end{equation}
which simply replaces $\bX'\bX$ by $\bX'\bX+\lambda\bI_p$ with a tuning parameter $\lambda>0$ in the least-squares estimator of (\ref{lse}).  From a regression point of view, the ridge estimator can be obtained in the following way. Let $\by^*=(\by',{\bf 0})'$ and $\bX^*=(\bX',\sqrt{\lambda}\bI_p)'$ be the augmented data, the ridge estimator is a solution to the following optimization problem:
\begin{equation}
    \wh\bbeta(\lambda)=\arg\min_{\bbeta\in R^p}\left\{\|\by^*-\bX^*\bbeta\|_2^2\right\}=\arg\min_{\bbeta\in R^p}\left\{\|\by-\bX\bbeta\|_2^2+\lambda\|\bbeta\|_2^2\right\}.
\end{equation}
From the expression of the ridge estimator for a given $\lambda>0$,
it is not hard to see that
\[\wh\bbeta(\lambda)\rightarrow\wh\bbeta_{lse},\,\, \text{as}\,\, \lambda\rightarrow 0,\]
 and 
\[\lambda\wh\bbeta(\lambda)\rightarrow\bX'\by,\,\,\text{as}\,\, \lambda\rightarrow \infty,\]
which is the componentwise regression estimator if each covariate is standardized. Therefore, a large $\lambda$ would reduce the variance of the estimator, but the bias may increase as $\lambda$ grows. 

In this paper, we only focus on a given $\lambda>0$ and investigate the bias-correction and inference issues for $\wh\bbeta(\lambda)$. For the purpose of comparisons with the proposed approach, we first specify the initial bias of the ridge estimator of (\ref{rg:e}) in the following theorem.

\begin{theorem}\label{thm1}
If $\by$ admits a linear structure as that in (\ref{v:hlm}) with $E\bve={\bf 0}$, then the bias of the ridge estimator $\wh\bbeta(\lambda)$ in (\ref{rg:e}) is
\begin{equation}\label{bt}
    \bb_{\lambda,0}=\bbeta-E\wh\bbeta(\lambda)=\lambda (\bX'\bX+\lambda\bI_p)^{-1}\bbeta,
\end{equation}
for any given $\lambda>0$ such that $(\bX\bX'+\lambda\bI_p)$ is invertible.
\end{theorem}
\begin{remark}
    The condition for the result in Theorem~\ref{thm1} to hold is the same as that for linear regression models. If the design matrix $\bX$ is random with either independent and identically distributed  ($i.i.d.$) or weakly dependent columns, we require $E(\bve|\bX)={\bf 0}$, and then, the bias of the ridge estimator conditioning on $\bX$ and a given $\lambda>0$ is
    \[\text{bias}[\wh\bbeta(\lambda)|\bX]=\bbeta-E[\wh\bbeta(\lambda)|\bX]=\lambda (\bX'\bX+\lambda\bI_p)^{-1}\bbeta,\]
    which is the same as that in (\ref{bt}). See also Ch. 29.6 in \cite{hansen2022econometrics} for details.
\end{remark}

From Theorem 1, the bias of the ridge estimator depends on the unknown true parameter $\bbeta$. Therefore, it is fundamentally challenging to make any statistical inference on the ridge estimator or to construct any confidence or prediction intervals involving the ridge estimator. Consequently, it is important to seek an effective way to correct or reduce the bias of a ridge estimator.

 In the next section, we will tackle this issue by proposing an iterative procedure to reduce the bias of the ridge estimators.






\subsection{Bias-Correction}\label{sec23}
The discussion in this section is divided into two key parts depending on whether $\bX'\bX$ is invertible or singular. For simplicity, in line with the setting  in \cite{wang2016high}, we assume that $\bX'\bX$ is invertible when $p< n$ and singular when $p>n$ throughout this article\footnote{An alternative framework is to consider the scenarios that $p/n\in (0,1)$ and $p/n\in(1,\infty)$, a common setting in random matrix theory, which is also helpful in establishing the asymptotic results if we further allow $n,p\rightarrow\infty$ later.}. This assumption is well-justified as we consider a fixed design matrix $\bX$ in this paper and  $\bX'\bX$ naturally maintains its invertibility when $p< n$. Furthermore, in extreme cases where highly correlated variables are present within $\bX$ (in a random sense), we may implement specific transformations based on prior knowledge or statistical methods such as the hierarchical clustering or the $k$-means algorithm to mitigate these correlations prior to conducting ridge regression; see the discussion in Section 4.1.2 of \cite{fan2008sure}. Consequently, we only focus on the bias-correction issue in this paper and rule out the case when some covariates in $\bX$ are highly correlated.







Note that the ridge estimator in (\ref{rg:e}) can be written as
\begin{equation}\label{decom:1}
    \wh\bbeta(\lambda)=\bbeta-\lambda(\bX'\bX+\lambda\bI_p)^{-1}\bbeta+(\bX'\bX+\lambda\bI_p)^{-1}\bX'\bve.
\end{equation}
The bias-correction procedure is based on  this expression and (\ref{bt}) in Theorem \ref{thm1}. The rationale for the procedure is as follows. Since $\bbeta$ in the bias term of (\ref{bt}) is unknown, we first replace it by $\wh\bbeta(\lambda)$ defined in (\ref{rg:e}) and construct a first-step de-biased ridge estimator as
 \begin{equation}\label{bc:0}
\wh\bbeta_{c,1}(\lambda)=\wh\bbeta(\lambda)+\lambda(\bX'\bX+\lambda\bI_p)^{-1}\wh\bbeta(\lambda).
 \end{equation}
Plugging $\wh\bbeta(\lambda)$ from (\ref{decom:1}) into (\ref{bc:0}), we obtain:
\begin{equation}\label{bc1:rp}
    \wh\bbeta_{c,1}(\lambda)=\bbeta-\lambda^2(\bX'\bX+\lambda\bI_p)^{-2}\bbeta+(\bX'\bX+\lambda\bI_p)^{-1}\bX'\bve+\lambda(\bX'\bX+\lambda\bI_p)^{-2}\bX'\bve.
\end{equation}
Consequently, the bias term of $\wh\bbeta_{c,1}(\lambda)$ is
\begin{equation}\label{bc:1}
     \bb_{\lambda,1}=\bbeta-E(\wh\bbeta_{c,1}(\lambda))=\lambda^2(\bX'\bX+\lambda\bI_p)^{-2}\bbeta.
 \end{equation}
It's apparent that the $\ell_2$-norm of the bias term $\bb_{\lambda,1}$ produced by $\wh\bbeta_{c,1}(\lambda)$ is smaller than that of the initial bias $\bb_{\lambda,0}$ in Theorem \ref{thm1} under mild conditions, implying that the bias  $\bb_{\lambda,0}$ has been partially corrected by $\wh\bbeta_{c,1}(\lambda)$. To see this, we conduct a singular-value decomposition on $\bX$ or a spectral decomposition on $\bX'\bX$, and the effectiveness of the bias-correction approach depends on two observations: (a) the eigenvalues of $\lambda(\bX'\bX+\lambda\bI_p)^{-1}$ are positive and strictly less than one; and (b) the eigenvalues of $\lambda^2(\bX'\bX+\lambda\bI_p)^{-2}$ becomes smaller in the de-biased estimator of (\ref{bc1:rp}) compared to those of  $\lambda(\bX'\bX+\lambda\bI_p)^{-1}$ in $\bb_{\lambda,0}$.


Following the first step, we replace the unknown vector $\bbeta$ in (\ref{bc:1}) by the ridge estimator $\wh\bbeta(\lambda)$ again, leading us to construct a second-step de-biased estimator:
\begin{equation}\label{bc2:rp}
    \wh\bbeta_{c,2}(\lambda)=\wh\bbeta_{c,1}(\lambda)+\lambda^2(\bX'\bX+\lambda\bI_p)^{-2}\wh\bbeta(\lambda)=\wh\bbeta(\lambda)+\sum_{j=1}^2\lambda^j(\bX'\bX+\lambda\bI_p)^{-j}\wh\bbeta(\lambda).
\end{equation}
By a similar argument, the bias term of $\wh\bbeta_{c,2}(\lambda)$ is
\begin{equation}\label{bc:2}
    \bb_{\lambda,2}=\bbeta-E(\wh\bbeta_{c,2}(\lambda))=\lambda^3(\bX'\bX+\lambda\bI_p)^{-3}\bbeta,
\end{equation}
where the eigenvalues of $\lambda^3(\bX'\bX+\lambda\bI_p)^{-3}$ are even smaller compared to those of $\lambda^2(\bX'\bX+\lambda\bI_p)^{-2}$ in $\bb_{\lambda,1}$. Consequently, we can  show that the $\ell_2$-norm of the bias $\bb_{\lambda,2}$ is smaller than that of $\bb_{\lambda,1}$ under the same framework. We  continue this procedure and denote
\begin{equation}\label{bck:rp}
    \wh\bbeta_{c,k}(\lambda)=\wh\bbeta(\lambda)+\sum_{j=1}^k\lambda^j(\bX'\bX+\lambda\bI_p)^{-j}\wh\bbeta(\lambda)
\end{equation}
as a de-biased estimator at the $k$-th step, where we define $\wh\bbeta_{c,0}(\lambda)=\wh\bbeta(\lambda)$ for $k=0$. To characterize the effect of the bias correction, we make an assumption on the singular-value decomposition (SVD) of $\bX$ first.
\begin{assumption}\label{asm1}
    For $p< n$, $\bX'\bX$ is invertible and the SVD of $\bX$ is $\bX=\bV_1\bD_1\bU_1'$, where $\bU_1'\bU_1=\bV_1'\bV_1=\bI_p$ and $\bD_1=\diag(d_1,...,d_p)$ with $d_i>0$ for $1\leq i\leq p$.
    \end{assumption}
Assumption~\ref{asm1} is intuitive for a fixed design $\bX$ with a large and fixed configuration of $(p,n)$. For example, the eigenvalues of $\bX'\bX$ are of order $n$ if the entries of $\bX$  are independent copies of a random variable with zero mean, unit variance, and
finite fourth moment if $p/n\in (0,1)$; see the Bai-Yin's law in \cite{bai1993limit}.  Consequently, the eigenvalues of $\lambda(\bX'\bX+\lambda\bI_p)^{-1}$ are strictly less than one, which ensures that our iterative procedure can substantially reduce the bias. 
In fact, we have the following theorem on the de-biased ridge estimator in (\ref{bck:rp}).
\begin{theorem}\label{thm2}
     If $p<n$ and $\bX'\bX$ is invertible. Under Assumption~\ref{asm1}, the bias of the de-biased ridge estimator $\wh\bbeta_{c,k}(\lambda)$ defined in (\ref{bck:rp}) is
     \begin{equation}\label{bc:k}
         \bb_{\lambda,k}=\bbeta-E(\wh\bbeta_{c,k}(\lambda))=\lambda^{k+1}(\bX'\bX+\lambda\bI_p)^{-(k+1)}\bbeta.
     \end{equation}
     Furthermore, for any configuration of $(p,n)$ with $\|\bbeta\|_2<C_{n,p}<\infty$, if the number of iterations $k$ satisfies $\max_{1\leq j\leq p}C_{n,p}(\frac{\lambda}{d_j^2+\lambda})^{k+1}\rightarrow 0$, we have
     \[\bb_{\lambda,k}\rightarrow {\bf 0},\,\, \text{as}\,\, k\rightarrow\infty.\]
\end{theorem}
\begin{remark}\label{rm2}
    (i) We observe that the assumptions required for Theorem~\ref{thm2} are quite minimal. We only need the fundamental assumptions inherent to linear regression models, ensuring that the bias term can be asymptotically eliminated with a sufficient number of iterations.\\
    (ii) The requirement for the $\ell_2$-norm of $\bbeta$ to be finite stems from the expectation of the response having a limited number of significant covariates in linear regression models. Without this restriction, as the number of covariates grows, each coefficient's contribution might become sufficiently small, allowing the response's variance to remain finite. It's important to highlight that the condition for the number of iterations $k$ can still be met even if $\|\bbeta\|_2\rightarrow\infty$ at a polynomial rate. This is because $(\frac{\lambda}{d_j^2+\lambda})^{k+1}$  decays exponentially, given that $|\frac{\lambda}{d_j^2+\lambda}|<1$ for $1\leq j\leq p$. Thus, the number of iterations $k$ can be selected to scale logarithmically with the dimension $p$.\\
    (iii) As discussed in the Introduction section, the asymptotic results in Theorem~\ref{thm2} are established for any given 
$(p,n)$ as 
$k\rightarrow\infty$.  This approach is also reasonable because it reflects common scenarios encountered in practical data analysis, where datasets often have fixed dimensions and sample sizes.
As a matter of fact, if Assumption~\ref{asm1} holds for  increasing $p$ and $n$ with $p<n$, the results remain applicable when $k\rightarrow\infty$ first, followed by
$n,p\rightarrow\infty$. In addition, under the framework in \cite{bai1993limit} that $d_i^{2}\asymp n$, the asymptotic results hold simultaneously as $n,p,k\rightarrow\infty$ if $p/n\in(0,1)$ and the penalty parameter $\lambda\asymp n$  because $\max_{1\leq j\leq p}|\frac{\lambda}{d_j^2+\lambda}|<1$ still hold in this setting.\\
(iv) Although the convergence in Theorem~\ref{thm1} is based on a given $\lambda>0$, it can be readily shown that 
\[\sup_{\lambda\in[\lambda_1,\lambda_2]}\|\bb_{\lambda,k}\|_2\rightarrow{0},\,\,\text{as}\,\, k\rightarrow\infty,\,\,\text{for}\,\,0<\lambda_1\leq \lambda_2<\infty,\]
implying that the convergence to zero is uniformly  for a range of $\lambda$. Similar argument also applies to the asymptotic results in Theorem~\ref{thm3}-\ref{thm5} below.
\end{remark}
Theorem~\ref{thm2} implies that we can  completely correct the bias term incurred by the ridge estimator $\wh\bbeta(\lambda)$ if $p<n$ so long as we conduct a sufficient number of iterations. This result is particularly useful for empirical data analysis when $n$ and $p$ are given.
To investigate the performance of the de-biased estimator $\wh\bbeta_{c,k}(\lambda)$ when $p>n$ under which $\bX'\bX$ is a singular matrix, we first perform a singular-value-decomposition on $\bX$. Suppose the true rank of $\bX$ is $\text{rank}(\bX)=p^*\leq  \min(p,n)=n$, where we can simply set $p^*=n$ since we deal with a given and fixed configuration of $(p,n)$ without including highly correlated covariates. However, the results in Theorem~\ref{thm3} still hold for $p^*<n$.
By abuse of notation, there exist semi-orthogonal matrices $\bV_1\in R^{n\times p^*}$ and $\bU_1\in R^{p\times p^*}$, and a diagonal matrix $\bD_1=\diag(d_1,...,d_{p^*})$ with $d_1\geq ...\geq d_{p^*}>0$ such that
\begin{equation}\label{svd:x}
\bX=\bV_1\bD_1\bU_1'\,\,\text{and}\,\,\bX'\bX=\bU_1\bD_1^2\bU_1'.
\end{equation}
Since $\bX'\bX+\lambda\bI_p$ is symmetric and invertible, there also exists an orthogonal complement matrix $\bU_2\in R^{p\times (p-p^*)}$ of $\bU_1$ such that
\begin{equation}\label{svd:xxi}
 \bX'\bX+\lambda\bI_p=\bU\bD\bU', \,\,\text{where}\,\, \bU=[\bU_1,\bU_2]\,\,\text{and}\,\,\bD=\diag(\bD_1^2+\lambda\bI_{p^*},\lambda\bI_{p-p^*}).
\end{equation}
We formulate the above description in Assumption~\ref{asm2} below.
\begin{assumption}\label{asm2}
    Rank($\bX$)$=p^*\leq \min(p,n)=n$ if $p>n$, and the design matrix $\bX$ has a SVD $\bX=\bV_1\bD_1\bU_1'$ such that  $\bX'\bX+\lambda\bI_p=\bU\bD\bU'$, where $\bD$, $\bD_1$, $\bU$, $\bU_1$,  $\bV_1$, and $\bU_2$ are defined as those in (\ref{svd:x}) and (\ref{svd:xxi}).
\end{assumption}
Then, we have the following theorem.
\begin{theorem}\label{thm3}
    If $p>n$ and $\bX'\bX$ is singular, but $\bX'\bX+\lambda\bI_p$ is invertible for a given $\lambda>0$. Under Assumption~\ref{asm2}, the bias term $\bb_{\lambda,k}$ of $\wh\bbeta_{c,k}(\lambda)$ is the same as (\ref{bc:k}) in Theorem~\ref{thm2}. If the number of iterations $k$ satisfies $\max_{1\leq j\leq p^*}C_{n,p}(\frac{\lambda}{d_j^2+\lambda})^{k+1}\rightarrow 0$ where the constant $C_{n,p}$ is the same as that in Theorem~\ref{thm2}, we have
    \[\bb_{\lambda,k}\rightarrow\bU_2\bU_2'\bbeta,\,\,\text{as}\,\, k\rightarrow \infty,\]
    where $\bU_2$ is defined in (\ref{svd:xxi}).
\end{theorem}
\begin{remark}\label{rm3}
   (i) The requirement for the number of iterations $k$ is the same as that in Theorem~\ref{thm2}, and therefore, we omit the illustrations to save space.\\
   (ii) The bias term in Theorem~\ref{thm3} corroborates the assertion in \cite{shao2012estimation} that ridge regression primarily estimates 
 $\bU_1\bU_1'\bbeta$ rather than $\bbeta$.\\
   (iii) Similar to Theorem~\ref{thm2}, the asymptotic results in Theorem~\ref{thm3} are established for any given configuration of $(p,n)$ as $k\rightarrow\infty$, because we can explicitly specify the bias term $\bU_2\bU_2'\bbeta$ for fixed $p$ and $n$. If we additionally allow $n,p\rightarrow\infty$, the result in Theorem~\ref{thm3} can be reformulated as
   \[ \bb_{\lambda,k}-\bU_2\bU_2'\bbeta\rightarrow{\bf 0},\,\,\text{as}\,\, n,p,k\rightarrow \infty,\]
   so long as $\max_{1\leq j\leq p^*}C_{n,p}(\frac{\lambda}{d_j^2+\lambda})^{k+1}\rightarrow 0$ and $d_j>0$ for $1\leq j\leq p^*$, where $p^*$ is the rank of $\bX$.\\
   (iv) It's possible that the rank of $\mathbf{X}$ is $p^*<p$ in the case of $p < n$. In such situations, we can obtain similar results as those outlined in Theorem~\ref{thm3}. The approach to handling this scenario is similar to that for $p>n$, and therefore, we omit further discussion on this and focus on the previously mentioned setting.
\end{remark}
A key insight from Theorem~\ref{thm3} is that there is a remaining bias term $\bU_2\bU_2'\bbeta$ that cannot be corrected by the proposed method. This is understandable since the projection of $\bbeta$ on the singular directions $\bU_2$ is not captured in Model (\ref{v:hlm}) according to the singular-value-decomposition of $\bX$ in 
(\ref{svd:x}). If the vector $\bbeta$ belongs to the space spanned by the columns of $\bU_1$, there will be no need to correct the bias since $\bU_2'\bbeta={\bf 0}$. Otherwise, it is challenging to make such a correction for the bias in Theorem~\ref{thm3}.  As a matter of fact, we have the following theorem regarding the bias term in Theorem~\ref{thm3}.
\begin{theorem}\label{thm4}
    Under the conditions in Theorem~\ref{thm3}, there is no linear transformation matrix $\bS\in R^{p\times n}$ such that $E[\bS\by]=\bU_2\bU_2'\bbeta$.
\end{theorem}
We focus on linear transformations of the data  in Theorem~\ref{thm4} because the least-squares estimator and the ridge estimator are all linear combinations of the data $\by$. Theorem~\ref{thm4} indicates that the remaining bias term $\bU_2\bU_2'\bbeta$ is unattainable through any linear transformation of the data $\by$, showing that the proposed approach does its best to 
correct the bias. The results in Theorems~\ref{thm2}--\ref{thm4} also indicate that our bias-correction method is an optimal one for any given $(p,n)$.

Simulation results (e.g., Figure~\ref{fig-4}) in Section~\ref{sec3} suggest that the uncorrectable bias $\bU_2\bU_2'\bbeta$ can be significant even when the number of nonzero elements in $\bbeta$ is small. This underscores that any method that only  corrects part of the bias may lead to poor inference performance.



\subsection{Ridge Screening}\label{sec24}
To further address the uncorrectable bias term identified in Theorem~\ref{thm3} and ensure valid statistical inferences when $p>n$, additional structures must be imposed on the parameters or covariates. Without these additional structures, as demonstrated in Theorem~\ref{thm4}, the bias-correction becomes challenging. In the subsequent analysis, we will use 
$C$ or 
$c$ to denote a generic constant, the specific value of which may vary across different contexts.

In this section, we propose a ridge-screening approach to select the significant variables in Model (\ref{v:hlm}). It is feasible for both scenarios when $p<n$ and $p>n$. Therefore, it can also be treated as a new variable selection approach especially useful in a high-dimensional setting, which is of independent interest to statisticians, econometricians, and data scientists. We only focus on the scenario when $p>n$ and $\bX\bX'$ is singular in such a situation. Note that the reason for the remaining bias term in Theorem~\ref{thm3} that cannot be corrected is that some projection directions of the coefficients $\bbeta$ are not captured in Model (\ref{v:hlm}). For example, this is the case when some parameters in $\bbeta$ are redundant if certain covariates in $\bX$ are strongly correlated.  Therefore, it is reasonable to make an assumption that some covariates in $\bX$ are not useful in the linear regression (\ref{v:hlm}), and hence, they can be dropped first before establishing valid ridge estimators. To embrace the sparsity assumption in high-dimensional data analysis, we assume the true parameter vector $\bbeta$ belongs to the following submodel class
\begin{equation}\label{submodel}
    \mathcal{M}_0=\{1\leq i\leq p, |\beta_i|\,\,\text{is not zero}\},
\end{equation}
where we assume its cardinality $|\mathcal{M}_0|=s^*< \min(p,n)$. It is obvious that we require $s^*< p^*$, which is the rank of $\bX$ defined in Assumption~\ref{asm2}. This 
cardinality assumption is natural and it reflects the idea that many coefficient parameters are relatively small among the $p$-dimensional vector $\bbeta$. We treat them as zero elements only for ease of exploitation.  According to the results in Theorem~\ref{thm2} and Theorem~\ref{thm3}, we can see that
\begin{equation}\label{rd:decom}
    E\wh\bbeta_{c,k}(\lambda)=\bbeta-\bb_{\lambda,k}=\bbeta-\bU_2\bU_2'\bbeta+o(1),
\end{equation}
as $k\rightarrow\infty$ for a given $\lambda>0$. 



It is intuitive to expect that the components of $\wh\bbeta_{c,k}(\lambda)$ corresponding to positions in the submodel $\mathcal{M}_0$ will be greater than those at positions in $\mathcal{M}_0^c$, the complement set of  $\mathcal{M}_0$. This is due to the following reason: Assuming the number of nonzero elements in $\bbeta$ is finite or relatively small compared to 
$p$, then the $\ell_2$-norm of the  $p$-dimensional dense vector $\bU_2\bU_2'\bbeta$ is also of finite or relatively small order.  Consequently, the magnitude of each projected coordinate in $\bU_2\bU_2'\bbeta$ is of a smaller order compared to the nonzero elements in $\bbeta$ on average.
Therefore, we propose a Ridge-Screening (RS) method that selects the submodel class
\begin{equation}\label{rs:sub}
    \mathcal{M}_k(\lambda^*)=\{1\leq i\leq p:|\wh\beta_{c,k,i}(\lambda^*)|\,\,\text{are among the largest $n^*$ of all $|\wh\beta_{c,k,i}(\lambda^*)|$'s}\}\footnote{We clarify the notation used here. For $p>n$, $s^*$ is the number of nonzero elements in $\bbeta$, $p^*$ denotes the rank of $\bX$, and $n^*$ is the number of covariates selected by the RS method. Obviously, we require that $s^*<n^*\leq p^*\leq\min(p,n)=n$ for $p>n$. We can also simply set $p^*=n$ since we deal with a fixed design.},
\end{equation}
where $\wh\bbeta_{c,k}(\lambda^*)=(\wh\beta_{c,k,1}(\lambda^*),...,\wh\beta_{c,k,p}(\lambda^*))'$ and we use a different penalty $\lambda^*>0$ to distinguish it from the one utilized in the subsequent step.  The ranking method to derive the submodel in (\ref{rs:sub}) is similar to the approach presented in Section 2.2 of \cite{fan2008sure}. However, the method in \cite{fan2008sure} is based on marginal correlations between the response and the features, while (\ref{rs:sub}) is a utilization of a de-biased estimator, which is more proximate to the true parameter than the conventional ridge estimator in Eq. (5) of \cite{fan2008sure}. Furthermore, our methodology is not constrained by a specific choice of  $\lambda^*$, and it can be chosen by an information criterion or a cross-validation method as discussed at the end of Section~\ref{sec24}.

In practice, we may select $n^*< \min(p,n)=n$ in (\ref{rs:sub}) through cross-validation along with $\lambda^*$ because the actual design matrix with $n^*$ columns of covariates is no longer singular in such cases. Additionally, we can show that the model with $n^*$ covariates asymptotically encompasses the submodel $\mathcal{M}_0$ in (\ref{submodel}) under mild conditions, i.e., the true submodel $\mathcal{M}_0$ in (\ref{submodel}) is nested within the one with $n^*$ covariates. 

Now, we restrict the design matrix to the submodel class $\mathcal{M}_k(\lambda^*)$ and denote the restricted design as $\bX_{\mathcal{M}_k}\in R^{n \times n^*}$, and the new ridge estimator is 
\begin{equation}\label{r:rd}
    \wh\bbeta_{\mathcal{M}_k}(\lambda)=(\bX_{\mathcal{M}_k}'\bX_{\mathcal{M}_k}+\lambda\bI_{n^*})^{-1}\bX_{\mathcal{M}_k}'\by,
\end{equation}
where $n^*< n$, and $\lambda$ can be different from the $\lambda^*$ used in the ridge screening approach in (\ref{rs:sub}). When 
$\lambda^*$
  and 
$n^*$
  are optimally selected using the method outlined at the end of Section~\ref{sec24} below, we can set 
$\lambda=\lambda^*$
  because it is optimal for the restricted ridge estimator with a given set of $n^*$ significant variables. We observe that the eigenvalues of $\bX_{\mathcal{M}_k}'\bX_{\mathcal{M}_k}$ are strictly greater than zero for a given configuration of $(p,n)$. This essentially reduces the scenario to the case when $p<n$ in Section \ref{sec23}. Consequently, we can employ the bias-correction procedure detailed in Section~\ref{sec23} over $l$ iterations to obtain a de-biased estimator:
\begin{equation}\label{db:l}
    \wh\bbeta_{\mathcal{M}_k,l}(\lambda)=\wh\bbeta_{\mathcal{M}_k}(\lambda)+\sum_{j=1}^l\lambda^j(\bX_{\mathcal{M}_k}'\bX_{\mathcal{M}_k}+\lambda\bI_{n^*})^{-j}\wh\bbeta_{\mathcal{M}_k}(\lambda).
\end{equation}
To establish the asymptotic properties of the RS method and the de-biased estimator for the restricted one in (\ref{db:l}), we make a few intuitive assumptions first.
\begin{assumption}\label{asm3}
    The nonzero singular values of $\bX$ in (\ref{svd:x}) are of order $\sqrt{n}$ and the penalty parameters $\lambda^*\asymp\lambda\asymp n$.
\end{assumption}
\begin{assumption}\label{asm30}
    For any submatrix $\bX_{\mathcal{M}_k}$ of $\bX$ with dimension $n^*< n$, all the eigenvalues of $\bX_{\mathcal{M}_k}'\bX_{\mathcal{M}_k}$, denoted as $d_{\mathcal{M}_k,j}^2$ for $1\leq j\leq n^*$, are of order $n$.
\end{assumption}
\begin{assumption}\label{asm4}
    For $i\in \mathcal{M}_0$ defined in (\ref{submodel}), $\min_{i\in\mathcal{M}_0}|\beta_i|\geq C_1n^{-\tau}$ for some $0< \tau< n$, and the magnitude of the $i$-th projected coordinate $|(\bU_2\bU_2'\bbeta)_i|\leq C_2|\beta_i|$ for $C_2<1$,  and $\|\bbeta\|_2^2\leq C_3s^*$, where $s^*$ is the number of nonzero elements in $\bbeta$.
\end{assumption}
\begin{assumption}\label{asm5}
    Assume $\bve$ is a sub-Gaussian random variable in the sense that
    \[P(|\bv'\bve|\geq x)\leq C\exp(-x^2),\]
    for any $\|\bv\|_2^2=c_*>0$ which is a finite positive constant.
\end{assumption}
Assumptions~\ref{asm3}-\ref{asm30} are natural conditions about the orders of the singular values of $\bX\in R^{n\times p}$ and $\bX_{\mathcal{M}_k}$, and the penalty parameter $\lambda^*$ (or $\lambda$) is comparable to the magnitude of $d_j^2$ (or $d_{\mathcal{M}_k,j}^2$). Similar to the illustrations for Assumption~\ref{asm1}, the magnitude in Assumption~\ref{asm3}-\ref{asm30} can be easily verified if the entries of $\bX$  are independent copies of a random variable with zero mean, unit variance, and
finite fourth moment under the setting  $p/n\in(1,\infty)$; see \cite{bai1993limit}. In fact, the orders specified in Assumptions~\ref{asm3}-\ref{asm30} are only employed to establish the validity of the ridge-screening method in Theorem~\ref{thm5} below, and they are not the only ones capable of achieving this, so long as the rates can be  properly controlled in the proof of Theorem~\ref{thm5}. 
For any fixed $(p,n)$ which can be large,  the efficacy of the bias-correction method and the inference methods in Section~\ref{sec25} below remain valid so long as the nonzero singular values and the penalty parameters are strictly greater than zero.
Assumption~\ref{asm4} indicates that the minimum nonzero element in $\bbeta$ cannot be too small, and the norm of the $i$-th projected coordinate $(\bU_2\bU_2'\bbeta)_i$ is bounded by the  magnitude of its original coordinate $|\beta_i|$ (up to a small constant).  This is actually a reasonable and intuitive assumption.  For instance, in Assumption~\ref{asm4}, if $\|\bbeta\|_2\leq C\sqrt{s^*}$ (at most) because there are only $s^*$ nonzero elements in $\bbeta$, then $\|\bU_1\bU_1'\bbeta\|_2=O_p(\sqrt{s^*})$, but it is a $p$-dimensional vector, meaning that the magnitude of each $(\bU_1\bU_1'\bbeta)_i$ is of order $\sqrt{s^*/p}$ on average. We may postulate that $\sqrt{s^*/p}=o(n^{-\tau})$ in such a case and
Assumption~\ref{asm4} is even slightly weaker than this situation as we can allow that the projected coordinate to be of the same order as that of the original one.
Assumption~\ref{asm5} is a general condition that includes Gaussian distributions as a special case.

We have the following theorem regarding the de-biased estimator in (\ref{db:l}) after the ridge-screening.
\begin{theorem}\label{thm5} Let Assumptions \ref{asm1}-\ref{asm5} hold.\\
    (i) Assuming the true parameter $\bbeta$ belongs to the submodel $\mathcal{M}_0$ in (\ref{submodel}). If $\frac{\log(p)}{n^{1-2\tau}}\rightarrow 0$, the number of iterations $k$ in the first stage satisfies $\max_{1\leq j\leq p^*}s^*(\frac{\lambda^*}{d_j^2+\lambda^*})^{k+1}\rightarrow 0$, and $n^*$, the number of selected elements in (\ref{rs:sub}),  satisfies that $n^*\geq s^*$ \text{when}  $p<n$, and
    \[\frac{n^*}{Cn^{-2\tau}s^*+Cn^{2\tau-2}p\log(p)}\rightarrow\infty\,\,(\text{as}\,\, n,p\rightarrow\infty)
    \quad\text{when}\quad p>n,\]
    we have
    \[P(\mathcal{M}_0\subset \mathcal{M}_{k}(\lambda^*))\rightarrow 1,\,\text{as $k\rightarrow\infty$},\]
    where we use $\subset$ in the sense that $\mathcal{M}_{k}(\lambda^*)$ may contain more parameters than $\mathcal{M}_0$.\\
    (ii) Conditioning on the event of $\{\mathcal{M}_0\subset \mathcal{M}_{k}(\lambda^*)\}$, for a properly chosen $\lambda>0$, the bias of the de-biased and restricted ridge estimator $\wh\bbeta_{\mathcal{M}_k,l}(\lambda)$ is 
    \begin{equation}\label{bc:kl}
         \bb_{\lambda,k,l}=\bbeta_{\mathcal{M}_k}-E(\wh\bbeta_{\mathcal{M}_k,l}(\lambda))=\lambda^{l+1}(\bX_{\mathcal{M}_k}'\bX_{\mathcal{M}_k}+\lambda\bI_{n^*})^{-(l+1)}\bbeta_{\mathcal{M}_k},
    \end{equation}
     where $\bbeta_{\mathcal{M}_k}\in R^{n^*}$ is the true value of $\bbeta$ restricted on the submodel $\mathcal{M}_k$. That is,  $\bbeta_{\mathcal{M}_k}$ consists of the nonzero elements in $\bbeta$ and some zero ones associated with their original indexes in $\mathcal{M}_k\setminus \mathcal{M}_0$.
    If the number of iterations $l$ satisfies $\max_{1\leq j\leq n^*}s^*(\frac{\lambda}{d_{\mathcal{M}_k,j}^2+\lambda})^{l+1}\rightarrow 0$, we have
     \[\bb_{\lambda,k,l}\rightarrow {\bf 0},\,\, \text{as}\,\, k,l\rightarrow\infty.\]
\end{theorem}
\begin{remark}
  (i)   Theorem~\ref{thm5} implies that the ridge-screening method is applicable in both  $p<n$ and $p>n$ scenarios. Remarkably, there's no need to specify $s^*$, which is the number of nonzero elements in the true $\bbeta$. For $p<n$, we can simply set $p^*=p$ as Assumption~\ref{asm2} and choose $n^*=p$ variables which consists of all the covariates. For $p>n$, we  may choose $n^*\asymp n$ under the assumption that $s^*/n\rightarrow 0$ and $p\log(p)/n^{3-2\tau}\rightarrow 0$ in an asymptotic sense, which is reasonable because $s^*$ is often small and we can adopt $p/n=c\in (1,\infty)$ in the setting of random matrix theory. \\
  (ii) Theorem~\ref{thm5} establishes that our proposed method from Section~\ref{sec23} can completely correct the bias of the restricted ridge estimators, which is particularly helpful when $p>n$. This, in conjunction with the results of Theorem~\ref{thm2}, confirms the versatility of our bias-correction approaches in addressing both scenarios where $p<n$ and $p>n$.\\
  (iii)   For $p > n$, our discoveries from Theorems \ref{thm4}-\ref{thm5} underscore the significance of initially reducing all covariates to $n^*$ significant ones, where $n^*< \min(p,n)$. This reduction, combined with the proposed bias-correction procedure for the restricted ridge estimators, enables subsequent valid statistical inferences using the de-biased ridge estimators, as detailed in Section \ref{sec25} below.
\end{remark}

To conclude this subsection, we provide a summary of the pseudo-code for the proposed bias-correction procedures in Algorithm \ref{pless:a1} and Algorithm \ref{plarge:a2}, corresponding to the scenarios of $p < n$ and $p > n$, respectively. For the convergence criterion in Algorithm~\ref{pless:a1}, we can choose a small threshold $\eta>0$, say $\eta=10^{-2}$, and the convergence of the algorithm is determined by checking whether the following inequality hold:
\begin{equation}\label{con:cri}
    \|\wh\bbeta_{c,k}(\lambda)-\wh\bbeta_{c,k-1}(\lambda)\|_2\leq \eta.
\end{equation}
It's important to note that the convergence is guaranteed because the sequence $\bA_k:=\sum_{j=1}^k\lambda^j(\bX'\bX+\lambda\bI_p)^{-j}$ converges, as demonstrated in the proof of  Theorem~\ref{thm2} in the Appendix. Additionally, we emphasize that the number of iterations is typically not large, as only a logarithmic order of the dimension is required to ensure convergence, as discussed in Remark~\ref{rm2}.


Finally, we briefly discuss the methods for selecting the unknown parameters in Algorithms~\ref{pless:a1}-\ref{plarge:a2} of the proposed approaches. Firstly, the sole unknown parameter in Algorithm~\ref{pless:a1} is $\lambda$. Throughout this paper, we assume $\lambda$ to be given, as our focus is mainly on the bias-correction issue for ridge estimators. Empirically, one can employ the widely-accepted information criteria (e.g., AIC or BIC) or utilize cross-validation methods to determine an appropriate $\lambda$. For a comprehensive understanding, readers are referred to Section 1.8 of \cite{van2023lecture}. Secondly, in the ridge-screening method of Algorithm~\ref{plarge:a2} when $p>n$, the parameters $(\lambda^*,n^*)$ are unknown. Here, one can employ the aforementioned information criteria or cross-validation methods, as discussed in Sections 1.8.1-1.8.3 of \cite{van2023lecture}, to simultaneously determine $(\lambda^*,n^*)$. This can be achieved by selecting a candidate for 
$\lambda^*$ and then determining the optimal $n^*$ such that the pair 
$(\lambda^*,n^*)$ minimizes the information criterion or yields the best prediction performance on test sets using the restricted ridge estimator $\wh\bbeta_{\mathcal{M}_k}(\lambda^*)$. The final choice of 
$(\lambda^*,n^*)$ can be determined by evaluating the performance across each grid point of the penalty parameters. After identifying the significant variables through the RS method, the parameter $\lambda$ used in bias-correction is equal to 
$\lambda^*$, as 
$\lambda^*$ represents the optimal choice for the associated 
$n^*$. Given that these methods are well-established in the literature, we omit further details here to save space.






\begin{algorithm}[ht]
\caption{Iterative bias-correction of ridge estimators when $p<n$}\label{pless:a1}
{\bf Input:} Design matrix $\bX\in R^{n\times p}$, response vector $\by\in R^{n}$, and penalty $\lambda>0$;
\begin{algorithmic}[1]
\State{Construct a ridge estimator $\wh\bbeta(\lambda)$, set $k=1$;}
 \While{Not Convergent}
      \State  {form $\wh\bbeta_{c,k}(\lambda)=\wh\bbeta(\lambda)+\sum_{j=1}^k\lambda^j(\bX'\bX+\lambda\bI_p)^{-j}\wh\bbeta(\lambda)$;}
       \State{$k\gets k+1$;}
\EndWhile
\State{$k\gets k-1$;}
\State{END}
\end{algorithmic}
{\bf Output:} A de-biased ridge estimator $\wh\bbeta_{c,k}(\lambda)$.
\end{algorithm}


\begin{algorithm}[ht]
\caption{Ridge-screening and bias-correction of ridge estimators when $p> n$}\label{plarge:a2}
{\bf Input:} Design matrix $\bX\in R^{n\times p}$, response vector $\by\in R^{n}$, and penalty parameters $\lambda^*,\lambda>0$;
\begin{algorithmic}[1]
\State{Apply Algorithm~\ref{pless:a1} and obtain an initial de-biased ridge estimator $\wh\bbeta_{c,k}(\lambda^*)$, set $k_1\gets k$; }
 \State{Identify the indexes of the largest $n^*< \min(n,p)$ elements among $|\wh\bbeta_{c,k_1}(\lambda^*)|$;}
\State{Define $\bX_{\mathcal{M}_{k_1}}$ as a restricted design matrix of $\bX$ on the indexes found in Step 2;}
\State{Initialize the new design $\bX_{\mathcal{M}_{k_1}}\in R^{n\times n^*}$, response vector $\by\in R^n$, and penalty $\lambda>0$;}
\State{Apply Algorithm~\ref{pless:a1} again, set $l\gets k$, where $k$ is the one in Step 6 of Algorithm~\ref{pless:a1};}
\State{END}
\end{algorithmic}
{\bf Output:} A de-biased ridge estimator $\wh\bbeta_{c,k_1,l}(\lambda)$.
\end{algorithm}



  \subsection{Inference}\label{sec25}
In this section, we briefly introduce the inference method for the de-biased ridge estimators in Sections~\ref{sec23} and \ref{sec24}. The following theorem can be derived immediately based on the proofs of Theorem~\ref{thm2} and Theorem \ref{thm5}.


\begin{theorem}\label{thm6}Assume $\bve\sim N({\bf 0},\bSigma_\ve)$, where $\bSigma_\ve$ is a diagonal covaraince matrix.\\
(i) If $p<n$ and $\bX'\bX$ is invertible. Under Assumption~\ref{asm1}, we have
\[\wh\bbeta_{c,k}(\lambda)-\bbeta\sim_d N(\bmu_{1,k}(\lambda),\bSigma_{1,k}(\lambda)),\]
where $\sim_d$ denotes the exact distribution, 
$\bmu_{1,k}(\lambda)=-\lambda^{k+1}(\bX'\bX+\lambda\bI_p)^{-(k+1)}\bbeta$,
and
\[\bSigma_{1,k}(\lambda)=\sum_{j=0}^k \lambda^j(\bX'\bX+\lambda\bI_p)^{-(j+1)}\bX'\bSigma_\ve\bX\sum_{j=0}^k \lambda^j(\bX'\bX+\lambda\bI_p)^{-(j+1)}.\]
Furthermore, for any given  configuration of $(p,n)$ with $p<n$, we have
\[\sqrt{n}(\wh\bbeta_{c,k}(\lambda)-\bbeta)\longrightarrow_d N\left({\bf 0},\bSigma_{1}(\lambda)\right),\,\,\text{as}\quad k\rightarrow\infty,\]
where $\bSigma_1(\lambda)$ is the asymptotic limit of $n\bSigma_{1,k}(\lambda)$.\\
(ii) Under Assumptions~\ref{asm2}-\ref{asm5} and a true model in (\ref{submodel}) when $p>n$. Suppose $k$ is sufficiently large in the ridge-screening such that the event $\{\mathcal{M}_0\subset \mathcal{M}_{k}(\lambda^*)\}$ holds. Then the estimator in (\ref{db:l}) has the following limiting distribution
\[\wh\bbeta_{\mathcal{M}_k,l}(\lambda)-\bbeta_{\mathcal{M}_k}\sim_d N(\bmu_{2,k,l},\bSigma_{2,k,l}(\lambda)),\]
where $\bmu_{2,k,l}(\lambda)=-\lambda^{l+1}(\bX_{\mathcal{M}_k}'\bX_{\mathcal{M}_k}+\lambda\bI_{n^*})^{-(l+1)}\bbeta_{\mathcal{M}_k}$,
and
\[\bSigma_{2,k,l}(\lambda)=\sum_{j=0}^l \lambda^j(\bX_{\mathcal{M}_k}'\bX_{\mathcal{M}_k}+\lambda\bI_{n^*})^{-(j+1)}\bX_{\mathcal{M}_k}'\bSigma_\ve\bX_{\mathcal{M}_k}\sum_{j=0}^l \lambda^j(\bX_{\mathcal{M}_k}'\bX_{\mathcal{M}_k}+\lambda\bI_{n^*})^{-(j+1)}.\]
Furthermore, for any given  configuration of $(p,n)$ with $p> n$, we have
\[\sqrt{n}(\wh\bbeta_{\mathcal{M}_k,l}(\lambda)-\bbeta_{\mathcal{M}_k})\longrightarrow_d N\left({\bf 0},\bSigma_{2}(\lambda)\right),\,\,\text{as}\quad k,l\rightarrow\infty,\]
where $\bSigma_2(\lambda)$ is the asymptotic limit of $n\bSigma_{2,k,l}(\lambda)$.\\
\end{theorem}
\begin{remark}\label{rm5}
(i) The Assumption of $\bve\sim N({\bf 0},\bSigma_\ve)$ can be relaxed to $\{\ve_i,i=1,...,T\}$ is a martingale difference sequence with finite variances, and the results in Theorem~\ref{thm6} still hold if we make use of martingale central limit theorems in \cite{hall2014martingale} as $n\rightarrow\infty$ under Assumptions~\ref{asm3}-\ref{asm30}.\\
   (ii) Equation (\ref{bck:rp}) and Theorem~\ref{thm6} indicate that
\begin{align}\label{dis:1}
  \sqrt{n} [\wh\bbeta(\lambda)-\bbeta&+\sum_{j=1}^k\lambda^j(\bX'\bX+\lambda\bI_p)^{-j}\wh\bbeta(\lambda)]\notag\\
   &\overset{\sim}{\longrightarrow}_d N\left({\bf 0},n\sum_{j=0}^k \lambda^j(\bX'\bX+\lambda\bI_p)^{-(j+1)}\bX'\bSigma_\ve\bX\sum_{j=0}^k \lambda^j(\bX'\bX+\lambda\bI_p)^{-(j+1)}\right),
\end{align}    
and 
\begin{align}\label{dis:2}
   &\sqrt{n}[\wh\bbeta_{\mathcal{M}_k}(\lambda)-\bbeta_{\mathcal{M}_k}+\sum_{j=1}^l\lambda^j(\bX_{\mathcal{M}_k}'\bX_{\mathcal{M}_k}+\lambda\bI_{n^*})^{-j}\wh\bbeta_{\mathcal{M}_k}(\lambda)]\notag\\
   &\overset{\sim}{\longrightarrow}_d N\left({\bf 0},n\sum_{j=0}^k \lambda^j(\bX_{\mathcal{M}_k}'\bX_{\mathcal{M}_k}+\lambda\bI_{n^*})^{-(j+1)}\bX_{\mathcal{M}_k}'\bSigma_\ve\bX_{\mathcal{M}_k}\sum_{j=0}^k \lambda^j(\bX_{\mathcal{M}_k}'\bX_{\mathcal{M}_k}+\lambda\bI_{n^*})^{-(j+1)}\right),
\end{align}   
for $p<n$ and $p>n$, respectively, where $\overset{\sim}{\longrightarrow}_d$ denotes approximate equivalence in distribution for sufficiently large $k$  and a given configuration of $(p,n)$. Therefore, we can make use of the approximations in (\ref{dis:1}) and (\ref{dis:2}) to make statistical inference.\\
(iii) Under the assumption that $\lambda\asymp n$ and the nonzero singular values of $\bX$ are of order $\sqrt{n}$,  we can easily show that the variance terms in (\ref{dis:1}) and (\ref{dis:2}) are of order $O(1)$. Consequently,
\[\wh\bbeta_{c,k}(\lambda)-\bbeta=O_p(n^{-1/2})\,\,\text{and}\,\,\wh\bbeta_{\mathcal{M}_k,l}(\lambda)-\bbeta_{\mathcal{M}_k}=O_p(n^{-1/2}),\]
implying that our de-biased estimators are convergent with standard rate $\sqrt{n}$.\\
(iv) In practice,  it is often assumed that  the error term $\bve$ is homoskedastic with $\bSigma_\ve=\sigma^2\bI_n$, which simplifies the inference. 
Consequently, for sufficiently large $k$ and $l$, a consistent estimator for $\sigma$ can be obtained as:
\[\wh\sigma=\sqrt{\frac{1}{n}\|\by-\bX\wh\bbeta_{c,k}(\lambda)\|_2^2}\,\,\text{or}\,\,\wh\sigma=\sqrt{\frac{1}{n}\|\by-\bX_{\mathcal{M}_k}\wh\bbeta_{\mathcal{M}_k,l}(\lambda)\|_2^2},\]
depending on whether $p<n$ or $p>n$. Then, the variance terms in (\ref{dis:1}) and (\ref{dis:2}) can be estimated from the data.
\end{remark}
Hence, by correcting the bias, we can proceed to make statistical inferences and construct confidence intervals using distributions in (\ref{dis:1}) and (\ref{dis:2}) without the need to identify the sparse structure in Model (\ref{hlm}). This is possible because all biases can be approximated by the data, and the covariance terms in the limiting distributions are of full rank, and they can be estimated from the available data.
\subsection{Time Series Ridge Regression}

In this section, we briefly illustrate the feasibility of the proposed bias-correction method in time series ridge regression. In particular, we consider the following Auto-Regressive (AR) model:
\begin{equation}\label{yt:ts}
    y_t=\beta_1 y_{t-1}+...+\beta_p y_{t-p}+\ve_t=\bx_t'\bbeta+\ve_t,t=p+1,...,n,
\end{equation}
where $\bx_t=(y_{t-1},...,y_{t-p})'$ is the covariate vector consisting of the $p$ lagged variables of $y_t$, and $\bbeta=(\beta_1,...,\beta_p)'$ is the associated parameter vector. We restrict our consideration to cases where $p<n$. It's uncommon to encounter situations where the number of lagged regressors in autoregressive (AR) models exceeds the sample size. In practical datasets, instances of AR models with orders surpassing, for instance, $10$, are quite rare.

Let $\wt\by=(y_{p+1},...,y_n)'$, $\wt\bX=(\bx_{p+1},...,\bx_{n})'$, and $\wt\bve=(\ve_{p+1},...,\ve_{p})'$, then, Model (\ref{yt:ts}) can be expressed as
\begin{equation}
    \wt\by_t=\wt\bX\bbeta+\wt\bve,
\end{equation}
where we assume $\cov(\wt\bve)=\sigma^2\bI_{n-p}$ for simplicity. For a proper $\lambda>0$, the de-biased estimator in (\ref{bck:rp}) can be written as
\[\wh\bbeta_{c,k}(\lambda)
   =\bbeta-\lambda^{k+1}(\wt\bX'\wt\bX+\lambda\bI_p)^{-(k+1)}\bbeta+\sum_{j=0}^k\lambda^j(\wt\bX'\wt\bX+\lambda\bI_p)^{-(j+1)}\wt\bX'\wt\bve.\]
For weakly stationary time series sequence $\{y_t\}$, the ergodic theorem guarantees that
\[\frac{1}{n}\wt\bX'\wt\bX=\frac{1}{n}\sum_{t=p+1}^n\bx_t\bx_{t}'\rightarrow_p E(\bx_t\bx_t'),\,\,\text{as}\,\, n\rightarrow\infty.\]
Suppose the covariance matrix $E(\bx_t\bx_t')$ admits a similar spectral decomposition as $\bX'\bX/n$ in Assumption~\ref{asm1} above, and  the the magnitude of the penalty $\lambda$ satisfies $\lambda\asymp n$ as that in Assumption~\ref{asm3}, we can show that
\[\sum_{j=0}^k\lambda^{j+1}(\wt\bX'\wt\bX+\lambda\bI_p)^{-(j+1)}=O_p(1),\,\,\text{and}\quad\frac{1}{\lambda}\wt\bX'\wt\bve=O_p(\frac{1}{\sqrt{n}}).\]
For large $n$, it follows that
\begin{equation}\label{bc:ts}
    \wh\bbeta_{c,k}(\lambda)-\bbeta=o_p(1),\,\,\text{as}\quad k\rightarrow\infty,
\end{equation}
and 
\begin{equation}\label{rate:ts}
    \sqrt{n}(\wh\bbeta_{c,k}(\lambda)-\bbeta)=\sqrt{n}\sum_{j=0}^k\lambda^j(\wt\bX'\wt\bX+\lambda\bI_p)^{-(j+1)}\wt\bX'\wt\bve=O_p(1)\,\,\text{as}\quad k\rightarrow\infty.
\end{equation}
Equation (\ref{bc:ts}) suggests that the bias-correction procedure remains applicable to time series regression with weakly dependent regressors. Additionally, Equation (\ref{rate:ts}) confirms that the asymptotic distribution from Theorem~\ref{thm6}(i) holds true for time series data as well.

\subsection{Constructions of Confidence and Prediction Intervals}\label{sec27}
In this section, we explore methods for constructing confidence and prediction intervals based on ridge regression. For simplicity, we assume  $\bSigma_\ve=\sigma^2\bI_n$ to facilitate the illustrations.

We begin by outlining the construction of confidence intervals for the mean response  $E(y|\bx_0)=\bx_0'\bbeta$ with a given covariate $\bx_0$ when $p<n$. Note that $\wh y_0=\bx_0'\wh\bbeta(\lambda)$ in a ridge regression, then, by (\ref{dis:1}) in Remark~\ref{rm5}(ii) with a sufficiently large $k$, we have
\begin{equation}\label{y0:hat:ci}
    \bx_0'\wh\bbeta(\lambda)-\bx_0'\bbeta+\bx_0'\sum_{j=1}^k\lambda^j(\bX'\bX+\lambda\bI_p)^{-j}\wh\bbeta(\lambda)\overset{\sim}{\longrightarrow}_d N\left({\bf 0},\bx_0'\bSigma_{1,k}(\lambda)\bx_0\right),
\end{equation}
where 
\[\bSigma_{1,k}(\lambda)=\sigma^2\sum_{j=0}^k \lambda^j(\bX'\bX+\lambda\bI_p)^{-(j+1)}\bX'\bX\sum_{j=0}^k \lambda^j(\bX'\bX+\lambda\bI_p)^{-(j+1)}.\]
In practice, we may replace $\sigma$ in $\bSigma_{1,k}(\lambda)$ by $\wh\sigma$, which is estimated from the data as that in Remark \ref{rm5}(iv), and we denote the estimated variance in (\ref{y0:hat:ci}) by $\wh\bSigma_{1,k}(\lambda)$. Denote $z_{\alpha}$ the critical value of a standard normal distribution such that its tail probability is $\alpha$, then, the $(1-\alpha)$-confidence interval for $\bx_0'\bbeta$  is $[L_1,U_1]$, where
\begin{equation}\label{l1}
   L_1=\bx_0'\wh\bbeta(\lambda)+\bx_0'\sum_{j=1}^k\lambda^j(\bX'\bX+\lambda\bI_p)^{-j}\wh\bbeta(\lambda)-z_{\alpha/2}\sqrt{\bx_0'\wh\bSigma_{1,k}(\lambda)\bx_0},
\end{equation}
and
\begin{equation}\label{u1}
   U_1=\bx_0'\wh\bbeta(\lambda)+\bx_0'\sum_{j=1}^k\lambda^j(\bX'\bX+\lambda\bI_p)^{-j}\wh\bbeta(\lambda)+z_{\alpha/2}\sqrt{\bx_0'\wh\bSigma_{1,k}(\lambda)\bx_0}.
\end{equation}
Next, we discuss the way to construct prediction intervals for a future value $y_{n+1}$ with a new covariate $\bx_{n+1}$. According to Ch. 2 of \cite{montgomery2012introduction}, by a similar argument as that for the confidence intervals, we can construct the $(1-\alpha)$-prediction intervals of the future value $y_{n+1}$ as $[L_2,U_2]$, where
\begin{equation}\label{l2}
   L_2=\bx_0'\wh\bbeta(\lambda)+\bx_0'\sum_{j=1}^k\lambda^j(\bX'\bX+\lambda\bI_p)^{-j}\wh\bbeta(\lambda)-z_{\alpha/2}\sqrt{\bx_0'\wh\bSigma_{1,k}(\lambda)\bx_0+\wh\sigma^2},
\end{equation}
and
\begin{equation}\label{u2}
   U_2=\bx_0'\wh\bbeta(\lambda)+\bx_0'\sum_{j=1}^k\lambda^j(\bX'\bX+\lambda\bI_p)^{-j}\wh\bbeta(\lambda)+z_{\alpha/2}\sqrt{\bx_0'\wh\bSigma_{1,k}(\lambda)\bx_0+\wh\sigma^2}.
\end{equation}
The values presented in (\ref{l1})-(\ref{u2}) are directly estimated from the data. Consequently, we are equipped to construct valid confidence and prediction intervals in ridge regression employing our proposed bias-correction techniques. Analogously, confidence and prediction intervals can also be established for models utilizing de-biased ridge estimators post ridge-screening when $p>n$. For instance, the 
 $(1-\alpha)$-prediction interval of the future value $y_{n+1}$  produced by the restricted ridge estimator is $[L_3,U_3]$ with
\begin{equation}\label{l3}
   L_3=\bx_{0,k}'\wh\bbeta_{\mathcal{M}_k}(\lambda)+\bx_{0,k}'\sum_{j=1}^l\lambda^j(\bX_{\mathcal{M}_k}'\bX_{\mathcal{M}_k}+\lambda\bI_{n^*})^{-j}\wh\bbeta_{\mathcal{M}_k}(\lambda)-z_{\alpha/2}\sqrt{\bx_{0,k}'\wh\bSigma_{2,k,l}(\lambda)\bx_{0,k}+\wh\sigma^2},
\end{equation}
and
\begin{equation}\label{u3}
U_3=\bx_{0,k}'\wh\bbeta_{\mathcal{M}_k}(\lambda)+\bx_{0,k}'\sum_{j=1}^l\lambda^j(\bX_{\mathcal{M}_k}'\bX_{\mathcal{M}_k}+\lambda\bI_{n^*})^{-j}\wh\bbeta_{\mathcal{M}_k}(\lambda)+z_{\alpha/2}\sqrt{\bx_{0,k}'\wh\bSigma_{2,k,l}(\lambda)\bx_{0,k}+\wh\sigma^2},
\end{equation}
 where $\bx_{0,k}$ is the $n^*$-dimensional sub-vector of the new covariate $\bx_0$ selected by the ridge-screening method.

\subsection{Bias-Variance Trade-off}
The bias-variance trade-off is a fundamental concept in machine learning and statistics. It refers to the delicate balance between two sources of error in a predictive model. It is a way of analyzing a learning algorithm's expected errors. Therefore, it would be interesting if we can derive the bias-variance trade-off of our de-biased estimators as we increase the number of iterations.

In this section, we study the MSE of the de-biased ridge estimator $\wh\bbeta_{c,k}(\lambda)$ in the $k$-th iteration  for $p<n$ with a given $\lambda>0$. The analysis of the scenario where $p>n$ is similar, and hence, we only derive it for the case when $p<n$. The MSE of the $k$-th de-biased ridge estimator is defined as 
\begin{align}\label{mse:bt}
    \text{MSE}(\wh\bbeta_{c,k}(\lambda))=&E[\wh\bbeta_{c,k}(\lambda)-\bbeta]'[\wh\bbeta_{c,k}(\lambda)-\bbeta]\notag\\
    =&E[\wh\bbeta_{c,k}(\lambda)-E\wh\bbeta_{c,k}(\lambda)]'[\wh\bbeta_{c,k}(\lambda)-E\wh\bbeta_{c,k}(\lambda)]+[E\wh\bbeta_{c,k}(\lambda)-\bbeta]'[E\wh\bbeta_{c,k}(\lambda)-\bbeta]\notag\\
    =&\var(\wh\bbeta_{c,k}(\lambda))+\text{bias}(\wh\bbeta_{c,k}(\lambda))^2.
\end{align}
For simplicity, we assume $\bSigma_\ve=\sigma^2\bI_n$. By Assumption~\ref{asm1} and Theorem~\ref{thm6}(i),
\begin{align}\label{var:btc}
    \var({\wh\bbeta_{c,k}(\lambda)})=&E[\bve'\bX\sum_{j=0}^k\lambda^j(\bX'\bX+\lambda\bI_p)^{-(k+1)}\sum_{j=0}^k\lambda^j(\bX'\bX+\lambda\bI_p)^{-(k+1)}\bX'\bve]\notag\\
    =&E[\bve'\bV_1\bD_{k}(\lambda)\bV_1'\bve]=\sigma^2\tr[\bD_{k}(\lambda)],
\end{align}
where $\bD_{k}(\lambda)=\diag(d_{k,1}(\lambda),...,d_{k,p}(\lambda))$ with
\[d_{k,i}(\lambda)=\frac{1}{d_i^2}\left[1-(\frac{\lambda}{\lambda+d_i^2})^{k+1}\right]^2.\]
By a similar argument, we can show that
\begin{align}\label{bias:sq}
 \text{bias}(\wh\bbeta_{c,k}(\lambda))^2=&  \bbeta'\lambda^{k+1}(\bX'\bX+\lambda\bI_p)^{-(k+1)}\lambda^{k+1}(\bX'\bX+\lambda\bI_p)^{-(k+1)}\bbeta\notag\\
 =&\bbeta'\bU_1\bLambda_{k}(\lambda)\bU_1'\bbeta,
\end{align}
where $\bLambda_{k}(\lambda)=\diag(\gamma_{k,1}(\lambda),...,\gamma_{k,p}(\lambda))$ with
\[\gamma_{k,i}(\lambda)=[\frac{\lambda}{\lambda+d_i^2}]^{2(k+1)}.\]
As the number of iterations $k$ increases, we can readily see that the bias term in (\ref{bias:sq}) decreases while the variance term in (\ref{var:btc}) increases. Therefore, it is interesting to see whether there is a bias-variance trade-off in the proposed bias-correction procedure. 

In the following theorem, we provide some sufficient conditions under which we have a bias-variance trade-off in the proposed method.
\begin{theorem}\label{thm7}
Let Assumption~\ref{asm1} hold and $\bU_1'\bbeta=(\delta_1,...,\delta_p)'$ where $p<n$.\\
    (i) If $\frac{\delta_i^2d_i^2}{\sigma^2}<1$ and $\frac{\delta_i^2d_i^2}{\sigma^2}+(\frac{\lambda}{\lambda+d_i^2})^{k^*+1}\geq 1$ for $1\leq i\leq p$ and some $k^*\geq 1$, then, as the number of iterations $k$ increases, the MSE of $\wh\bbeta_{c,k}(\lambda)$ will initially decrease to its minimum value and subsequently rise to a stable level. In particular, the minimum of the MSE can be achieved at the $k$-th iteration for some $k\in [\lfloor k_1 \rfloor,\lfloor k_2\rfloor]$, where
    \[k_1=\min\left\{\frac{\log(1-\frac{\delta_i^2d_i^2}{\sigma^2})}{\log(\frac{\lambda}{\lambda+d_i^2})}-1:1\leq i\leq p\right\},k_2=\max\left\{\frac{\log(1-\frac{\delta_i^2d_i^2}{\sigma^2})}{\log(\frac{\lambda}{\lambda+d_i^2})}-1:1\leq i\leq p\right\},\]
and $\lfloor x\rfloor$ is the largest integer that does not exceed $x$.\\
    (ii) If $\frac{\delta_i^2d_i^2}{\sigma^2}>1$ for $1\leq i\leq p$, then, as the number of iterations $k$ increases, the MSE of $\wh\bbeta_{c,k}(\lambda)$ will decrease to a stable level.
\end{theorem}
\begin{remark}
    (i) Theorem~\ref{thm7} indicates there exists a bias-variance trade-off in the proposed bias-correction procedure under certain conditions. That is, if the design matrix and the penalty meet the criteria outlined in Theorem~\ref{thm7}(i), the MSE of the de-biased estimators can attain a minimum value by balancing the bias and variance terms as the number of iterations $k$, or equivalently, the "model complexity", increases.\\
    (ii) Theorem~\ref{thm7}(ii) suggests the possibility of a monotonically decreasing MSE as the number of iterations $k$ increases. This phenomenon arises because the increase in the variance term is not comparable to the decrease in the bias term. Consequently, this finding is intriguing as it indicates the potential for a de-biased estimator to achieve an even smaller MSE than classical ridge estimators.\\
    (iii) We highlight that the conditions specified in Theorem \ref{thm7}(ii) can be easily satisfied under certain scenarios. For instance, assuming $d_i^2\asymp {n}$ as detailed in Assumption~\ref{asm3}, and $\delta_i^2\asymp O(p^{-1})$ or $O(1)$ depending on whether $\bbeta$ is a sparse or dense vector with bounded elements (given that each entry in $\bU_1$ is of order $p^{-1/2}$, it becomes evident that $\delta_i^2d_i^2/\sigma^2$ is at least of order $O(n/p)$, which can exceed $1$. Similarly, the condition in Theorem~\ref{thm7}(i) can also be satisfied if $d_i$ is of a smaller rate. We omit the details to save space. 
The simulation results presented in Section~\ref{sec3} further corroborate our findings.
    
\end{remark}
From Theorem 2 in \cite{theobald1974generalizations}, the MSE of the classical ridge estimator can be smaller than that of the least-squares estimator for certain properly chosen $\lambda>0$. In our framework, we observe a similar paradigm concerning the number of iterations $k>0$, as described in the following corollary
\begin{corollary}
Under the conditions outlined in Theorem~\ref{thm7}, there exists an iteration number 
$k$ such that
    \[\text{MSE}(\wh\bbeta_{c,k}(\lambda))<\text{MSE}(\wh\bbeta_{c,0}(\lambda))=\text{MSE}(\wh\bbeta(\lambda)),\]
     suggesting that the proposed de-biased estimators can further minimize the MSEs while also correcting a portion of the bias compared to classical ridge estimators.
\end{corollary}



  \section{Monte Carlo Simulations }\label{sec3}
In this section, we conduct Monte-Carlo experiments to illustrate the proposed procedure. We consider two scenarios where $p<n$ and $p>n$ and study the effect of the bias-correction procedure and the approximation of the asymptotic normalities established in Section~\ref{sec25}.\\

{\noindent\bf Example 1.} In this example, we consider the data-generating process in (\ref{v:hlm}) for different settings of $(p,n)$ with $p<n$. For each configuration of  $(p,n)$, we set the seed number in \texttt{R} software as (1234) and first generate an $n\times p$ matrix $\bH$, where its elements are drawn independently from a Uniform distribution $U(-2,2)$. We then perform a singular-value decomposition on $\bH$ and obtain its left and right singular vectors $\bM\in R^{n\times p}$ and $\bN\in R^{p\times p}$, then we let $\bX=\bM\bN'$. The first $p/2$ elements of $\bbeta$ are generated from $U(-2,-1)$ independently, and the remaining $p/2$ elements are from $U(1,2)$. In each replication, the noise $\bve$ is generated from multivariate normal distribution $N({\bf 0},\bI_n)$. We consider the scenarios where $(p,n)=(50,100)$, $(50,400)$, $(100,200)$, and $(100,500)$. The choices of $\lambda$ are $\lambda=0.05n$, $0.1n$, $0.3n$, and $0.5n$. 1000 replications are used in each setting throughout the experiments.

In Table~\ref{Table-a1}, we report the estimation errors of the ridge estimators and the de-biased ones. For each configuration of $(p,n,\lambda)$, we define the empirical mean squared  errors (MSEs) as 
\begin{equation}\label{sse0}
    MSE(\bb_\lambda)=\frac{1}{1000}\sum_{j=1}^{1000}\|\wh\bbeta^{(j)}(\lambda)-\bbeta\|_2^2
\end{equation}
and
\begin{equation}\label{ssek}
      MSE(\bb_{\lambda,k})=\frac{1}{1000}\sum_{j=1}^{1000}\|\wh\bbeta_{c,k}^{(j)}(\lambda)-\bbeta\|_2^2,  
\end{equation}
where $\wh\bbeta^{(j)}(\lambda)$ and $\wh\bbeta_{c,k}^{(j)}(\lambda)$ are the ridge estimators and the de-biased one with $k$ iterations in the $j$-th replication, respectively. The standard errors $\wh\sigma_0$ and $\wh\sigma_{100}$
are estimated via
\begin{equation}\label{sigma:e}
    \wh\sigma_0=\sqrt{\frac{1}{1000}\sum_{j=1}^{1000}\frac{1}{n}\|\by-\bX\wh\bbeta^{(j)}(\lambda)\|_2^2}\,\,\text{and}\,\, \wh\sigma_{100}=\sqrt{\frac{1}{1000}\sum_{j=1}^{1000}\frac{1}{n}\|\by-\bX\wh\bbeta_{c,100}^{(j)}(\lambda)\|_2^2}.
\end{equation}
In other words, we evaluate the estimations of the standard errors by the ridge estimators and the de-biased ones.

For each setting in Table~\ref{Table-a1}, we see that the MSE first decreases as the number of iterations increases, and then increases as we use more iterations to correction the bias terms. This is understandable because it is in line with the bias-variance trade-off in the machine learning literature. See, for example, \cite{hastie2009elements}. Specifically, we plot the MSEs for $(p,n)=(50,100)$ and $(100,200)$ with $\lambda=0.05n$ in Figure~\ref{fig-0}, where we can clearly see that the MSEs can achieve a minimum point as we increase the number of iterations, or equivalently, the model complexity, and the MSE will further increase as we try to completely correct the biases. Finally, the MSE becomes stable, which is consistent with our asymptotic theory. Moreover, the standard error estimated using the debiased ridge estimator is closer to the true one (unity), showing the effectiveness of our bias-correction procedure.


\begin{table}[htp]
\caption{ Empirical mean squared errors (MSEs) when $p<n$ in Example 1, where the MSEs are defined in (\ref{sse0}) and (\ref{ssek}) for $\bb_{\lambda}$ and $\bb_{\lambda,k}$, respectively. The number of iterations $k=1,5,10,20,50,100$, and $\wh\sigma_0$ and $\wh\sigma_{100}$ are estimated by (\ref{sigma:e}). 1000 replications are used in the experiments.} 
          \label{Table-a1}
{\begin{center}
\begin{tabular}{cccccccccc}
\toprule
&\multicolumn{9}{c}{$\lambda=0.05n$}\\
\cline{2-10}
$(p,n)$&$\bb_{\lambda}$&$\bb_{\lambda,1}$&$\bb_{\lambda,5}$&$\bb_{\lambda,10}$&$\bb_{\lambda,20}$&$\bb_{\lambda,50}$&$\bb_{\lambda,100}$&$\wh\sigma_0$&$\wh\sigma_{100}$\\
\hline
(50,100)&78.8&58.3&34.3&39.2&47.7&49.8&49.9&1.29&1.19\\
(50,400)&105.3&95.9&67.9&48.4&35.6&42.9&49.3&1.11&1.06\\
(100,200)&192.4&161.3&92.8&70.6&79.0&98.5&100.0&1.47&1.23\\
(100,500)&217.4&201.4&151.2&111.4&76.7&79.0&96.3&1.15&1.05\\
\midrule
&\multicolumn{9}{c}{$\lambda=0.1n$}\\
\cline{2-10}
$(p,n)$&$\bb_{\lambda}$&$\bb_{\lambda,1}$&$\bb_{\lambda,5}$&$\bb_{\lambda,10}$&$\bb_{\lambda,20}$&$\bb_{\lambda,50}$&$\bb_{\lambda,100}$&$\wh\sigma_0$&$\wh\sigma_{100}$\\
\hline
(50,100)&92.6&77.6&44.8&34.5&39.2&49.1&49.8&1.34&1.19\\
(50,400)&110.4&105.2&87.2&70.3&49.4&35.1&42.9&1.12&1.05\\
(100,200)&210.5&191.5&135.5&96.4&70.9&85.7&98.5&1.50&1.23\\
(100,500)&210.4&217.2&186.5&155.8&113.8&71.6&79.0&1.16&1.03\\
\midrule
&\multicolumn{9}{c}{$\lambda=0.3n$}\\
\cline{2-10}
$(p,n)$&$\bb_{\lambda}$&$\bb_{\lambda,1}$&$\bb_{\lambda,5}$&$\bb_{\lambda,10}$&$\bb_{\lambda,20}$&$\bb_{\lambda,50}$&$\bb_{\lambda,100}$&$\wh\sigma_0$&$\wh\sigma_{100}$\\
\hline
(50,100)&104.6&98.1&76.8&58.6&40.2&36.7&46.4&1.39&1.18\\
(50,400)&114.1&112.2&105.1&97.1&83.2&55.8&37.9&1.12&1.04\\
(100,200)&224.3&217.1&191.0&163.9&124.3&75.4&74.1&1.52&1.20\\
(100,500)&231.9&228.8&217.1&203.5&179.5&127.6&85.3&1.16&1.03\\
\midrule
&\multicolumn{9}{c}{$\lambda=0.5n$}\\
\cline{2-10}
$(p,n)$&$\bb_{\lambda}$&$\bb_{\lambda,1}$&$\bb_{\lambda,5}$&$\bb_{\lambda,10}$&$\bb_{\lambda,20}$&$\bb_{\lambda,50}$&$\bb_{\lambda,100}$&$\wh\sigma_0$&$\wh\sigma_{100}$\\
\hline
(50,100)&107.3&103.2&88.6&74.0&54.1&34.7&39.2&1.40&1.14\\
(50,400)&114.9&113.7&109.3&104.1&94.6&72.3&50.3&1.12&1.05\\
(100,200)&227.2&222.8&206.0&187.3&156.1&99.8&71.2&1.53&1.20\\
(100,500)&233.1&231.3&224.1&215.4&199.4&159.8&115.9&1.16&1.06\\
\bottomrule
\end{tabular}
  \end{center}}
\end{table}

\begin{figure}[ht]
\begin{center}
{\includegraphics[width=14cm,height=8cm]{bias-variance-TO.pdf}}
\caption{  The bias-variance trade-off reflected by the empirical MSEs of the de-biased ridge estimators for different number of iterations in Example 1, where we consider $\lambda=0.05n$ for $(p,n)=(50,100)$ and $(100,200)$ in (a) and (b), respectively.  1000 replications are used in the experiments.  }\label{fig-0}
\end{center}
\end{figure}



Furthermore, we study the consistency of the de-biased estimators using our proposed method. For simplicity, we investigate the averaging estimation errors (AEEs) of the ridge estimators and the de-biased ones, which are defined, respectively, as
\begin{equation}\label{aee0}
    AEE(\bb_\lambda)=\frac{1}{\sqrt{p}}\|\frac{1}{1000}\sum_{j=1}^{1000}\wh\bbeta^{(j)}(\lambda)-\bbeta\|_2
\end{equation}
and
\begin{equation}\label{aeek}
    AEE(\bb_{\lambda,k})=\frac{1}{\sqrt{p}}\|\frac{1}{1000}\sum_{j=1}^{1000}\wh\bbeta_{c,k}^{(j)}(\lambda)-\bbeta\|_2,
\end{equation}
where the ones in the $\ell_2$-norm are the empirical versions of the biases. The AEEs are reported in Table~\ref{Table-a2}.
From Table~\ref{Table-a2}, we can see a decreasing pattern for each $(p,n)$ as the number of iterations increases, implying that our de-biased estimators are consistent to the true parameters for sufficiently large $k$.







\begin{table}[htp]
\caption{Empirical averaging estimation errors (AEEs) when $p<n$ in Example 1, where the AEEs are defined in (\ref{aee0}) and (\ref{aeek}) for $\bb_{\lambda}$ and $\bb_{\lambda,k}$, respectively. The number of iterations $k=1,5,10,20,50$, and $100$. 1000 replications are used in the experiments.} 
          \label{Table-a2}
{\begin{center}
\begin{tabular}{cccccccc}
\toprule
&\multicolumn{7}{c}{$\lambda=0.05n$}\\
\cline{2-8}
$(p,n)$&$\bb_{\lambda}$&$\bb_{\lambda,1}$&$\bb_{\lambda,5}$&$\bb_{\lambda,10}$&$\bb_{\lambda,20}$&$\bb_{\lambda,50}$&$\bb_{\lambda,100}$\\
\hline
(50,100)&1.24&1.01&0.50&0.20&0.04&0.03&0.03\\
(50,400)&1.45&1.38&1.14&0.89&0.55&0.13&0.03\\
(100,200)&1.38&1.26&0.86&0.53&0.21&0.03&0.03\\
(100,500)&1.47&1.42&1.21&1.00&0.67&0.21&0.04\\
\midrule
&\multicolumn{7}{c}{$\lambda=0.1n$}\\
\cline{2-8}
$(p,n)$&$\bb_{\lambda}$&$\bb_{\lambda,1}$&$\bb_{\lambda,5}$&$\bb_{\lambda,10}$&$\bb_{\lambda,20}$&$\bb_{\lambda,50}$&$\bb_{\lambda,100}$\\
\hline
(50,100)&1.36&1.23&0.84&0.52&0.20&0.03&0.03\\
(50,400)&1.49&1.45&1.31&1.16&0.91&0.44&0.13\\
(100,200)&1.45&1.38&1.14&0.89&0.55&0.13&0.03\\
(100,500)&1.50&1.47&1.36&1.23&1.01&0.56&0.21\\
\midrule
&\multicolumn{7}{c}{$\lambda=0.3n$}\\
\cline{2-8}
$(p,n)$&$\bb_{\lambda}$&$\bb_{\lambda,1}$&$\bb_{\lambda,5}$&$\bb_{\lambda,10}$&$\bb_{\lambda,20}$&$\bb_{\lambda,50}$&$\bb_{\lambda,100}$\\
\hline
(50,100)&1.45&1.40&1.23&1.04&0.75&0.28&0.05\\
(50,400)&1.51&1.50&1.45&1.39&1.28&1.00&0.66\\
(100,200)&1.50&1.47&1.38&1.27&1.08&0.66&0.29\\
(100,500)&1.52&1.51&1.47&1.42&1.33&1.09&0.78\\
\midrule
&\multicolumn{7}{c}{$\lambda=0.5n$}\\
\cline{2-8}
$(p,n)$&$\bb_{\lambda}$&$\bb_{\lambda,1}$&$\bb_{\lambda,5}$&$\bb_{\lambda,10}$&$\bb_{\lambda,20}$&$\bb_{\lambda,50}$&$\bb_{\lambda,100}$\\
\hline
(50,100)&1.47&1.44&1.33&1.20&0.98&0.54&0.20\\
(50,400)&1.52&1.51&1.48&1.44&1.37&1.18&0.92\\
(100,200)&1.51&1.49&1.43&1.36&1.24&0.92&0.56\\
(100,500)&1.53&1.52&1.49&1.47&1.41&1.25&1.02\\
\bottomrule
\end{tabular}
  \end{center}}
\end{table}

Finally, we investigate the performance of the inference approach based on the de-biased estimators in Theorem~\ref{thm6} for $p<n$. For simplicity, we only consider the case when $(p,n)=(100,200)$ and $\lambda=0.3n$, and we can produce similar results for other settings. Let $\be_i$ be the $i$-th standard unit vector of $R^p$, 
\[\btheta_1=(0.8,-1,0.5,{\bf 0}_{p-3}')',\,\,\text{and}\,\,\btheta_2=(-1,0.5,0.8,{\bf 0}_{p-3}')',\]
where ${\bf 0}_s$ is an $s$-dimensional  vector of zeros. Figure~\ref{fig-1} presents the histograms of $\sqrt{n}\be_1'(\wh\bbeta(\lambda)-\bbeta)$, $\sqrt{n}\be_2'(\wh\bbeta(\lambda)-\bbeta)$, $\sqrt{n}\btheta_1'(\wh\bbeta(\lambda)-\bbeta)$, and $\sqrt{n}\btheta_2'(\wh\bbeta(\lambda)-\bbeta)$ obtained from 1000 experiments. From Figure~\ref{fig-1}, we see that the biases of the empirical means in all the histogram plots are significantly large and diverges from zero, implying that the traditional ridge estimators are not appropriate to make statistical inferences without the information of the bias terms. In contrast, we plot the empirical histograms of the de-biased counterparts in Figure~\ref{fig-2} under the same setting.  From Figure~\ref{fig-2}, we see that the finite sample performance is quite satisfactory, and the bias effect of the de-biased estimators is very small. In addition, the curve of the normal distribution in Theorem~\ref{thm6}(i) is added to the corresponding histograms, where the standard error term $\sigma$ is estimated from the data as that in Remark~\ref{rm5}(iv) and Table~\ref{Table-a1}. From these curves, we can further confirm that our inference method is valid for the de-biased ridge estimators, indicating that valid inferences based on the  ridge estimators with our bias-correction method can be made in practice.\\



\begin{figure}[ht]
\begin{center}
{\includegraphics[width=14cm,height=8cm]{ridge-hist-p100n200lbd0.3.pdf}}
\caption{  Empirical histograms of (a) $\sqrt{n}\be_1'(\wh\bbeta(\lambda)-\bbeta)$; (b)  $\sqrt{n}\be_2'(\wh\bbeta(\lambda)-\bbeta)$; (c) $\sqrt{n}\btheta_1'(\wh\bbeta(\lambda)-\bbeta)$; and (d) $\sqrt{n}\btheta_2'(\wh\bbeta(\lambda)-\bbeta)$ in Example 1, where $(p,n)=(100,200)$ and $\lambda=0.3n$. 1000 replications are used in the experiments.  }\label{fig-1}
\end{center}
\end{figure}

\begin{figure}[ht]
\begin{center}
{\includegraphics[width=14cm,height=8cm]{debiased-p100n200lbd0.3.pdf}}
\caption{Empirical histograms of (a) $\sqrt{n}\be_1'(\wh\bbeta_{c,k}(\lambda)-\bbeta)$; (b)  $\sqrt{n}\be_2'(\wh\bbeta_{c,k}(\lambda)-\bbeta)$; (c) $\sqrt{n}\btheta_1'(\wh\bbeta_{c,k}(\lambda)-\bbeta)$; and (d) $\sqrt{n}\btheta_2'(\wh\bbeta_{c,k}(\lambda)-\bbeta)$ in Example 1. We choose 
$k=120$ for the de-biased estimators with $(p,n)=(100,200)$ and $\lambda=0.3n$. A density curve is plotted based on the limiting distribution outlined in Theorem~\ref{thm6}(i).  1000 replications are used in the experiments. }\label{fig-2}
\end{center}
\end{figure}


{\noindent\bf Example 2.} In this example, we consider the scenario where $p>n$. The seed number used in this example is the same as that in Example 1. Each row of the design matrix $\bX$ is generated independently from multivariate standard normal distribution $N({\bf 0},\bI_p)$. For each configuration of $(p,n)$, the first $5$ elements in $\bbeta$ are generated from $U(-5,-2)$,  the $6$th to the $10$th element are from $U(2,5)$, and the remaining ones are zeros. In other words, we consider the sparse vector $\bbeta$ with $p^*=10$ in (\ref{submodel}). The dimensions and the sample sizes are $(p,n)=(150,120), (150,140), (220,180)$, and $(220,200)$, and we consider $\lambda=0.1n$, $0.3n$, and $0.8n$ in each replication for any given $(p,n)$. For simplicity, we choose $n^*=40$ in the ridge screening of (\ref{rs:sub}) with $k=100$ iterations. $1000$ replications are used for each configuration of $(p,n)$ throughout the experiments.



In Table~\ref{Table-a3}, we report the empirical MSEs of the ridge estimators and the de-biased ones before applying the ridge-screening method, and those after the ridge-screening with $k=100$ iterations in the first stage of bias-corrections.
Specifically, the upper panel in Table~\ref{Table-a3} for each $\lambda$ presents 
the MSEs of the ridge estimators and the de-biased ones using 100 iterations without applying the ridge-screening approach. We see that the MSEs of the de-biased estimators become quite stable as we increase the number of iterations, which is consistent with our findings that there is a term that cannot be corrected from the data. In addition, the estimated standard errors are not close to unity, which is the true one in the experiments. In the lower panel of Table~\ref{Table-a3} for each $\lambda$, we apply the ridge-screening approach with $k=100$ and sort out the largest $n^*=40$ components of $\wh\bbeta_{c,100}(\lambda)$, and then obtain the restricted ridge estimators. We further apply the bias-correction approach and the de-biased estimators are obtained in another $100$ iterations. From the lower panels of Table~\ref{Table-a3} for each $\lambda$, we see that the bias of the de-biased estimators decreases sharply to a relatively stable value for each configuration of $(p,n)$, which is understandable since the selected model consists of more parameters than those in the true one. In addition, the estimation errors are significantly smaller than those without the ridge-screening approach, and the estimated standard errors after bias correction are closer to the true one than the methods without using the ridge-screening and the bias-correction approaches. We also note that the estimated standard errors are still larger than the true one (unity) when $\lambda=0.8n$ after the ridge-screening, but they are much closer to the true ones than all the ones produced by the original ridge estimators. In addition, this can be optimized by choosing more appropriate tuning parameter $\lambda$ (e.g. $\lambda=0.1n$ or $0.3n$) such that the estimated standard errors are close to one. Furthermore, we also note that there is also a bias-variance trade-off in the MSEs of the de-biased estimators with or without the ridge-screening. Figure~\ref{fig-20} plots the the empirical MSEs of the de-biased ridge estimators for different number of iterations for $\lambda=0.5n$ and $(p,n)=(150,120)$, where (a) plots the MSEs of the de-baised ridge estimators  before applying the ridge-screening method, and (b) provides the MSEs of the de-biased  ridge estimators after the ridge-screening.  We can clearly see that there is an optimal number of iteration that minimizes the MSE, which is in line with our asymptotic results in Theorem~\ref{thm7}.

We also study the performance of the proposed ridge-screening method in Table~\ref{Table-a30}, where the empirical probability (EP) is calculated by
\begin{equation}
 EP(\mathcal{M}_0\subset \mathcal{M}_{k}(\lambda^*))=\frac{1}{1000}\sum_{i=1}^{1000}\frac{|\mathcal{M}_0\cap \mathcal{M}_{k}^i(\lambda^*)|}{10},
\end{equation}
where $\mathcal{M}_{k}^i(\lambda^*)$ is the recovery in the $i$-th experiment and $|\mathcal{M}_0\cap \mathcal{M}_{k}^i(\lambda^*)|$ is the cardinality of the set $\mathcal{M}_0\cap \mathcal{M}_{k}^i(\lambda^*)$. We can see from Table~\ref{Table-a30} that our method provides satisfactory performance in variable selections{\footnote{The comparison results with other variable selection approaches, such as the LASSO, are not provided in this experiment due to the empirical probability of correct recoveries by the proposed RS method being $100\





\begin{table}[htp]\scriptsize
\caption{Empirical mean squared  errors (MSEs) when $p> n$ in Example 2, where the MSEs are similarly defined as those in (\ref{sse0}) and (\ref{ssek}) for $\bb_{\lambda}$, $\bb_{\lambda,k}$, and $\bb_{\lambda,k,l}$. For each $\lambda^*=\lambda$, the upper panel reports the MSEs before ridge-screening, and the lower one presents the MSEs after ridge-screening with $k=100$ and $n^*=40$ in the variable selection. The number of iterations is set to $k=1,5,10,20,50$, and $100$ for the first-stage de-biased estimation, and  $l=1,5,10,20,50,$ and $100$ for the second-stage bias-correction following the ridge-screening.  $\wh\sigma_0$, $\wh\sigma_{100}$, $\wh\sigma_{k,0}$ and $\wh\sigma_{k,100}$ are similarly estimated by the method in (\ref{sigma:e}). 1000 replications are used in the experiments.} 
          \label{Table-a3}
{\begin{center}
\begin{tabular}{cccccccccc}
\toprule
&\multicolumn{9}{c}{$\lambda^*=0.1n$ (before ridge-screening)}\\
\cline{2-10}
$(p,n)$&$\bb_{\lambda}$&$\bb_{\lambda,1}$&$\bb_{\lambda,5}$&$\bb_{\lambda,10}$&$\bb_{\lambda,20}$&$\bb_{\lambda,50}$&$\bb_{\lambda,100}$&$\wh\sigma_0$&$\wh\sigma_{100}$\\
\hline
(150,120)&30.64&26.70&28.64&28.73&28.73&28.73&28.73&1.66&2.14\\
(150,140)&14.84&10.87&11.67&11.68&11.68&11.68&11.68&2.00&2.07\\
(220,180)&50.10&45.30&46.08&46.12&46.12&46.12&46.12&1.75&2.24\\
(220,200)&30.90&24.09&25.02&25.04&25.04&25.04&25.04&1.82&2.03\\
\midrule
&\multicolumn{9}{c}{$\lambda=0.1n$, $k=100$ (after ridge-screening)}\\
\cline{2-10}
$(p,n)$&$\bb_{\lambda,k}$&$\bb_{\lambda,k,1}$&$\bb_{\lambda,k,5}$&$\bb_{\lambda,k,10}$&$\bb_{\lambda,k,20}$&$\bb_{\lambda,k,50}$&$\bb_{\lambda,k,100}$&$\wh\sigma_{k,0}$&$\wh\sigma_{k,100}$\\
\hline
(150,120)&11.34&5.44&5.50&5.50&5.50&5.50&5.50&1.37&0.84\\
(150,140)&8.95&5.52&5.62&5.62&5.62&5.62&5.62&1.25&0.82\\
(220,180)&10.89&3.51&3.48&3.48&3.48&3.48&3.48&1.58&0.91\\
(220,200)&8.49&3.40&3.45&3.45&3.45&3.45&3.45&1.50&0.90\\
\midrule
\midrule
&\multicolumn{9}{c}{$\lambda^*=0.3n$ (before ridge-screening)}\\
\cline{2-10}
$(p,n)$&$\bb_{\lambda}$&$\bb_{\lambda,1}$&$\bb_{\lambda,5}$&$\bb_{\lambda,10}$&$\bb_{\lambda,20}$&$\bb_{\lambda,50}$&$\bb_{\lambda,100}$&$\wh\sigma_0$&$\wh\sigma_{100}$\\
\hline
(150,120)&39.96&31.59&34.59&35.23&35.25&35.25&35.25&2.48&3.46\\
(150,140)&23.83&15.88&18.09&18.39&18.39&18.39&18.39&2.33&3.10\\
(220,180)&60.94&51.27&52.98&53.43&53.45&53.45&53.45&2.63&3.68\\
(220,200)&45.13&32.61&35.01&35.45&35.46&35.46&35.46&2.86&3.62\\
\midrule
&\multicolumn{9}{c}{$\lambda=0.3n$, $k=100$ (after ridge-screening)}\\
\cline{2-10}
$(p,n)$&$\bb_{\lambda,k}$&$\bb_{\lambda,k,1}$&$\bb_{\lambda,k,5}$&$\bb_{\lambda,k,10}$&$\bb_{\lambda,k,20}$&$\bb_{\lambda,k,50}$&$\bb_{\lambda,k,100}$&$\wh\sigma_{k,0}$&$\wh\sigma_{k,100}$\\
\hline
(150,120)&24.95&12.36&12.55&12.55&12.55&12.55&12.55&2.60&1.31\\
(150,140)&17.94&9.56&10.04&10.05&10.05&10.05&10.05&2.39&1.21\\
(220,180)&31.66&13.39&12.73&12.73&12.73&12.73&12.73&3.01&1.44\\
(220,200)&24.13&9.39&9.51&9.51&9.51&9.51&9.51&3.00&1.30\\
\midrule
\midrule
&\multicolumn{9}{c}{$\lambda^*=0.8n$ (before ridge-screening)}\\
\cline{2-10}
$(p,n)$&$\bb_{\lambda}$&$\bb_{\lambda,1}$&$\bb_{\lambda,5}$&$\bb_{\lambda,10}$&$\bb_{\lambda,20}$&$\bb_{\lambda,50}$&$\bb_{\lambda,100}$&$\wh\sigma_0$&$\wh\sigma_{100}$\\
\hline
(150,120)&53.47&40.79&44.04&47.60&48.03&48.04&48.04&3.89&5.72\\
(150,140)&36.98&24.48&29.14&31.97&32.23&32.23&32.23&3.90&5.35\\
(220,180)&76.32&61.94&64.70&68.21&68.59&68.60&68.60&4.15&6.17\\
(220,200)&64.95&46.37&50.70&54.71&55.11&55.11&55.11&4.63&6.53\\
\midrule
&\multicolumn{9}{c}{$\lambda=0.8n$, $k=100$ (after ridge-screening)}\\
\cline{2-10}
$(p,n)$&$\bb_{\lambda,k}$&$\bb_{\lambda,k,1}$&$\bb_{\lambda,k,5}$&$\bb_{\lambda,k,10}$&$\bb_{\lambda,k,20}$&$\bb_{\lambda,k,50}$&$\bb_{\lambda,k,100}$&$\wh\sigma_{k,0}$&$\wh\sigma_{k,100}$\\
\hline
(150,120)&44.82&27.37&26.15&26.49&26.50&26.50&26.50&4.23&2.75\\
(150,140)&31.97&17.88&18.98&19.42&19.43&19.43&19.43&4.22&2.53\\
(220,180)&59.55&35.56&32.39&32.55&32.56&32.56&32.56&4.80&2.86\\
(220,200)&51.11&26.92&25.79&26.20&26.21&26.21&26.21&5.14&2.93\\
\bottomrule
\end{tabular}
  \end{center}}
\end{table}
\begin{figure}[ht]
\begin{center}
{\includegraphics[width=14cm,height=8cm]{bv-tradeoff-k100-plargen.pdf}}
\caption{  The bias-variance trade-off reflected by the empirical MSEs of the de-biased ridge estimators for different number of iterations in Example 2, where we consider $\lambda=0.5n$ for $(p,n)=(150,120)$, where (a) plots the MSEs of the de-baised ridge estimators  before applying the ridge-screening method, and (b) provides the MSEs of the de-biased  ridge estimators after the ridge-screening.  1000 replications are used in the experiments.  }\label{fig-20}
\end{center}
\end{figure}

\begin{table}[htp]\footnotesize
\caption{Empirical probability (EP) of correct recoveries of the true models for $\lambda^*=0.1n$, $0.3n$, and $0.8n$, where we conduct 100 iterations in the bias-correction procedure and set $n^*=40$ in the ridge-screening. 1000 replications are used in the experiments.} 
          \label{Table-a30}
{\begin{center}
\begin{tabular}{ccccc}
\toprule
\multicolumn{5}{c}{$\lambda^*=0.1n,k=100$}\\
\hline
&\multicolumn{4}{c}{$(p,n)$}\\
\cline{2-5}
EP&(150,200)&(150,140)&(220,180)&(220,200)\\
\hline
$EP(\mathcal{M}_0\subset \mathcal{M}_{k}(\lambda^*))$&100\\midrule
\multicolumn{5}{c}{$\lambda^*=0.3n,k=100$}\\
\hline
&\multicolumn{4}{c}{$(p,n)$}\\
\cline{2-5}
EP&(150,200)&(150,140)&(220,180)&(220,200)\\
\hline
$EP(\mathcal{M}_0\subset \mathcal{M}_{k}(\lambda^*))$&100\\midrule
\multicolumn{5}{c}{$\lambda^*=0.8n,k=100$}\\
\hline
&\multicolumn{4}{c}{$(p,n)$}\\
\cline{2-5}
EP&(150,200)&(150,140)&(220,180)&(220,200)\\
\hline
$EP(\mathcal{M}_0\subset \mathcal{M}_{k}(\lambda^*))$&100\\bottomrule
\end{tabular}
  \end{center}}
\end{table}


Similar to the experiments in Example 1, we present the empirical averaging estimation errors (AEEs) in Table~\ref{Table-a4}. The settings of $(p,n)$, $\lambda^*$, $\lambda$, $k$, and $l$
are the same as those in Table~\ref{Table-a3}. From Table~\ref{Table-a4}, we see that the averaging estimation errors before ridge-screening are also quite stable as we increase the number of iterations which is similar to the findings in Table~\ref{Table-a3}. After we apply the ridge-screening approach to the de-biased estimators in the upper panel of Table~\ref{Table-a4} with $k=100$ for each $\lambda^*$, the bias terms (in the lower panel for each $\lambda$) decrease substantially from the ones before, and they also decrease to stable values as we increase the number of iterations in the second-stage bias correction. Notably, the bias terms are all significantly smaller than those in the upper panel without the ridge-screening procedure. This is in agreement with our theoretical results in Theorem~\ref{thm5} and Theorem~\ref{thm7}.

\begin{table}[htp]\scriptsize
\caption{Empirical averaging estimation errors (AEEs) when $p> n$ in Example 2, where the AEEs are similarly defined as those in (\ref{sse0}) and (\ref{ssek}) for $\bb_{\lambda}$, $\bb_{\lambda,k}$, and $\bb_{\lambda,k,l}$. For each $\lambda$, the upper panel reports the AEES before ridge-screening, and the lower one presents the AEEs after ridge-screening with $k=100$ and $n^*=40$ in the variable selection. The number of iterations is set to $k=1,5,10,20,50$, and $100$ for the first-stage de-biased estimation, and  $l=1,5,10,20,50,$ and $100$ for the second-stage bias-correction following the ridge-screening.   1000 replications are used in the experiments.} 
          \label{Table-a4}
{\begin{center}
\begin{tabular}{cccccccc}
\toprule
&\multicolumn{7}{c}{$\lambda^*=0.1n$ (before ridge-screening)}\\
\cline{2-8}
$(p,n)$&$\bb_{\lambda}$&$\bb_{\lambda,1}$&$\bb_{\lambda,5}$&$\bb_{\lambda,10}$&$\bb_{\lambda,20}$&$\bb_{\lambda,50}$&$\bb_{\lambda,100}$\\
\hline
(150,120)&0.44 &0.41& 0.42& 0.42& 0.42& 0.42 &0.42\\
(150,140)&0.30& 0.24& 0.25& 0.25& 0.25& 0.25& 0.25\\
(220,180)&0.47 &0.44& 0.44& 0.44& 0.44 &0.44 &0.44\\
(220,200)&0.37 &0.32 &0.32& 0.32 &0.32& 0.32& 0.32\\
\midrule
&\multicolumn{7}{c}{$\lambda=0.1n$, $k=100$ (after ridge-screening)}\\
\cline{2-8}
$(p,n)$&$\bb_{\lambda,k}$&$\bb_{\lambda,k,1}$&$\bb_{\lambda,k,5}$&$\bb_{\lambda,10}$&$\bb_{\lambda,k,20}$&$\bb_{\lambda,k,50}$&$\bb_{\lambda,k,100}$\\
\hline
(150,120)&0.26& 0.17& 0.17& 0.17 &0.17& 0.17 &0.17\\
(150,140)&0.22& 0.16 &0.16 &0.16& 0.16& 0.16& 0.16\\
(220,180)& 0.21& 0.11 &0.11& 0.11& 0.11& 0.11& 0.11\\
(220,200)&0.19& 0.11& 0.11 &0.11 &0.11& 0.11 &0.11\\
\midrule
\midrule
&\multicolumn{7}{c}{$\lambda^*=0.3n$ (before ridge-screening)}\\
\cline{2-8}
$(p,n)$&$\bb_{\lambda}$&$\bb_{\lambda,1}$&$\bb_{\lambda,5}$&$\bb_{\lambda,10}$&$\bb_{\lambda,20}$&$\bb_{\lambda,50}$&$\bb_{\lambda,100}$\\
\hline
(150,120)&0.51 &0.45& 0.47& 0.47& 0.47& 0.47 &0.47\\
(150,140)&0.39& 0.31& 0.33& 0.33& 0.33& 0.33& 0.33\\
(220,180)&0.52 &0.48& 0.48& 0.48& 0.48 &0.48 &0.48\\
(220,200)&0.45 &0.38 &0.39& 0.39 &0.39& 0.39& 0.39\\
\midrule
&\multicolumn{7}{c}{$\lambda=0.3n$, $k=100$ (after ridge-screening)}\\
\cline{2-8}
$(p,n)$&$\bb_{\lambda,k}$&$\bb_{\lambda,k,1}$&$\bb_{\lambda,k,5}$&$\bb_{\lambda,10}$&$\bb_{\lambda,k,20}$&$\bb_{\lambda,k,50}$&$\bb_{\lambda,k,100}$\\
\hline
(150,120)&0.40& 0.27& 0.27& 0.27 &0.27& 0.27 &0.27\\
(150,140)&0.34& 0.24 &0.24 &0.25& 0.25& 0.25& 0.25\\
(220,180)& 0.37& 0.24 &0.23& 0.23& 0.23& 0.23& 0.23\\
(220,200)&0.33& 0.20& 0.20 &0.20 &0.20& 0.20 &0.20\\
\midrule
\midrule
&\multicolumn{7}{c}{$\lambda^*=0.8n$ (before ridge-screening)}\\
\cline{2-8}
$(p,n)$&$\bb_{\lambda}$&$\bb_{\lambda,1}$&$\bb_{\lambda,5}$&$\bb_{\lambda,10}$&$\bb_{\lambda,20}$&$\bb_{\lambda,50}$&$\bb_{\lambda,100}$\\
\hline
(150,120)&0.60 &0.52& 0.53& 0.56& 0.56& 0.56 &0.56\\
(150,140)&0.50& 0.40& 0.43& 0.45& 0.45& 0.45& 0.45\\
(220,180)&0.59 &0.53& 0.54& 0.55& 0.55 &0.55 &0.55\\
(220,200)&0.54 &0.46 &0.47& 0.49 &0.49& 0.49& 0.49\\
\midrule
&\multicolumn{7}{c}{$\lambda=0.8n$, $k=100$ (after ridge-screening)}\\
\cline{2-8}
$(p,n)$&$\bb_{\lambda,k}$&$\bb_{\lambda,k,1}$&$\bb_{\lambda,k,5}$&$\bb_{\lambda,10}$&$\bb_{\lambda,k,20}$&$\bb_{\lambda,k,50}$&$\bb_{\lambda,k,100}$\\
\hline
(150,120)&0.54& 0.42& 0.41& 0.41 &0.41& 0.41 &0.41\\
(150,140)&0.46& 0.34 &0.35 &0.35& 0.35& 0.35& 0.35\\
(220,180)& 0.52& 0.40 &0.38& 0.38& 0.38& 0.38& 0.38\\
(220,200)&0.48& 0.34& 0.33&0.33 &0.33& 0.33 &0.33\\
\bottomrule
\end{tabular}
  \end{center}}
\end{table}

Finally, we study the performance of the inference method in Theorem~\ref{thm6}(ii). We only consider the case when $(p,n)=(220,200)$, $\lambda^*=\lambda=0.1n$, and $n^*=40$, and we can produce similar results for other settings. Define
\[\bgamma_1=(0.8,-1,0.5,{\bf 0}_{p-3}')',\quad\bgamma_2=(0,-1,0.5,0.8,{\bf 0}_{p-4}')',\]
\[\bgamma_3=({\bf 0}_3',-0.9,0.4,-0.8,{\bf 0}_{p-6}')',\,\,\text{and}\,\,\bgamma_4=({\bf 0}_4',0.5,0.7,-0.8,{\bf 0}_{p-7}')'.\]
We plot the empirical histograms of  $\sqrt{n}\bgamma_1'(\wh\bbeta(\lambda)-\bbeta)$,   $\sqrt{n}\bgamma_2'(\wh\bbeta(\lambda)-\bbeta)$, $\sqrt{n}\bgamma_3'(\wh\bbeta(\lambda)-\bbeta)$, and  $\sqrt{n}\bgamma_4'(\wh\bbeta(\lambda)-\bbeta)$ in Figure~\ref{fig-3}, from which it is observed that most of the estimators are not centered around zero. Subsequently, we apply the bias-correction procedure to the original ridge estimators, and the empirical histograms of the bias-corrected estimators are presented in Figure~\ref{fig-4} as compared to Figure~\ref{fig-3}. From Figure~\ref{fig-4}, it can be seen that while some estimators become more centered around zero (e.g., (a) and (b)) due to the bias-correction procedure, the validity of inference without ridge-screening cannot be assured, as (c) and (d) still exhibit significant biases.

Furthermore, we apply the ridge-screening approach to the de-biased estimators depicted in Figure~\ref{fig-4}, subsequently obtaining restricted ridge estimators along with their de-biased counterparts. Figure~\ref{fig-5} showcases the empirical histograms of: (a) $\sqrt{n}\bgamma_1'(\wh\bbeta_{\mathcal{M}_k,l}^*(\lambda)-\bbeta)$; (b)  $\sqrt{n}\bgamma_2'(\wh\bbeta_{\mathcal{M}_k,l}^*(\lambda)-\bbeta)$; (c) $\sqrt{n}\bgamma_3'(\wh\bbeta_{\mathcal{M}_k,l}^*(\lambda)-\bbeta)$; and (d) $\sqrt{n}\bgamma_4'(\wh\bbeta_{\mathcal{M}_k,l}^*(\lambda)-\bbeta)$, where $\wh\bbeta_{\mathcal{M}_k,l}^*(\lambda)$ is a $p$-dimensional vector with $\wh\bbeta_{\mathcal{M}_k,l}(\lambda)$ as its sub-vector and the remaining elements are set to zero. Similar to the curves in Figure~\ref{fig-2}, the density curve in Figure~\ref{fig-5} is plotted based on the limiting distribution specified in Theorem~\ref{thm6}(ii). From Figure~\ref{fig-5}, it is evident that the de-biased restricted ridge estimators predominantly cluster around zero. The density curves for the corresponding estimators are quite satisfactory, suggesting the validity of the proposed inference approach.

\begin{figure}[ht]
\begin{center}
{\includegraphics[width=14cm,height=8cm]{ridge-plargen-nobc-p220n200.pdf}}
\caption{Empirical histograms of (a) $\sqrt{n}\bgamma_1'(\wh\bbeta(\lambda)-\bbeta)$; (b)  $\sqrt{n}\bgamma_2'(\wh\bbeta(\lambda)-\bbeta)$; (c) $\sqrt{n}\bgamma_3'(\wh\bbeta(\lambda)-\bbeta)$; and (d) $\sqrt{n}\bgamma_4'(\wh\bbeta(\lambda)-\bbeta)$ in Example 2. We set $(p,n)=(220,200)$ and $\lambda=0.1n$. 1000 replications are used in the experiments.  }\label{fig-3}
\end{center}
\end{figure}


\begin{figure}[ht]
\begin{center}
{\includegraphics[width=14cm,height=8cm]{ridge-plargen-afterbc-p220n200-noRS.pdf}}
\caption{Empirical histograms of (a) $\sqrt{n}\bgamma_1'(\wh\bbeta_{c,k}(\lambda)-\bbeta)$; (b)  $\sqrt{n}\bgamma_2'(\wh\bbeta_{c,k}(\lambda)-\bbeta)$; (c) $\sqrt{n}\bgamma_3'(\wh\bbeta_{c,k}(\lambda)-\bbeta)$; and (d) $\sqrt{n}\bgamma_4'(\wh\bbeta_{c,k}(\lambda)-\bbeta)$ in Example 2. We set $(p,n)=(220,200)$, $k=100$, and $\lambda=0.1n$. 1000 replications are used in the experiments.  }\label{fig-4}
\end{center}
\end{figure}


\begin{figure}[ht]
\begin{center}
{\includegraphics[width=14cm,height=8cm]{rd-rs100-p220n200lbd01.pdf}}
\caption{Empirical histograms of (a) $\sqrt{n}\bgamma_1'(\wh\bbeta_{\mathcal{M}_k,l}^*(\lambda)-\bbeta)$; (b)  $\sqrt{n}\bgamma_2'(\wh\bbeta_{\mathcal{M}_k,l}^*(\lambda)-\bbeta)$; (c) $\sqrt{n}\bgamma_3'(\wh\bbeta_{\mathcal{M}_k,l}^*(\lambda)-\bbeta)$; and (d) $\sqrt{n}\bgamma_4'(\wh\bbeta_{\mathcal{M}_k,l}^*(\lambda)-\bbeta)$ in Example 2, where $\wh\bbeta_{\mathcal{M}_k,l}^*(\lambda)$ is a $p$-dimensional vector with $\wh\bbeta_{\mathcal{M}_k,l}(\lambda)$ as its sub-vector and the other elements being zero. We set $(p,n)=(220,200)$, $k=100$, $n^*=40$, $l=100$, and $\lambda=\lambda^*=0.1n$. The density curve is plotted based on the limiting distribution in Theorem~\ref{thm6}(ii). 1000 replications are used in the experiments. }\label{fig-5}
\end{center}
\end{figure}


  
\section{Empirical Application} \label{sec40}

In this section, we apply the proposed method to forecast the U.S. macroeconomic series and study the out-of-sample prediction intervals using ridge regression. We consider the widely used macroeconomic variables studied by \cite{Stock2002}, \cite{mccracken2016fred}, \cite{giannone2021economic}, and \cite{gao2022modeling,gao2023supervised}, among many others. The data are obtained from the FRED-MD data base which are maintained by St. Louis Fed. See \url{https://research.stlouisfed.org/econ/mccracken/fred-databases/}. There are 127 variables in the online data set, and we remove 4 of them because of missing values therein.  Consequently, we consider the 123 macro variables spanning from July 1962 to December
2019 as all the series have no missing values during this period, which is the same as the setting in \cite{gao2023supervised}. The detailed variables and transformation codes to ensure the stationarity of each
macro variable are provided in Table IA.I of \cite{gao2023supervised}. The sample size is $n = 690$.


Similar to \cite{Stock2002b}, \cite{bai2006}, and \cite{cheng2015forecasting}, we adopt the following factor-augmented regression model: 
\begin{equation}
\begin{array}{l}
     \bX_t= \bLambda\bff_t+\be_t, \\
     y_{t+h}=\alpha_1y_{t}+...+\alpha_{q}y_{t-q+1}+\gamma_1f_{1,t}+...+\gamma_rf_{r,t}+\ve_{t+h}, t=q,...,n-h,
\end{array}
\end{equation}
where $\bX_t$ consists of 122 macroeconomic variables, $y_{t}$ is the remaining target one, and $\bff_t=(f_{1,t},...,f_{r,t})'$ is an $r$-dimensional latent factor process which can be estimated by applying Principal Component Analysis (PCA) to $\bX_t$. We set $q=10$ and $r=60$, and apply the ridge regression to estimate the parameters $\alpha_i$ and $\gamma_j$ for $1\leq i\leq q$ and $1\leq j\leq r$ with estimated factors. Hence, the number of covariates in the ridge regression is $p=q+r=70$. We focus on the prediction of  the consumer price index: all (CPI-All), which is an important index related to the inflation. See also \cite{Stock2002} and \cite{gao2022modeling} for similar studies. 

First, we set $\lambda\in\{0.05n,0.1n,0.2n,...,1.5n\}$ which contains 16 candidates for the penalty parameter.  We split the sample into two subsamples, where the first one consists of the first $80\
Next, for each configuration of $(\wh\lambda,h)=(0.8n,1)$ and $(0.7n,2)$, we execute Algorithm~\ref{pless:a1}. We observe that the bias-correction procedure terminates within 3 to 5 iterations when setting $\eta=O(10^{-4})$  in equation (\ref{con:cri}). For simplicity, we opt for $k=10$ in the bias-correction, a choice deemed sufficiently large. Following the guidelines provided in Section~\ref{sec27}, we compute pointwise prediction intervals for the data points in the testing set. The standard errors for each prediction are determined using the method described in Remark~\ref{rm5}(iv). We present the $95\









\begin{figure}[ht]
\begin{center}
{\includegraphics[width=14cm,height=7cm]{CPI-h1.pdf}}
\caption{Out-of-sample pointwise prediction intervals for the monthly CPI-All from July 1, 2008 to Dec. 12, 2019, constructed using (\ref{l2})-(\ref{u2}) at a $95\\end{center}
\end{figure}

\begin{figure}[ht]
\begin{center}
{\includegraphics[width=14cm,height=7cm]{CPI-h2.pdf}}
\caption{Out-of-sample pointwise prediction intervals for the monthly CPI-All from July 1, 2008 to Dec. 12, 2019, constructed using (\ref{l2})-(\ref{u2}) at a $95\\end{center}
\end{figure}
Finally, we investigate the effectiveness of the ridge-screening (RS) approach in out-of-sample predictions. We take the 1-step ahead prediction as an example. For each $\lambda^*\in\{0.05n,0.1n,0.2n,...,1.5n\}$, which is the same as our previous candidate set, we adopt $k=10$  and vary $n^*$ from $10$ to $70$  in (\ref{rs:sub}), that is, we select the largest $10$ to $70$ variables based on their magnitudes from the set $\{|\wh\bbeta_{c,10,1}(\lambda^*)|,...,|\wh\bbeta_{c,10,70}(\lambda^*)|\}$ and compute their corresponding out-of-sample forecast errors. By minimizing the out-of-sample MSFEs, we determined that the optimal parameters are $(\wh\lambda^*,n^*)=(0.5n,31)$. This implies that the submodel, consisting of covariates corresponding to the largest 31 elements in $|\wh\bbeta_{c,10}(0.5n)|$, yields the smallest out-of-sample MSFEs. Consequently, we chose $\wh\lambda=\wh\lambda^*=0.5n$ for the subsequent bias-correction procedure in the restricted ridge estimators, conducting it with $l=10$ iterations. Figure~\ref{fig-10} displays the $95\


\begin{figure}[ht]
\begin{center}
{\includegraphics[width=14cm,height=7cm]{CPI-h1-RS.pdf}}
\caption{Out-of-sample pointwise prediction intervals for the monthly CPI-All from July 1, 2008 to Dec. 12, 2019, constructed using (\ref{l3})-(\ref{u3}) at $95\\end{center}
\end{figure}

		
		
		
		
		
		

		
		
	
		
			\section{Conclusion} \label{sec4}

   A fundamental question in ridge regression revolves around the possibility of correcting the bias term using available data, alongside ensuring the validity of statistical inferences without affecting the predictive performance of the original ridge estimator. This paper tackles these long-standing challenges by introducing a straightforward and readily implementable iterative procedure, along with a ridge-screening method. In cases where the dimension $p$ is smaller than the sample size $n$, our procedure effectively corrects the bias term under very mild assumptions. However, when $p>n$ and the projection matrix becomes singular, we propose a ridge-screening approach to eliminate coefficients that are relatively insignificant compared to others. We then specifically focus on the restricted model, which retains a sufficient number of parameters to encapsulate most of the information regarding the response vector. Our bias-correction procedure can be further applied to the restricted model, enabling the correction of the bias. Moreover, the ridge-screening method offers a novel approach to variable selection, which is of independent interest beyond bias correction. We derive the limiting distributions of the de-biased estimators, facilitating the making of statistical inferences. Simulated  and real data examples are used to corroborate the effectiveness and validity of our proposed methodology. The proposed inference solution for ridge regression serves as an illustrative example addressing the inference challenges of regularized machine learning methods outlined in Section 2.8 of \cite{athey2019machine}  without adversely affecting the predictive performance.  



		
		
	
	
	
		\singlespacing
\bibliographystyle{econometrica-3}
\let\oldbibliography\thebibliography
\renewcommand{\thebibliography}[1]{  \oldbibliography{#1}  \setlength{\itemsep}{3pt}}
	{\footnotesize
		\bibliography{reference}
	}

 \onehalfspacing

\newpage
\setcounter{page}{1}
\appendix
\restoregeometry

 \setcounter{equation}{0}
	\renewcommand{\theequation}{A.\arabic{equation}}
  \section*{Online Appendix: Proofs of the Theorems}\label{sec5}
\onehalfspacing We will use $C$ or $c$ to denote a generic constant the value of which may change at different places.
\vskip 0.5cm

 {\bf Proof of Theorem \ref{thm1}.} To see the bias of the ridge estimator, it follows from (\ref{v:hlm}) that
\begin{equation}\label{decom}
    \wh\bbeta(\lambda)=\bbeta-\lambda(\bX'\bX+\lambda\bI_p)^{-1}\bbeta+(\bX'\bX+\lambda\bI_p)^{-1}\bX'\bve.
\end{equation}
Conditioning on $\bX$, by the assumption that $E\bve={\bf 0}$, it follows immediately that
\[\bbeta-E(\wh\bbeta(\lambda))=\lambda(\bX'\bX+\lambda\bI_p)^{-1}\bbeta.\]
This completes the proof. $\Box$
\vskip 0.6 cm
{\bf Proof of Theorem~\ref{thm2}}. By (\ref{decom}) and an elementary argument, $\wh\bbeta_{c,k}(\lambda)$ can be written as
\begin{align}\label{btck}
   \wh\bbeta_{c,k}(\lambda)=&\wh\bbeta(\lambda)+\sum_{j=1}^{k}\lambda^j(\bX'\bX+\lambda\bI_p)^{-j}\wh\bbeta(\lambda)\notag\\
   =&\bbeta-\lambda^{k+1}(\bX'\bX+\lambda\bI_p)^{-(k+1)}\bbeta+\sum_{j=0}^k\lambda^j(\bX'\bX+\lambda\bI_p)^{-(j+1)}\bX'\bve.
\end{align}
Then, it follows that
\[\bbeta-E\wh\bbeta_{c,k}=\lambda^{k+1}(\bX'\bX+\lambda\bI_p)^{-(k+1)}\bbeta.\]
When $p<n$ and $\bX'\bX$ is invertible, by Assumption~\ref{asm1}, the singular-value decomposition of $\bX'\bX$ is
\[\bX'\bX=\bU_1\bD_1^2\bU_1,\,\,\bD_1=\diag(d_1,...,d_p),\,\, d_j>0.\]
Then, it is not hard to see that
\[(\bX'\bX+\lambda\bI_p)^{-(k+1)}=\bU_1(\bD_1^2+\lambda\bI_p)^{-(k+1)}\bU_1',\]
and
\[\bb_{\lambda,k}=\lambda^{k+1}\bU_1(\bD_1^2+\lambda\bI_p)^{-(k+1)}\bU_1'\bbeta.\]
Note that, for $1\leq j\leq p$, each diagonal element in $\lambda^{k+1}(\bD_1^2+\lambda\bI_p)^{-(k+1)}$ is
\[\left(\frac{\lambda}{d_j^2+\lambda}\right)^{k+1}\rightarrow 0,\,\, \text{as $k\rightarrow\infty$},\]
at a rate of exponential decay. Therefore, for any given configuration of $(p,n)$ with $p<n$, if the number of iteration $k$ satisfies $\max_{1\leq j\leq p}C_{n,p}(\frac{\lambda}{d_j^2+\lambda})^{k+1}\rightarrow 0$ with $\|\bbeta\|_2<C_{p,n}$, we can show that
\[\|\bb_{\lambda}\|_2\rightarrow 0,\,\,\text{as $k\rightarrow\infty$}.\]
As a matter of fact, the aforementioned result applies to any given and fixed $(p, n)$. Furthermore, when considering the scenario where both $p$ and $n$ tend towards infinity in an asymptotic framework, it is not hard to see that the above convergence also holds under the condition that $d_j^2\asymp \lambda \asymp n$ in relation to the sample size $n$ within an asymptotic setting. 
This completes the proof. $\Box$
\vskip 0.6cm

{\bf Proof of Theorem~\ref{thm3}}. Note that
\[\bb_{\lambda,k}=\lambda^{k+1}(\bX'\bX+\lambda\bI_p)^{-(k+1)}\bbeta.\]
When $p>n$, by Assumption~\ref{asm2}, the singular-value decomposition of $(\bX'\bX+\lambda\bI_p)$ is
\[\bX'\bX+\lambda\bI_p=\bU\bD\bU',\,\,\bU=[\bU_1,\bU_2],\,\,\bD=\diag(\bD_1^2+\lambda\bI_{p^*},\lambda\bI_{p-p^*}).\]
Then, 
\begin{align}\label{blk2}
 \bb_{\lambda,k}=&\lambda^{k+1}\left\{\bU_1(\bD_1+\lambda\bI_{p^*})^{-(k+1)}\bU_1'+\lambda^{-(k+1)}\bU_2\bU_2'\right\}\bbeta\notag\\
 =&\bU_1\lambda^{k+1}(\bD_1+\lambda\bI_{p^*})^{-(k+1)}\bU_1'\bbeta+\bU_2\bU_2'\bbeta.
\end{align}
By a similar argument as that in the proof of Theorem~\ref{thm2}, the first term in (\ref{blk2}) vanishes as $k\rightarrow \infty$. Hence, 
\[ \bb_{\lambda,k}\rightarrow \bU_2\bU_2'\bbeta,\,\,\text{as $k\rightarrow\infty$}.\]
If we further allow $n,p\rightarrow\infty$, under a similar setting in the proof of Theorem~\ref{thm2} above, we can show that
\[ \bb_{\lambda,k}-\bU_2\bU_2'\bbeta\rightarrow {\bf 0},\,\,\text{as $n,p,k\rightarrow\infty$},\]
which is discussed in Remark~\ref{rm3}(iii).
This completes the proof. $\Box$
\vskip 0.6cm

{\bf Proof of Theorem~\ref{thm4}}. We prove it by contradiction. Suppose $\bX'\bX$ is singular and there exists a transformation matrix $\bS$ such that
\begin{equation}\label{sy}
 E(\bS\bY)=\bU_2\bU_2'\bbeta.
\end{equation}
In other words, the remaining bias term in Theorem~\ref{thm3} can be corrected by $\bS\bY$.
(\ref{sy}) implies that
\begin{equation}\label{syy}
E(\bS\bX)\bbeta=\bU_2\bU_2'\bbeta.    
\end{equation}
Since a $p$-dimensional space can be spanned by the columns of $[\bU_1,\bU_2]$, there exist vectors $\balpha_1\in R^{p^*}$ and $\balpha_2\in R^{p-p^*}$ such that
\[\bbeta=\bU_1\balpha_1+\bU_2\balpha_2.\]
We only talk about the case when $\balpha_1\neq{\bf 0}$ and $\balpha_2\neq 0 $ since either $\balpha_1={\bf 0}$ or $\balpha_2\neq 0 $ will lead the bias term to be zero, implying that we do not need to correct it.
We plug it into (\ref{syy}) and obtain
\begin{equation}\label{eq:sy}
 E(\bS\bX)\bU_1\balpha_1=\bU_2\balpha_2.
\end{equation}
Note that $\bU_1\balpha_1$ is a vector in the hyperplane spanned by the columns of $\bU_1$, and $\bU_2\balpha_2$ is a vector in the hyperplane spanned by the columns of $\bU_2$. Since $\bU_1$ is orthogonal to $\bU_2$, we cannot find any linear transformation matrix $\bS$ such that the equation of (\ref{eq:sy}) holds. This contradicts our assumption. This completes the proof. $\Box$
\vskip 0.6 cm

{\bf Proof of Theorem~\ref{thm5}}. (i) We first consider the case when $p<n$. By (\ref{btck}) in the proof of Theorem~\ref{thm2} above,
\[\wh\bbeta_{c,k}(\lambda^*)=\bbeta+\sum_{j=0}^k{\lambda^*}^j(\bX'\bX+\lambda^*\bI_p)^{-(j+1)}\bX'\bve+o(1)=\bbeta+\bgamma+o(1),\]
where $o(1)$ can be made arbitrarily small at an exponential rate, and $\bbeta$ has $s^*$ nonzero element, which can be smaller than $p$. We use the SVD of $\bX'\bX$ when $p<n$ as that in the proof of Theorem~\ref{thm2} and obtain
\[\sum_{j=0}^k{\lambda^*}^{j+1}(\bX'\bX+\lambda^*\bI_p)^{-(j+1)}=\bU_1\bD_1^*\bU_1',\]
where $\bD_1^*=\diag(d_1^*,...,d_p^*)$ with
\[d_i^*=\sum_{j=0}^k\left(\frac{\lambda^*}{\lambda^*+d_i^2}\right)^{j+1}=\frac{\lambda^*}{d_i^2}+o(1)\asymp O(1),\,\,1\leq i\leq p,\]
where $o(1)$ is also an arbitrarily small term at an exponential rate, and $\lambda^*$ and $d_i^2$ are all of order $n$ by Assumption \ref{asm3}. Then, 
\[\bgamma=\sum_{j=0}^k{\lambda^*}^j(\bX'\bX+\lambda^*\bI_p)^{-(j+1)}\bX'\bve=\bU_1\bD_1^*\bU_1'\bV\frac{\bD_1}{\sqrt{\lambda^*}}\bU_1'\frac{\bve}{\sqrt{\lambda^*}}.\]
Note that the $\ell_2$-norm of each row of $\bU_1\bD_1^*\bU_1'\bV\frac{\bD_1}{\sqrt{\lambda^*}}\bU_1'$ is
\[\|\be_i'\bU_1\bD_1^*\bU_1'\bV\frac{\bD_1}{\sqrt{\lambda^*}}\bU_1'\|_2^2\leq C<\infty,\]
where we used the fact that each element in the diagonal matrices $\bD_1^*$ and $\frac{\bD_1}{\sqrt{\lambda^*}}$ is bounded from above and below. By Assumption \ref{asm5}, we have
\[\max_{1\leq i\leq p}|\gamma_i|=O_p(\sqrt{\frac{1}{\lambda^*}}\sqrt{\log(p)}=O_p(\sqrt{\frac{{\log(p)}}{n}}).\]
For $i\in\mathcal{M}_0$, by Assumption \ref{asm4} and $\log(p)/n^{1-2\tau}\rightarrow 0$,
\[\min_{i\in\mathcal{M}_0}|\wh\beta_{c,k,i}(\lambda^*)|\geq \min_{i\in\mathcal{M}_0}|\beta_i|-O_p(\sqrt{\frac{{\log(p)}}{n}})\geq C n^{-\tau},\]
and
\[\max_{i\in\mathcal{M}_0^c}|\wh\beta_{c,k,i}(\lambda^*)|\leq C\sqrt{\frac{{\log(p)}}{n}}.\]
It is obvious that 
\[\min_{i\in\mathcal{M}_0}|\wh\beta_{c,k,i}(\lambda^*)|>\max_{i\in\mathcal{M}_0^c}|\wh\beta_{c,k,i}(\lambda^*)|,\]
with probability tending to one. This implies that the magnitudes of the parameters   $\wh\beta_{c,k,i}(\lambda^*)$'s for $i \in \mathcal{M}_0$ are of larger orders than the remaining ones. Therefore, we prove that
\[P(\mathcal{M}_0\subset\mathcal{M}_k(\lambda^*))\rightarrow 1,\,\, \text{as}\,\, k\rightarrow\infty,\]
if we choose $n^*\geq s^*$ in (\ref{rs:sub}) of the main article. For instance, we may simply choose $n^*=p$ for $p<n$, which creates a dense model suitable for bias-correction.\\

Next, we consider the case when $p>n$. Note that Theorem~\ref{thm3} implies that
\begin{equation}\label{betk:rp}
    \wh\bbeta_{c,k}(\lambda^*)=\bbeta-\bU_2\bU_2'\bbeta+\sum_{j=0}^k{\lambda^*}^j(\bX'\bX+\lambda^*\bI_p)^{-(j+1)}\bX'\bve+o(1)=\bbeta-\bdelta+\bgamma+o(1),
\end{equation}
where the term $o(1)$ can be arbitrarily small at a rate of exponential decay as $k\rightarrow\infty$.
The idea of proving the result is as follows. We first investigate the lower bound of $(\min_{i\in\mathcal{M}_0}|\wh\beta_{c,k,i}(\lambda^*)|)^2$ and the upper bound of $\|\wh\bbeta_{c,k,i}(\lambda^*)\|_2^2$. If the number of coordinates selected in $\mathcal{M}_k(\lambda^*)$ is greater than 
the ratio between the upper bound and the lower bound, the submodel $\mathcal{M}_k(\lambda^*)$ must contain the true $\mathcal{M}_0$ as a subset.

For $i\in\mathcal{M}_0$, by Assumption \ref{asm4}, the magnitude of the $i$-th coordinate of $\bU_2\bU_2'\bbeta$ is less than that of $\bbeta$. If $\log(p)/n^{1-2\tau}\rightarrow 0$, we have 
\[\min_{i\in \mathcal{M}_0}|\wh\beta_{c,k,i}(\lambda^*)|\geq C_2n^{-\tau}-C_1n^{-\tau}-O_p(\sqrt{\log(p)/n})\geq O_p(n^{-\tau}),\]
where $C_2>C_1>0$. In addition,
\[\|\wh\bbeta_{c,k}(\lambda^*)\|_2^2\leq C\|\bU_1\bU_1'\bbeta\|_2^2+C\|\sum_{j=0}^k{\lambda^*}^j(\bX'\bX+\lambda^*\bI_p)^{-(j+1)}\bX'\bve\|_2^2\leq Cs^*+Cn^{-2}p\log(p).\]
Then,
\begin{equation}\label{ratio}
    \frac{\|\wh\bbeta_{c,k}(\lambda^*)\|_2^2}{(\min_{i\in \mathcal{M}_0}|\wh\beta_{c,k,i}(\lambda^*)|)^2}\leq Cn^{-2\tau}s^*+Cn^{2\tau-2}p\log(p).
\end{equation}
The upper bound in (\ref{ratio}) is important for us to show the result because it implies that the number of elements in $|\wh\bbeta_{c,k}(\lambda^*)|$ that are greater than $\min_{i\in \mathcal{M}_0}|\wh\beta_{c,k,i}(\lambda^*)|$ is at most $Cn^{-2\tau}s^*+Cn^{2\tau-2}p\log(p)$. Note that all elements with indexes $i\in \mathcal{M}_0$ are greater than or equal to  $\min_{i\in \mathcal{M}_0}|\wh\beta_{c,k,i}(\lambda^*)|$. Therefore, if the choice of  $n^*$ in (\ref{rs:sub}) satisfies 
\[\frac{n^*}{Cn^{-2\tau}s^*+Cn^{2\tau-2}p\log(p)}\rightarrow\infty,\]
or equivalently,  the number of largest elements selected in $|\wh\bbeta_{c,k}(\lambda^*)|$, $n^*$, is more than the total number of parameters in $|\wh\bbeta_{c,k}(\lambda^*)|$ that is greater than $\min_{i\in \mathcal{M}_0}|\wh\beta_{c,k,i}(\lambda^*)|$,
we must have that $\mathcal{M}_k(\lambda^*)$ consists of all the indexes in $\mathcal{M}_0$. This proves (i).\\

(ii) Since $\mathcal{M}_0\subset\mathcal{M}_k(\lambda^*)$ with probability tending to one, on the event of  $\{\mathcal{M}_0\subset\mathcal{M}_k(\lambda^*)\}$, the proof is the same as that for Theorem~\ref{thm2}. This completes the proof. $\Box$

\vskip 0.6cm
{\bf Proof of Theorem~\ref{thm6}}. By (\ref{btck}), we can obtain that
\[ \wh\bbeta_{c,k}(\lambda)-\bbeta=-\lambda^{k+1}(\bX'\bX+\lambda\bI_p)^{-(k+1)}\bbeta+\sum_{j=0}^k\lambda^j(\bX'\bX+\lambda\bI_p)^{-(j+1)}\bX'\bve,\]
and
\[ \wh\bbeta_{\mathcal{M}_k,l}(\lambda)-\bbeta_{\mathcal{M}_k}=-\lambda^{l+1}(\bX_{\mathcal{M}_k}'\bX_{\mathcal{M}_k}+\lambda\bI_{n^*})^{-(l+1)}\bbeta_{\mathcal{M}_k}+\sum_{j=0}^l\lambda^j(\bX_{\mathcal{M}_k}'\bX_{\mathcal{M}_k}+\lambda\bI_{n^*})^{-(j+1)}\bX_{\mathcal{M}_k}'\bve.\]
Under the assumption that $\bve\sim N({\bf 0},\bSigma_\ve)$, the exact distributions in Theorem~\ref{thm6}(i)-(ii) are straightforward. 

Moreover, by a similar argument as that in the proof of Theorem~\ref{thm2} above, it is not hard to see that
\[\sqrt{n}\bmu_{1,k}(\lambda)=o(1)\quad\text{and}\quad \sqrt{n}\bmu_{2,k,l}(\lambda)=o(1),\]
where $o(1)$ can be arbitrarily small at an exponential rate if $k$ and $l$ are sufficiently large. Consequently, for any given and fixed $(p,n)$, the means $\sqrt{n}\bmu_{1,k}(\lambda)$ and $\sqrt{n}\bmu_{2,k,l}(\lambda)$ are asymptotically vanishing as $k,l\rightarrow\infty$, and the asymptotic normal distributions in Theorem~\ref{thm6} hold. 

As discussed in Remark~\ref{rm5}(i), if the normality assumption that $\bve\sim N({\bf 0},\bSigma_\ve)$ is  relaxed to the case that $\{\ve_i,i=1,...,T\}$ is a martingale difference sequence with finite variances, under the assumption that $\lambda\asymp\lambda^*\asymp n$ and the nonzero singular values of $\bX$ are of order $\sqrt{n}$,  the limiting distributions in Theorem~\ref{thm6} can also be established as $n\rightarrow\infty$, following the argument in \cite{hall2014martingale}. We omit the details. This completes the proof of Theorem~\ref{thm6}. $\Box$

	\vskip 0.6 cm
{\bf Proof of Theorem~\ref{thm7}}. (i)	Assume $\bSigma_\ve=\sigma^2\bI_n$,	by Assumption~\ref{asm1} and (\ref{mse:bt}), the MSE of $\wh\bbeta_{c,k}(\lambda)$ can be expressed as
\begin{align}\label{mse:beck:i}
    \text{MSE}(\wh\bbeta_{c,k}(\lambda))=&\bbeta'\lambda^{k+1}(\bX'\bX+\lambda\bI_p)^{-(k+1)}\lambda^{k+1}(\bX'\bX+\lambda\bI_p)^{-(k+1)}\bbeta\notag\\
    &+E[\bve'\bX\sum_{j=0}^k\lambda^j(\bX'\bX+\lambda\bI_p)^{-(k+1)}\sum_{j=0}^k\lambda^j(\bX'\bX+\lambda\bI_p)^{-(k+1)}\bX'\bve]\notag\\
    =&\bbeta'\bU_1\bLambda_{k}(\lambda)\bU_1'\bbeta+E[\bve'\bV_1\bD_{k}(\lambda)\bV_1'\bve]\notag\\
    =&\bbeta'\bU_1\bLambda_{k}(\lambda)\bU_1'\bbeta+\sigma^2\tr[\bD_{k}(\lambda)],
\end{align}
		where $\bLambda_{k}(\lambda)=\diag(\gamma_{k,1}(\lambda),...,\gamma_{k,p}(\lambda))$ with
\[\gamma_{k,i}(\lambda)=[\frac{\lambda}{\lambda+d_i^2}]^{2(k+1)},\]
and $\bD_{k}(\lambda)=\diag(d_{k,1}(\lambda),...,d_{k,p}(\lambda))$ with
\[d_{k,i}(\lambda)=\frac{1}{d_i^2}\left[1-(\frac{\lambda}{\lambda+d_i^2})^{k+1}\right]^2.\]
Let $\bU_1'\bbeta=(\delta_1,...,\delta_p)'$, then, 	
\begin{equation}\label{mse:btck:final}
    \text{MSE}(\wh\bbeta_{c,k}(\lambda))=\sum_{i=1}^p\delta_i^2[\frac{\lambda}{\lambda+d_i^2}]^{2(k+1)}+\sigma^2\sum_{i=1}^p\frac{1}{d_i^2}\left[1-(\frac{\lambda}{\lambda+d_i^2})^{k+1}\right]^2.
\end{equation}
Note that $\text{MSE}(\wh\bbeta(\lambda))=\text{MSE}(\wh\bbeta_{c,0}(\lambda))$, and the MSE will become stable if $k\rightarrow\infty$, then, for any finite $k\geq 1$, define 
\[f_i(k)=\delta_i^2[\frac{\lambda}{\lambda+d_i^2}]^{2(k+1)}+\sigma^2\frac{1}{d_i^2}\left[1-(\frac{\lambda}{\lambda+d_i^2})^{k+1}\right]^2,\]
and $\text{MSE}(\wh\bbeta_{c,k}(\lambda))=\sum_{i=1}^pf_i(k)$. It suffices to investigate the derivatives of $f_{i}(k)$ for $1\leq i\leq p$ and finite $k>0$. Note that
\begin{equation}\label{fip}
    f_i'(k)=2\delta_i^2(k+1)\log(\frac{\lambda}{\lambda+d_i^2})-\frac{2\sigma^2}{d_i^2}(k+1)[1-(\frac{\lambda}{\lambda+d_i^2})^{k+1}]\log(\frac{\lambda}{\lambda+d_i^2}).
\end{equation}
  Then $f_i'(k)<0$ if and only if
\begin{equation}\label{cond:iff}
    \frac{\delta_i^2d_i^2}{\sigma^2}>1-(\frac{\lambda}{\lambda+d_i})^{k+1}.
\end{equation}
It follows from (\ref{cond:iff}) that
\begin{equation}\label{k:upper}
    k\leq\frac{\log(1-\frac{\delta_i^2d_i^2}{\sigma^2})}{\log(\frac{\lambda}{\lambda+d_i^2})}-1, 1\leq i\leq p.
\end{equation}
On the other hand, we also expect that $k\geq k^*\geq 1$ such that we can achieve a minimum  before the MSE increases, then we also require that
\[\frac{\delta_i^2d_i^2}{\sigma^2}+(\frac{\lambda}{\lambda+d_i^2})^{k^*+1}\geq 1.\]
Since the MSE consists of $p$ terms in the summation, then it achieves the global minimum at $k$ which is between the minimum and the maximum integers of all the inflection points in the upper bounds of (\ref{k:upper}) for all $1\leq i\leq p$. This completes the proof of Theorem~\ref{thm7}(i).\\

(ii) The result of Theorem~\ref{thm7}(ii) is straightforward  from the inequality in (\ref{cond:iff}). We omit the details here. This completes the proof. $\Box$

	

\end{document}
	
	\begin{titlepage}
	
	\begin{center}
	{\Huge Internet Appendix for\\Understanding Timeline Systematic Risk: A Sparse Factor Approach in Time Horizons}
	\end{center}

	 \thispagestyle{empty}
		\vspace{0.5cm}
		
		\begin{abstract}
			The Internet Appendix collects the proofs and additional results that support the main text. 
			\vspace{1cm}
			
			\noindent\textbf{Keywords:} Factor Analysis, Principal Components, Sparsity, Large-Dimension, Panel Data	
			
			\noindent\textbf{JEL classification:} C14, C38, C55, G12
		\end{abstract}
	\end{titlepage}
	
	
	
	\setcounter{page}{1}
	
	\setcounter{section}{0}
	\setcounter{subsection}{0}
	
	\renewcommand{\thesection}{IA.\Alph{section}}
	\renewcommand{\thesubsection}{\thesection.\arabic{subsection}}
	
	\setcounter{equation}{0}
	\renewcommand{\theequation}{\thesection.\arabic{equation}}
	
	
	
	
	\renewcommand{\theequation}{IA.\arabic{equation}}	\renewcommand{\thefigure}{IA.\arabic{figure}} \setcounter{figure}{0}
	\renewcommand{\thetable}{IA.\Roman{table}} \setcounter{table}{0}
	
	
	
	
	
	
	
	

\section{Proofs of the Theorems}

 We will use $C$ or $c$ to denote a generic constant the value of which may change at different places.
\vskip 0.5cm

 {\bf Proof of Theorem \ref{thm1}.} To see the bias of the ridge estimator, it follows from (\ref{v:hlm}) that
\begin{equation}\label{decom}
    \wh\bbeta(\lambda)=\bbeta-\lambda(\bX'\bX+\lambda\bI_p)^{-1}\bbeta+(\bX'\bX+\lambda\bI_p)^{-1}\bX'\bve.
\end{equation}
Conditioning on $\bX$, by the assumption that $E\bve={\bf 0}$, it follows immediately that
\[\bbeta-E(\wh\bbeta(\lambda))=\lambda(\bX'\bX+\lambda\bI_p)^{-1}\bbeta.\]
This completes the proof. $\Box$
\vskip 0.5 cm
{\bf Proof of Theorem~\ref{thm2}}. By (\ref{decom}) an an elementary argument, $\wh\bbeta_{c,k}(k)$ can be written as
\begin{align}\label{btck}
   \wh\bbeta_{c,k}(\lambda)=&\wh\bbeta(\lambda)+\sum_{j=1}^{l}\lambda^j(\bX'\bX+\lambda\bI_p)^{-j}\wh\bbeta(\lambda)\notag\\
   =&\bbeta-\lambda^{k+1}(\bX'\bX+\lambda\bI_p)^{-(k+1)}\bbeta+\sum_{j=0}^k\lambda^j(\bX'\bX+\lambda\bI_p)^{-(j+1)}\bX'\bve.
\end{align}
Then, it follows that
\[\bbeta-E\wh\bbeta_{c,k}=\lambda^{k+1}(\bX'\bX+\lambda\bI_p)^{-(k+1)}\bbeta.\]
When $p\leq n$ and $\bX'\bX$ is invertible, the singular-value decomposition of $\bX'\bX$ is
\[\bX'\bX=\bU_1\bD_1^2\bU_1,\,\,\bD_1=\diag(d_1,...,d_p),\,\, d_j>0.\]
Then, it is not hard to see that
\[(\bX'\bX+\lambda\bI_p)^{-(k+1)}=\bU_1(\bD_1^2+\lambda\bI_p)^{-(k+1)}\bU_1',\]
and
\[\bb_{\lambda,k}=\lambda^{k+1}\bU_1(\bD_1^2+\lambda\bI_p)^{-(k+1)}\bU_1'\bbeta.\]
Note that each diagonal element in $\lambda^{k+1}(\bD_1^2+\lambda\bI_p)^{-(k+1)}$ is
\[\left(\frac{\lambda}{d_j^2+\lambda}\right)^{k+1}\rightarrow 0,\,\, \text{as $k\rightarrow\infty$},\]
at a rate of exponential decay. Therefore, for any given configuration of $(p,n)$ with $p\leq n$, we can show that
\[\|\bb_{\lambda}\|_2\rightarrow 0,\,\,\text{as $k\rightarrow\infty$}.\]
This completes the proof. $\Box$
\vskip 0.5cm

{\bf Proof of Theorem~\ref{thm3}}. Note that
\[\bb_{\lambda,k}=\lambda^{k+1}(\bX'\bX+\lambda\bI_p)^{-(k+1)}\bbeta.\]
When $p>n$, the singular-value decomposition of $(\bX'\bX+\lambda\bI_p)$ is
\[\bX'\bX+\lambda\bI_p=\bU\bD\bU',\,\,\bU=[\bU_1,\bU_2],\,\,\bD=\diag(\bD_1^2+\lambda\bI_{p^*},\lambda\bI_{p-p^*}).\]
Then, 
\begin{align}\label{blk2}
 \bb_{\lambda,k}=&\lambda^{k+1}\left\{\bU_1(\bD_1+\lambda\bI_{p^*})^{-(k+1)}\bU_1+\lambda^{-(k+1)}\bU_2\bU_2'\right\}\bbeta\notag\\
 =&\bU_1\lambda^{k+1}(\bD_1+\lambda\bI_{p^*})^{-(k+1)}\bU_1\bbeta+\bU_2\bU_2'\bbeta.
\end{align}
By a similar argument as that in the proof of Theorem~\ref{thm2}, the first term in (\ref{blk2}) vanishes as $k\rightarrow \infty$. Hence, 
\[ \bb_{\lambda,k}\rightarrow \bU_2\bU_2'\bbeta,\,\,\text{as $k\rightarrow\infty$}.\]
This completes the proof. $\Box$
\vskip 0.5cm

{\bf Proof of Theorem~\ref{thm4}}. We prove it by contradiction. Suppose $\bX'\bX$ is singular and there exists a transformation matrix $\bS$ such that
\begin{equation}\label{sy}
 E(\bS\bY)=\bU_2\bU_2\bbeta.
\end{equation}
In other words, the remaining bias term in Theorem~\ref{thm3} can be corrected by $\bS\bY$.
(\ref{sy}) implies that
\begin{equation}\label{syy}
E(\bS\bX)\bbeta=\bU_2\bU_2'\bbeta.    
\end{equation}
Since a $p$-dimensional space can be spanned by the columns of $[\bU_1,\bU_2]$, there exist vectors $\balpha_1\in R^{p^*}$ and $\balpha_2\in R^{p-p^*}$ such that
\[\bbeta=\bU_1\balpha_1+\bU_2\balpha_2.\]
We only talk about the case when $\balpha_1\neq{\bf 0}$ and $\balpha_2\neq 0 $ since either $\balpha_1={\bf 0}$ or $\balpha_2\neq 0 $ will lead the bias term to be zero, implying that we do not need to correct it.
We plug it into (\ref{syy}) and obtain
\begin{equation}\label{eq:sy}
 E(\bS\bX)\bU_1\balpha_1=\bU_2\balpha_2.
\end{equation}
Note that $\bU_1\balpha_1$ is a vector in the hyperplane spanned by the columns of $\bU_1$, and $\bU_2\balpha_2$ is a vector in the hyperplane spanned by the columns of $\bU_2$. Since $\bU_1$ is orthogonal to $\bU_2$, we cannot find any linear transformation matrix $\bS$ such that the equation of (\ref{eq:sy}) holds. This contradicts our assumption. This completes the proof. $\Box$
\vskip 0.5 cm

{\bf Proof of Theorem~\ref{thm5}}. (i) We first consider the case when $p\leq n$. By Theorem~\ref{thm2},
\[\wh\bbeta_{c,k}(\lambda)=\bbeta+\sum_{j=0}^k\lambda^j(\bX'\bX+\lambda\bI_p)^{-(j+1)}\bX'\bve+o(1)=\bbeta+\gamma+o(1),\]
where $o(1)$ can be arbitrarily small at an exponential rate. We use the SVD of $\bX'\bX$ when $p\leq n$ as that in the proof of Theorem~\ref{thm2} and obtain
\[\sum_{j=0}^k\lambda^{j+1}(\bX'\bX+\lambda\bI_p)^{-(j+1)}=\bU_1\bD_1^*\bU_1,\]
where $\bD_1^*=\diag(d_1^*,...,d_p^*)$ with
\[d_i^*=\sum_{j=0}^k\left(\frac{\lambda}{\lambda+d_i^2}\right)^{j+1}=\frac{\lambda}{d_i^2}+o(1)\asymp O(1),\]
where $o(1)$ is also an arbitrarily small term at an exponential rate, and $\lambda$ and $d_i^2$ are all of order $n$ by Assumption ?. Then, 
\[\bgamma=\bU_1\bD_1^*\bU_1'\bV\frac{\bD_1}{\sqrt{\lambda}}\bU_1'\frac{\bve}{\sqrt{\lambda}}.\]
Note that the $l_2$-norm of each row of $\bU_1\bD_1^*\bU_1'\bV\frac{\bD_1}{\sqrt{\lambda}}\bU_1'$ is
\[\|\be_i'\bU_1\bD_1^*\bU_1'\bV\frac{\bD_1}{\sqrt{\lambda}}\bU_1'\|_2^2\leq C<\infty,\]
where we used the fact that each element in the diagonal matrices $\bD_1^*$ and $\frac{\bD_1}{\sqrt{\lambda}}$ is bounded from above and below. By Assumption ?, we have
\[\max_{1\leq i\leq p}|\gamma_i|=O_p(\sqrt{\frac{1}{\lambda}}\sqrt{\frac{\log(p)}{n}})=O_p(\frac{\sqrt{\log(p)}}{n}).\]
For $i\in\mathcal{M}_0$, by Assumption ?
\[\min_{i\in\mathcal{M}_0}|\wh\beta_{c,k,i}(\lambda^*)|\geq \min_{i\in\mathcal{M}_0}|\beta_i|-O_p(\frac{\sqrt{\log(p)}}{n})\geq C n^{-\tau},\]
and
\[\max_{i\in\mathcal{M}_0^c}|\wh\beta_{c,k,i}|\leq C\frac{\sqrt{\log(p)}}{n}.\]
It is obvious that 
\[\min_{i\in\mathcal{M}_0}|\wh\beta_{c,k,i}(\lambda^*)|>\max_{i\in\mathcal{M}_0^c}|\wh\beta_{c,k,i}(\lambda^*)|,\]
with probability tending to one. This implies that
\[P(\mathcal{M}_0\subset\mathcal{M}_k(\lambda^*))\rightarrow 1,\,\, \text{as}\,\, k\rightarrow\infty.\]
Next, we consider the case when $p>n$. Note that Theorem~\ref{thm3} implies that
\begin{equation}\label{betk:rp}
    \wh\bbeta_{c,k}(\lambda)=\bbeta-\bU_2\bU_2'\bbeta+\sum_{j=0}^k\lambda^j(\bX'\bX+\lambda\bI_p)^{-(j+1)}\bX'\bve+o(1)=\bbeta-\bdelta+\bgamma+o(1),
\end{equation}
where the term $o(1)$ can be arbitrarily small at a rate of exponential decay as $k\rightarrow\infty$.
The idea of proving the result is as follows. We first investigate the lower bound of $(\min_{i\in\mathcal{M}_0}|\wh\beta_{c,k,i}(\lambda^*)|)^2$ and the upper bound of $\|\wh\bbeta_{c,k,i}(\lambda^*)\|_2^2$, if the number of coordinates selected in $\mathcal{M}_k(\lambda^*)$ is greater than 
the ratio between the upper bound and the lower bound, the submodel $\mathcal{M}_k(\lambda^*)$ must contain the true $\mathcal{M}_0$ as a subset.

For $i\in\mathcal{M}_0$, by Assumption ?, the magnitude of the $i$-th coordinate of $\bU_2\bU_2'\bbeta$ is less than that of $\bbeta$, therefore, 
\[\min_{i\in \mathcal{M}_0}|\wh\beta_{c,k,i}(\lambda)|\geq C_2n^{-\tau}-C_1n^{-\tau}-O_p(\sqrt{\log(p)}/n)\geq O_p(n^{-\tau}),\]
where $C_2>C_1>0$. In addition,
\[\|\wh\bbeta_{c,k}(\lambda^*)\|_2^2\leq C\|\bU_1\bU_1'\bbeta\|_2^2+C\|\sum_{j=0}^k\lambda^j(\bX'\bX+\lambda\bI_p)^{-(j+1)}\bX'\bve\|_2^2\leq Cp^*+Cn^{-2}p\log(p).\]
Then,
\begin{equation}\label{ratio}
    \frac{\|\wh\bbeta_{c,k}(\lambda^*)\|_2^2}{(\min_{i\in \mathcal{M}_0}|\wh\beta_{c,k,i}(\lambda)|)^2}\leq Cn^{-2\tau}p^*+Cn^{2\tau-2}p\log(p).
\end{equation}
If 
\[\frac{n^*}{Cn^{-2\tau}p^*+Cn^{2\tau-2}p\log(p)}\rightarrow\infty,\]
we must have that $\mathcal{M}_k(\lambda^*)$ consists of all the indexes in $\mathcal{M}_0$. This proves (i).

(ii) Since $\mathcal{M}_0\subset\mathcal{M}_k(\lambda^*)$ with probability tending to one, on the event of  $\{\mathcal{M}_0\subset\mathcal{M}_k(\lambda^*)\}$, the proof is the same as that for Theorem~\ref{thm2}. This completes the proof. $\Box$

\vskip 0.5cm
{\bf Proof of Theorem~\ref{thm6}}. By (\ref{btck}), we can obtain that
\[ \wh\bbeta_{c,k}(\lambda)-\bbeta=\sum_{j=0}^k\lambda^j(\bX'\bX+\lambda\bI_p)^{-(j+1)}\bX'\bve+o(1),\]
and
\[ \wh\bbeta_{\mathcal{M}_k,l}(\lambda)-\bbeta=\sum_{j=0}^l\lambda^j(\bX_{\mathcal{M}_k}'\bX_{\mathcal{M}_k}+\lambda\bI_p)^{-(j+1)}\bX_{\mathcal{M}_k}'\bve+o(1),\]
where $o(1)$ can be arbitrarily small at an exponential rate, and $k$ and $l$ are sufficiently large.
The results follow immediately. $\Box$





\end{document}